\documentclass[10pt, a5paper]{report}

\usepackage[T1]{fontenc}%
\usepackage[utf8]{inputenc}% encodage utf8
\usepackage[francais]{babel}% texte français
\usepackage{modules-livret}% style du livret
\usepackage{hyperref}
\usepackage{colortbl}

\usepackage{multicol}
%\setlength{\columnseprule}{0.5pt}
\setlength{\columnsep}{20pt}

% % % % % % % % % % % % % % % % % % % % % % % % % % % % % % % % % % % % % % % % % % % % % % % % % % % % % % % 
\begin{document}

%---------------------- % % % Personnalisation des couleurs % % % ----------- BLEU --------
\definecolor{couleurFonce}{RGB}{0,92,133} % Couleur du Code APOGEE
\definecolor{couleurClaire}{RGB}{100,151,186} % Couleur du fond de la bande
\definecolor{couleurTexte}{RGB}{255,255,255} % Couleur du texte de la bande
%------------------------------------------------------------------------------------------

\begin{enumerate}
\item Protocoles et sécurité:\\
Les attaques réseau. Les méthodes de protection (inetd, tcpwrappers, arp).
Protection des accès distants (PKI, SSO). Switch. VLAN. Routeurs.
Pare-feux, pare-feux applicatifs, proxy.
IPSec, VPN.
SSL, TLS, SSH.
NIDS.
\item Mobilité dans les réseaux sans fil de type 802.xx:\\
Réseaux mobiles et protocoles IP (IPV6, HMIP, LERS, SIP).
Réseaux mobiles adhoc (par exemple MANET).
Réseaux mobiles de type NEMO (Network Mobility).
\item Mobilité dans les réseaux téléphoniques:\\
Générations de mobiles (GSM, GPRS, 3G, 3G+, UMTS).
\item Sécurité des réseaux sans fils:\\
Sécurité des réseaux GSM (authentification et chiffrement, sécurité des cartes SIM, interception d'appel, sécurité des services de DATA (SMS, MMS)).
Intégrité, confidentialité et disponibilité des données sur les réseaux sans fil (filtrage des adresses MAC, WEP/WPA/WPA2, AAA : Radius, Portail Captif, VPN). 
\end{enumerate}


   blabla sur une colonne, c'est plus sérieux. C'est le
   style qui est généralement utilisé pour écrire des
   articles.
   blabla sur une colonne, c'est plus sérieux. C'est le
   style qui est généralement utilisé pour écrire des
   articles.
   blabla sur \href{http://www.free.fr}{http://www.free.fr}, c'est plus sérieux. C'est le
   style qui est généralement utilisé pour écrire des
   articles.
   blabla sur une colonne, c'est plus sérieux. C'est le
   style qui est généralement utilisé pour écrire des
   articles.

\begin{multicols}{3}[Titre sur une seule colonne blabla blabla blabla blabla blabla .]
   3~colonnes équilibrées, 3~colonnes équilibrées, 3~colonnes
   équilibrées, 3~colonnes équilibrées, 3~colonnes équilibrées,
   3~colonnes équilibrées, 3~colonnes équilibrées, 3~colonnes équilibrées
\end{multicols}

\begin{multicols}{2}[\section{Titre numéroté blabla blabla blabla blabla .}]
   blabla sur deux colonnes, c'est plus sérieux. C'est le
   style qui est généralement utilisé pour écrire des
   articles.
   blabla sur deux colonnes, c'est plus sérieux. C'est le
   style qui est généralement utilisé pour écrire des
   articles.
   blabla sur deux colonnes, c'est plus sérieux. C'est le
   style qui est généralement utilisé pour écrire des
   articles.
   blabla sur deux colonnes, c'est plus sérieux. C'est le
   style qui est généralement utilisé pour écrire des
   articles.
\end{multicols}

   blabla sur une colonne, c'est plus sérieux. C'est le
   style qui est généralement utilisé pour écrire des
   articles.
   blabla sur une colonne, c'est plus sérieux. C'est le
   style qui est généralement utilisé pour écrire des
   articles.
   blabla sur une colonne, c'est plus sérieux. C'est le
   style qui est généralement utilisé pour écrire des
   articles.
   blabla sur une colonne, c'est plus sérieux. C'est le
   style qui est généralement utilisé pour écrire des
   articles.


\arrayrulecolor{couleurFonce}% Couleur des lignes séparatrices du tableau
\renewcommand{\arraystretch}{1.2}% Coeff appliqué à la hauteur des cellules
\begin{tabular}{c|p{6cm}|p{1cm}|p{1cm}|p{1cm}|p{1cm}|}
\cline{2-6}

&
\cellcolor{couleurFonce} \color{white}\bfseries Intitul\'e & \cellcolor{couleurFonce} \color{white}\bfseries ECTS & \cellcolor{couleurFonce} \color{white}\bfseries CM & \cellcolor{couleurFonce} \color{white}\bfseries TD & \cellcolor{couleurFonce} \color{white}\bfseries TP\\ \cline{2-6}

\hline \multirow{7}{*}{\rotatebox{90}{\color{couleurFonce}\bfseries  SEMESTRE 1}}
 & \cellcolor{couleurClaire} \color{couleurTexte} Algorithmique et programmation 1 & \cellcolor{couleurClaire} 6 & \cellcolor{couleurClaire} 45 & \cellcolor{couleurClaire}  &  \cellcolor{couleurClaire} 15 \\ \cline{2-6}
 & Environnement informatique 1 & 3 & 24 &  &  \\ \cline{2-6}
 & \cellcolor{couleurClaire} Introduction au raisonnement mathématiques & \cellcolor{couleurClaire} 6 & \cellcolor{couleurClaire} 60 & \cellcolor{couleurClaire}  &  \cellcolor{couleurClaire}  \\ \cline{2-6}
 & Suites et fonctions réelles & 6 & 60 &  &  \\ \cline{2-6}
 & \cellcolor{couleurClaire} Arithmétique & \cellcolor{couleurClaire} 3 & \cellcolor{couleurClaire} 24 & \cellcolor{couleurClaire}  &  \cellcolor{couleurClaire}  \\ \cline{2-6}
 & Anglais 1 & 3 & 25 &  &  \\ \cline{2-6}
 & \cellcolor{couleurClaire} Unité d'ouverture & \cellcolor{couleurClaire} 3 & \cellcolor{couleurClaire}  & \cellcolor{couleurClaire} 22 &  \cellcolor{couleurClaire}  \\ \cline{2-6}
\hline \multirow{7}{*}{\rotatebox{90}{\color{couleurFonce}\bfseries  SEMESTRE 2}}
 & Algorithmique et programmation 2 & 6 & 60 &  &  \\ \cline{2-6}
 & \cellcolor{couleurClaire} Outils mathématiques pour l'informatique & \cellcolor{couleurClaire} 6 & \cellcolor{couleurClaire} 50 & \cellcolor{couleurClaire}  &  \cellcolor{couleurClaire}  \\ \cline{2-6}
 & Modélisation & 3 & 24 &  &  \\ \cline{2-6}
 & \cellcolor{couleurClaire} Projet informatique & \cellcolor{couleurClaire} 3 & \cellcolor{couleurClaire}  & \cellcolor{couleurClaire}  &  \cellcolor{couleurClaire} 24 \\ \cline{2-6}
 & Mathématiques & 6 & 60 &  &  \\ \cline{2-6}
 & \cellcolor{couleurClaire} Anglais 2 & \cellcolor{couleurClaire} 3 & \cellcolor{couleurClaire} 25 & \cellcolor{couleurClaire}  &  \cellcolor{couleurClaire}  \\ \cline{2-6}
 & Préparation au C2I & 3 &  & 22 &  \\ \cline{2-6}
\hline
\end{tabular}

\vspace{0.5cm}




\section{Licence 3 Informatique : Parcours MIAGE}

\arrayrulecolor{couleurFonce}% Couleur des lignes séparatrices du tableau
\renewcommand{\arraystretch}{1.2}% Coeff appliqué à la hauteur des cellules
%\rowcolors[\hline]{ligneDeÃÅbut}{couleurPaire}{couleurImpaire}% Alternance de couleur (need package xcolor)
\begin{tabular}{c|m{6cm}|cm{1cm}|cm{1cm}|cm{1cm}|cm{1cm}|}
\cline{2-6}

&
\cellcolor{couleurFonce} \color{white}\bfseries Intitul\'e & \cellcolor{couleurFonce} \color{white}\bfseries ECTS & \cellcolor{couleurFonce} \color{white}\bfseries CM & \cellcolor{couleurFonce} \color{white}\bfseries TD & \cellcolor{couleurFonce} \color{white}\bfseries TP\\ \cline{2-6}

\hline \multirow{10}{*}{\rotatebox{90}{\color{couleurFonce}\bfseries  SEMESTRE 3}}
 & \cellcolor{couleurClaire} \color{couleurTexte} \mbox{Algorithmique} \mbox{et} \mbox{programmation} \mbox{3} \mbox{(programmation} \mbox{orientée} \mbox{objet)}  & \cellcolor{couleurClaire} \color{couleurTexte} 5 & \cellcolor{couleurClaire} \color{couleurTexte} 20 & \cellcolor{couleurClaire} \color{couleurTexte} 40 & \cellcolor{couleurClaire} \color{couleurTexte}  \\ \cline{2-6}
 & \color{black} \mbox{Bases} \mbox{de} \mbox{données} \mbox{et} \mbox{Internet}  & \color{black} 4 & \color{black} 10 & \color{black}  & \color{black} 40CTP \\ \cline{2-6}
 & \cellcolor{couleurClaire} \color{couleurTexte} \mbox{Environnement} \mbox{informatique} \mbox{2} \mbox{(Atelier} \mbox{de} \mbox{l’informaticien)}  & \cellcolor{couleurClaire} \color{couleurTexte} 4 & \cellcolor{couleurClaire} \color{couleurTexte}  & \cellcolor{couleurClaire} \color{couleurTexte}  & \cellcolor{couleurClaire} \color{couleurTexte} 40CTP \\ \cline{2-6}
 & \color{black} \mbox{Architecture} \mbox{des} \mbox{ordinateurs}  & \color{black} 4 & \color{black} 8 & \color{black} 16 & \color{black} 6 \\ \cline{2-6}
 & \cellcolor{couleurClaire} \color{couleurTexte} \mbox{Application} \mbox{de} \mbox{l'algèbre}  & \cellcolor{couleurClaire} \color{couleurTexte} 5 & \cellcolor{couleurClaire} \color{couleurTexte}  & \cellcolor{couleurClaire} \color{couleurTexte} 50CTD & \cellcolor{couleurClaire} \color{couleurTexte}  \\ \cline{2-6}
 & \color{black} \mbox{Anglais} \mbox{3}  & \color{black} 3 & \color{black}  & \color{black} 25CTD & \color{black}  \\ \cline{2-6}
 & \cellcolor{couleurClaire} \color{couleurTexte} \mbox{Unité} \mbox{d'ouverture}  & \cellcolor{couleurClaire} \color{couleurTexte} 3 & \cellcolor{couleurClaire} \color{couleurTexte}  & \cellcolor{couleurClaire} \color{couleurTexte} -22 & \cellcolor{couleurClaire} \color{couleurTexte}  \\ \cline{2-6}
 & \color{black} \mbox{Projet} \mbox{personnel} \mbox{et} \mbox{professionnel}  & \color{black} 2 & \color{black} 12 & \color{black}  & \color{black}  \\ \cline{2-6}
\hline \multirow{10}{*}{\rotatebox{90}{\color{couleurFonce}\bfseries  SEMESTRE 4}}
 & \cellcolor{couleurClaire} \color{couleurTexte} \mbox{Programmation} \mbox{fonctionnelle}  & \cellcolor{couleurClaire} \color{couleurTexte} 6 & \cellcolor{couleurClaire} \color{couleurTexte} 20 & \cellcolor{couleurClaire} \color{couleurTexte} 40 & \cellcolor{couleurClaire} \color{couleurTexte}  \\ \cline{2-6}
 & \color{black} \mbox{Algorithmique} \mbox{et} \mbox{combinatoire} \mbox{des} \mbox{structures} \mbox{discrètes}  & \color{black} 6 & \color{black} 20 & \color{black} 30 & \color{black}  \\ \cline{2-6}
 & \cellcolor{couleurClaire} \color{couleurTexte} \mbox{Projet} \mbox{informatique} \mbox{2} \mbox{(Conception} \mbox{et} \mbox{projet)}  & \cellcolor{couleurClaire} \color{couleurTexte} 5 & \cellcolor{couleurClaire} \color{couleurTexte} 10 & \cellcolor{couleurClaire} \color{couleurTexte}  & \cellcolor{couleurClaire} \color{couleurTexte} 30 \\ \cline{2-6}
 & \color{black} \mbox{Probabilités}  & \color{black} 5 & \color{black}  & \color{black} 40CTD & \color{black} 30 \\ \cline{2-6}
 & \cellcolor{couleurClaire} \color{couleurTexte} \mbox{Anglais} \mbox{4}  & \cellcolor{couleurClaire} \color{couleurTexte} 3 & \cellcolor{couleurClaire} \color{couleurTexte}  & \cellcolor{couleurClaire} \color{couleurTexte} 25CTD & \cellcolor{couleurClaire} \color{couleurTexte}  \\ \cline{2-6}
 & \color{black} \mbox{Bases} \mbox{du} \mbox{système} \mbox{comptable}  & \color{black} 5 & \color{black}  & \color{black} 30CTD & \color{black}  \\ \cline{2-6}
 & \cellcolor{couleurClaire} \color{couleurTexte} \mbox{Programmation} \mbox{impérative}  & \cellcolor{couleurClaire} \color{couleurTexte} 5 & \cellcolor{couleurClaire} \color{couleurTexte} 10 & \cellcolor{couleurClaire} \color{couleurTexte} 20 & \cellcolor{couleurClaire} \color{couleurTexte}  \\ \cline{2-6}
\hline
\end{tabular}

%\begin{tabularx}{15cm}{|c|p{4cm}|X|}
%    \hline
%    1 & The two gentlemen of Verona & The Taming of the Shrew \\
%    \hline
%    2 & The comedy of errors & Love's Labour's Lost \\
%    \hline
%    3 & The Merchant of venice & The Merry Wives of Windsor \\
%    \hline
%\end{tabularx}

%\begin{tabular}{l >{\columncolor{lightgray}} l l}
%    1 &   & 3 \\
%    a &   & c \\
%    I & \multirow{-3}{1em}{2 A II} & III
% \end{tabular}
 
\end{document}
