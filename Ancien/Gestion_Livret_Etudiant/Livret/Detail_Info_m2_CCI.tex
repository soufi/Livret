%---------------------- % % % Personnalisation des couleurs % % % ----------- ROUGE --------
\definecolor{couleurFonce}{RGB}{160,0,10} % Couleur du Code APOGEE
\definecolor{couleurClaire}{RGB}{225,135,140} % Couleur du fond de la bande
\definecolor{couleurTexte}{RGB}{255,255,255} % Couleur du texte de la bande
%------------------------------------------------------------------------------------------


%==========================================================================================
% Semestre 3
%==========================================================================================
\module[codeApogee={UE 30},
titre={Initiation}, 
CODEUE={1}, 
COURS={8}, 
TD={}, 
TP={8}, 
CTD={}, 
TOTAL={16}, 
SEMESTRE={Semestre 3}, 
COEFF={0}, 
ECTS={0}, 
MethodeEval={Contrôle continue et terminal}, 
ModalitesCCSemestreUn={CC et CT}, 
ModalitesCCSemestreDeux={CT}, 
%CalculNFSessionUne={$\frac{(CC+2*CT)}{3}$}, 
%CalculNFSessionDeux={CT}, 
NoteEliminatoire={7}, 
nomPremierResp={Frédéric Dabrowski}, 
emailPremierResp={frederic.dabrowski@univ-orleans.fr}, 
nomSecondResp={}, 
emailSecondResp={}, 
langue={Français}, 
nbPrerequis={0}, 
descriptionCourte={true}, 
descriptionLongue={true}, 
objectifs={true}, 
ressources={true}, 
bibliographie={false}] 
{
Unité obligatoire. 
} 
{
\begin{itemize} 
  \item Fondements de l\'informatique : codage des données, traitements, structure générale d\'un ordinateur
  \item Utilisation des systèmes d\'exploitation de type Unix.
  \item Outils logiciels bureautique et internet. 
\end{itemize} 
} 
{} 
{\begin{itemize} 
  \ObjItem Remettre à un même niveau de base tous les étudiants et les familiariser avec le fonctionnement des ordinateurs et les outils informatiques usuels.
\end{itemize} 
} 
%{Adresse de la plateforme de cours en ligne de la formation : \url{celene.univ-orleans.fr/abc}} 
{Ressources}
{Biblio}
 
\vfill

%==========================================================================================
\module[codeApogee={UE 31},
titre={Algorithmique}, 
CODEUE={1}, 
COURS={15}, 
TD={15}, 
TP={}, 
CTD={}, 
TOTAL={30}, 
SEMESTRE={Semestre 3}, 
COEFF={3}, 
ECTS={3}, 
MethodeEval={Contrôle continue et terminal}, 
ModalitesCCSemestreUn={CC et CT}, 
ModalitesCCSemestreDeux={CT}, 
%CalculNFSessionUne={$\frac{(CC+2*CT)}{3}$}, 
%CalculNFSessionDeux={CT}, 
NoteEliminatoire={7}, 
nomPremierResp={Bich DAO}, 
emailPremierResp={Bich.DAO@univ-orleans.fr}, 
nomSecondResp={}, 
emailSecondResp={}, 
langue={Français}, 
nbPrerequis={1}, 
descriptionCourte={true}, 
descriptionLongue={true}, 
objectifs={true}, 
ressources={true}, 
bibliographie={false}] 
{
Unité obligatoire. 
} 
{
\begin{itemize} 
  \item Notions : variables, affectation, conditionnelle, itération, récursion.
  \item Algorithmes simples sur les tableaux.
  \item Algorithmes de tri. 
\end{itemize} 
} 
{Mathématiques élémentaires} 
{\begin{itemize} 
  \ObjItem Comprendre le fonctionnement d\'un algorithme donné.
  \ObjItem Concevoir des algorithmes pour un problème simple donné.
  \ObjItem Savoir utiliser des structures de données.
\end{itemize} 
} 
{Ressources} 
{Biblio} 
 
\vfill

%==========================================================================================
\module[codeApogee={UE 32},
titre={Bases de données}, 
CODEUE={1}, 
COURS={20}, 
TD={25}, 
TP={25}, 
CTD={}, 
TOTAL={70}, 
SEMESTRE={Semestre 3}, 
COEFF={6}, 
ECTS={6}, 
MethodeEval={Contrôle continue et terminal}, 
ModalitesCCSemestreUn={CC et CT}, 
ModalitesCCSemestreDeux={CT}, 
%CalculNFSessionUne={$\frac{(CC+2*CT)}{3}$}, 
%CalculNFSessionDeux={CT}, 
NoteEliminatoire={7}, 
nomPremierResp={Nicolas OLLINGER}, 
emailPremierResp={Nicolas.OLLINGER@univ-orleans.fr}, 
nomSecondResp={}, 
emailSecondResp={}, 
langue={Français}, 
nbPrerequis={1}, 
descriptionCourte={true}, 
descriptionLongue={true}, 
objectifs={true}, 
ressources={true}, 
bibliographie={false}] 
{
Unité obligatoire. 
} 
{
\begin{itemize} 
  \item Structure fonctionnelle et Architecture d\'un SGBD.
  \item Problématique de la modélisation logique des données.
  \item Modèle relationnel, Langage SQL.
  \item Interrogation de données du modèle relationnel. 
\end{itemize} 
} 
{Module Initiation} 
{\begin{itemize} 
  \ObjItem Donner aux étudiants les grandes lignes des diverses fonctionnalités d\'un SGBD relationnel.   
  \ObjItem Introduire les techniques de modélisation de données.  
  \ObjItem Apprendre et maîtriser le langage de manipulation et d\'interrogation de bases de données : SQL.
\end{itemize} 
} 
{Ressources} 
{Biblio} 
 
\vfill

%==========================================================================================
\module[codeApogee={UE 33}, 
titre={Systèmes}, 
CODEUE={1}, 
COURS={10}, 
TD={10}, 
TP={15}, 
CTD={}, 
TOTAL={35}, 
SEMESTRE={Semestre 3}, 
COEFF={3}, 
ECTS={3}, 
MethodeEval={Contrôle continue et terminal}, 
ModalitesCCSemestreUn={CC et CT}, 
ModalitesCCSemestreDeux={CT}, 
%CalculNFSessionUne={$\frac{(CC+2*CT)}{3}$}, 
%CalculNFSessionDeux={CT}, 
NoteEliminatoire={7}, 
nomPremierResp={Sophie ROBERT}, 
emailPremierResp={Sophie.ROBERT@univ-orleans.fr}, 
nomSecondResp={}, 
emailSecondResp={}, 
langue={Français}, 
nbPrerequis={1}, 
descriptionCourte={true}, 
descriptionLongue={true}, 
objectifs={true}, 
ressources={true}, 
bibliographie={false}] 
{
Unité obligatoire. 
} 
{
\begin{itemize} 
  \item Architecture de systèmes d\'exploitation.
  \item Utilisation d\'Unix.
  \item Administration Unix. 
\end{itemize} 
} 
{Module Initiation} 
{\begin{itemize} 
  \ObjItem Utilisation et administration de systèmes d\'exploitation.
\end{itemize} 
} 
{Ressources} 
{Biblio} 
 
\vfill

%==========================================================================================
\module[codeApogee={UE 34}, 
titre={Réseaux}, 
CODEUE={1}, 
COURS={10}, 
TD={10}, 
TP={15}, 
CTD={}, 
TOTAL={35}, 
SEMESTRE={Semestre 3}, 
COEFF={3}, 
ECTS={3}, 
MethodeEval={Contrôle continue et terminal}, 
ModalitesCCSemestreUn={CC et CT}, 
ModalitesCCSemestreDeux={CT}, 
%CalculNFSessionUne={$\frac{(CC+2*CT)}{3}$}, 
%CalculNFSessionDeux={CT}, 
NoteEliminatoire={7}, 
nomPremierResp={Nicolas OLLINGER}, 
emailPremierResp={Nicolas.OLLINGER@univ-orleans.fr}, 
nomSecondResp={}, 
emailSecondResp={}, 
langue={Français}, 
nbPrerequis={1}, 
descriptionCourte={true}, 
descriptionLongue={true}, 
objectifs={true}, 
ressources={true}, 
bibliographie={false}] 
{
Unité obligatoire. 
} 
{
\begin{itemize} 
  \item Architecture des réseaux : structure en couches, protocoles, services.
  \item Réseaux locaux sous UDP-TCP/IP, Ethernet.
  \item Protocoles de routage : RIP, OSPF, BGP.
  \item Principaux protocoles Internet : DNS (annuaire de noms de domaines) SMTP (mail), FTP (transfert de fichiers), HTTP (web) ... 
\end{itemize} 
} 
{Module Initiation}
{\begin{itemize} 
  \ObjItem Principes et pratique des réseaux locaux informatiques.
\end{itemize} 
} 
{Ressources} 
{Biblio} 
 
\vfill

%==========================================================================================
\module[codeApogee={UE 35}, 
titre={Programmation objet 1}, 
CODEUE={1}, 
COURS={20}, 
TD={25}, 
TP={25}, 
CTD={}, 
TOTAL={70}, 
SEMESTRE={Semestre 3}, 
COEFF={6}, 
ECTS={6}, 
MethodeEval={Contrôle continue et terminal}, 
ModalitesCCSemestreUn={CC et CT}, 
ModalitesCCSemestreDeux={CT}, 
%CalculNFSessionUne={$\frac{(CC+2*CT)}{3}$}, 
%CalculNFSessionDeux={CT}, 
NoteEliminatoire={7}, 
nomPremierResp={Bich DAO}, 
emailPremierResp={Bich.DAO@univ-orleans.fr}, 
nomSecondResp={}, 
emailSecondResp={}, 
langue={Français}, 
nbPrerequis={1}, 
descriptionCourte={true}, 
descriptionLongue={true}, 
objectifs={true}, 
ressources={true}, 
bibliographie={false}] 
{
Unité obligatoire. 
} 
{
\begin{itemize} 
  \item Introduction à la programmation, algorithmes de base
  \item Programmation objet: classe, objet, état, méthode, communication entre objets.
  \item Récurrence. 
  \item Tableaux, Tris.
  \item Interfaces.
  \item Héritage.
  \item Exceptions.
  \item Entrées/sorties.
  \item Introduction à la notation UML. 
\end{itemize} 
} 
{Module Initiation} 
{\begin{itemize} 
  \ObjItem Introduction à la programmation objet dans un langage impératif. 
  \ObjItem Mise en oeuvre d\'algorithmes dans ce langage. 
\end{itemize} 
} 
{Ressources} 
{Biblio} 
 
\vfill

%==========================================================================================
\module[codeApogee={UE 36}, 
titre={Projet 1}, 
CODEUE={1}, 
COURS={}, 
TD={}, 
TP={}, 
CTD={}, 
TOTAL={0}, 
SEMESTRE={Semestre 3}, 
COEFF={3}, 
ECTS={3}, 
%MethodeEval={Contrôle continue et terminal}, 
ModalitesCCSemestreUn={Rapport et soutenance de projet}, 
ModalitesCCSemestreDeux={Pas de 2nde session}, 
%CalculNFSessionUne={$\frac{(CC+2*CT)}{3}$}, 
%CalculNFSessionDeux={CT}, 
NoteEliminatoire={7}, 
nomPremierResp={Laure KAHLEM}, 
emailPremierResp={Laure.KAHLEM@univ-orleans.fr}, 
nomSecondResp={}, 
emailSecondResp={}, 
langue={Français}, 
nbPrerequis={1}, 
descriptionCourte={true}, 
descriptionLongue={true}, 
objectifs={true}, 
ressources={true}, 
bibliographie={false}] 
{
Unité obligatoire. 
} 
{
Réalisation d\'une applications mettant en oeuvre les notions vues dans les modules Algorithmique, Programmation objet 1, Bases de données, Systèmes et Réseaux. 
} 
{Module Initiation} 
{\begin{itemize} 
  \ObjItem Mise en oeuvre dans un exemple d\'application réaliste des concepts vus dans les modules informatique de la période considérée.
\end{itemize} 
} 
{Ressources} 
{Biblio} 
 
\vfill

%==========================================================================================
\module[codeApogee={UE 37}, 
titre={Simulation de  gestion d\'entreprise}, 
CODEUE={1}, 
COURS={}, 
TD={24}, 
TP={}, 
CTD={}, 
TOTAL={24}, 
SEMESTRE={Semestre 3}, 
COEFF={3}, 
ECTS={3}, 
MethodeEval={Contrôle continue et terminal}, 
ModalitesCCSemestreUn={CC et CT}, 
ModalitesCCSemestreDeux={CT}, 
%CalculNFSessionUne={$\frac{(CC+2*CT)}{3}$}, 
%CalculNFSessionDeux={CT}, 
NoteEliminatoire={7}, 
nomPremierResp={Chaker HAOUET}, 
emailPremierResp={Chaker.HAOUET@univ-orleans.fr}, 
nomSecondResp={}, 
emailSecondResp={}, 
langue={Français}, 
nbPrerequis={1}, 
descriptionCourte={true}, 
descriptionLongue={true}, 
objectifs={true}, 
ressources={true}, 
bibliographie={false}] 
{
Unité obligatoire. 
} 
{
Simulation visant à amener les groupes (chacun représentant une entreprise en concurrence avec les autres), après avoir formalisé une stratégie, 
à prendre des décisions d\'ordre commercial, de production, d\'investissement et de financement. Dans ce cadre, ils mettent en \oe uvre la 
plupart des outils financiers et prévisionnels connus. 
} 
{Le jeu fait appel aux connaissances des étudiants ainsi qu\'à la réflexion et l\'imagination.} 
{\begin{itemize} 
  \ObjItem Au terme de cette simulation, les étudiants doivent pouvoir gérer le temps, travailler en groupe, gérer les conflits, ... 
  \ObjItem Connaitre le monde de l\'entreprise.
\end{itemize} 
} 
{Ressources} 
{Biblio} 
 
\vfill

%==========================================================================================
\module[codeApogee={UE 38}, 
titre={Anglais}, 
CODEUE={1}, 
COURS={}, 
TD={20}, 
TP={}, 
CTD={}, 
TOTAL={20}, 
SEMESTRE={Semestre 3}, 
COEFF={3}, 
ECTS={3}, 
MethodeEval={Contrôle continue et terminal}, 
ModalitesCCSemestreUn={CC et CT}, 
ModalitesCCSemestreDeux={CT}, 
%CalculNFSessionUne={$\frac{(CC+2*CT)}{3}$}, 
%CalculNFSessionDeux={CT}, 
NoteEliminatoire={7}, 
nomPremierResp={Cédric SARRE}, 
emailPremierResp={Cedric.SARRE@univ-orleans.fr}, 
nomSecondResp={}, 
emailSecondResp={}, 
langue={Français}, 
nbPrerequis={1}, 
descriptionCourte={true}, 
descriptionLongue={true}, 
objectifs={true}, 
ressources={true}, 
bibliographie={false}] 
{
Unité obligatoire. 
} 
{
Etude des techniques de présentation orale : amélioration de la prononciation, organisation du discours, guidage de l\'auditoire, élaboration d\'aides visuelles. 
} 
{Anglais non professionnel} 
{\begin{itemize} 
  \ObjItem S\'exprimer couramment et efficacement dans une langue riche, souple et nuancée dans le domaine de la spécialité, exposer son opinion de manière claire, détaillée et structurée.
\end{itemize} 
} 
{Ressources} 
{Biblio} 
 
\vfill

%==========================================================================================
% Semestre 3
%==========================================================================================
\module[codeApogee={UE 41}, 
titre={Applications internet}, 
CODEUE={1}, 
COURS={20}, 
TD={24}, 
TP={24}, 
CTD={}, 
TOTAL={68}, 
SEMESTRE={Semestre 4}, 
COEFF={5}, 
ECTS={5}, 
MethodeEval={Contrôle continue et terminal}, 
ModalitesCCSemestreUn={CC et CT}, 
ModalitesCCSemestreDeux={CT}, 
%CalculNFSessionUne={$\frac{(CC+2*CT)}{3}$}, 
%CalculNFSessionDeux={CT}, 
NoteEliminatoire={7}, 
nomPremierResp={AbdelAli ED-DBALI}, 
emailPremierResp={AbdelAli.ED-DBALI@univ-orleans.fr}, 
nomSecondResp={}, 
emailSecondResp={}, 
langue={Français}, 
nbPrerequis={0}, 
descriptionCourte={true}, 
descriptionLongue={true}, 
objectifs={true}, 
ressources={true}, 
bibliographie={false}] 
{
Unité obligatoire. 
} 
{
\begin{itemize} 
  \item Les langages HTML, XHTML, feuilles de style CSS.
  \item Applications réparties : le modèle client/serveur.
  \item Langages pour les pages web dynamiques : PHP, MySQL
  \item XML. 
\end{itemize} 
} 
{} 
{\begin{itemize} 
  \ObjItem Développer et maintenir des sites et applications intra/internet.
\end{itemize} 
} 
{Ressources} 
{Biblio} 
 
\vfill

%==========================================================================================
\module[codeApogee={UE 42}, 
titre={Génie logiciel}, 
CODEUE={1}, 
COURS={20}, 
TD={24}, 
TP={24}, 
CTD={}, 
TOTAL={68}, 
SEMESTRE={Semestre 4}, 
COEFF={5}, 
ECTS={5}, 
MethodeEval={Contrôle continue et terminal}, 
ModalitesCCSemestreUn={CC et CT}, 
ModalitesCCSemestreDeux={CT}, 
%CalculNFSessionUne={$\frac{(CC+2*CT)}{3}$}, 
%CalculNFSessionDeux={CT}, 
NoteEliminatoire={7}, 
nomPremierResp={Jean-Michel COUVREUR}, 
emailPremierResp={Jean-Michel.COUVREUR@univ-orleans.fr}, 
nomSecondResp={}, 
emailSecondResp={}, 
langue={Français}, 
nbPrerequis={0}, 
descriptionCourte={true}, 
descriptionLongue={true}, 
objectifs={true}, 
ressources={true}, 
bibliographie={false}] 
{
Unité obligatoire. 
} 
{
\begin{itemize} 
  \item Modélisation UML : diagrammes de classes, de séquences, d\'états-transition et de cas d\'utilisation.
  \item Pratique d\'un atelier de génie logiciel.
  \item Méthodologie d\'analyse et de conception objet.
  \item Introduction aux patrons de conception (design patterns).
  \item Introduction à la gestion de projets.
\end{itemize} 
} 
{} 
{\begin{itemize} 
  \ObjItem Acquisition des connaissances de bases d\'UML, d\'une méthodologie permettant d\'analyser un problème, d\'en réaliser la modélisation, puis d\'élaborer une mise en oeuvre sous forme d\'une application informatique. 
  \ObjItem Connaissance des notions de la gestion de projets. 
\end{itemize} 
} 
{Ressources} 
{Biblio} 
 
\vfill

%==========================================================================================
\module[codeApogee={UE 43}, 
titre={Programmation objet 2}, 
CODEUE={1}, 
COURS={20}, 
TD={24}, 
TP={24}, 
CTD={}, 
TOTAL={68}, 
SEMESTRE={Semestre 4}, 
COEFF={5}, 
ECTS={5}, 
MethodeEval={Contrôle continue et terminal}, 
ModalitesCCSemestreUn={CC et CT}, 
ModalitesCCSemestreDeux={CT}, 
%CalculNFSessionUne={$\frac{(CC+2*CT)}{3}$}, 
%CalculNFSessionDeux={CT}, 
NoteEliminatoire={7}, 
nomPremierResp={Laure KAHLEM}, 
emailPremierResp={Laure.KAHLEM@univ-orleans.fr}, 
nomSecondResp={}, 
emailSecondResp={}, 
langue={Français}, 
nbPrerequis={1}, 
descriptionCourte={true}, 
descriptionLongue={true}, 
objectifs={true}, 
ressources={true}, 
bibliographie={false}] 
{
Unité obligatoire. 
} 
{
\begin{itemize} 
  \item Généricité.
  \item Classes internes.
  \item Implantation de structures de données.
  \item Collections des bibliothèques standards.
  \item Programmation événementielle.
  \item Interface graphique. 
\end{itemize} 
} 
{Modules Algorithmique, Programmation objet 1} 
{\begin{itemize} 
  \ObjItem Savoir développer une application avec un langage orienté objet utilisant des structures de données récursives, interface graphique et interface avec une base de donnée.
\end{itemize} 
} 
{Ressources} 
{Biblio} 
 
\vfill

%==========================================================================================
\module[codeApogee={UE 44}, 
titre={Projet 2}, 
CODEUE={1}, 
COURS={}, 
TD={}, 
TP={}, 
CTD={}, 
TOTAL={0}, 
SEMESTRE={Semestre 4}, 
COEFF={5}, 
ECTS={5}, 
%MethodeEval={}, 
ModalitesCCSemestreUn={Rapport et soutenance de projet}, 
ModalitesCCSemestreDeux={Pas de 2nde session}, 
%CalculNFSessionUne={$\frac{(CC+2*CT)}{3}$}, 
%CalculNFSessionDeux={CT}, 
NoteEliminatoire={7}, 
nomPremierResp={AbdelAli ED-DBALI}, 
emailPremierResp={AbdelAli.ED-DBALI@univ-orleans.fr}, 
nomSecondResp={}, 
emailSecondResp={}, 
langue={Français}, 
nbPrerequis={0}, 
descriptionCourte={true}, 
descriptionLongue={true}, 
objectifs={true}, 
ressources={true}, 
bibliographie={false}] 
{
Unité obligatoire. 
} 
{
Développer une application web en mettant en \oe uvre les concepts appris en génie logiciel et applications internet. 
} 
{} 
{\begin{itemize} 
  \ObjItem Mise en oeuvre dans un exemple d\'application réaliste des concepts vus dans les modules informatique de la période considérée.
\end{itemize} 
} 
{Ressources} 
{Biblio} 
 
\vfill

%==========================================================================================
\module[codeApogee={UE 45}, 
titre={Stage}, 
CODEUE={1}, 
COURS={10}, 
TD={}, 
TP={}, 
CTD={}, 
TOTAL={10}, 
SEMESTRE={Semestre 4}, 
COEFF={10}, 
ECTS={10}, 
%MethodeEval={}, 
ModalitesCCSemestreUn={Rapport et soutenance de stage}, 
ModalitesCCSemestreDeux={Pas de 2nde session}, 
%CalculNFSessionUne={$\frac{(CC+2*CT)}{3}$}, 
%CalculNFSessionDeux={CT}, 
NoteEliminatoire={7}, 
nomPremierResp={Nicolas OLLINGER}, 
emailPremierResp={Nicolas.OLLINGER@univ-orleans.fr}, 
nomSecondResp={}, 
emailSecondResp={}, 
langue={Français}, 
nbPrerequis={0}, 
descriptionCourte={true}, 
descriptionLongue={true}, 
objectifs={true}, 
ressources={true}, 
bibliographie={false}] 
{
Unité obligatoire.
} 
{
Stage de 4 à 6 mois consécutifs dans une entreprise, suivi par un enseignant et donnant lieu à la rédaction d\'un mémoire puis 
d\'une soutenance en présence d\'un jury mixte d\'enseignants et de responsables de l\'entreprise.\\
Le sujet est variable selon le stage. 
} 
{} 
{\begin{itemize} 
  \ObjItem Permettre d\'une part de mettre en pratique les enseignements dispensés pendant l\'année et d\'autre part de familiariser 
  les étudiants avec la vie professionnelle afin de permettre leur insertion.
\end{itemize} 
} 
{Ressources} 
{Biblio} 
 
\vfill

%==========================================================================================
\end{document}
