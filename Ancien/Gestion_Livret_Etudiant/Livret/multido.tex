%%% Save file as: MULTIDO.TEX/MULTIDO.STY      Source: FILESERV@SHSU.BITNET
%%% NOTE: The documentation for this TeX macro/LaTeX style option appears
%%%       in multido.doc, a LaTeX file which follows \endinput
%%% The following checksum includes all SHSU-generated topmatter
%%% checksum = "61709 563 2028 19405"
%% BEGIN multido.tex/multido.sty
%%
\def\fileversion{1.3}
\def\filedate{92/11/06}
%%
%% COPYRIGHT 1992, by Timothy Van Zandt, tvz@Princeton.EDU
%%
%% DESCRIPTION:
%%   multido.tex/multido.sty contains a loop macro, \multido, that supports
%%   fixed-point addition and has a nice interface. Among other things, it
%%   is useful for pictures and putting numbers on axes. multido.tex
%%   is compatible with most TeX macro packages, including Plain TeX, LaTeX,
%%   AmSTeX, and Ams-LaTeX.
%%
%% INSTALLATION:
%%   Put this file where your TeX looks for inputs, under the name multido.tex.
%%   Name a copy multido.sty to use as a LaTeX style option, or create a file
%%   multido.sty with the lines:
%%     \input multido.tex
%%     \endinput
%%
%% DOCUMENTATION:
%%   See multido.doc, which might be appended to this file.
%%
%% COPYING:
%%   Copying of part or all of this file is allowed under the following
%%   conditions only:
%%   (1) You may freely distribute unchanged copies of the file. Please
%%       include the documentation when you do so.
%%   (2) You may modify a renamed copy of the file, but only for personal
%%       use or use within an organization.
%%   (3) You may copy fragments from the file, for personal use or for
%%       distribution, as long as credit is given where credit is due.
%%
%%   You are NOT ALLOWED to take money for the distribution or use of
%%   this file or modified versions or fragments thereof, except for
%%   a nominal charge for copying etc.
%%
%% CODE:
% These macros use TeX primitives, plus the Plain TeX commands:
%   \dimen@, \dimen@i, \count@, \newcount, \newtoks, \@ne, \tw@, \@z@.
%
% Check if file is loaded, announce file on terminal, and take care of @:
%
\expandafter\ifx\csname multido@\endcsname\relax\else\expandafter\endinput\fi
\message{\space\space v\fileversion\space\space\filedate\space\space <tvz>}
\edef\MultidoAtCode{\the\catcode`\@}\catcode`\@=11
%
% These are some hacks borrowed from PSTricks, which are loaded if PSTricks is
% not being used. "\@dimtonum" strips the value of "#1", a dimension register,
% of the "pt", and assigns the result to "#2", a command sequence.
% 
\expandafter\ifx\csname @pstrickserr\endcsname\relax
  \def\@empty{}
  \def\@nnil{\@nil}
  \def\@dimtonum#1#2{\edef#2{\@@dimtonum#1}}
  \def\@@dimtonum#1{\expandafter\@@@dimtonum\the#1}
  {\let\@nnil\expandafter\catcode`\p=12\catcode`\t=12
  \global\@nnil\def\csname @@@dimtonum\endcsname#1pt{#1}}
\fi
%
% "\multido@count" stores the number of repetitions. "\multidocount" keeps
% track of the iteration. These are also used locally as scratch counters by
% "\FPadd@". "\multido@stuff" is used to store the <stuff> that is to be
% repeated.
%
\newcount\multido@count
\newcount\multidocount
\newtoks\multido@stuff
%
\def\multido{\multido@{}{\begingroup}{\endgroup}}
\def\mmultido{\multido@{\multido@stepvar}{\begingroup}{\endgroup}}
\def\Multido{\multido@{}{}{}}
\def\MMultido{\multido@{\multido@stepvar}{}{}}
%
% "\multido@initvar" processes the variable declarations, initializing the
% value of the variables and defining "\multido@stepvar" to increment the
% variables with each repetition. "\multido@count" is set to the number of
% repetitions.
%
\long\def\multido@#1#2#3#4#5#6{%
  #2%
    \multido@count=#5\relax
    \ifnum\multido@count=\z@\else\multido@@{#1}{#4}{#6}\fi
  #3%
  \ignorespaces}
\def\multido@@#1#2#3{%
  \def\multido@stepvar{}%
  \def\do{\noexpand\do\noexpand}%
  \multido@initvar#2,\@nil,%
  \let\do\noexpand
  \edef\multido@stepvar{\multido@stepvar}%
  \ifnum\multido@count<\z@\multido@count=-\multido@count\fi
  \multidocount=1\relax
  #1%
  \multido@stuff{#3}%
  \multido@loop}
\def\multido@loop{%
  \the\multido@stuff
  \ifnum\multidocount<\multido@count
    \advance\multidocount\@ne
    \multido@stepvar
    \expandafter\multido@loop
  \fi}
\def\multidostop{\multidocount=\multido@count}
\def\multido@badvar{%
  \multido@count=0
  \errhelp{\multido command will be skipped.}%
  \errmessage{Bad \string\multido\space variable declaration}}
\def\multido@initvar#1,{%
  \def\multido@temp{#1}%
  \ifx\multido@temp\@nnil\else
    \ifx\multido@temp\@empty\else
      \multido@@initvar#1\@nil=+\@nil\relax
    \fi
    \expandafter\multido@initvar
  \fi}
\def\multido@@initvar#1=#2+#3\@nil#4\relax{%
  \ifx\@empty#4\@empty
    \multido@badvar
  \else
    \multido@vartype#1\@empty
    \ifx\multido@temp\relax
      \multido@badvar
    \else
      \multido@temp{#2}{#3}#1%
    \fi
  \fi}
\def\multido@vartype#1{%
  \ifcat\noexpand\@nil\noexpand#1%
     \expandafter\multido@@vartype\string#1\@nil
  \else
    \let\multido@temp\relax
  \fi}
\def\multido@@vartype#1#2#3\@nil{%
  \expandafter\let\expandafter\multido@temp
    \csname multido@init@#2\endcsname}
\def\MultidoCheckNames{%
  \let\multido@@@vartype\multido@vartype
  \def\multido@vartype##1{%
    \ifx\undefined##1%
      \multido@@@vartype{##1}%
    \else
      \errhelp{\multido command will be skipped.}%
      \errmessage{Multido variable \string##1 already defined}%
      \let\multido@temp\relax
    \fi}}
%
% For each variable type, we must now define \multido@init<vartype>. The
% syntax of these macros is:
%
%   \multido@init@<vartype>{initial value}{increment}{variable}
%
% and the outcome should be:
%
% 1. Set <variable> to the initial value.
% 2. Invoke \multido@addtostep{step stuff}, where <step stuff> is whatever
% should be done to increment the variable. <step stuff> is expanded first,
% with expansion suppressed by "\do".
%
\def\multido@addtostep#1{\edef\multido@stepvar{\multido@stepvar#1}}
%
% Variable type for dimensions ("d", "D"):
%
\def\multido@init@d#1#2#3{%
  \expandafter\ifx\csname pssetlength\endcsname\relax
    \dimen@=#1\relax\dimen@i=#2\relax
  \else
    \pssetlength\dimen@{#1}\pssetlength\dimen@i{#2}%
  \fi
  \edef#3{\number\dimen@ sp}%
  \ifnum\multido@count<\z@\dimen@i=-\dimen@i\fi
  \multido@addtostep{\do\multido@step@d{\do#3}{\number\dimen@i sp}}}
\def\multido@step@d#1#2{%
  \dimen@=#1\advance\dimen@#2
  \edef#1{\number\dimen@ sp}}%
\def\multido@init@D{\multido@init@d}
%
% Variable type for integers ("i", "I"):
%
\def\multido@init@i#1#2#3{%
  \count@=#1\relax
  \edef#3{\the\count@}%
  \count@=#2\relax
  \ifnum\multido@count<\z@\count@=-\count@\fi
  \multido@addtostep{\do\multido@step@i{\do#3}{\the\count@}}}
\def\multido@step@i#1#2{%
  \count@=#1\advance\count@ by #2
  \edef#1{\the\count@}}
\def\multido@init@I{\multido@init@i}
%
% Variable type for reals ("r", "R").
%
\def\multido@init@r#1#2#3{%
  \dimen@=#1pt
  \@dimtonum\dimen@#3%
  \dimen@=#2pt
  \ifnum\multido@count<\z@\dimen@=-\dimen@\fi
  \multido@addtostep{\do\multido@step@r{\do#3}{\number\dimen@ sp}}}
\def\multido@step@r#1#2{%
  \dimen@=#1pt\advance\dimen@#2
  \@dimtonum\dimen@#1}
\def\multido@init@R{\multido@init@r}
%
% Variable type for numbers ("n", "N").
%
\def\multido@init@n#1#2#3{%
  \edef#3{#1}%
  \ifnum\multido@count<\z@\expandafter\FPsub\else\expandafter\FPadd\fi
  {0}{#2}\multido@temp
  \multido@addtostep{\do\FPadd{\do#3}{\multido@temp}{\do#3}}}
\def\multido@init@N{\multido@init@n}
%
% Simple fixed-point addition couldn't be harder. "\dimen@" and "\dimen@i" are
% used as scratch {\em counters} (to avoid creating a new counter). "\count@",
% "\multido@count" and "\multidocount" are also used as scratch counters.
%
% This is probably pretty close to optimal, given the requirements that
% leading and trailing spaces be OK in the arguments, and that there need not
% be any numbers to the left of the decimal.
%
\def\FPadd#1#2#3{%
  \edef\multido@temp{#1..\noexpand\@nil#2}%
  \expandafter\FPadd@\multido@temp..\@nil
  \let#3\multido@temp}
\def\FPsub#1#2{%
  \edef\multido@temp{\noexpand\FPsub@#2\noexpand\@empty}%
  \FPadd{#1}{\multido@temp}}
\def\FPsub@#1{\ifx-#1\else-#1\fi}
\def\FPadd@#1.#2.#3\@nil#4.#5.#6\@nil{%
  \begingroup
    \def\multido@temp{\let\next\relax}%
    \let\next\z@
    \afterassignment\multido@temp\count@=#1\next
    \dimen@i=0#2sp\relax
    \let\next\z@
    \afterassignment\multido@temp\multido@count=#4\next
    \multidocount=0#5\relax
    \dimen@=\number\count@ sp
    \count@=\@ne
    \FPadd@@@#500000000\@nil
    \ifnum#11<\z@
      \count@=-\tw@
      \dimen@=-\dimen@
    \fi
    \ifnum#41<\z@
      \count@=-\count@
      \multido@count=-\multido@count
    \fi
    \ifnum\count@>\z@
      \advance\multido@count\dimen@
      \advance\multidocount\dimen@i
      \ifnum\multidocount<\multido@temp\relax
        \advance\multidocount\multido@temp\relax
      \else
        \advance\multido@count\@ne
      \fi
      \advance\count@-3
    \else
      \advance\multido@count-\dimen@
      \advance\multidocount-\dimen@i
      \ifnum\multido@count<\z@
        \multido@count=-\multido@count
        \multidocount=-\multidocount
        \advance\count@\@ne
      \else
        \ifnum\multido@count=\z@
          \ifnum\multidocount<\z@
            \multidocount=-\multidocount
            \advance\count@\@ne
          \fi
        \fi
      \fi
      \ifnum\multidocount<\z@
        \advance\multidocount\multido@temp
        \advance\multido@count-\@ne
      \fi
      \advance\multidocount\multido@temp\relax
    \fi
    \expandafter\FPadd@@\the\multidocount\@empty\@nil
  \endgroup}
\def\FPadd@@#1#2#3\@nil{\xdef\multido@temp{%
  \ifnum\count@=-1-\fi
  \the\multido@count\ifx#2\@empty\else.#2#3\fi}}
\def\FPadd@@@#1#2#3#4#5#6#7#8#9\@nil{\def\multido@temp{1#9}}
%
\catcode`\@=\MultidoAtCode\relax
\endinput
%% END multido.tex/multido.sty

%% BEGIN multido.doc
%%
%% Documentation for multido.tex/multido.sty.
%% Run through LaTeX, with or without the NFSS.
%%
%% If you find this file appended to multido.tex, copy it to a separate file
%% or simply add "\let\endinput\relax" to the beginning of multido.tex.
%%
\documentstyle[12pt]{article}
\input multido.tex

\makeatletter

%% PAGE PARAMETERS:
\setlength{\parindent}{0pt}          % paragraph indent
\setlength{\parskip}{4pt plus 1pt minus 1pt}
\setlength{\topmargin}{-2.5pc}
\setlength{\headheight}{12pt}        % height of running head
\setlength{\headsep}{30pt}           % distance between header and text
\setlength{\textheight}{54pc}        % height of text on page
%%
%% DATES, VERSIONS AND TITLES:
%%
\def\expanddate#1/#2/#3/{\year=19#1 \month=#2 \day=#3}
\begingroup
  \expandafter\expanddate\filedate/
  \xdef\thefiledate{\today}
\endgroup
\date{Version \fileversion\\ \thefiledate}
\def\@maketitle{%
 \begin{center}
   {\Large\bf \@title \par}
   \vskip 1.2em {\lineskip .5em
   \begin{tabular}[t]{c}\@author\end{tabular}\par} 
   \vskip .8em {\@date}%
  \end{center}
  \par
  \vskip .5em}
%%
%% PAGE STYLE:
%%
\pagestyle{myheadings}
\markright{Documentation for multido.tex \hfill
  v.\fileversion\hskip 1em \thefiledate\hfill}%
%%
%% VERBATIM:
%%
\begingroup
  \catcode`\{=12 \catcode`\}=12
  \catcode`\(=1 \catcode`\)=2
  \catcode`\+=0 \catcode`\\=12
  +gdef+|(\)          % \| = character \
  +gdef+{({)          % \{ = character {
  +gdef+}(})          % \} = character }
+endgroup
\def\Backslash{\protect\|}
%%
%% Define some functional font commands:
%%
\def\MainFont{\tt}      % For macro definitions.
\def\UsageFont{\tt}           % For in-line macro names.
\def\InlineFont{\tt}          % For other in-line snipets of code.
\def\MetaFont{\rm\it}            % For meta arguments.
%%
%% Short-cuts for font commands:
%%
\def\s#1{\string#1}
\def\t#1{{\InlineFont\string#1}}   % For other in-line code.
\def\m#1{{\MetaFont #1\/}}         % For just putting things in italics
\def\M#1{{\InlineFont\{\m{#1}\}}}  % Arg is in italic, enclosed in tt braces.
\let\N\t                           % For in-line macro names.
\catcode`\<=13 \def<#1>{\m{#1}}    % <meta>
%%
%% This is for listing macro definitions in a quote-like environment.
%%
\begingroup
  \catcode`\[=1 \catcode`\]=2
  \catcode`\{=13 \catcode`\}=12
  \gdef\@MD[%
    \catcode`\{=13 \catcode`\}=12
    \def{##1}[\{[\MetaFont ##1]\}]]
\endgroup
\def\MD{%
  \quote
  \begingroup
    \@MD
    \def\end{\endgroup\end}%
    \def\\{\@centercr\s}%
    \MainFont
    \s}
\def\endMD{\endquote}
%%
%% END PREAMBLE:
%%
\catcode`\"=13
\def"{\verb"}
\catcode`\@=12
\begin{document}
%%
%%%%%%%%%%%%%%%%%%%%%%%%%%%%%%%%%%%%%%%%%%%%%%%%%%%%%%%%%%%%%%%%%%%%%%%%%%%%%

\title{Documentation for multido.tex:\\[2pt]
  A loop macro for Generic \protect\TeX}
\author{Timothy Van Zandt%
  \thanks{Author's address: Department of Economics, Princeton University,
  Princeton, NJ 08544-1021, USA. Internet: {\tt tvz@Princeton.EDU}}}
\maketitle

 "multido.tex"/"multido.sty" contains the \N\multido\ macro, which is a loop
facility for Generic TeX. This macro happens to be useful for drawing
pictures, and was originally developed for the PSTricks
package,\footnote{PSTricks is an extensive collection of PostScript-based
macros for Generic TeX. It is available from the /pub directory at
Princeton.EDU, and \TeX\ archives.} but you can use it for other purposes as
well.

 A special feature is support of fixed-point addition. For example, PSTricks
uses the \N\multido\ to put numbers on axes, much like in the following
\LaTeX\ example:
\begin{verbatim}
    \setlength{\unitlength}{1cm}
    \small
    \begin{picture}(8,1)(0,-.5)
      \put(0,0){\vector(1,0){8}}
      \multido{\i=0+1,\n=0+0.25}{8}{%
        \put(\i,-.1){\line(0,1){.2}}
        \put(\i,-.2){\makebox(0,0)[t]{\n}}}
    \end{picture}
\end{verbatim}
\begin{center}
    \setlength{\unitlength}{1cm}
    \small
    \begin{picture}(8,1)(0,-.5)
      \put(0,0){\vector(1,0){8}}
      \multido{\i=0+1,\n=0+0.25}{8}{%
        \put(\i,-.1){\line(0,1){.2}}
        \put(\i,-.2){\makebox(0,0)[t]{\n}}}
    \end{picture}
\end{center}

The general syntax for \N\multido\ is:
\begin{MD}
  \multido{variables}{repetitions}{stuff}
\end{MD}
 <stuff> is whatever you want repeated; it can be any balanced \TeX\ input.
<repetitions> is the number times <stuff> is repeated.

 The first argument is the interesting one. <variables> is a comma-separated
list of variable declarations.\footnote{Don't use commas to mark the decimal
point within the \protect<variables> argument, as they will be confused for
delimiters.} Each variable declaration is of the form:
\begin{center}\tt
  <variable> = <initial value> + <increment>
\end{center}
 <variable> is a command sequence that can be used in <stuff>. It is initially
set to <initial value>, and is then incremented by <increment> with each
repetition.

 The first letter of the variable name determines the variable type. There are
four variable types:
\begin{description}
 \item[Dimension (d or D)] The initial value and the increment should be
dimensions (lengths, in \LaTeX\ parlance). The substitution text is a
dimension, with "sp" units. E.g., "\dx=4cm+5pt".\footnote{For PSTricks users,
the unit is optional.}

 \item[Number (n or N)] The initial value and increment should be integers or
numbers with the same number of digits to the right of the decimal. The one
exception is that it is always OK for the initial value to be an integer.
There can be at most 8 digits on each side of the decimal. The substitution
text is a number, with fixed-point addition. E.g., "\n=3+7.05",
"\Nx=5.30+-1.25".

 \item[Integer (i or I)] The initial value and increment should be integers.
This gives the same result as using a number variable, but it is faster. E.g.,
"\I=2+-1".

 \item[Real (r or R)] The initial value and increment should be integers or
numbers with at most 4 digits on each side of the decimal. The substitution
text is a number, but with floating point addition and occasional small
errors. This gives a less satisfactory result than using a number variable,
but it is faster. E.g., "\ry=4.2+1.05".
\end{description}

 Here are some examples that illustrate how the substitution text is
determined:
\begin{quote}
  "\multido{}{10}{\TeX\ }"\\[3pt]
    \hbox to 2em{}\multido{}{10}{\TeX\ }{}\\[8pt]
  "\multido{\d=2pt+3pt}{5}{\d, }"\\[3pt]
    \hbox to 2em{}\multido{\d=2pt+3pt}{5}{\d, }{}\\[8pt]
  "\multido{\n=2+3}{10}{\n, }"\\[3pt]
    \hbox to 2em{}\multido{\n=2+3}{10}{\n, }{}\\[8pt]
  "\multido{\i=2+-3}{10}{\i, }"\\[3pt]
    \hbox to 2em{}\multido{\i=2+-3}{10}{\i, }{}\\[8pt]
  "\multido{\r=2+3.05}{6}{\r, }"\\[3pt]
    \hbox to 2em{}\multido{\r=2+3.05}{6}{\r, }{}\\[8pt]
  "\multido{\n=2.00+-3.05}{8}{\n, }"\\[3pt]
    \hbox to 2em{}\multido{\n=2.00+-3.05}{8}{\n, }{}
\end{quote}

Here are some details about the choice of names:
\begin{itemize}
 \item Your computer won't explode if you use names that conflict with \TeX\
internal commands, but you might want to check name conflicts if you get
inexplicable errors. The command \N\MultidoCheckNames\ can be useful in this
case. It causes \N\multido\ to report an error whenever you use a variable
name that is already defined. But see the next item.
 \item The whole \N\multido\ loop is grouped. This means, e.g., that although
"\i" is a Plain \TeX\ command sequence (giving a dotless ``\i''), you can use
the variable "\i" if you do not use any dotless i's in <stuff> (and if you do
not use \N\MultidoCheckNames).
\end{itemize}

Here are a few more details:
\begin{itemize}
\item \N\Multido\ commands can be nested.
 \item Spaces after a \N\multido\ command are ignored. This makes \N\multido\
more hospitable for pictures.
 \item Spaces between the various parts of the <variables> argument are
ignored.
\end{itemize}

 And finally here a few special features, some of which are of interest mainly
macro writers and other \TeX nicians:
\begin{itemize}
 \item The material that is repeated is not grouped, so that you can insert
your own recursive routines.
 \item There is a variant, "\mmultido", which works just like "\multido"
except that the variables are all incremented once before starting.
 \item If you use "\Multido" or "\MMultido" instead of "\multido" or
"\mmultido", resp., then the whole loop is not grouped. This can be useful,
e.g., for making entries in an alignment environment. However, these cannot be
nested within any "\multido" macro.
 \item If the number of repetitions is a negative number, the variables are
incremented backwards.
 \item The count register "\multidocount" keeps track of the number of current
iteration.
 \item The command "\multidostop" causes the "\multido" loop to quit at the
end of the current iteration.
\item Fixed point addition is performed by \N\FPadd\ and \N\FPsub:
\begin{MD}
  \FPadd{num1}{num2}{command}\\
  \FPsub{num1}{num2}{command}
\end{MD}
 <num2> is added to or subtracted from <num1>, and the answers is stored in
the command sequence given as the third argument. The rules about decimals and
so on that apply to number variables apply here as well. E.g., after
\begin{verbatim}
    \FPsub{1.75}{-0.15}{\answer}
\end{verbatim}
the definition of "\answer" is 1.90.
\end{itemize}

\subsection*{Changes:}

\begin{description}
 \item[V1.1] Fixed bug in \N\FPadd\ that gave wrong answer for, e.g., $3.4 +
-0.2.$

\item[V1.2] Made unit optional for dimension variables when using PSTricks.

 \item[V1.3] Now 0 repetitions really gets 0 repetitions. "\def\multido@"
changed to "\long\def\multido@".
\end{description}

\end{document}
%% END multido.doc
