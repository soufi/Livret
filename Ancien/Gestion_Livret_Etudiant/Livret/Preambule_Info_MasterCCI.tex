\documentclass[10pt, a5paper]{report}

\usepackage[T1]{fontenc}%
\usepackage[utf8]{inputenc}% encodage utf8
\usepackage[francais]{babel}% texte français
\usepackage[top=10mm, bottom=20mm, foot=5mm, left=7mm, offset=7mm, textwidth=350pt]{geometry}
%\usepackage{wrapfig}
%\usepackage{graphics}
%\usepackage{setspace}
%\usepackage{tikz}
%\usepackage{array}
%\usepackage[explicit]{titlesec}
\usepackage{url}
\usepackage{init-preambule}
\usepackage[final]{pdfpages}

% Définir les couleur du document
%---------------------- % % % Personnalisation des couleurs % % % ----------- ROUGE --------
\definecolor{couleurFonce}{RGB}{160,0,10} % Couleur du Code APOGEE
\definecolor{couleurClaire}{RGB}{225,135,140} % Couleur du fond de la bande
\definecolor{couleurTexte}{RGB}{255,255,255} % Couleur du texte de la bande
%------------------------------------------------------------------------------------------


\begin{document}
\begin{spacing}{1.5}

\chapter*{Master CCI}

\footnotesize
\section*{Objectifs}

\input{Extra/Resp_Nicolas_OLLINGER}

\letterine{L}e Master CCI (Compétences Complémentaires en Informatique) permet à des étudiants de niveau M1 ou M2 dans une discipline autre que l'informatique d'acquérir des compétences 
scientifiques et techniques similaires à celles d'un titulaire d'une licence d'informatique, avec en plus des notions de gestion de projet 
de niveau Master.

L'objectif du Master CCI est de former des cadres pluridisciplinaires capables de collaborer aussi bien avec les spécialistes de leur 
discipline d'origine qu'avec des informaticiens, afin de mettre en place des solutions de modélisation ou de développement en lien avec 
cette discipline d'origine.

L'accent est mis sur des compétences en réseau, en système, en programmation Web et en gestion de projet informatique. 
Idéalement, les titulaires d'un Master CCI sont ainsi par exemple capables de concevoir et de gérer des plates-formes Web dont le contenu  
est en lien avec leur autre domaine de compétence (que ce soit la biologie, la finance, ou tout autre domaine). 
Maîtriser à la fois le contenu d'un projet informatique et les modalités de sa réalisation est une double compétence fortement appréciée 
des professionnels.

Parmi les métiers possibles, on peut citer : administrateur réseau, administrateur système, gestionnaire de site Web, gestionnaire de 
systèmes d'information, intégrateur, chef de projet informatique.

\section*{Organisation}

\letterine{L}e Master CCI n'est ouvert que pour la deuxième année (M2), sur dossier, pour les titulaires d'un M1 ou d'un M2 dans une discipline autre 
que l'informatique. C'est un Master à visée uniquement professionnelle, avec un parcours unique (sans options).

Ce diplôme est ouvert à la formation continue : chaque année, il accueille des candidats qui souhaitent enrichir ou mettre à jour leurs 
compétences en informatique ou se reconvertir, souvent avec un dispositif de financement aidé. La gestion du volet administratif de la 
formation continue est prise en charge par le SeFCo (Service de la Formation Continue et de l'apprentissage : \url{http://www.univ-orleans.fr/sefco}).

Un stage d'au moins 4 mois en milieu professionnel, co-encadré par un universitaire et par un professionnel, et formalisé par une convention, 
est obligatoire à la fin de la formation. Les cours se terminent pour cela fin mars. Pour valider le travail réalisé, les étudiants doivent 
rédiger un mémoire et faire une soutenance devant un jury constitué des encadrants et de membres de l'équipe pédagogique.

\subsection*{Enseignements}

%---------------------- % % % Personnalisation des couleurs % % % ----------- ROUGE --------
\definecolor{couleurFonce}{RGB}{160,0,10} % Couleur du Code APOGEE
\definecolor{couleurClaire}{RGB}{225,135,140} % Couleur du fond de la bande
\definecolor{couleurTexte}{RGB}{255,255,255} % Couleur du texte de la bande
%------------------------------------------------------------------------------------------

\arrayrulecolor{couleurFonce}% Couleur des lignes séparatrices du tableau
\renewcommand{\arraystretch}{1.2}% Coeff appliqué à la hauteur des cellules
%\rowcolors[\hline]{ligneDébut}{couleurPaire}{couleurImpaire}% Alternance de couleur (need package xcolor)
\begin{tabular}{c|m{6cm}|cm{1cm}|cm{1cm}|cm{1cm}|cm{1cm}|}
\cline{2-6}

&
\cellcolor{couleurFonce} \color{white}\bfseries Intitul\'e & \cellcolor{couleurFonce} \color{white}\bfseries ECTS & \cellcolor{couleurFonce} \color{white}\bfseries CM & \cellcolor{couleurFonce} \color{white}\bfseries TD & \cellcolor{couleurFonce} \color{white}\bfseries TP\\ \cline{2-6}

\hline \multirow{6}{*}{\rotatebox{90}{\color{couleurFonce}\bfseries SEMESTRE 3}}
 & \color{black} \mbox{Initiation}  & \color{black} 0 & \color{black} 8 & \color{black}  & \color{black} 8 \\ \cline{2-6}
 & \cellcolor{couleurClaire} \color{couleurTexte} \mbox{Algorithmique}  & \cellcolor{couleurClaire} \color{couleurTexte} 3 & \cellcolor{couleurClaire} \color{couleurTexte} 15 & \cellcolor{couleurClaire} \color{couleurTexte} 15 & \cellcolor{couleurClaire} \color{couleurTexte}  \\ \cline{2-6}
 & \color{black} \mbox{Bases} \mbox{de} \mbox{données}  & \color{black} 6 & \color{black} 20 & \color{black} 25 & \color{black} 25 \\ \cline{2-6}
 & \cellcolor{couleurClaire} \color{couleurTexte} \mbox{Systèmes}  & \cellcolor{couleurClaire} \color{couleurTexte} 3 & \cellcolor{couleurClaire} \color{couleurTexte} 10 & \cellcolor{couleurClaire} \color{couleurTexte} 10 & \cellcolor{couleurClaire} \color{couleurTexte} 15 \\ \cline{2-6}
 & \color{black} \mbox{Réseaux}  & \color{black} 3 & \color{black} 10 & \color{black} 10 & \color{black} 15 \\ \cline{2-6}
 & \cellcolor{couleurClaire} \color{couleurTexte} \mbox{Programmation} \mbox{objet} \mbox{1}  & \cellcolor{couleurClaire} \color{couleurTexte} 6 & \cellcolor{couleurClaire} \color{couleurTexte} 20 & \cellcolor{couleurClaire} \color{couleurTexte} 25 & \cellcolor{couleurClaire} \color{couleurTexte} 25 \\ \cline{2-6}
 & \color{black} \mbox{Projet} \mbox{1}  & \color{black} 3 & \color{black}  & \color{black}  & \color{black}  \\ \cline{2-6}
 & \cellcolor{couleurClaire} \color{couleurTexte} \mbox{Simulation} \mbox{d'entreprise} \mbox{de} \mbox{gestion}  & \cellcolor{couleurClaire} \color{couleurTexte} 3 & \cellcolor{couleurClaire} \color{couleurTexte}  & \cellcolor{couleurClaire} \color{couleurTexte} 24 & \cellcolor{couleurClaire} \color{couleurTexte}  \\ \cline{2-6}
 & \color{black} \mbox{Anglais}  & \color{black} 3 & \color{black}  & \color{black} 20 & \color{black}  \\ \cline{2-6}
\hline \multirow{6}{*}{\rotatebox{90}{\color{couleurFonce}\bfseries SEMESTRE 4}}
 & \color{black} \mbox{Applications} \mbox{internet}  & \color{black} 5 & \color{black} 20 & \color{black} 24 & \color{black} 24 \\ \cline{2-6}
 & \cellcolor{couleurClaire} \color{couleurTexte} \mbox{Génie} \mbox{logiciel}  & \cellcolor{couleurClaire} \color{couleurTexte} 5 & \cellcolor{couleurClaire} \color{couleurTexte} 20 & \cellcolor{couleurClaire} \color{couleurTexte} 24 & \cellcolor{couleurClaire} \color{couleurTexte} 24 \\ \cline{2-6}
 & \color{black} \mbox{Programmation} \mbox{objet} \mbox{2}  & \color{black} 5 & \color{black} 20 & \color{black} 24 & \color{black} 24 \\ \cline{2-6}
 & \cellcolor{couleurClaire} \color{couleurTexte} \mbox{Projet} \mbox{2}  & \cellcolor{couleurClaire} \color{couleurTexte} 5 & \cellcolor{couleurClaire} \color{couleurTexte}  & \cellcolor{couleurClaire} \color{couleurTexte}  & \cellcolor{couleurClaire} \color{couleurTexte}  \\ \cline{2-6}
 & \color{black} \mbox{Stage}  & \color{black} 10 & \color{black} 10 & \color{black}  & \color{black}  \\ \cline{2-6}
\hline
\end{tabular}



\section*{Détail des enseignements}



\end{spacing}

\end{document}
