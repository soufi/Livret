\definecolor{couleurFonce}{RGB}{0,92,133} %couleur du code apogee 
\definecolor{couleurClaire}{RGB}{100,151,186} %couleur du fond de la bande 
\definecolor{couleurTexte}{RGB}{255,255,255} %couleur du texte de la bande 
\begin{document}

\module[codeApogee={UE38 },
titre={Anglais},
COURS={0},
TD={20},
TP={0},
CTD={0},
TOTAL={20},
SEMESTRE={Semestre 0},
COEFF={3},
ECTS={3},
MethodeEval={},
ModalitesCCSemestreUn={},
ModalitesCCSemestreDeux={},
CalculNFSessionUne={$frac{(CC+2*CT)}{3}$},
CalculNFSessionDeux={},
NoteEliminatoire={0},
nomPremierResp={ },
emailPremierResp={},
nomSecondResp={Cédric SARRE},
emailSecondResp={Cedric.SARRE@univ-orleans.fr},
langue={Français_},
%INTRO ou description courte
{},
%Description Longue
{Etude des techniques de présentation orale : amélioration de la prononciation, organisation du discours, guidage de l'auditoire, élaboration d'aides visuelles.},
%Prerequis
{Anglais non professionnel},
%Objectif
{\begin{itemize}
\ObjItem S'exprimer couramment et efficacement dans une langue riche, souple et nuancée dans le domaine de la spécialité, exposer son opinion de manière claire, détaillée et structurée.
\end{itemize}},
%Ressources
{},
%Biblio
{Biblio},

\vfill


\module[codeApogee={},
titre={Simulation et stratégie d'entreprise},
COURS={0},
TD={24},
TP={0},
CTD={0},
TOTAL={24},
SEMESTRE={Semestre 0},
COEFF={3},
ECTS={3},
MethodeEval={},
ModalitesCCSemestreUn={},
ModalitesCCSemestreDeux={},
CalculNFSessionUne={$frac{(CC+2*CT)}{3}$},
CalculNFSessionDeux={},
NoteEliminatoire={0},
nomPremierResp={ },
emailPremierResp={},
nomSecondResp={Chaker HAOUET},
emailSecondResp={Chaker.HAOUET@univ-orleans.fr},
langue={Français_},
%INTRO ou description courte
{Unité obligatoire.},
%Description Longue
{Les étudiants sont mis en situation de gérer une entreprise à travaers des décisions d'ordre commercial, financier et de production. Ces entreprises sont en concurrence sur le marché, et sont en mesure d'évaluer régulièrement leurs résultats à l'aide des documents financiers et d'études de positionnement. Ainsi cette situation de gestion d'entreprise est l'occasion d'appliquer les principaux concepts en statégies et marketing, et d'élaborer des tableaux de bord afin de guider les étudiants dans leurs décisions et d'en mesurer les impacts financier.},
%Prerequis
{},
%Objectif
{\begin{itemize}
\ObjItem Connaissance du monde de l'entreprise.
\end{itemize}},
%Ressources
{},
%Biblio
{Biblio},

\vfill


\module[codeApogee={},
titre={Imagerie opérationnelle},
COURS={74},
TD={51},
TP={0},
CTD={0},
TOTAL={125},
SEMESTRE={Semestre 0},
COEFF={10},
ECTS={10},
MethodeEval={},
ModalitesCCSemestreUn={},
ModalitesCCSemestreDeux={},
CalculNFSessionUne={$frac{(CC+2*CT)}{3}$},
CalculNFSessionDeux={},
NoteEliminatoire={0},
nomPremierResp={Rachid JENNANE},
emailPremierResp={Rachid.JENNANE@univ-orleans.fr},
nomSecondResp={ },
emailSecondResp={},
langue={Français_},
%INTRO ou description courte
{Unité orbligatoire.\
Cette unité se passe sur le site de l'école Polytech'Orléans dans la spécialité "Écotechnologies Électroniques et Optiques".},
%Description Longue
{\begin{enumerate}
\item  Analyse d'images
 egin{itemize}
 \item  Choisir les outils logiciels adaptés à une problématique
 \item  Savoir segmenter une image
 \item  Résoudre un problème mal posé par des méthodes inverses
 \item  Détecter des contours par modèles déformables
 \item  Reconnaitre des formes dans une image
 \item  Classifier des objets dans des bases d'images
 \item  Tatouer une image pour cacher des informations
 \item  Synthétiser des images texturées
 end{itemize}
\item  Traitements vidéo
 egin{itemize}
 \item  Indexer une vidéo par le contenu
 \item  Analyser une séquence vidéo
 \item  Suivre une cible dans une séquence vidéo
 \item  Modéliser la prise de vue et le déplacement d'une caméra 
 \item  Faire un panorama avec une mosaïque d'images
 \item  Exploiter la réalité augmentée 
 \item  Reconstruire des objets 3D par tomographie 
 end{itemize}
\item  Tests, contrôle et validation
 egin{itemize}
 \item  Analyse multivariable (ACP) et réduction de dimensionnalité
 \item  Savoir choisir des vecteurs tests, une base de données, une vérité terrain
 \item  Choisir des critères de validation
 \item  Réaliser un plan de contrôles
 end{itemize}
\item  Fusion de données
 egin{itemize}
 \item  Fusionner des données par approches probabiliste, floue et fonctions de croyance
 \item  Traiter des données sur GPU pour télévision 3D
 \item  Embarquer un traitement d'image 
 \item  Traiter des images couleurs
 \item  Fouille de données pour l'extraction de connaissances
 end{itemize}
\end{enumerate}},
%Prerequis
{},
%Objectif
{\begin{itemize}
\item Maîtriser les aspects théoriques des méthodes de traitement des images.
\item Être capable de d'établir des plans de tests pertinents pour valider les techniques de vision et d'imagerie mises en oe uvre.
\item Être capable de fusionner les informations en provenance de différents capteurs et savoir prendre des décisions.
\item Objectif
\end{itemize}},
%Ressources
{},
%Biblio
{Biblio},

\vfill


\module[codeApogee={},
titre={Projet},
COURS={0},
TD={0},
TP={0},
CTD={0},
TOTAL={0},
SEMESTRE={Semestre 0},
COEFF={3},
ECTS={3},
MethodeEval={},
ModalitesCCSemestreUn={},
ModalitesCCSemestreDeux={},
CalculNFSessionUne={$frac{(CC+2*CT)}{3}$},
CalculNFSessionDeux={},
NoteEliminatoire={0},
nomPremierResp={Rachid JENNANE},
emailPremierResp={Rachid.JENNANE@univ-orleans.fr},
nomSecondResp={Sophie ROBERT},
emailSecondResp={Sophie.ROBERT@univ-orleans.fr},
langue={Français_},
%INTRO ou description courte
{},
%Description Longue
{Réalisation d'une application en rapport avec les UE du semestre.},
%Prerequis
{Maîtrise des techniques de développement de logiciels.},
%Objectif
{\begin{itemize}
\ObjItem Mise en pratique des principes et techniques étudiés dans les unités d'enseignement.
\end{itemize}},
%Ressources
{},
%Biblio
{Biblio},

\vfill


\module[codeApogee={},
titre={Informatique ambiante},
COURS={61},
TD={0},
TP={64},
CTD={0},
TOTAL={125},
SEMESTRE={Semestre 0},
COEFF={10},
ECTS={10},
MethodeEval={},
ModalitesCCSemestreUn={},
ModalitesCCSemestreDeux={},
CalculNFSessionUne={$frac{(CC+2*CT)}{3}$},
CalculNFSessionDeux={},
NoteEliminatoire={0},
nomPremierResp={Rémy LECOGNE},
emailPremierResp={Remy.LECOGNE@univ-orleans.fr},
nomSecondResp={ },
emailSecondResp={},
langue={Français_},
%INTRO ou description courte
{Unité orbligatoire.\
Cette unité se passe sur le site de l'école Polytech'Orléans dans la spécialité "},
%Description Longue
{\begin{enumerate}
\item  Réseaux de communication
 egin{itemize}
 \item  Connaitre les différentes technologies de communication sans fil.
 \item  Sélectionner la technologie la plus adaptée à une situation donnée.
 \item  Mettre en place un système de communication sans fils (Bluetooth, Wifi, RFID, ....).
 \item  Identifier les différents systèmes d'exploitation et leurs limites (cas des systèmes mobiles).
 end{itemize} 
\item  Informatique Graphique
 egin{itemize}
 \item  Comprendre les architectures (matérielles et logicielles) permettant une programmation parallèle.
 \item  Réaliser des programmes déployés sur GPU.
 \item  Mettre en place des interfaces ergonomiques.
 \item  Utiliser les bibliothèques usuelles de génération et de visualisation de graphismes 2D et 3D.
 end{itemize} 
\item  Design logiciel
 egin{itemize}
 \item  Comprendre et appliquer les méthodes de conception et de qualité logicielle.
 \item  Mettre en oe uvre des procédures de test logiciel.
 \item  Connaitre les failles de sécurité liées au développement logiciel ou aux réseaux de communication.
 end{itemize}
\end{enumerate}},
%Prerequis
{},
%Objectif
{\begin{itemize}
\item Comprendre et mettre en place des transferts de données via des réseaux de communication sans fils.
\item Réaliser des programmes bien construits, fiables et sécurisés.
\item Maitriser les architectures et programmations parallèles.
\item Mettre en en place des programmes ergonomiques et visuels (utilisation de graphismes 2D ou 3D).
\end{itemize}},
%Ressources
{},
%Biblio
{Biblio},

\vfill


\module[codeApogee={},
titre={Management opérationnel},
COURS={16},
TD={24},
TP={16},
CTD={0},
TOTAL={56},
SEMESTRE={Semestre 0},
COEFF={4},
ECTS={4},
MethodeEval={},
ModalitesCCSemestreUn={},
ModalitesCCSemestreDeux={},
CalculNFSessionUne={$frac{(CC+2*CT)}{3}$},
CalculNFSessionDeux={},
NoteEliminatoire={0},
nomPremierResp={Jean-Jacques YVERNAULT},
emailPremierResp={Jean-Jacques.YVERNAULT@univ-orleans.fr},
nomSecondResp={ },
emailSecondResp={},
langue={Français_},
%INTRO ou description courte
{Unité orbligatoire.\
Cette unité se passe sur le site de l'école Polytech'Orléans.},
%Description Longue
{Management de la qualite? et du de?veloppement durable:
Styles de management et e?volution des missions de l'inge?nieur. La notion de responsabilite? d'un poste. 
La relation client-fournisseur interne et l'arbitrage. La relation client-fournisseur externe : ne?gocier des achats des ventes.
Les liens entre le stage d'inge?nieur et le management.
Ro?le et missions:
Typologie des comportements au sein d'une e?quipe. Re?unions d'information et de re?solution de proble?mes. 
Entretiens de management et d'e?valuation. Donner des directives. Motiver ses colle?gues. Ge?rer les cas difficiles et les conflits. 
S'organiser, faire le suivi. Ge?rer le stress.
Travailler en e?quipe:
Me?thodes et outils du management de qualite? et de la re?solution de proble?me. De?veloppement durable : 
de?marche inte?gre?e, indicateur et pre?vention des risques.},
%Prerequis
{},
%Objectif
{\begin{itemize}
\item Optimiser son comportement, son relationnel et son organisation pour tenir et de?velopper son ro?le d'inge?nieur au sein d'une entreprise.
\item Acque?rir les me?thodes de l'animation d'e?quipe et de la ne?gociation.
\item Comprendre les ressorts de la motivation.
\item Participer aux re?unions et entretiens avec efficacite?.
\item Utiliser les outils de management de la qualite? et du de?veloppement durable.
\item Valoriser son stage.
\end{itemize}},
%Ressources
{},
%Biblio
{Biblio},

\vfill


\module[codeApogee={},
titre={Initiation à la recherche},
COURS={57},
TD={0},
TP={0},
CTD={0},
TOTAL={57},
SEMESTRE={Semestre 0},
COEFF={7},
ECTS={7},
MethodeEval={},
ModalitesCCSemestreUn={},
ModalitesCCSemestreDeux={},
CalculNFSessionUne={$frac{(CC+2*CT)}{3}$},
CalculNFSessionDeux={},
NoteEliminatoire={0},
nomPremierResp={Rachid JENNANE},
emailPremierResp={Rachid.JENNANE@univ-orleans.fr},
nomSecondResp={Sophie ROBERT},
emailSecondResp={Sophie.ROBERT@univ-orleans.fr},
langue={Français_},
%INTRO ou description courte
{Unité conseillée pour ceux qui se destinent à la recherche.},
%Description Longue
{\begin{itemize}
Initiation au stage recherche :

\item introduction d'outils pour aborder un stage de recherche en laboratoire
\item présentation du cycle de tutoriaux, des thématiques, des possibilités de poursuites en thèse et plus largement du milieu de la recherche académique ou industrielle
\item présentation des projets académiques proposés au semestre 4
 
Cycle de tutoriaux :

\item 2 tutoriaux longs (d'une durée totale de 9h; soit 2 fois 3 séances de 1h30) seront axés sur une thématique préalablement
choisie et pour laquelle un renforcement est sollicité par le laboratoire.
\item 20 tutoriaux courts (de 1h30 chacun) articulés autour de thématiques telles que la résolution par contraintes,
l'apprentissage, extraction de connaissances, le parallélisme, la réalité virtuelle, la sécurité et sûreté des logiciels,
les modèles de calculs, l'algorithmique et la théorie des graphes, ...
 
Ces tutoriaux se voudront à la fois introductifs et concrets, mais ils apporteront également des connaissances pointues sur des domaines maîtrisés par les intervenants.
\end{itemize}},
%Prerequis
{Avoir une connaissance générale de l'informatique.},
%Objectif
{\begin{itemize}
\item L'objectif est d'initier l'étudiant à une démarche scientifique et de le familiariser à un travail de recherche bibliographique. 
\item Les tutoriaux ont pour objectif d'appréhender quelques thématiques de recherche et d'introduire des techniques récentes ou fondamentales.
\end{itemize}},
%Ressources
{},
%Biblio
{Biblio},

\vfill


\module[codeApogee={},
titre={Fouille d'images},
COURS={20},
TD={15},
TP={0},
CTD={0},
TOTAL={35},
SEMESTRE={Semestre 0},
COEFF={3},
ECTS={3},
MethodeEval={},
ModalitesCCSemestreUn={},
ModalitesCCSemestreDeux={},
CalculNFSessionUne={$frac{(CC+2*CT)}{3}$},
CalculNFSessionDeux={},
NoteEliminatoire={0},
nomPremierResp={ },
emailPremierResp={},
nomSecondResp={Christel VRAIN},
emailSecondResp={Christel.VRAIN@univ-orleans.fr},
langue={Français_},
%INTRO ou description courte
{Unité obligatoire.},
%Description Longue
{Ce module explore les différentes techniques et compétences nécessaires à la fouille d'images, depuis la description synthétique des images jusqu'aux techniques d'apprentissage automatique.
La description synthétique des images consiste à extraire un nombre restreint de descripteurs numériques, représentatifs du contenu de l'image, la décrivant sur un plan local ou global (orientations ou couleurs dominantes, texture...). Nous étudierons ou rappellerons différentes méthodes d'extraction de descripteurs, tels que les histogrammes, les matrices de cooccurence ou encore les ondelettes. Nous verrons également comment extraire les points d'intérêt au sein des images.
Par ailleurs, nous présenterons différentes facettes de l'apprentissage automatique, d'abord de manière générale, puis dans le cadre de leur application aux images.
Nous aborderons la notion de distance ou similarité, nous montrerons comment elle peut s'appliquer pour des recherches locales (images similaires, classification par plus proche voisin...) ou globale (structuration de l'espace des images, clustering...).
Nous étudierons l'impact de connaissances a priori sur l'efficacité des méthodes (approches non supervisées, supervisées, semi-supervisées).},
%Prerequis
{},
%Objectif
{\begin{itemize}
\ObjItem Apporter à l'étudiant une double compétence dans les techniques d'apprentissage en général et dans leur application aux images en particulier.
\end{itemize}},
%Ressources
{},
%Biblio
{Biblio},

\vfill


\module[codeApogee={},
titre={Visualisation avancée},
COURS={20},
TD={15},
TP={0},
CTD={0},
TOTAL={35},
SEMESTRE={Semestre 0},
COEFF={3},
ECTS={3},
MethodeEval={},
ModalitesCCSemestreUn={},
ModalitesCCSemestreDeux={},
CalculNFSessionUne={$frac{(CC+2*CT)}{3}$},
CalculNFSessionDeux={},
NoteEliminatoire={0},
nomPremierResp={ },
emailPremierResp={},
nomSecondResp={Sébastien LIMET},
emailSecondResp={Sebastien.LIMET@univ-orleans.fr},
langue={Français_},
%INTRO ou description courte
{Unité obligatoire.},
%Description Longue
{La complexité sémantique et la massivité des données issues de mesures scientifiques, de simulations numériques ou d'immenses bases de données disponibles sur le réseau, rendent indispensable le recours à la médiation visuelle pour en permettre une appréhension la plus riche possible.  La mise en oeuvre de  techniques de visualisation élaborées conduit à utiliser des architectures parallèles et distribués pour faire face à la complexité des traitements numériques en amont ou propre au rendu visuel. Cette puissance de traitement peut être mise en oeuvre pour simplifier le rendu afin de l'adapter à un rendu nomade, mais elle  peut aussi adapter les données en post-traitement pour que celles-ci soient analysées via un vaste environnement de Réalité Virtuelle multi-écrans plus ou moins distant sur le réseau.
Nous présentons dans ce cours  les fondements du pipeline graphique parallèle, les différentes techniques de rendu scientifique, les moyens d'adapter le rendu nomade aux gros volumes de données complexes et enfin nous abordons la visualisation scientifique utilisant les techniques avancée de Réalité Virtuelle au service de la performance.},
%Prerequis
{Module Calcul intensif, Module programmation graphique. Notions en Réseaux.Architecture des systèmes},
%Objectif
{\begin{itemize}
\item Comprendre différentes techniques de visualisation d'information scientifique. Comprendre le fonctionnement d'une application graphique nomade.
\item Aborder sur des exemples les principes des applications  de visualisation scientifique portants sur des données massives de type geo-scientifique ou biologie moléculaire.
\end{itemize}},
%Ressources
{},
%Biblio
{Biblio},

\vfill


\module[codeApogee={},
titre={Stage},
COURS={0},
TD={0},
TP={0},
CTD={0},
TOTAL={0},
SEMESTRE={Semestre 0},
COEFF={12},
ECTS={12},
MethodeEval={},
ModalitesCCSemestreUn={},
ModalitesCCSemestreDeux={},
CalculNFSessionUne={$frac{(CC+2*CT)}{3}$},
CalculNFSessionDeux={},
NoteEliminatoire={0},
nomPremierResp={ },
emailPremierResp={},
nomSecondResp={Sophie ROBERT},
emailSecondResp={Sophie.ROBERT@univ-orleans.fr},
langue={Français_},
%INTRO ou description courte
{Unité obligatoire.},
%Description Longue
{\begin{itemize}
\item Un stage en entreprise à temps complet de 4 à 6 mois ou
\item Un stage de recherche  à  temps complet de 4 à 6 mois dans un laboratoire au sein d'une équipe de
recherche confronte l'étudiant au monde de la recherche et lui permet à  la fois d'approfondir et d'individualiser
la formation de base. Bien qu'il soit conseillé de faire le stage en laboratoire de recherche, le stage peut se dérouler
dans un service de recherche et développement d'une entreprise.

La recherche du stage est à l'initiative de l'étudiant.
Cependant, le sujet doit être validé par les responsables de la formation. Le stage fait l'objet d'une convention
engageant l'entreprise ou le laboratoire, l'université et l'étudiant.
\end{itemize}},
%Prerequis
{},
%Objectif
{\begin{itemize}
\ObjItem Appliquer tous les concepts vu durant le master.
\end{itemize}},
%Ressources
{},
%Biblio
{Biblio},

\vfill


\module[codeApogee={},
titre={Programmation multi-coe urs},
COURS={20},
TD={15},
TP={0},
CTD={0},
TOTAL={35},
SEMESTRE={Semestre 0},
COEFF={3},
ECTS={3},
MethodeEval={},
ModalitesCCSemestreUn={},
ModalitesCCSemestreDeux={},
CalculNFSessionUne={$frac{(CC+2*CT)}{3}$},
CalculNFSessionDeux={},
NoteEliminatoire={0},
nomPremierResp={Sylvain JUBERTIE},
emailPremierResp={Sylvain.JUBERTIE@univ-orleans.fr},
nomSecondResp={ },
emailSecondResp={},
langue={Français_},
%INTRO ou description courte
{Unité obligatoire.},
%Description Longue
{Ce cours porte sur l'exploitation des différents niveaux de parallélisme présents dans la quasi-totalité des architectures actuelles. Ces niveaux (multi-coeurs, unités vectorielles et cartes graphiques) seront d'abord présentés, en particulier, les problématiques de programmation liées aux spécificités de ces architectures seront étudiés (hyperthreading, pipeline, cache, modèle mémoire, alignement).
Après une introduction aux structures de données adaptées au parallélisme en mémoire partagée, la programmation de ces architectures sera étudiée au travers d'exemples touchant au calcul scientifique et au traitement d'images. La programmation des processeurs multi-coeurs reposera sur l'utilisation des Pthreads, d'OpenMP, d'Intel TBB et sur une présentation du concept de transaction. La programmation de cartes graphiques reposera sur l'utilisation de CUDA et les jeux d'instructions SSE et Altivec seront utilisés pour la programmation des unités vectorielles intégrées dans les processeurs. Une vision plus
haut-niveau sera donnée au travers de la librairie OpenCL. Finalement, l'accent sera mis sur la combinaison de ces différents niveaux de parallélisme, la mesure des performances et l'adéquation
entre problèmes et choix d'architectures/algorithmes adaptés.},
%Prerequis
{Programmation impérative
Architecture des ordinateur},
%Objectif
{\begin{itemize}
\item Capacité à exploiter correctement et efficacement les différents niveau de parallèlisme présents dans les architectures actuelles.
\item Capacité à choisir une architecture en fonction d'un problème donné.
\item Capacité à utiliser ces compétences dans les domaines du calcul scientifique et du traitement d'images.
\end{itemize}},
%Ressources
{},
%Biblio
{Biblio},

\vfill


\module[codeApogee={},
titre={Projet},
COURS={0},
TD={0},
TP={0},
CTD={0},
TOTAL={0},
SEMESTRE={Semestre 0},
COEFF={6},
ECTS={6},
MethodeEval={},
ModalitesCCSemestreUn={},
ModalitesCCSemestreDeux={},
CalculNFSessionUne={$frac{(CC+2*CT)}{3}$},
CalculNFSessionDeux={},
NoteEliminatoire={0},
nomPremierResp={Rachid JENNANE},
emailPremierResp={Rachid.JENNANE@univ-orleans.fr},
nomSecondResp={Sophie ROBERT},
emailSecondResp={Sophie.ROBERT@univ-orleans.fr},
langue={Français_},
%INTRO ou description courte
{},
%Description Longue
{Réalisation d'une application en rapport avec les UE du semestre.},
%Prerequis
{Maîtrise des techniques de développement de logiciels.},
%Objectif
{\begin{itemize}
\ObjItem Mise en pratique des principes et techniques étudiés dans les unités d'enseignement.
\end{itemize}},
%Ressources
{},
%Biblio
{Biblio},

\vfill


\module[codeApogee={},
titre={Préparation au stage recherche},
COURS={0},
TD={0},
TP={0},
CTD={0},
TOTAL={0},
SEMESTRE={Semestre 0},
COEFF={6},
ECTS={6},
MethodeEval={},
ModalitesCCSemestreUn={},
ModalitesCCSemestreDeux={},
CalculNFSessionUne={$frac{(CC+2*CT)}{3}$},
CalculNFSessionDeux={},
NoteEliminatoire={0},
nomPremierResp={},
emailPremierResp={},
nomSecondResp={},
emailSecondResp={},
langue={Français_},
%INTRO ou description courte
{Unité conseillée pour ceux qui se destinent à la recherche.},
%Description Longue
{\begin{itemize}
\item Réalisation d'un état de l'art ou/et d'une expérimentation dans un domaine précis de l'informatique.
\item Initiation à la recherche.
 
Les étudiants assistent à 4h de cours pour avoir les prérequis pour ce module.
\end{itemize}},
%Prerequis
{},
%Objectif
{\begin{itemize}
\ObjItem Savoir réaliser un état de l'art dans un domaine spécialisé de la recherche en informatique et être à même d'amorcer une démarche scientifique.
\end{itemize}},
%Ressources
{},
%Biblio
{Biblio},

\vfill


\end{document}
