\definecolor{couleurFonce}{RGB}{0,92,133} %couleur du code apogee 
\definecolor{couleurClaire}{RGB}{100,151,186} %couleur du fond de la bande 
\definecolor{couleurTexte}{RGB}{255,255,255} %couleur du texte de la bande 
\begin{document}

\module[codeApogee={},
titre={Unité Libre},
COURS={0},
TD={24},
TP={0},
CTD={0},
TOTAL={24},
SEMESTRE={Semestre 0},
COEFF={3},
ECTS={3},
MethodeEval={},
ModalitesCCSemestreUn={},
ModalitesCCSemestreDeux={},
CalculNFSessionUne={$frac{(CC+2*CT)}{3}$},
CalculNFSessionDeux={},
NoteEliminatoire={0},
nomPremierResp={},
emailPremierResp={},
nomSecondResp={},
emailSecondResp={},
langue={Français_},
%INTRO ou description courte
{Unité obligatoire.},
%Description Longue
{\begin{itemize}
L'unité Libre est à choisir, en début du semestre, parmi la centaine d'enseignements dédiés à cet usage et offerts par toutes
les composantes de l'université (Sciences, Droit-Economie-Gestion, Sport).\
Voici quelques exemples d'unités Libres,:

  \item Sport.
  \item Droit de l'informatique.
  \item Problèmes économiques contemporains.
  \item Histoire du cinéma, histoire des arts.
  \item Enseigner : posture et identité professionnelles.
  \item Lecture critique du réchauffement climatique.
  \item Maîtriser son expression ; les enjeux de la communication orale : le corps, l'espace, la voix.
\end{itemize}},
%Prerequis
{},
%Objectif
{\begin{itemize}
\ObjItem Comprendre comment ce qu'on apprend dans le cadre d'un diplôme déjà très spécialisé s'insère dans le large champ des connaissances
 et des savoirs auxquels on sera confronté dans son expérience professionnelle ou personnelle.
\end{itemize}},
%Ressources
{},
%Biblio
{Biblio},

\vfill


\module[codeApogee={},
titre={Anglais 6},
COURS={0},
TD={25},
TP={0},
CTD={0},
TOTAL={25},
SEMESTRE={Semestre 0},
COEFF={3},
ECTS={3},
MethodeEval={},
ModalitesCCSemestreUn={},
ModalitesCCSemestreDeux={},
CalculNFSessionUne={$frac{(CC+2*CT)}{3}$},
CalculNFSessionDeux={},
NoteEliminatoire={0},
nomPremierResp={ },
emailPremierResp={},
nomSecondResp={Hervé Perreau},
emailSecondResp={herve.perreau@univ-orleans.fr},
langue={Français_},
%INTRO ou description courte
{Unité obligatoire.},
%Description Longue
{Travail de compréhension et d’expression à partir de documents authentiques longs et/ou complexes portant sur des innovations technologiques, des découvertes et avancées scientifiques.},
%Prerequis
{Avoir suivi l'unité "Anglais 5" ou un volume d'heures de formation équivalente.},
%Objectif
{\begin{itemize}
\ObjItem Comprendre l’information exprimée dans des messages complexes sur le domaine des Sciences et Technologies et s’exprimer sur ce même domaine à l’oral avec un degré suffisant de spontanéité et de fluidité (niveau européen,: B2).
\end{itemize}},
%Ressources
{},
%Biblio
{Biblio},

\vfill


\end{document}
