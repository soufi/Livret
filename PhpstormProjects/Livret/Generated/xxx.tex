\documentclass[10pt, a5paper]{report}

\usepackage[T1]{fontenc}%
\usepackage[utf8]{inputenc}% encodage utf8
\usepackage[francais]{babel}% texte franaçais
\usepackage[final]{pdfpages}
\usepackage{modules-livret}% style du livret
\usepackage{url}
%\usepackage{init-preambule}
\pagestyle{empty}

% % % % % % % % % % % % % % % % % % % % % % % % % % % % % % % % % % % % % % % % % % % % % % % % % % % % % % % 
\begin{document}
\definecolor{couleurFonce}{RGB}{0,92,133} %couleur du code apogee 
\definecolor{couleurClaire}{RGB}{100,151,186} %couleur du fond de la bande 
\definecolor{couleurTexte}{RGB}{255,255,255} %couleur du texte de la bande 
\begin{document}

\module[codeApogee={UE34 },
titre={Réseaux},
COURS={10},
TD={10},
TP={15},
CTD={0},
TOTAL={35},
SEMESTRE={Semestre 0},
COEFF={3},
ECTS={3},
MethodeEval={},
ModalitesCCSemestreUn={},
ModalitesCCSemestreDeux={},
CalculNFSessionUne={$frac{(CC+2*CT)}{3}$},
CalculNFSessionDeux={},
NoteEliminatoire={0},
nomPremierResp={ },
emailPremierResp={},
nomSecondResp={Nicolas OLLINGER},
emailSecondResp={Nicolas.OLLINGER@univ-orleans.fr},
langue={Français_},
%INTRO ou description courte
{},
%Description Longue
{\begin{itemize}

 
  \item Architecture des réseaux : structure en couches, protocoles, services.
  \item Réseaux locaux sous UDP-TCP/IP, Ethernet.
  \item Protocoles de routage : RIP, OSPF, BGP.
  \item Principaux protocoles Internet : DNS (annuaire de noms de domaines) SMTP (mail), FTP (transfert de fichiers), HTTP (web) ... 
 

\end{itemize}},
%Prerequis
{Module Initiation
},
%Objectif
{\begin{itemize}
 
  \ObjItem Principes et pratique des réseaux locaux informatiques.
 

\end{itemize}},
%Ressources
{},
%Biblio
{Biblio
},

\vfill


\module[codeApogee={UE36 },
titre={Projet 1},
COURS={0},
TD={0},
TP={0},
CTD={0},
TOTAL={0},
SEMESTRE={Semestre 0},
COEFF={3},
ECTS={3},
MethodeEval={},
ModalitesCCSemestreUn={},
ModalitesCCSemestreDeux={},
CalculNFSessionUne={$frac{(CC+2*CT)}{3}$},
CalculNFSessionDeux={},
NoteEliminatoire={0},
nomPremierResp={ },
emailPremierResp={},
nomSecondResp={Laure Kahlem},
emailSecondResp={laure.kahlem@univ-orleans.fr},
langue={Français_},
%INTRO ou description courte
{},
%Description Longue
{Réalisation d'une applications mettant en oeuvre les notions vues dans les modules Algorithmique, Programmation objet 1, Bases de données, Systèmes et Réseaux.},
%Prerequis
{Module Initiation},
%Objectif
{\begin{itemize}
\ObjItem Mise en oeuvre dans un exemple d'application réaliste des concepts vus dans les modules informatique de la période considérée.
\end{itemize}},
%Ressources
{},
%Biblio
{Biblio},

\vfill


\module[codeApogee={UE30 },
titre={Initiation},
COURS={8},
TD={0},
TP={8},
CTD={0},
TOTAL={16},
SEMESTRE={Semestre 0},
COEFF={0},
ECTS={0},
MethodeEval={},
ModalitesCCSemestreUn={},
ModalitesCCSemestreDeux={},
CalculNFSessionUne={$frac{(CC+2*CT)}{3}$},
CalculNFSessionDeux={},
NoteEliminatoire={0},
nomPremierResp={ },
emailPremierResp={},
nomSecondResp={Frédéric DABROWSKI},
emailSecondResp={Frederic.DABROWSKI@univ-orleans.fr},
langue={Français_},
%INTRO ou description courte
{},
%Description Longue
{\begin{itemize}
\item Fondements de l'informatique : codage des données, traitements, structure générale d'un ordinateur
  \item Utilisation des systèmes d'exploitation de type Unix.
  \item Outils logiciels bureautique et internet.
\end{itemize}},
%Prerequis
{},
%Objectif
{\begin{itemize}
\ObjItem Remettre à un même niveau de base tous les étudiants et les familiariser avec le fonctionnement des ordinateurs et les outils informatiques usuels.
\end{itemize}},
%Ressources
{},
%Biblio
{Biblio},

\vfill


\module[codeApogee={UE31 },
titre={Algorithmique},
COURS={15},
TD={15},
TP={0},
CTD={0},
TOTAL={30},
SEMESTRE={Semestre 0},
COEFF={3},
ECTS={3},
MethodeEval={},
ModalitesCCSemestreUn={},
ModalitesCCSemestreDeux={},
CalculNFSessionUne={},
CalculNFSessionDeux={},
NoteEliminatoire={0},
nomPremierResp={ },
emailPremierResp={},
nomSecondResp={Bich DAO},
emailSecondResp={Bich.DAO@univ-orleans.fr},
langue={Français_},
%INTRO ou description courte
{},
%Description Longue
{\begin{itemize}
\item Notions : variables, affectation, conditionnelle, itération, récursion.
  \item Algorithmes simples sur les tableaux.
  \item Algorithmes de tri.
\end{itemize}},
%Prerequis
{Mathématiques élémentaires},
%Objectif
{\begin{itemize}
\item Comprendre le fonctionnement d'un algorithme donné.
  \item Concevoir des algorithmes pour un problème simple donné.
  \item Savoir utiliser des structures de données.
\end{itemize}},
%Ressources
{},
%Biblio
{Biblio},

\vfill


\module[codeApogee={UE38 },
titre={Anglais},
COURS={0},
TD={20},
TP={0},
CTD={0},
TOTAL={20},
SEMESTRE={Semestre 0},
COEFF={3},
ECTS={3},
MethodeEval={},
ModalitesCCSemestreUn={},
ModalitesCCSemestreDeux={},
CalculNFSessionUne={$frac{(CC+2*CT)}{3}$},
CalculNFSessionDeux={},
NoteEliminatoire={0},
nomPremierResp={ },
emailPremierResp={},
nomSecondResp={Cédric SARRE},
emailSecondResp={Cedric.SARRE@univ-orleans.fr},
langue={Français_},
%INTRO ou description courte
{},
%Description Longue
{Etude des techniques de présentation orale : amélioration de la prononciation, organisation du discours, guidage de l'auditoire, élaboration d'aides visuelles.},
%Prerequis
{Anglais non professionnel},
%Objectif
{\begin{itemize}
\ObjItem S'exprimer couramment et efficacement dans une langue riche, souple et nuancée dans le domaine de la spécialité, exposer son opinion de manière claire, détaillée et structurée.
\end{itemize}},
%Ressources
{},
%Biblio
{Biblio},

\vfill


\module[codeApogee={UE37 },
titre={Simulation de  gestion d'entreprise},
COURS={0},
TD={24},
TP={0},
CTD={0},
TOTAL={24},
SEMESTRE={Semestre 0},
COEFF={3},
ECTS={3},
MethodeEval={},
ModalitesCCSemestreUn={},
ModalitesCCSemestreDeux={},
CalculNFSessionUne={$frac{(CC+2*CT)}{3}$},
CalculNFSessionDeux={},
NoteEliminatoire={0},
nomPremierResp={ },
emailPremierResp={},
nomSecondResp={Chaker HAOUET},
emailSecondResp={Chaker.HAOUET@univ-orleans.fr},
langue={Français_},
%INTRO ou description courte
{},
%Description Longue
{Simulation visant à amener les groupes (chacun représentant une entreprise en concurrence avec les autres), après avoir formalisé une stratégie, 
à prendre des décisions d'ordre commercial, de production, d'investissement et de financement. Dans ce cadre, ils mettent en oe uvre la 
plupart des outils financiers et prévisionnels connus.},
%Prerequis
{Le jeu fait appel aux connaissances des étudiants ainsi qu'à la réflexion et l'imagination.},
%Objectif
{\begin{itemize}
\item Au terme de cette simulation, les étudiants doivent pouvoir gérer le temps, travailler en groupe, gérer les conflits, ... 
  \item Connaitre le monde de l'entreprise.
\end{itemize}},
%Ressources
{},
%Biblio
{Biblio},

\vfill


\module[codeApogee={UE32 },
titre={Bases de données},
COURS={20},
TD={25},
TP={25},
CTD={0},
TOTAL={70},
SEMESTRE={Semestre 0},
COEFF={6},
ECTS={6},
MethodeEval={},
ModalitesCCSemestreUn={},
ModalitesCCSemestreDeux={},
CalculNFSessionUne={$frac{(CC+2*CT)}{3}$},
CalculNFSessionDeux={},
NoteEliminatoire={0},
nomPremierResp={ },
emailPremierResp={},
nomSecondResp={Nicolas OLLINGER},
emailSecondResp={Nicolas.OLLINGER@univ-orleans.fr},
langue={Français_},
%INTRO ou description courte
{},
%Description Longue
{\begin{itemize}
\item Structure fonctionnelle et Architecture d'un SGBD.
  \item Problématique de la modélisation logique des données.
  \item Modèle relationnel, Langage SQL.
  \item Interrogation de données du modèle relationnel.
\end{itemize}},
%Prerequis
{Module Initiation},
%Objectif
{\begin{itemize}
\item Donner aux étudiants les grandes lignes des diverses fonctionnalités d'un SGBD relationnel.   
  \item Introduire les techniques de modélisation de données.  
  \item Apprendre et maîtriser le langage de manipulation et d'interrogation de bases de données : SQL.
\end{itemize}},
%Ressources
{},
%Biblio
{Biblio},

\vfill


\module[codeApogee={UE33 },
titre={Systèmes},
COURS={10},
TD={10},
TP={15},
CTD={0},
TOTAL={35},
SEMESTRE={Semestre 0},
COEFF={3},
ECTS={3},
MethodeEval={},
ModalitesCCSemestreUn={},
ModalitesCCSemestreDeux={},
CalculNFSessionUne={$frac{(CC+2*CT)}{3}$},
CalculNFSessionDeux={},
NoteEliminatoire={0},
nomPremierResp={ },
emailPremierResp={},
nomSecondResp={Sophie ROBERT},
emailSecondResp={Sophie.ROBERT@univ-orleans.fr},
langue={Français_},
%INTRO ou description courte
{},
%Description Longue
{\begin{itemize}
\item Architecture de systèmes d'exploitation.
  \item Utilisation d'Unix.
  \item Administration Unix.
\end{itemize}},
%Prerequis
{Module Initiation},
%Objectif
{\begin{itemize}
\ObjItem Utilisation et administration de systèmes d'exploitation.
\end{itemize}},
%Ressources
{},
%Biblio
{Biblio},

\vfill


\module[codeApogee={},
titre={Programmation objet 1},
COURS={20},
TD={25},
TP={25},
CTD={0},
TOTAL={70},
SEMESTRE={Semestre 0},
COEFF={6},
ECTS={6},
MethodeEval={},
ModalitesCCSemestreUn={},
ModalitesCCSemestreDeux={},
CalculNFSessionUne={$frac{(CC+2*CT)}{3}$},
CalculNFSessionDeux={},
NoteEliminatoire={0},
nomPremierResp={ },
emailPremierResp={},
nomSecondResp={Bich DAO},
emailSecondResp={Bich.DAO@univ-orleans.fr},
langue={Français_},
%INTRO ou description courte
{Unité obligatoire.},
%Description Longue
{\begin{itemize}
\item Introduction à la programmation, algorithmes de base
  \item Programmation objet: classe, objet, état, méthode, communication entre objets.
  \item Récurrence. 
  \item Tableaux, Tris.
  \item Interfaces.
  \item Héritage.
  \item Exceptions.
  \item Entrées/sorties.
  \item Introduction à la notation UML.
\end{itemize}},
%Prerequis
{Module Initiation},
%Objectif
{\begin{itemize}
\item Introduction à la programmation objet dans un langage impératif. 
  \item Mise en oeuvre d'algorithmes dans ce langage.
\end{itemize}},
%Ressources
{},
%Biblio
{Biblio},

\vfill


\module[codeApogee={UE41 },
titre={Applications internet},
COURS={20},
TD={24},
TP={24},
CTD={0},
TOTAL={68},
SEMESTRE={Semestre 0},
COEFF={5},
ECTS={5},
MethodeEval={},
ModalitesCCSemestreUn={},
ModalitesCCSemestreDeux={},
CalculNFSessionUne={$frac{(CC+2*CT)}{3}$},
CalculNFSessionDeux={},
NoteEliminatoire={0},
nomPremierResp={ },
emailPremierResp={},
nomSecondResp={AbdelAli ED-DBALI},
emailSecondResp={AbdelAli.ED-DBALI@univ-orleans.fr},
langue={Français_},
%INTRO ou description courte
{},
%Description Longue
{\begin{itemize}

 
  \item Les langages HTML, XHTML, feuilles de style CSS.
  \item Applications réparties : le modèle client/serveur.
  \item Langages pour les pages web dynamiques : PHP, MySQL
  \item XML. 
 

\end{itemize}},
%Prerequis
{
},
%Objectif
{\begin{itemize}
 
  \ObjItem Développer et maintenir des sites et applications intra/internet.
 

\end{itemize}},
%Ressources
{},
%Biblio
{Biblio
},

\vfill


\module[codeApogee={UE44 },
titre={Projet 2},
COURS={0},
TD={0},
TP={0},
CTD={0},
TOTAL={0},
SEMESTRE={Semestre 0},
COEFF={5},
ECTS={5},
MethodeEval={},
ModalitesCCSemestreUn={},
ModalitesCCSemestreDeux={},
CalculNFSessionUne={$frac{(CC+2*CT)}{3}$},
CalculNFSessionDeux={},
NoteEliminatoire={0},
nomPremierResp={ },
emailPremierResp={},
nomSecondResp={AbdelAli ED-DBALI},
emailSecondResp={AbdelAli.ED-DBALI@univ-orleans.fr},
langue={Français_},
%INTRO ou description courte
{},
%Description Longue
{Développer une application web en mettant en oe uvre les concepts appris en génie logiciel et applications internet.},
%Prerequis
{},
%Objectif
{\begin{itemize}
\ObjItem Mise en oeuvre dans un exemple d'application réaliste des concepts vus dans les modules informatique de la période considérée.
\end{itemize}},
%Ressources
{},
%Biblio
{Biblio},

\vfill


\module[codeApogee={UE42 },
titre={Génie logiciel},
COURS={20},
TD={24},
TP={24},
CTD={0},
TOTAL={68},
SEMESTRE={Semestre 0},
COEFF={5},
ECTS={5},
MethodeEval={},
ModalitesCCSemestreUn={},
ModalitesCCSemestreDeux={},
CalculNFSessionUne={$frac{(CC+2*CT)}{3}$},
CalculNFSessionDeux={},
NoteEliminatoire={0},
nomPremierResp={ },
emailPremierResp={},
nomSecondResp={Jean-Michel COUVREUR},
emailSecondResp={Jean-Michel.COUVREUR@univ-orleans.fr},
langue={Français_},
%INTRO ou description courte
{},
%Description Longue
{\begin{itemize}
\item Modélisation UML : diagrammes de classes, de séquences, d'états-transition et de cas d'utilisation.
  \item Pratique d'un atelier de génie logiciel.
  \item Méthodologie d'analyse et de conception objet.
  \item Introduction aux patrons de conception (design patterns).
  \item Introduction à la gestion de projets.
\end{itemize}},
%Prerequis
{},
%Objectif
{\begin{itemize}
\item Acquisition des connaissances de bases d'UML, d'une méthodologie permettant d'analyser un problème, d'en réaliser la modélisation, puis d'élaborer une mise en oeuvre sous forme d'une application informatique. 
  \item Connaissance des notions de la gestion de projets.
\end{itemize}},
%Ressources
{},
%Biblio
{Biblio},

\vfill


\module[codeApogee={},
titre={Stage},
COURS={10},
TD={0},
TP={0},
CTD={0},
TOTAL={10},
SEMESTRE={Semestre 0},
COEFF={10},
ECTS={10},
MethodeEval={},
ModalitesCCSemestreUn={},
ModalitesCCSemestreDeux={},
CalculNFSessionUne={$frac{(CC+2*CT)}{3}$},
CalculNFSessionDeux={},
NoteEliminatoire={0},
nomPremierResp={ },
emailPremierResp={},
nomSecondResp={Nicolas OLLINGER},
emailSecondResp={Nicolas.OLLINGER@univ-orleans.fr},
langue={Français_},
%INTRO ou description courte
{},
%Description Longue
{Stage de 4 à 6 mois consécutifs dans une entreprise, suivi par un enseignant et donnant lieu à la rédaction d'un mémoire puis 
d'une soutenance en présence d'un jury mixte d'enseignants et de responsables de l'entreprise.\
Le sujet est variable selon le stage.},
%Prerequis
{},
%Objectif
{\begin{itemize}
\ObjItem Permettre d'une part de mettre en pratique les enseignements dispensés pendant l'année et d'autre part de familiariser 
  les étudiants avec la vie professionnelle afin de permettre leur insertion.
\end{itemize}},
%Ressources
{},
%Biblio
{Biblio},

\vfill


\module[codeApogee={UE43 },
titre={Programmation objet 2},
COURS={20},
TD={24},
TP={24},
CTD={0},
TOTAL={68},
SEMESTRE={Semestre 0},
COEFF={5},
ECTS={5},
MethodeEval={},
ModalitesCCSemestreUn={},
ModalitesCCSemestreDeux={},
CalculNFSessionUne={$frac{(CC+2*CT)}{3}$},
CalculNFSessionDeux={},
NoteEliminatoire={0},
nomPremierResp={ },
emailPremierResp={},
nomSecondResp={Laure Kahlem},
emailSecondResp={laure.kahlem@univ-orleans.fr},
langue={Français_},
%INTRO ou description courte
{},
%Description Longue
{\begin{itemize}
\item Généricité.
  \item Classes internes.
  \item Implantation de structures de données.
  \item Collections des bibliothèques standards.
  \item Programmation événementielle.
  \item Interface graphique.
\end{itemize}},
%Prerequis
{Modules Algorithmique, Programmation objet 1},
%Objectif
{\begin{itemize}
\ObjItem Savoir développer une application avec un langage orienté objet utilisant des structures de données récursives, interface graphique et interface avec une base de donnée.
\end{itemize}},
%Ressources
{},
%Biblio
{Biblio},

\vfill


\end{document}
% % % % % % % % % % % % % % % % % % % % % % % % % % % % % % % % % % % % % % % % % % % % % % % % % % % % % % % 
\end{document}
