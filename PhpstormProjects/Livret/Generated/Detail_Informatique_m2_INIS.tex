\definecolor{couleurFonce}{RGB}{0,92,133} %couleur du code apogee 
\definecolor{couleurClaire}{RGB}{100,151,186} %couleur du fond de la bande 
\definecolor{couleurTexte}{RGB}{255,255,255} %couleur du texte de la bande 
\begin{document}

\module[codeApogee={},
titre={Pratique des contraintes},
COURS={20},
TD={15},
TP={0},
CTD={0},
TOTAL={35},
SEMESTRE={Semestre 0},
COEFF={4},
ECTS={4},
MethodeEval={},
ModalitesCCSemestreUn={},
ModalitesCCSemestreDeux={},
CalculNFSessionUne={$frac{(CC+2*CT)}{3}$},
CalculNFSessionDeux={},
NoteEliminatoire={0},
nomPremierResp={ },
emailPremierResp={},
nomSecondResp={Bich DAO},
emailSecondResp={Bich.DAO@univ-orleans.fr},
langue={Français_},
%INTRO ou description courte
{Unité obligatoire. A choisir pour le parcours WIN : Web, Intelligence et Nomadisme.},
%Description Longue
{Ce module s'inscrit dans une démarche déclarative et descriptive pour modéliser et résoudre des problèmes combinatoires complexes et professionellement pertinents.
On y montre l'application des contraintes dans un éventail de problèmes réels, en mettant l'accent sur la pratique de la modélisation et l'utilisation des outils.
Il s'inscrit dans la continuité du module PLC de M1 qui présente le paradigme de la programmation logique et offre une introduction aux contraintes.},
%Prerequis
{Programmation en logique et par contraintes (vu en M1).},
%Objectif
{\begin{itemize}
\ObjItem Modélisation et résolution de problèmes par approche déclarative.
\end{itemize}},
%Ressources
{},
%Biblio
{Biblio},

\vfill


\module[codeApogee={},
titre={Sécurité des applications nomades},
COURS={20},
TD={15},
TP={0},
CTD={0},
TOTAL={35},
SEMESTRE={Semestre 0},
COEFF={4},
ECTS={4},
MethodeEval={},
ModalitesCCSemestreUn={},
ModalitesCCSemestreDeux={},
CalculNFSessionUne={$frac{(CC+2*CT)}{3}$},
CalculNFSessionDeux={},
NoteEliminatoire={0},
nomPremierResp={},
emailPremierResp={},
nomSecondResp={},
emailSecondResp={},
langue={Français_},
%INTRO ou description courte
{Unité obligatoire.},
%Description Longue
{Ce cours porte sur la sécurité des applications J2ME (Java 2 Mobile Edition) et se décompose en deux parties.
La première partie traite des problèmes liés à la configuration de la politique de sécurité de la machine virtuelle
(security manager, chargeur de classe, contrôle  d'accès, signature de classes, ...) et des bonnes pratiques de programmation.
Plusieurs aspects du langage Java (héritage, modificateurs,   sérialisation, JNI...) pouvant avoir un impact sur la sécurité des  applications seront étudiés.
En particulier, l'accent sera mis, au travers d'une étude de la spécification du langage, sur les pratiques de développement Java conduisant à la production d'un code robuste.
La seconde partie du cours portera sur le code exécuté par la machine virtuelle et la spécification de cette dernière.
En particulier, les mécanismes de vérification de bytecode mis en oeuvre par la machine virtuelle (principalement basés sur la sureté du typage)
et les techniques d'analyse sous-jacentes seront étudiés. Finalement, ces techniques d'analyse seront généralisées afin de permettre
leur application à des propriétés de sécurité plus précises.},
%Prerequis
{Développement d'application nomades.
Programmation Java.},
%Objectif
{\begin{itemize}
\item Capacité à configurer correctement une machine virtuelle Java en fonction d'une politique de sécurité donnée.
\item Maîtrise des subtilités du langage Java ayant un impact sur la sécurité des applications.
\item Connaissance des propriétés de sureté du code assurées par les machines virtuelle Java et des techniques d'analyse sous-jacentes.
\item Application de ces techniques à des propriétés de sécurité spécifiques.
\end{itemize}},
%Ressources
{},
%Biblio
{Biblio},

\vfill


\end{document}
