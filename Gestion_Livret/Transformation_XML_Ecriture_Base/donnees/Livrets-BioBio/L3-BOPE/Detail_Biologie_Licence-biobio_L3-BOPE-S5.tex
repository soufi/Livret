\documentclass[10pt, a5paper]{report}

\usepackage[T1]{fontenc}%
\usepackage[utf8]{inputenc}% encodage utf8
\usepackage[francais]{babel}% texte français
\usepackage[final]{pdfpages}
\usepackage{modules-livret}% style du livret
\usepackage{url}
%\usepackage{init-preambule}
\pagestyle{empty}

% % % % % % % % % % % % % % % % % % % % % % % % % % % % % % % % % % % % % % % % % % % % % % % % % % % % % % % 
\begin{document}

%---------------------- % % % Personnalisation des couleurs % % % ----------- Vert Licence --------
\definecolor{couleurFonce}{RGB}{18,92,40} % Couleur du Code APOGEE
\definecolor{couleurClaire}{RGB}{28,161,68} % Couleur du fond de la bande
\definecolor{couleurTexte}{RGB}{255,255,255} % Couleur du texte de la bande
%------------------------------------------------------------------------------------------


%==========================================================================================
% Semestre 5
%==========================================================================================
\module[codeApogee={SOL5BO01},
titre={Biodiversité}, 
COURS={4}, 
TD={4}, 
TP={40}, 
CTD={},
CTP={}, 
TOTAL={48}, 
SEMESTRE={Semestre 5}, 
COEFF={5}, 
ECTS={5}, 
MethodeEval={TP},
ModalitesCCSemestreUn={RNE et RSE : CC+CT 1h+1h},
ModalitesCCSemestreDeux={RNE et RSE : CT 1h+1h},
CalculNFSessionUne={50\% + 50\%},
CalculNFSessionDeux={50\% + 50\%},
NoteEliminatoire={}, 
nomPremierResp={Christiane Depierreux}, 
emailPremierResp={christiane.depierreux@univ-orleans.fr}, 
nomSecondResp={Géraldine Roux}, 
emailSecondResp={geraldine.roux@univ-orleans.fr}, 
langue={Français}, 
nbPrerequis={1}, 
descriptionCourte={true}, 
descriptionLongue={true}, 
objectifs={true}, 
ressources={true}, 
bibliographie={false}] 
% ******* Texte introductif
{
Parcours BOPE / BBV / BGST
} 
% ******* Contenu détaillé
{
\begin{itemize}
\item Inventaire et diversité taxonomique des principaux groupes d’animaux. Floristique et systématique du monde végétal. Clefs pour la détermination des taxons rencontrés. Mesure des indices de diversité.
\item Observations sur le terrain (un groupe TP terrain encadré par deux enseignants) et analyses en laboratoire (TP)
\end{itemize}
} 
% ******* Pré-requis
{Bases en systématique des arthropodes et en écologie des peuplements.
} 
% ******* Objectifs
{\begin{itemize} 
  \ObjItem Permettre à l’étudiant de découvrir des milieux naturels de notre région, participer à l’inventaire de la faune et de la flore par la réalisation d’un herbier personnel et d’une boite de référence d’insectes.
\end{itemize} 
} 
% ******* Ressources pédagogiques
{} 
% ******* Bibliographie éventuelle
{Biblio}
 
\vfill

\module[codeApogee={SOL5BO02},
titre={Ecologie fonctionnelle}, 
COURS={14}, 
TD={8}, 
TP={14}, 
CTD={},
CTP={}, 
TOTAL={36}, 
SEMESTRE={Semestre 5}, 
COEFF={4}, 
ECTS={4}, 
MethodeEval={Ecrit/Oral},
ModalitesCCSemestreUn={RNE et RSE : CM(CT) + TP/TD (CT oral) 2h+20 min},
ModalitesCCSemestreDeux={RNE et RSE : CM(CT) + TP/TD (CT oral) 2h+20 min},
CalculNFSessionUne={60\% CM + 40\% TD+TP},
CalculNFSessionDeux={60\% CM + 40\% TD+TP},
NoteEliminatoire={}, 
nomPremierResp={Aurélien Sallé}, 
emailPremierResp={aurelien.salle@univ-orleans.fr}, 
nomSecondResp={}, 
emailSecondResp={}, 
langue={Français}, 
nbPrerequis={1}, 
descriptionCourte={false}, 
descriptionLongue={true}, 
objectifs={true}, 
ressources={true}, 
bibliographie={false}] 
% ******* Texte introductif
{
} 
% ******* Contenu détaillé
{
\begin{itemize}
\item Le fonctionnement des écosystèmes ; modalités des transferts de matières et d’énergie (puits, sources, zones de stockage), au travers des réseaux trophiques ; quelques dysfonctionnements et leurs remèdes ; analyse d’un milieu en liaison
avec une perturbation ; notion de pédologie.
\item Un groupe de TP terrain encadré par deux enseignants
\end{itemize}
} 
% ******* Pré-requis
{Notions d'écologie générale.
} 
% ******* Objectifs
{\begin{itemize} 
  \ObjItem Appréhension théorique et pratique du fonctionnement des écosystèmes et de l’impact des activités anthropiques sur le cycle du carbone et l’effet de serre.
\end{itemize} 
} 
% ******* Ressources pédagogiques
{} 
% ******* Bibliographie éventuelle
{Biblio}
 
\vfill
\module[codeApogee={SOL5BO03},
titre={Evolution et adaptation des angiospermes}, 
COURS={14}, 
TD={12}, 
TP={10}, 
CTD={},
CTP={}, 
TOTAL={36}, 
SEMESTRE={Semestre 5}, 
COEFF={4}, 
ECTS={4}, 
MethodeEval={Ecrit/TP},
ModalitesCCSemestreUn={RNE et RSE : Ecrit (CT) + TP (CC+CT) 1h30+2h},
ModalitesCCSemestreDeux={RNE et RSE : Ecrit (CT) + TP (CT) 1h30+2h},
CalculNFSessionUne={66\% Ecrit + 33\% TP},
CalculNFSessionDeux={66\% Ecrit + 33\% TP},
NoteEliminatoire={}, 
nomPremierResp={Christiane Depierreux}, 
emailPremierResp={christiane.depierreux@univ-orleans.fr}, 
nomSecondResp={}, 
emailSecondResp={}, 
langue={Français}, 
nbPrerequis={1}, 
descriptionCourte={true}, 
descriptionLongue={true}, 
objectifs={true}, 
ressources={true}, 
bibliographie={false}] 
% ******* Texte introductif
{
Parcours BOPE / BBV / BGST
} 
% ******* Contenu détaillé
{
Evolution et adaptation des végétaux au milieu : Compléments sur la reproduction et l’évolution de l’appareil reproducteur des angiospermes. Notions de classification. Adaptations morphologiques et anatomiques des végétaux aux différentes contraintes environnementales. Perception des signaux de contrainte : lumière, température, gravité...
} 
% ******* Pré-requis
{Notions d’organisation, d’anatomie et de reproduction des embryophytes.
} 
% ******* Objectifs
{\begin{itemize} 
  \ObjItem Acquérir les connaissances sur l’évolution et l’adaptation des plantes aux milieux.
\end{itemize} 
} 
% ******* Ressources pédagogiques
{} 
% ******* Bibliographie éventuelle
{Biblio}
 
\vfill
\module[codeApogee={SOL5BO04},
titre={Lois de probabilités et estimations de paramètres usuels}, 
COURS={}, 
TD={}, 
TP={}, 
CTD={24},
CTP={}, 
TOTAL={24}, 
SEMESTRE={Semestre 5}, 
COEFF={3}, 
ECTS={3}, 
MethodeEval={Ecrit/Oral},
ModalitesCCSemestreUn={RNE et RSE : CT (Ecrit) 1h},
ModalitesCCSemestreDeux={RNE et RSE : CT (Oral) 15 min},
CalculNFSessionUne={100\% CT},
CalculNFSessionDeux={100\% CT},
NoteEliminatoire={}, 
nomPremierResp={Franck Brignolas}, 
emailPremierResp={franck.brignolas@univ-orleans.fr}, 
nomSecondResp={}, 
emailSecondResp={}, 
langue={Français}, 
nbPrerequis={0}, 
descriptionCourte={true}, 
descriptionLongue={true}, 
objectifs={true}, 
ressources={true}, 
bibliographie={false}] 
% ******* Texte introductif
{
Parcours BOPE / BMC / BBV
} 
% ******* Contenu détaillé
{
Notion de variable aléatoire (VA) ; VA qualitatives, VA quantitatives discrètes et continues ; principales lois de probabilité et leur utilisation en biologie ; estimation de paramètres et intervalles de confiance.
} 
% ******* Pré-requis
{
} 
% ******* Objectifs
{\begin{itemize} 
  \ObjItem Acquisition de connaissances de bases en statistique.
\end{itemize} 
} 
% ******* Ressources pédagogiques
{} 
% ******* Bibliographie éventuelle
{Biblio}
 
\vfill
\module[codeApogee={SOL5BO05},
titre={Bases anatomiques des grandes fonctions animales}, 
COURS={14}, 
TD={}, 
TP={10}, 
CTD={},
CTP={}, 
TOTAL={24}, 
SEMESTRE={Semestre 5}, 
COEFF={3}, 
ECTS={3}, 
MethodeEval={Ecrit/Oral},
ModalitesCCSemestreUn={RNE et RSE : CT (Ecrit) 1h},
ModalitesCCSemestreDeux={RNE et RSE : CT (Oral) 1h},
CalculNFSessionUne={100\%},
CalculNFSessionDeux={100\%},
NoteEliminatoire={}, 
nomPremierResp={Valérie Altemayer}, 
emailPremierResp={valerie.altemayer@univ-orleans.fr}, 
nomSecondResp={}, 
emailSecondResp={}, 
langue={Français}, 
nbPrerequis={1}, 
descriptionCourte={true}, 
descriptionLongue={true}, 
objectifs={true}, 
ressources={true}, 
bibliographie={false}] 
% ******* Texte introductif
{
Parcours BOPE/ BMC / BGST / PLURI
} 
% ******* Contenu détaillé
{
\begin{itemize}
\item Cours : Organisation des tissus : tissus épithéliaux – conjonctifs – musculaires. Description anatomique et histologique des appareils : circulatoire – respiratoire – digestif excréteur et reproducteur. 
\item TP illustration du cours à partir de préparations et de coupes histologiques – Etude de l’organisation de tous les appareils présentés en cours chez une souris. Les TP pourront être enseignés en anglais.
\end{itemize}
} 
% ******* Pré-requis
{Avoir de bonnes bases en biologie cellulaire.
} 
% ******* Objectifs
{\begin{itemize} 
  \ObjItem Acquisition de l’anatomie et de l’histologie des grandes fonctions avant d’appréhender la physiologie.
\end{itemize} 
} 
% ******* Ressources pédagogiques
{} 
% ******* Bibliographie éventuelle
{Biblio}
 
\vfill
\module[codeApogee={SOL5IP01},
titre={Insertion professionnelle}, 
COURS={10}, 
TD={9}, 
TP={}, 
CTD={},
CTP={}, 
TOTAL={19}, 
SEMESTRE={Semestre 5}, 
COEFF={1}, 
ECTS={1}, 
MethodeEval={Ecrit / Oral},
ModalitesCCSemestreUn={RNE et RSE : CT(Ecrit) 1h30},
ModalitesCCSemestreDeux={RNE et RSE : CT(Ecrit) 1h / CT (Oral) 15 min },
CalculNFSessionUne={100\% CT},
CalculNFSessionDeux={50\% Ecrit + 50\% Oral},
NoteEliminatoire={}, 
nomPremierResp={Olivier Richard}, 
emailPremierResp={olivier.richard@univ-orleans.fr}, 
nomSecondResp={}, 
emailSecondResp={}, 
langue={Français}, 
nbPrerequis={0}, 
descriptionCourte={true}, 
descriptionLongue={true}, 
objectifs={true}, 
ressources={true}, 
bibliographie={false}] 
% ******* Texte introductif
{
Parcours BOPE / BMC / BBV
} 
% ******* Contenu détaillé
{
\begin{itemize}
\item (4h par des intervenants du domaine) Découverte de l’entreprise privée (dans le domaine des Sciences de la Vie) : rôle économique, organisation, fonctionnement, types de métiers, modes de recrutement, droit du travail (4h par des intervenants du domaine) Découverte de l’entreprise publique (dans le domaine des Sciences de la vie : CNRS/INRA/Université etc ...) : statuts, buts, organisation hiérarchiques et carrières, modes de financement, modes de recrutement.
\item Travail en groupe (TD groupe de 20) Initiation à la rédaction d’un CV, d’une lettre de motivation dans le cadre d’une demande de stage ou d’inscription en Master. Première approche de la situation de l’entretien de recrutement (niveau L, objectif le stage ou l’insertion en M1). Sensibilisation aux moyens pour rechercher l’information sur les stages et les emplois (secteur privé). Utilisation des ressources en ligne sur les métiers de la fonction publique.
\end{itemize}
} 
% ******* Pré-requis
{
} 
% ******* Objectifs
{\begin{itemize} 
  \ObjItem Rendre actif la démarche étudiante pour une insertion après la Licence (emploi, formation complémentaire, Master). Connaissance du tissu économique dans le domaine des Sciences de la Vie. Réflexion de l’étudiant sur son projet personnel. Elaboration d’un CV niveau L, d’une demande de stage niveau L.
\end{itemize} 
} 
% ******* Ressources pédagogiques
{} 
% ******* Bibliographie éventuelle
{Biblio}
 
\vfill

\module[codeApogee={SOL5AG35},
titre={Anglais 5}, 
COURS={}, 
TD={24}, 
TP={}, 
CTD={},
CTP={}, 
TOTAL={24}, 
SEMESTRE={Semestre 5}, 
COEFF={3}, 
ECTS={3}, 
MethodeEval={Ecrit},
ModalitesCCSemestreUn={RNE : CC 3h ; RSE : CT 1h30},
ModalitesCCSemestreDeux={RNE et RSE : CT 1h30},
CalculNFSessionUne={100\%},
CalculNFSessionDeux={100\%},
NoteEliminatoire={}, 
nomPremierResp={Hervé Perreau}, 
emailPremierResp={herve.perreau@univ-orleans.fr}, 
nomSecondResp={}, 
emailSecondResp={}, 
langue={Français/Anglais}, 
nbPrerequis={1}, 
descriptionCourte={true}, 
descriptionLongue={true}, 
objectifs={true}, 
ressources={true}, 
bibliographie={false}] 
% ******* Texte introductif
{
Parcours BOPE / BMC/ BBV / BGST / PLURI
} 
% ******* Contenu détaillé
{
Travail de compréhension et d’expression orale à partir de documents authentiques longs et/ou complexes, portant sur des innovations technologiques, des découvertes ou avancées scientifiques.
} 
% ******* Pré-requis
{Avoir suivi Anglais 3 + 4 ou environ 500 heures de formation équivalente.
} 
% ******* Objectifs
{\begin{itemize} 
  \ObjItem Comprendre l’information exprimée dans des messages complexes sur le domaine des Sciences et Technologies et s’exprimer sur ce même domaine à l’écrit dans un registre de langue approprié.
\end{itemize} 
} 
% ******* Ressources pédagogiques
{} 
% ******* Bibliographie éventuelle
{Biblio}
 
\vfill
\module[codeApogee={SOL5BO06},
titre={Croissance et développement des végétaux}, 
COURS={24}, 
TD={}, 
TP={12}, 
CTD={},
CTP={}, 
TOTAL={36}, 
SEMESTRE={Semestre 5}, 
COEFF={4}, 
ECTS={4}, 
MethodeEval={Ecrit/TP},
ModalitesCCSemestreUn={RNE : Ecrit CT 2h / TP CC ; RSE : Ecrit CT 2h / TP CT},
ModalitesCCSemestreDeux={RNE et RSE : Ecrit/TP CT 2h/1h},
CalculNFSessionUne={Ecrit 66\% + TP 33\%},
CalculNFSessionDeux={Ecrit 66\% + TP 33\%},
NoteEliminatoire={}, 
nomPremierResp={Daniel Hagège}, 
emailPremierResp={daniel.hagege@univ-orleans.fr}, 
nomSecondResp={}, 
emailSecondResp={}, 
langue={Français}, 
nbPrerequis={0}, 
descriptionCourte={true}, 
descriptionLongue={true}, 
objectifs={true}, 
ressources={true}, 
bibliographie={true}] 
% ******* Texte introductif
{
Parcours BOPE / BBV
} 
% ******* Contenu détaillé
{
Les phytohormones (les principales familles, biosynthèse et dégradation, transport, rôles physiologiques, utilisations pratiques) - La graine : de la mise en réserve à la mobilisation - La germination - La floraison (vernalisation, photopériodisme, thermopériodisme, théories de la floraison, morphogenèse florale)
} 
% ******* Pré-requis
{
} 
% ******* Objectifs
{\begin{itemize} 
  \ObjItem Acquisition des connaissances de base de la physiologie du développement des plantes.
\end{itemize} 
} 
% ******* Ressources pédagogiques
{
} 
% ******* Bibliographie éventuelle
{Heller et al.-T1 et 2 (Dunod) ; Mazliak et al. (Hermann) ; Laval-Martin et Mazliak (Hermann) ; Guignard- Biochimie végétale (Dunod 2000) ; Luttge, Kluge et Bauer, Botanique (Tec et Doc 1996) ; Anderson-Beardall : Molecular activities of plant cells. (Blackwell Scientific Pub.) ; Campbell : Biologie (De Boeck Université) ; Nultsch : Botanique générale (De Boeck Université) ; Taiz and Zeiger : Plant physiology (Sinauer) ; Buchanan, Gruissem and Jones, Biochemistry and molecular biology (Amer. Soc. Plant Physiol.)
}
 
\vfill
\module[codeApogee={SOL5BO07},
titre={Entomologie générale}, 
COURS={14}, 
TD={10}, 
TP={12}, 
CTD={},
CTP={}, 
TOTAL={36}, 
SEMESTRE={Semestre 5}, 
COEFF={4}, 
ECTS={4}, 
MethodeEval={Ecrit/Oral},
ModalitesCCSemestreUn={RNE : Ecrit CT 2h / TP CC ; RSE : Ecrit CT 2h / TP CT},
ModalitesCCSemestreDeux={RNE et RSE : CT(Ecrit+Oral) 2h+15min},
CalculNFSessionUne={75\% Ecrit + 25\% TP},
CalculNFSessionDeux={60\% Ecrit + 40\% TP},
NoteEliminatoire={}, 
nomPremierResp={Stéphanie Bankhead-Dronnet}, 
emailPremierResp={stephanie.bankhead@univ-orleans.fr}, 
nomSecondResp={}, 
emailSecondResp={}, 
langue={Français}, 
nbPrerequis={1}, 
descriptionCourte={false}, 
descriptionLongue={true}, 
objectifs={true}, 
ressources={true}, 
bibliographie={false}] 
% ******* Texte introductif
{
} 
% ******* Contenu détaillé
{
Morphologie, anatomie et physiologie de l’insecte : appareil digestif et digestion, tissu adipeux, appareil excréteur et osmorégulation, appareil circulatoire, système trachéen et respiration, systèmes sensoriels et nerveux, appareils reproducteurs. Développement post-embryonnaire, mue et métamorphoses. Physiologie du développement : diapause, contrôle endocrine. Systématique générale ; Eusocialité des insectes ; Insectes parasites.
} 
% ******* Pré-requis
{Caractères généraux des Euarthropodes.
} 
% ******* Objectifs
{\begin{itemize} 
  \ObjItem Connaître la systématique des Insectes jusqu’à l’Ordre et sous-Ordre, les caractères morphologiques et les grandes fonctions, et la place des insectes dans le monde animal.
\end{itemize} 
} 
% ******* Ressources pédagogiques
{} 
% ******* Bibliographie éventuelle
{Biblio}
 
\vfill
\module[codeApogee={SOL5BO08},
titre={Télédétection et cartographie végétale}, 
COURS={2}, 
TD={10}, 
TP={12}, 
CTD={},
CTP={}, 
TOTAL={24}, 
SEMESTRE={Semestre 5}, 
COEFF={3}, 
ECTS={3}, 
MethodeEval={Oral},
ModalitesCCSemestreUn={RNE et RSE : CT(E+TP) 15min+20min},
ModalitesCCSemestreDeux={RNE et RSE : CT 20min},
CalculNFSessionUne={50\% + 50\%},
CalculNFSessionDeux={100\%},
NoteEliminatoire={}, 
nomPremierResp={Cécile Vincent}, 
emailPremierResp={cecile.vincent@univ-orleans.fr}, 
nomSecondResp={}, 
emailSecondResp={}, 
langue={Français}, 
nbPrerequis={1}, 
descriptionCourte={false}, 
descriptionLongue={true}, 
objectifs={true}, 
ressources={false}, 
bibliographie={false}] 
% ******* Texte introductif
{
} 
% ******* Contenu détaillé
{
Acquérir des connaissances de base sur le mode d’organisation, la création, la gestion et l’analyse d’une base de données à références spatiales (Systèmes d’Informations Géographiques). Comprendre les principes de base de la télédétection
et les potentialités offertes par cette technique pour des applications environnementales. Application à une étude de cas sur le terrain avec relevé floristique et mesures des variables environnementales.
} 
% ******* Pré-requis
{Notions écologie, télédétection, botanique.
} 
% ******* Objectifs
{\begin{itemize} 
  \ObjItem Utiliser les outils de télédétection pour le prétraitement, traitement d’images et extraction de l’information à partir de données de télédétection.
\end{itemize} 
} 
% ******* Ressources pédagogiques
{} 
% ******* Bibliographie éventuelle
{Biblio}
 
\vfill
\module[codeApogee={SOL5BO09},
titre={Génétique quantitative}, 
COURS={12}, 
TD={12}, 
TP={}, 
CTD={},
CTP={}, 
TOTAL={24}, 
SEMESTRE={Semestre 5}, 
COEFF={3}, 
ECTS={3}, 
MethodeEval={Ecrit/Oral},
ModalitesCCSemestreUn={RNE : CC(TD) / CT(Ecrit+Oral) 1h+15min ; RSE : CT(Ecrit+Oral) 1h+15min},
ModalitesCCSemestreDeux={RNE et RSE : CT(Ecrit+Oral) 1h+15min},
CalculNFSessionUne={RNE : 33\% + 33\% + 33\% (TD+Ecrit+Oral) ; RSE : 50\% + 50\% (Ecrit+Oral)},
CalculNFSessionDeux={50\% + 50\% (Ecrit+Oral)},
NoteEliminatoire={}, 
nomPremierResp={Arnaud Menuet}, 
emailPremierResp={arnaud.menuet@univ-orleans.fr}, 
nomSecondResp={}, 
emailSecondResp={}, 
langue={Français}, 
nbPrerequis={1}, 
descriptionCourte={false}, 
descriptionLongue={true}, 
objectifs={true}, 
ressources={false}, 
bibliographie={false}] 
% ******* Texte introductif
{
} 
% ******* Contenu détaillé
{
Etude de la transmission des caractères quantitatifs. Notions d’héritabilité. Illustration de l’analyse des caractères quantitatifs chez l’homme. La sélection artificielle et les nouveaux outils moléculaires. De la sélection naturelle à la dérive génétique.
} 
% ******* Pré-requis
{Génétique formelle de première année de Licence.
} 
% ******* Objectifs
{\begin{itemize} 
  \ObjItem Montrer comment l’analyse formelle chez un organisme ne se limite pas aux caractères qualitatifs.
\end{itemize} 
} 
% ******* Ressources pédagogiques
{} 
% ******* Bibliographie éventuelle
{Biblio}
 
\vfill

\module[codeApogee={UEL},
titre={Maths prépa concours : techniques de calcul en mathématiques}, 
COURS={}, 
TD={20}, 
TP={}, 
CTD={},
CTP={}, 
TOTAL={20}, 
SEMESTRE={Semestre 5}, 
COEFF={3}, 
ECTS={3}, 
%MethodeEval={???}, 
ModalitesCCSemestreUn={Cf. modalités de contrôle de connaissances des UE Libres}, 
ModalitesCCSemestreDeux={Cf. modalités de contrôle de connaissances des UE Libres}, 
%CalculNFSessionUne={Examen 67 \% ; TP 33 \%}, 
%CalculNFSessionDeux={Examen 67 \% ; TP 33 \%}, 
NoteEliminatoire={}, 
nomPremierResp={Emmanuel Cepa}, 
emailPremierResp={emmanuel.cepa@univ-orleans.fr}, 
nomSecondResp={}, 
emailSecondResp={}, 
langue={Français}, 
nbPrerequis={1}, 
descriptionCourte={true}, 
descriptionLongue={true}, 
objectifs={true}, 
ressources={false}, 
bibliographie={false}] 
% ******* Texte introductif
{
UE pouvant être prise aussi en semestre 3
} 
% ******* Contenu détaillé
{
Les thèmes suivants seront abordés. Les notions minimales de cours seront données afin de privilégier les exemples.
\begin{itemize}
\item Fonctions numériques de la variable réelle / Intégrales / Equations différentielles
\item Suites / Algèbre Linéaire /Probabilités
\end{itemize}
}
% ******* Pré-requis
{Maths niveau BAC S, forte motivation.} 
% ******* Objectifs
{\begin{itemize} 
  \ObjItem Permettre à des étudiants d’acquérir des connaissances utiles pour passer les concours d’accès aux écoles nationales vétérinaires (concours ENV- voie B), les concours d’ingénieurs agro (concours ENSA voie B) mais également certains concours administratifs.
\end{itemize} 
} 
% ******* Ressources pédagogiques
{} 
% ******* Bibliographie éventuelle
{Biblio}%==========================================================================================
%===================================================================================
\end{document}
