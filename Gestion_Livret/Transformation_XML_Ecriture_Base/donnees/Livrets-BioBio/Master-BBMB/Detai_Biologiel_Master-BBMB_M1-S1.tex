\documentclass[10pt, a5paper]{report}

\usepackage[T1]{fontenc}%
\usepackage[utf8]{inputenc}% encodage utf8
\usepackage[francais]{babel}% texte français
\usepackage[final]{pdfpages}
\usepackage{modules-livret}% style du livret
\usepackage{url}
%\usepackage{init-preambule}
\pagestyle{empty}

% % % % % % % % % % % % % % % % % % % % % % % % % % % % % % % % % % % % % % % % % % % % % % % % % % % % % % % 
\begin{document}

%---------------------- % % % Personnalisation des couleurs % % % ----------- Vert Licence --------
\definecolor{couleurFonce}{RGB}{108,5,5} % Couleur du Code APOGEE
\definecolor{couleurClaire}{RGB}{147,97,97} % Couleur du fond de la bande
\definecolor{couleurTexte}{RGB}{255,255,255} % Couleur du texte de la bande
%------------------------------------------------------------------------------------------


%==========================================================================================
% Semestre 1
%==========================================================================================
\module[codeApogee={SOM1BH01},
titre={Expression du génôme eucaryote}, 
COURS={24}, 
TD={14}, 
TP={10}, 
CTD={},
CTP={}, 
TOTAL={48}, 
SEMESTRE={Semestre 1}, 
COEFF={5}, 
ECTS={5}, 
MethodeEval={Ecrit/TP},
ModalitesCCSemestreUn={RNE et RSE : CT(E) 2h / CC(TP)},
ModalitesCCSemestreDeux={RNE et RSE : CT(E) 2h / CC(TP)},
CalculNFSessionUne={Ecrit 75\% + TP 25 \%},
CalculNFSessionDeux={Ecrit 75\% + TP 25 \%},
NoteEliminatoire={7}, 
nomPremierResp={Alain Legrand}, 
emailPremierResp={alain.legrand@cnrs-orleans.fr}, 
nomSecondResp={}, 
emailSecondResp={}, 
langue={Français}, 
nbPrerequis={1}, 
descriptionCourte={true}, 
descriptionLongue={true}, 
objectifs={true}, 
ressources={false}, 
bibliographie={false}] 
% ******* Texte introductif
{
Spécialités BBMB et BOPE / Passerelle MESC2A
} 
% ******* Contenu détaillé
{
Régulation au niveau de l’ADN : Régulation transcriptionnelle (mode d’activation des facteurs de transcription), post-transcriptionnelle (coiffe des ARNm et polyadénylation), traductionnelle (initiation coiffe dépendante et indépendante), Epissage alternatif, constitutif et en trans), structure génique complexes , méthodologie ( étude du transcriptome, hybridation différentielle, puces à ADN, criblage virtuel de banque de données, RT-PCR quantitative, technique de double hybride).
}
% ******* Pré-requis
{18 ECTS de biologie moléculaire
} 
% ******* Objectifs
{\begin{itemize} 
  \ObjItem Approfondir la connaissance des mécanismes de régulation de l'expression génique au niveau transcriptionnel, post transcriptionnel et traductionnel. Présentation des techniques les plus récentes utilisées dans le domaine de recherche.
\end{itemize} 
} 
% ******* Ressources pédagogiques
{} 
% ******* Bibliographie éventuelle
{Biblio}
 
\vfill
%===================================================================================
\module[codeApogee={SOM1AG41},
titre={Anglais scientifique 1}, 
COURS={}, 
TD={24}, 
TP={}, 
CTD={},
CTP={}, 
TOTAL={24}, 
SEMESTRE={Semestre 1}, 
COEFF={3}, 
ECTS={3}, 
MethodeEval={Ecrit/Oral},
ModalitesCCSemestreUn={RNE et RSE : CT(E) 1h30 / CC(O) 1h30},
ModalitesCCSemestreDeux={RNE et RSE : CT(E) 1h30},
CalculNFSessionUne={Ecrit 50\% + oral 50 \%},
CalculNFSessionDeux={Ecrit 100\%},
NoteEliminatoire={7}, 
nomPremierResp={Lupka Mihajlovska}, 
emailPremierResp={lupka.mihajlovska@univ-orleans.fr}, 
nomSecondResp={}, 
emailSecondResp={}, 
langue={Anglais}, 
nbPrerequis={1}, 
descriptionCourte={true}, 
descriptionLongue={true}, 
objectifs={true}, 
ressources={false}, 
bibliographie={false}] 
% ******* Texte introductif
{
Spécialités BBMB et BOPE / Passerelle MESC2A
} 
% ******* Contenu détaillé
{
Remise à niveau, Conversation en anglais avec les moyens audio-visuels modernes (e-mail, video-projecteur…)}
% ******* Pré-requis
{Bon niveau d’ordre général pratique, oral (conversation, téléphone, voyage) et écrite (synthèse de lecture, s’exprimer simplement mais clairement). Assez bonne connaissance de la langue spécifique de son domaine scientifique et technique.
} 
% ******* Objectifs
{\begin{itemize} 
  \ObjItem Savoir présenter un rapport de travail en anglais et commenter le déroulement d’une opération.
\end{itemize} 
} 
% ******* Ressources pédagogiques
{} 
% ******* Bibliographie éventuelle
{Biblio}
 
\vfill
%===================================================================================
\module[codeApogee={SOM1BH02},
titre={Biologie Moléculaire et Cellulaire Expérimentales}, 
COURS={}, 
TD={}, 
TP={48}, 
CTD={},
CTP={}, 
TOTAL={48}, 
SEMESTRE={Semestre 1}, 
COEFF={5}, 
ECTS={5}, 
MethodeEval={Ecrit/TP},
ModalitesCCSemestreUn={RNE et RSE : CC(TP)},
ModalitesCCSemestreDeux={RNE et RSE : CT écrit 1h},
CalculNFSessionUne={TP 100\%},
CalculNFSessionDeux={Ecrit 100\%},
NoteEliminatoire={7}, 
nomPremierResp={Lucile Mollet}, 
emailPremierResp={lucile.mollet@cnrs-orleans.fr}, 
nomSecondResp={}, 
emailSecondResp={}, 
langue={Français}, 
nbPrerequis={1}, 
descriptionCourte={true}, 
descriptionLongue={true}, 
objectifs={true}, 
ressources={false}, 
bibliographie={false}] 
% ******* Texte introductif
{
Spécialités BBMB et BOPE / Passerelle MESC2A
} 
% ******* Contenu détaillé
{
Construction de vecteurs d’expression avec gènes rapporteurs pour l’analyse de la domiciliation de protéines de fusion (gène GFP). Cultures cellulaire et transfection de vecteurs d’expression permettant l’analyse de l’activité de promoteurs (gène rapporteur luciférase, luminométrie). PCR : application à la détection de mutations ponctuelles. Méthode semi-quantitative pour la quantification d’ARN spécifiques. Interactions protéine/protéine \textit{in vitro} par chromatographie d'affinité : techniques du GST pull-down et analyse par western blot.
}
% ******* Pré-requis
{18 ECTS de biologie moléculaire
} 
% ******* Objectifs
{\begin{itemize} 
  \ObjItem Cette unité entièrement consacrée au travail expérimental vise à permettre d'acquérir une bonne pratique de quelques techniques fondamentales utilisées en biologies moléculaire et cellulaire.
\end{itemize} 
} 
% ******* Ressources pédagogiques
{} 
% ******* Bibliographie éventuelle
{Biblio}
 
\vfill
%===================================================================================
\module[codeApogee={SOM1BO02},
titre={Biostat 1 : Initiation à "R"}, 
COURS={}, 
TD={24}, 
TP={}, 
CTD={},
CTP={}, 
TOTAL={24}, 
SEMESTRE={Semestre 1}, 
COEFF={3}, 
ECTS={3}, 
MethodeEval={Ecrit},
ModalitesCCSemestreUn={RNE et RSE : CT 1h},
ModalitesCCSemestreDeux={RNE et RSE : CT 1h},
CalculNFSessionUne={Ecrit 100\%},
CalculNFSessionDeux={Ecrit 100\%},
NoteEliminatoire={7}, 
nomPremierResp={Franck Brignolas}, 
emailPremierResp={franck.brignolas@univ-orleans.fr}, 
nomSecondResp={}, 
emailSecondResp={}, 
langue={Français}, 
nbPrerequis={1}, 
descriptionCourte={true}, 
descriptionLongue={true}, 
objectifs={true}, 
ressources={false}, 
bibliographie={false}] 
% ******* Texte introductif
{
Spécialités BBMB et BOPE / Passerelle MESC2A
} 
% ******* Contenu détaillé
{
- Présentation de "R" : objets de données, fonctions et opérateurs ; manipulations des objets de données ; fonctions de statistique descriptives ; création de graphiques usuels.
- Utilisation de "R" : lois de probabilités ; tests statistiques : comparaison d’effectifs, de proportions, de moyennes, de médianes, de variances et de distribution ; corrélations entre variables ; régression linéaire.
}
% ******* Pré-requis
{UE Biométrie de licence 3 (ou équivalent)
} 
% ******* Objectifs
{\begin{itemize} 
  \ObjItem Maîtriser les tests statistiques couramment utilisés dans les Sciences de la Vie.
\ObjItem Acquérir une autonomie dans l’analyse des données en utilisant le logiciel "R".
\end{itemize} 
} 
% ******* Ressources pédagogiques
{} 
% ******* Bibliographie éventuelle
{Biblio}
 
\vfill
%===================================================================================
\module[codeApogee={SOM1BH03},
titre={Dynamique et régulations cellulaires}, 
COURS={28}, 
TD={20}, 
TP={}, 
CTD={},
CTP={}, 
TOTAL={48}, 
SEMESTRE={Semestre 1}, 
COEFF={5}, 
ECTS={5}, 
MethodeEval={Ecrit/Oral},
ModalitesCCSemestreUn={RNE et RSE : CT(E) 2h / CT(O)},
ModalitesCCSemestreDeux={RNE et RSE : CT(E) 2h / CT(O)},
CalculNFSessionUne={Ecrit 75\% + oral 25 \%},
CalculNFSessionDeux={Ecrit 75\% + oral 25 \%},
NoteEliminatoire={7}, 
nomPremierResp={Stéphane Charpentier}, 
emailPremierResp={stephane.charpentier@cnrs-orleans.fr}, 
nomSecondResp={}, 
emailSecondResp={}, 
langue={Français}, 
nbPrerequis={1}, 
descriptionCourte={true}, 
descriptionLongue={true}, 
objectifs={true}, 
ressources={false}, 
bibliographie={false}] 
% ******* Texte introductif
{
Spécialité BBMB et Passerelle MESC2A
} 
% ******* Contenu détaillé
{
Contrôle du cycle cellulaire : Généralités. Progression et point de contrôle. Techniques d’analyse : gènes essentiels du contrôle, signaux (MPF, complexe cycline-CDK, inhibiteur de CDK). Mécanismes de l’apoptose : introduction et principes de base. Voies moléculaires : mitochondries et récepteurs membranaires. Régulation des signaux apoptotiques. Signalisation trans-membranaire : diffusions simple et facilitée, transport passif et actif, canaux et transporteurs. Transduction du signal : récepteurs (définition, méthodes d’étude, agoniste, antagoniste, agoniste partiel, agoniste inverse…), effecteurs, seconds messagers… Récepteurs métabotropiques (tyrosine kinase, guanylate cyclase, couplés aux protéines G…) : structure (oligomérisation domaines d'interaction…), voies de transduction (MAP kinases, JAK/STAT, cyclases, phospholipases…), mécanismes de désensibilisation.
}
% ******* Pré-requis
{Biologie cellulaire 1 + 6 ECTS de biochimie générale
} 
% ******* Objectifs
{\begin{itemize} 
  \ObjItem Approfondir la connaissance des mécanismes biochimiques impliqués dans les transports transmembranaires et  dans la transduction du signal dans les cellules. L’intégration de ces mécanismes moléculaires fondamentaux dans le contrôle du cycle cellulaire, de l’apoptose et de la tumorisation sera plus particulièrement développée.
\end{itemize} 
} 
% ******* Ressources pédagogiques
{} 
% ******* Bibliographie éventuelle
{Biblio}
 
\vfill
%===================================================================================
\module[codeApogee={SOM1BH04},
titre={Immunologie 1}, 
COURS={24}, 
TD={}, 
TP={}, 
CTD={},
CTP={}, 
TOTAL={24}, 
SEMESTRE={Semestre 1}, 
COEFF={3}, 
ECTS={3}, 
MethodeEval={Ecrit},
ModalitesCCSemestreUn={RNE et RSE : CT(E) 2h},
ModalitesCCSemestreDeux={RNE et RSE : CT(E) 2h},
CalculNFSessionUne={Ecrit 100\%},
CalculNFSessionDeux={Ecrit 100\%},
NoteEliminatoire={7}, 
nomPremierResp={François Erard}, 
emailPremierResp={francois.erard@cnrs-orleans.fr}, 
nomSecondResp={}, 
emailSecondResp={}, 
langue={Français}, 
nbPrerequis={1}, 
descriptionCourte={true}, 
descriptionLongue={true}, 
objectifs={true}, 
ressources={false}, 
bibliographie={false}] 
% ******* Texte introductif
{
Spécialité BBMB et Passerelle MESC2A
} 
% ******* Contenu détaillé
{
Immunité naturelle ou innée : Reconnaissance des pathogènes, et réponses inflammatoires. Rôle des cellules myéloïdes, NK, NKT et dendritiques dans le contrôle de l’infection. 
}
% ******* Pré-requis
{Notions de base en immunologie.
} 
% ******* Objectifs
{\begin{itemize} 
  \ObjItem L’objectif majeur de ce module est de permettre aux étudiants une compréhension aisée des mécanismes cellulaires et moléculaires les plus subtils du fonctionnement normal du système immunitaire et de ses déficiences..
\end{itemize} 
} 
% ******* Ressources pédagogiques
{} 
% ******* Bibliographie éventuelle
{Biblio}
 
\vfill
%===================================================================================
\module[codeApogee={SOM1BH05},
titre={Biologie structurale}, 
COURS={24}, 
TD={}, 
TP={}, 
CTD={},
CTP={}, 
TOTAL={24}, 
SEMESTRE={Semestre 1}, 
COEFF={3}, 
ECTS={3}, 
MethodeEval={Ecrit},
ModalitesCCSemestreUn={RNE et RSE : CT(E) 2h},
ModalitesCCSemestreDeux={RNE et RSE : CT(E) 2h},
CalculNFSessionUne={Ecrit 100\%},
CalculNFSessionDeux={Ecrit 100\%},
NoteEliminatoire={7}, 
nomPremierResp={Chantal Pichon}, 
emailPremierResp={chantal.pichon@cnrs-orleans.fr}, 
nomSecondResp={}, 
emailSecondResp={}, 
langue={Français}, 
nbPrerequis={1}, 
descriptionCourte={true}, 
descriptionLongue={true}, 
objectifs={true}, 
ressources={false}, 
bibliographie={false}] 
% ******* Texte introductif
{
Spécialité BBMB
} 
% ******* Contenu détaillé
{
Introduction à la biologie structurale. A l’aide d’exemples, présenter les apports des techniques de spectroscopie de microscopie et d’imagerie conduisant à la connaissance et la compréhension des structures et des interactions des molécules biologiques ainsi que des organites cellulaires.
}
% ******* Pré-requis
{Connaissances de bases en biologie cellulaire et en biochimie.
} 
% ******* Objectifs
{\begin{itemize} 
  \ObjItem Connaissance de bases fondamentales et connaissance du domaine d’applicabilité des différentes techniques de spectroscopie, de microscopie et d’imagerie utilisées pour l’étude du vivant.
\end{itemize} 
} 
% ******* Ressources pédagogiques
{} 
% ******* Bibliographie éventuelle
{Biblio}
 
\vfill
%===================================================================================
\module[codeApogee={SOM1BH06},
titre={Transfert de gènes}, 
COURS={24}, 
TD={}, 
TP={}, 
CTD={},
CTP={}, 
TOTAL={24}, 
SEMESTRE={Semestre 1}, 
COEFF={3}, 
ECTS={3}, 
MethodeEval={Ecrit},
ModalitesCCSemestreUn={RNE et RSE : CT(E) 1h},
ModalitesCCSemestreDeux={RNE et RSE : CT(E) 1h},
CalculNFSessionUne={Ecrit 100\%},
CalculNFSessionDeux={Ecrit 100\%},
NoteEliminatoire={7}, 
nomPremierResp={Patrick Baril}, 
emailPremierResp={patrick.baril@cnrs-orleans.fr}, 
nomSecondResp={Chantal Pichon}, 
emailSecondResp={chantal.pichon@cnrs-orleans.fr}, 
langue={Français}, 
nbPrerequis={1}, 
descriptionCourte={true}, 
descriptionLongue={true}, 
objectifs={true}, 
ressources={false}, 
bibliographie={false}] 
% ******* Texte introductif
{
Spécialité BBMB
} 
% ******* Contenu détaillé
{
Historique de la thérapie génique. Transfert de gènes \textit{in vitro} et \textit{in vivo}. Gènes thérapeutiques et marqueurs génétiques. Vecteurs chimiques et Vecteurs viraux, Méthodes physiques. Inhibition de l’expression génique par des acides nucléiques: oligonucléotide antisens, leurres ADN ou ARN, ribozymes, ARN interférentiels, microARN. Exemples d’applications thérapeutiques. Présentation des méthodologies de base en transgénèse : Applications potentielles pour la recherche académique et privée ; les normes et les contraintes liées à l’utilisation du modèle animal.
}
% ******* Pré-requis
{Notions de base en biochimie, biologie moléculaire, biologie cellulaire et immunologie.
} 
% ******* Objectifs
{\begin{itemize} 
  \ObjItem Connaissance des stratégies de thérapies génique et cellulaire, description des outils de transfert d’acides nucléiques et exemples d’application thérapeutiques. Apprentissage des  méthodologies de transgénèse : applications potentielles pour la recherche académique et la recherché industrielle; normes et restrictions reliées à l’utilisation des modèles animaux, lois éthiques. 
\end{itemize} 
} 
% ******* Ressources pédagogiques
{} 
% ******* Bibliographie éventuelle
{Biblio}
 
\vfill
%===================================================================================
\module[codeApogee={SOM1BH07},
titre={Assurance qualité et réglementation BPL}, 
COURS={48}, 
TD={}, 
TP={}, 
CTD={},
CTP={}, 
TOTAL={48}, 
SEMESTRE={Semestre 1}, 
COEFF={5}, 
ECTS={5}, 
MethodeEval={Ecrit},
ModalitesCCSemestreUn={RNE et RSE : CT(E) 1h},
ModalitesCCSemestreDeux={RNE et RSE : CT(E) 1h},
CalculNFSessionUne={Ecrit 100\%},
CalculNFSessionDeux={Ecrit 100\%},
NoteEliminatoire={7}, 
nomPremierResp={William Même}, 
emailPremierResp={william.meme@cnrs-orleans.fr}, 
nomSecondResp={}, 
emailSecondResp={}, 
langue={Français}, 
nbPrerequis={0}, 
descriptionCourte={true}, 
descriptionLongue={true}, 
objectifs={true}, 
ressources={false}, 
bibliographie={false}] 
% ******* Texte introductif
{
Passerelle MESC2A
} 
% ******* Contenu détaillé
{
Système de management de la qualité, contrôle qualité et assurance qualité (HACCP, Normes ISO, BRC, IFS). La réalisation du produit. La gestion de la qualité. L’accréditation et la certification. Les audits. Bonnes Pratiques de Laboratoire (BPL). Risques physiques, chimiques et biologiques. Risques bio-industriels. Réglementation sur l’utilisation confinée.
}
% ******* Pré-requis
{
} 
% ******* Objectifs
{\begin{itemize} 
  \ObjItem Organiser et mettre en œuvre les différentes procédures garantissant la qualité des produits. Superviser et suivre le contrôle des matières premières, des moyens de production, des produits semi-finis et des produits finis. Participer à l'amélioration des procédés de fabrication, de l'organisation de la production et des équipements productifs. 
\end{itemize} 
} 
% ******* Ressources pédagogiques
{} 
% ******* Bibliographie éventuelle
{Biblio}
 
\vfill
%===================================================================================
\module[codeApogee={SOM1IP00},
titre={Outils de recherche d'emploi}, 
COURS={24}, 
TD={}, 
TP={}, 
CTD={},
CTP={}, 
TOTAL={24}, 
SEMESTRE={Semestre 1}, 
COEFF={2}, 
ECTS={2}, 
MethodeEval={Ecrit},
ModalitesCCSemestreUn={RNE et RSE : CC},
ModalitesCCSemestreDeux={Pas de seconde session},
%CalculNFSessionUne={Ecrit 75\% + TP 25 \%},
%CalculNFSessionDeux={Ecrit 75\% + TP 25 \%},
NoteEliminatoire={7}, 
nomPremierResp={William Même}, 
emailPremierResp={william.meme@cnrs-orleans.fr}, 
nomSecondResp={Olivier Richard}, 
emailSecondResp={olivier.richard@univ-orleans.fr}, 
langue={Français}, 
nbPrerequis={0}, 
descriptionCourte={true}, 
descriptionLongue={true}, 
objectifs={true}, 
ressources={false}, 
bibliographie={false}] 
% ******* Texte introductif
{
Passerelle MESC2A
} 
% ******* Contenu détaillé
{
Définition du projet professionnel personnalisé : stratégies de recherche de stage et d’emploi en entreprise, rédaction du CV et lettres de motivation, préparation de l’entretien.
}
% ******* Pré-requis
{18 ECTS de biologie moléculaire
} 
% ******* Objectifs
{\begin{itemize} 
  \ObjItem Sensibiliser les étudiants à la connaissance de l’entreprise. Préparer les étudiants à la recherche d'un stage et à la recherche d'un emploi.
\end{itemize} 
} 
% ******* Ressources pédagogiques
{} 
% ******* Bibliographie éventuelle
{Biblio}
 
\vfill
%===================================================================================

\end{document}
