\documentclass[10pt, a5paper]{report}

\usepackage[T1]{fontenc}%
\usepackage[utf8]{inputenc}% encodage utf8
\usepackage[francais]{babel}% texte français
\usepackage[final]{pdfpages}
\usepackage{modules-livret}% style du livret
\usepackage{url}
%\usepackage{init-preambule}
\pagestyle{empty}

% % % % % % % % % % % % % % % % % % % % % % % % % % % % % % % % % % % % % % % % % % % % % % % % % % % % % % % 
\begin{document}

%---------------------- % % % Personnalisation des couleurs % % % ----------- Vert Licence --------
\definecolor{couleurFonce}{RGB}{108,5,5} % Couleur du Code APOGEE
\definecolor{couleurClaire}{RGB}{147,97,97} % Couleur du fond de la bande
\definecolor{couleurTexte}{RGB}{255,255,255} % Couleur du texte de la bande
%------------------------------------------------------------------------------------------


%==========================================================================================
% Semestre 2
%==========================================================================================
\module[codeApogee={SOM2AG42},
titre={Anglais Scientifique et communication}, 
COURS={}, 
TD={24}, 
TP={}, 
CTD={},
CTP={}, 
TOTAL={24}, 
SEMESTRE={Semestre 2}, 
COEFF={3}, 
ECTS={3}, 
MethodeEval={Ecrit/Oral},
ModalitesCCSemestreUn={RNE et RSE : CT(E) 1h30 / CC(O) 15 min},
ModalitesCCSemestreDeux={RNE et RSE : CT(O) 20 min},
CalculNFSessionUne={Ecrit 75\% + Oral 50\%},
CalculNFSessionDeux={Oral\%},
NoteEliminatoire={7}, 
nomPremierResp={Lucile Mollet}, 
emailPremierResp={lucile.mollet@cnrs-orleans.fr}, 
nomSecondResp={Olivier Richard}, 
emailSecondResp={olivier.richard@univ-orleans.fr}, 
langue={Français}, 
nbPrerequis={1}, 
descriptionCourte={true}, 
descriptionLongue={true}, 
objectifs={true}, 
ressources={false}, 
bibliographie={false}] 
% ******* Texte introductif
{
Spécialités BBMB et BOPE / Passerelle MESC2A
} 
% ******* Contenu détaillé
{
Faire un bilan du stage de l’année passée et résumer les visites d’entreprises en anglais. Commenter le déroulement d’un atelier technologique en anglais. Se préparer à des entretiens en anglais pour une embauche (thèse ou poste R \& D)
}
% ******* Pré-requis
{Bon niveau d’ordre général pratique, oral et écrite, assez bonne connaissance de la langue spécifique de son domaine scientifique et technique.
} 
% ******* Objectifs
{\begin{itemize} 
  \ObjItem Savoir présenter un rapport de travail en anglais et commenter le déroulement d’une expérimentation.
\end{itemize} 
} 
% ******* Ressources pédagogiques
{} 
% ******* Bibliographie éventuelle
{Biblio}
 
\vfill
%==========================================================================================
\module[codeApogee={SOM2IP00},
titre={Ouverture à l'international}, 
COURS={}, 
TD={12}, 
TP={}, 
CTD={},
CTP={}, 
TOTAL={12}, 
SEMESTRE={Semestre 2}, 
COEFF={1}, 
ECTS={1}, 
%MethodeEval={},
%ModalitesCCSemestreUn={RNE et RSE : CT(E) 1h30 / CC(O) 15 min},
%ModalitesCCSemestreDeux={RNE et RSE : CT(O) 20 min},
%CalculNFSessionUne={Ecrit 75\% + Oral 50\%},
%CalculNFSessionDeux={Oral\%},
%NoteEliminatoire={7}, 
nomPremierResp={Lupka Mihajlovska}, 
emailPremierResp={lupka.mihajlovska@univ-orleans.fr}, 
nomSecondResp={}, 
emailSecondResp={}, 
langue={Anglais}, 
nbPrerequis={1}, 
descriptionCourte={true}, 
descriptionLongue={true}, 
objectifs={true}, 
ressources={false}, 
bibliographie={false}] 
% ******* Texte introductif
{
Spécialités BBMB et BOPE / Passerelle MESC2A
} 
% ******* Contenu détaillé
{
Savoir communiquer dans le milieu professionnel (CV, lettre, téléphone, entretien de recrutement, participer à une réunion), valider son niveau d’anglais par une certification en langues, niveau B2 (TOEIC, CLES).
}
% ******* Pré-requis
{Bon niveau d’ordre général pratique.
} 
% ******* Objectifs
{\begin{itemize} 
  \ObjItem Préparer l’étudiant à faire un entretien  d’embauche dans la langue anglaise (savoir communiquer dans le milieu professionnel).
\end{itemize} 
} 
% ******* Ressources pédagogiques
{} 
% ******* Bibliographie éventuelle
{Biblio}
 
\vfill
%==========================================================================================
\module[codeApogee={SOM2PR01},
titre={Projet Professionnel et connaissance de l'entreprise}, 
COURS={}, 
TD={24}, 
TP={}, 
CTD={},
CTP={}, 
TOTAL={24}, 
SEMESTRE={Semestre 2}, 
COEFF={3}, 
ECTS={3}, 
MethodeEval={Ecrit},
ModalitesCCSemestreUn={RNE et RSE : CC rapport},
ModalitesCCSemestreDeux={RNE et RSE : Pas de seconde session},
CalculNFSessionUne={Ecrit 100\%},
%CalculNFSessionDeux={Oral\%},
NoteEliminatoire={7}, 
nomPremierResp={Olivier Richard William Même Philippe Herrandez}, 
emailPremierResp={olivier.richard@univ-orleans.fr william.meme@univ-orleans.fr philippe.herrandez@univ-orleans.fr}, 
nomSecondResp={}, 
emailSecondResp={}, 
langue={Français}, 
nbPrerequis={0}, 
descriptionCourte={true}, 
descriptionLongue={true}, 
objectifs={true}, 
ressources={false}, 
bibliographie={false}] 
% ******* Texte introductif
{
Spécialités BBMB et BOPE / Passerelle MESC2A
} 
% ******* Contenu détaillé
{
Construction d’un projet professionnel ; visites d’entreprises, de collectivités locales, organismes ; préparation de CV, techniques de recherche d’emploi, informations sur les concours de recrutement de chercheurs et enseignants-chercheurs, ingénieurs…, table ronde sur les exigences du milieu professionnel, avec participation de représentants du publique et du privé. Connaissance de l’entreprise (grandes fonctions, grands types d’activités); connaissance des différents secteurs de recherche (public et privé); notions d’économie et de gestion (éléments de base); conférences et témoignages de parcours professionnels;  contacts avec le milieu professionnel.
}
% ******* Pré-requis
{Bon niveau d’ordre général pratique.
} 
% ******* Objectifs
{\begin{itemize} 
  \ObjItem faire réfléchir l’étudiant sur ses motivations et ses ambitions professionnelles ; le faire se positionner par rapport à la continuation en thèse ou à l’entrée dans la vie active ; sensibiliser les étudiants à la connaissance des métiers de la recherche.
\end{itemize} 
} 
% ******* Ressources pédagogiques
{} 
% ******* Bibliographie éventuelle
{Biblio}
 
\vfill
%==========================================================================================
\module[codeApogee={SOM2ST01},
titre={Stage}, 
COURS={}, 
TD={12}, 
TP={}, 
CTD={},
CTP={}, 
TOTAL={12}, 
SEMESTRE={Semestre 2}, 
COEFF={3}, 
ECTS={3}, 
MethodeEval={Ecrit/Oral},
ModalitesCCSemestreUn={RNE et RSE : CT(Rapport)+ Oral + Note Maître de stage},
ModalitesCCSemestreDeux={RNE et RSE : Pas de seconde session},
%CalculNFSessionUne={Ecrit 100\%},
%CalculNFSessionDeux={Oral\%},
NoteEliminatoire={7}, 
nomPremierResp={François lieutier}, 
emailPremierResp={francois.lieutier@univ-orleans.fr}, 
nomSecondResp={}, 
emailSecondResp={}, 
langue={Français},
nbPrerequis={0}, 
descriptionCourte={true}, 
descriptionLongue={true}, 
objectifs={true}, 
ressources={false}, 
bibliographie={false}] 
% ******* Texte introductif
{
Spécialités BBMB et BOPE / Passerelle MESC2A
} 
% ******* Contenu détaillé
{
Stage de 2 mois en laboratoire de recherche, ou en entreprise ou collectivité locale : participation à la construction d’un protocole, la réalisation d’une expérience ou d’un projet ; mise en forme des résultats ; traitement des données ; interprétation ; présentation orale et écrite.
}
% ******* Pré-requis
{Bon niveau d’ordre général pratique.
} 
% ******* Objectifs
{\begin{itemize} 
  \ObjItem Prise de contact avec le monde professionnel de la recherche et le travail de chercheur.
\end{itemize} 
} 
% ******* Ressources pédagogiques
{} 
% ******* Bibliographie éventuelle
{Biblio}
 
\vfill
%==========================================================================================
\module[codeApogee={SOM2BO05},
titre={Agronomie et amélioration des plantes}, 
COURS={25}, 
TD={5}, 
TP={18}, 
CTD={},
CTP={}, 
TOTAL={48}, 
SEMESTRE={Semestre 2}, 
COEFF={5}, 
ECTS={5}, 
MethodeEval={Ecrit/Oral/TP},
ModalitesCCSemestreUn={RNE et RSE : CT (CM+TD) Ecrit 2h + Oral + CC (TP)},
ModalitesCCSemestreDeux={RNE et RSE : CT (CM+TD) Ecrit 2h + (TP) Ecrit 1h},
CalculNFSessionUne={Ecrit 50\% + Oral 25\% + TP 25\%},
CalculNFSessionDeux={Ecrit 50\% + Oral 25\% + TP 25\%},
NoteEliminatoire={7}, 
nomPremierResp={Stéphane Maury}, 
emailPremierResp={stephane.maury@univ-orleans.fr}, 
nomSecondResp={}, 
emailSecondResp={}, 
langue={Français},
nbPrerequis={0}, 
descriptionCourte={false}, 
descriptionLongue={true}, 
objectifs={true}, 
ressources={false}, 
bibliographie={false}] 
% ******* Texte introductif
{
} 
% ******* Contenu détaillé
{
Amélioration des plantes cultivées, processus de propagation des espèces végétales d’intérêt agronomique  (reproduction sexuée, multiplication végétative traditionnelle, in vitro, hors sol, initiation à la transgénèse) et utilisation des plantes pour la production de molécules à visée pharmacologique ou agroalimentaire (cultures cellulaires et ingénierie métabolique).
}
% ******* Pré-requis
{
} 
% ******* Objectifs
{\begin{itemize} 
  \ObjItem Connaissances des méthodologies d’amélioration, production et exploitation des plantes notamment cultivées.
\end{itemize} 
} 
% ******* Ressources pédagogiques
{} 
% ******* Bibliographie éventuelle
{Biblio}
 
\vfill
%==========================================================================================
\module[codeApogee={SOM2BO06},
titre={Influence de l’environnement sur le développement des plantes}, 
COURS={28}, 
TD={4}, 
TP={16}, 
CTD={},
CTP={}, 
TOTAL={48}, 
SEMESTRE={Semestre 2}, 
COEFF={5}, 
ECTS={5}, 
MethodeEval={Ecrit/Oral/TP},
ModalitesCCSemestreUn={RNE et RSE : CT(E) 2h + oral + CC TP},
ModalitesCCSemestreDeux={RNE et RSE : CT(E) 2h + oral + CT TP 1h},
CalculNFSessionUne={Ecrit 40\% + Oral 40\% + TP 20\%},
CalculNFSessionDeux={Ecrit 40\% + Oral 40\% + TP 20\%},
NoteEliminatoire={7}, 
nomPremierResp={Daniel Hagège}, 
emailPremierResp={daniel.hagege@univ-orleans.fr}, 
nomSecondResp={}, 
emailSecondResp={}, 
langue={Français},
nbPrerequis={1}, 
descriptionCourte={false}, 
descriptionLongue={true}, 
objectifs={true}, 
ressources={false}, 
bibliographie={false}] 
% ******* Texte introductif
{
} 
% ******* Contenu détaillé
{
Quelques mécanismes fondamentaux du contrôle du développement des plantes : Perception de l’environnement ; lumière et photomorphogenèse ; vernalisation ; gravitropisme.  Croissance et développement ; rôle du métabolisme secondaire ; lignine ; rôle des phytohormones et modes d’actions moléculaires.
}
% ******* Pré-requis
{Connaissance générale sur la physiologie,  la croissance et le développement des végétaux.
} 
% ******* Objectifs
{\begin{itemize} 
  \ObjItem Connaissances approfondies des mécanismes physiologiques, biochimiques et moléculaires qui régissent les interactions entre les plantes et leur environnement.
\end{itemize} 
} 
% ******* Ressources pédagogiques
{} 
% ******* Bibliographie éventuelle
{Biblio}
 
\vfill
%==========================================================================================
\module[codeApogee={SOM2BO07},
titre={Ecotoxicologie et phytoremédiation}, 
COURS={28}, 
TD={12}, 
TP={8}, 
CTD={},
CTP={}, 
TOTAL={48}, 
SEMESTRE={Semestre 2}, 
COEFF={5}, 
ECTS={5}, 
MethodeEval={Ecrit/TP},
ModalitesCCSemestreUn={RNE et RSE : CT(CM) Ecrit 2h + CC (TP,TD)},
ModalitesCCSemestreDeux={RNE et RSE : CT(E) 2h},
CalculNFSessionUne={CT 66\% + CC 33 \%},
CalculNFSessionDeux={Ecrit 100\%},
NoteEliminatoire={7}, 
nomPremierResp={Domenico Morabito}, 
emailPremierResp={domenico.morabito@univ-orleans.fr}, 
nomSecondResp={}, 
emailSecondResp={}, 
langue={Français},
nbPrerequis={1}, 
descriptionCourte={false}, 
descriptionLongue={true}, 
objectifs={true}, 
ressources={false}, 
bibliographie={false}] 
% ******* Texte introductif
{
} 
% ******* Contenu détaillé
{
Connaissances modernes sur les principaux types de pollution terrestre, de leurs effets sur les êtres vivants et les écosystèmes, et des possibilités de réhabilitation des milieux pollués. Exemples majeurs de pollution. Les polluants et leur mode d’action : pollution domestique, pollution industrielle, pollution agricole, pollution atmosphérique. Biotransformation, bioaccumulation et biomagnification des polluants. La réhabilitation des milieux pollués et les possibilités offertes par la phytoremédiation. Etude de cas concrets. Risques liés à l’utilisation des OGM.
}
% ******* Pré-requis
{Ecologie générale et Physiologie.
} 
% ******* Objectifs
{\begin{itemize} 
  \ObjItem Acquisition des  connaissances récentes liées à l’effet de la pollution sur la santé humaine et les écosystèmes.
\end{itemize} 
} 
% ******* Ressources pédagogiques
{} 
% ******* Bibliographie éventuelle
{Biblio}
 
\vfill
%==========================================================================================
\module[codeApogee={SOM2BO08},
titre={Agents de perturbation dans les écosystèmes terrestres}, 
COURS={24}, 
TD={8}, 
TP={16}, 
CTD={},
CTP={}, 
TOTAL={48}, 
SEMESTRE={Semestre 2}, 
COEFF={5}, 
ECTS={5}, 
MethodeEval={Ecrit/TP},
ModalitesCCSemestreUn={RNE et RSE : CT(CM) Ecrit 2h + CC (TP,TD)},
ModalitesCCSemestreDeux={RNE et RSE : CT(E) 2h},
CalculNFSessionUne={CT 66\% + CC 33 \%},
CalculNFSessionDeux={Ecrit 100\%},
NoteEliminatoire={7}, 
nomPremierResp={Aurélien Sallé}, 
emailPremierResp={aurelien.salle@univ-orleans.fr}, 
nomSecondResp={}, 
emailSecondResp={}, 
langue={Français},
nbPrerequis={1}, 
descriptionCourte={false}, 
descriptionLongue={true}, 
objectifs={true}, 
ressources={false}, 
bibliographie={false}] 
% ******* Texte introductif
{
} 
% ******* Contenu détaillé
{
Perturbations biotiques et abiotiques, phytophagie, guildes de ravageurs, phytopathologie, gestion chimique et biologique des ravageurs, perturbations abiotiques.
}
% ******* Pré-requis
{Ecologie générale, notions d’entomologie générale.
} 
% ******* Objectifs
{\begin{itemize} 
  \ObjItem Donner à l’étudiant un panorama des agents de perturbations biotiques et abiotiques dans les écosystèmes terrestres, des problèmes qu’ils posent et des modalités de gestion.
\end{itemize} 
} 
% ******* Ressources pédagogiques
{} 
% ******* Bibliographie éventuelle
{Biblio}
 
\vfill
%==========================================================================================
\end{document}
