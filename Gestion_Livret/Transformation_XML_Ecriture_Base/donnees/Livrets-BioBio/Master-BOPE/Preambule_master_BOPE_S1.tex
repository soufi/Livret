\documentclass[10pt, a5paper]{report}

\usepackage[T1]{fontenc}%
\usepackage[utf8]{inputenc}% encodage utf8
\usepackage[francais]{babel}% texte français
\usepackage[top=10mm, bottom=20mm, foot=5mm, left=7mm, offset=7mm, textwidth=350pt]{geometry}
\usepackage{url}
\usepackage{init-preambule}
\usepackage[final]{pdfpages}

% Définir les couleur du document
%---------------------- % % % Personnalisation des couleurs % % % ----------- Vert Licence --------
\definecolor{couleurFonce}{RGB}{108,5,5} % Couleur du Code APOGEE
\definecolor{couleurClaire}{RGB}{147,97,97} % Couleur du fond de la bande
\definecolor{couleurTexte}{RGB}{255,255,255} % Couleur du texte de la bande
%------------------------------------------------------------------------------------------


\begin{document}
\begin{spacing}{1.5}

\chapter*{Master mention Sciences Biologiques}

\footnotesize
\section*{Informations générales}

\input{Extra/Resp_ChantalPICHON}
\letterine{C}ette formation a pour objectif de fournir aux étudiants des connaissances sur les systèmes biologiques de manière intégrée : de l’échelle moléculaire à celle de l’écosystème, aussi bien en termes de structure que de fonction. Cette mention propose à l’étudiant,  après un socle commun au sein de la première année, poursuivi en seconde année, d’acquérir en fonction de son projet professionnel une formation à l’échelle des populations et des organismes ou bien à dominante cellulaire et moléculaire. 

\subsection*{Objectifs et organisation du Master}

\letterine{L}’objectif de l’offre de formation est de proposer des enseignements qui s’appuient très fortement à la fois sur les recherches menées dans les laboratoires  de l’Institut Thmatiques Pluridisciplinaire (ITP) Sciences Biologiques, Chimie du Vivant (Centre de Biophysique Moléculaire (CBM), Immunologie et Neurogénétique Expérimentales et Moléculaire (INEM), Institut de Chimie Organique et Analytique (ICOA),  Laboratoire de Biologie des Ligneux et des Grandes Cultures (LBLGC), Imagerie Multimodale Multiéchelle et Modélisation du Tissu Osseux et articulaire (I3MTO) ainsi que l’INRA d’Orléans et le Cemagref) mais aussi sur le tissu économique de la Région Centre.  
Le but est donc de former des étudiants capables d’acquérir des connaissances sur les systèmes biologiques de manière intégrée, de l’échelle moléculaire à l’écosystème aussi bien en termes de structure que de fonction. Cette mention propose à l’étudiant en fonction de sa sensibilité et des métiers visés, de s’orienter soit vers un enseignement supérieur plutôt à l’échelle cellulaire et moléculaire soit vers un enseignement supérieur plutôt à l’échelle de l’organisme et des populations. Cette formation s’organise autour d’un socle commun au sein de la première année.

\begin{itemize}
\item Les aspects cellulaires et moléculaires sont abordées principalement dans la spécialité \textbf{Biochimie Biologie Moléculaire et Biotechnologie (BBMB)} portés par les laboratoires de recherche  CNRS, CBM, ICOA INEM… en lien avec les pôles de compétitivité Cosmetic Valley, et Pharma Valley du cluster pôle Pharma inclus dans l’alliance Pharma Valley. 
\item Les aspects développés aux échelles de l’organisme et des populations sont abordés dans la spécialité \textbf{Biologie des Organismes, des Populations et des Ecosystèmes (BOPE)} supportés par les laboratoires Universitaire (LBLGC), INRA (Zoologie Forestière, amélioration génétique et physiologie forestière), CEMAGREF (Unité écosystèmes forestiers), IRD, en lien avec le pôle de compétitivité DREAM et le cluster de recherche RESONAT.
\item Pour les étudiants dont leur projet professionnel est d’accéder à des métiers autour de l’éducation et de la diffusion scientifique, il est proposé la spécialité:« Métiers de l’enseignement secondaire en Sciences de la Vie et de la Terre et de la diffusion de sciences et des techniques pour les scolaires » (MED-SVT). Cette spécialité vise à la formation des étudiants aux métiers de l’enseignement et de la diffusion des connaissances scientifiques et techniques en milieu scolaire. En particulier, il fournira aux étudiants à la fois une préparation aux épreuves du concours en leur apportant les compétences disciplinaires exigées selon le programme du concours du CAPES SVT, mais également une formation professionnelle en leur permettant d’acquérir les connaissances théoriques en didactique, nécessaires au métier d’enseignant, et pratiques par des stages en établissement et une analyse des expériences vécues.  
Cette spécialité est d’ores et déjà validée pour la période 2011-2015 et ne sera pas détaillée davantage dans ce dossier.
\item Enfin pour les étudiants désireux d’obtenir une double compétence, le Master Compétence complémentaire en informatique (CCI) est proposé. Le Master CCI est une mention de tous les autres Masters de l'établissement sauf ceux d'informatique et de Miage, dont les étudiants ont déjà acquis l'essentiel des compétences qui y sont enseignées. Le Master CCI est une appellation nationale et elle est traitée à part dans une autre livret.
\end{itemize}

\subsection*{Conditions d'admission et poursuites d'études}

\letterine{L}e master est accessible en 1ère année aux titulaires d’un diplôme de Licence en Biologie,  Biochimie ou Chimie selon la spécialité (validation des UE de chimie avec une moyenne supérieure ou égale à 10).
Une commission de validation des acquis sera constituée chaque année par le responsable des parcours ou de spécialité. La composition de celle-ci sera désignée par l’équipe de pilotage.

L’accès à la 2ème année du Master se fait :
- de droit pour les étudiants ayant validé en 1ère session d'examen la première année du Master de la même spécialité
- sur entretien pour tout étudiant ayant validé la première année du Master de la même spécialité à l'issue de la 2ème session d'examen;
- sur dossier et entretien pour tout étudiant ayant validé les 60 premiers ECTS d'un autre Master spécialisé dans le même domaine que la spécialité.
Ce master est également ouvert à toute personne après Validation d’acquis ou de VAE après étude du dossier et entretien (UE).
Les spécialités du master Sciences Biologiques sont résolument tournées vers le milieu professionnel et avec une ouverture affirmée vers la Recherche. Pour les différents parcours, une poursuite d’études en thèse sera envisageable grâce à une initiation à la recherche basée sur un stage de longue durée en laboratoire de recherche public ou privé et sur les modules projets de recherche menés durant les 2 années de master. 
L’option TBI va privilégier la voie professionnelle pour les diplômés avec une entrée dans la vie active au niveau bac+5.  

\subsection*{Glossaire}

\subsubsection*{Unités d’enseignement :} 

Ensemble d’enseignements comprenant des cours, TD, TP ou autres travaux personnels. Chaque unité est affectée d’ECTS et fait l’objet d’un contrôle de connaissances.
\begin{itemize}
\item[\textit{Unités d’ossature}] : unités d’enseignement obligatoires correspondant à la formation de l’étudiant dans la licence  pour le parcours choisi.
\item[\textit{Unités optionnelles}] : unités à choisir dans une sélection d’unités proposées pour personnaliser son parcours.
\item[\textit{Unités libres}] : UE permettant à l’étudiant de compléter sa formation (culture générale, méthodologie universitaire, éléments de professionnalisation, stages, compléments de langue, utilisation des ressources documentaires…). Unités à choisir dans le livret des UE libres de l’ensemble des composantes de l’Université.
\end{itemize}

\subsubsection*{Crédits ECTS (European Credits Transfer System = Système Européen de Crédits de Transfert)}

\begin{itemize}
\item A chaque unité d’enseignement (UE) est affectée une valeur en crédits qui correspond au volume du horaire de l’unité. Les crédits sont attribués quand l’unité est validée (note > ou = à 10/20).
\item Chaque semestre validé correspond à 30 crédits. Une licence correspond donc à 180 crédits (6 semestres).
\item Ces crédits représentent une monnaie d’échange et sont : 
\begin{itemize}
\item Transférables dans toute autre université européenne ;
\item Capitalisables c’est-à-dire définitivement acquis quelle que soit la durée du parcours de l’étudiant. 
\end{itemize}
\end{itemize}

\subsubsection*{Grade :}

Les grades universitaires sanctionnent les divers niveaux de l’enseignement supérieur. Depuis le décret du 8 avril 2002 ce sont : le baccalauréat, la licence, le master et le doctorat.

\subsubsection*{Diplôme :}

A chaque grade correspond un titre ou un diplôme. Sont conservées les possibilités de délivrer deux diplômes intermédiaires, ne correspondant pas à un des grades précédents :
\begin{itemize}
\item celui de maîtrise (correspondant aux 60 crédits ECTS des semestres 1 et 2 de master) – (article 9 de l’arrêté du 25 avril 2002 relatif au diplôme de master),
\item celui de DEUG (correspondant aux 120 crédits ECTS des semestres 1, 2, 3 et 4 de licence) – (article 2 de l’arrêté du 23 avril 2002 relatif aux études universitaires conduisant au grade de licence).
\end{itemize}

\textbf{Ces deux derniers diplômes sont délivrés dans la mention retenue sans indication de parcours ou de spécialité, mais accompagnés d’un « supplément au diplôme ».}

\subsubsection*{Supplément au diplôme} 

(décret du 8 avril 2002 portant application au système français d’enseignement supérieur de la construction de l’Espace européen de l’enseignement supérieur. Art. 2). C’est une annexe descriptive au diplôme,  destinée à assurer la lisibilité des connaissances et aptitudes acquises dans le cadre de la mobilité internationale.

\subsubsection*{Equipe de formation :} Chaque mention de licence ou master est pilotée par une équipe de formation qui veille à l’adéquation de l’organisation des études avec les objectifs fixés, organise l’évaluation générale des formations et élabore  un bilan annuel de la formation.


\subsubsection*{Directeur des études :} en contact direct avec les étudiants et la scolarité de l’UFR, il anime l’équipe pédagogique.
Jury : Constitué pour chaque semestre et présidé par le Directeur des études, il a en charge le recueil des notes, l’établissement des moyennes et la validation du semestre.


\subsubsection*{Jury :} Constitué pour chaque semestre et présidé par le Directeur des études, il a en charge le recueil des notes, l’établissement des moyennes et la validation du semestre. Un jury d’année réuni après chaque session a en charge la délivrance de l’année et pour le M2 la délivrance finale du diplôme.

\subsubsection*{Compensations :}

\begin{itemize}
\item Pour chaque semestre, si la note globale moyenne est supérieure ou égale à 10/20, le semestre est validé et lui sont associés 30 ECTS.
\item La compensation annuelle s'organise dès la première sessions entre les deux semestres consécutifs d'une même année universitaire : soit le semestre 1 avec le semestre 2.
\item La compensation ne peut avoir lieu, si l’une des \textbf{notes d’unité est < 7/20}, ce qui signifie qu’une telle note est éliminatoire pour la session.
\end{itemize}


\section*{La spécialité Biologie des Organismes, des Populations et des Ecosystèmes (BOPE) }

\input{Extra/Resp_FrancoisLIEUTIER}

\subsection*{Objectifs et organisation}

\letterine{L}L’objectif pédagogique général est de donner à l’étudiant une culture générale approfondie sur la structure et le fonctionnement des écosystèmes terrestres et leurs relations (historiques et fonctionnelles) avec les activités humaines, assortie d’une double spécialité : forêts et grandes cultures. En plus de l’apport de connaissances de bases, on développera les capacités d’observation et les compétences de l’étudiant à établir des liens entre structure et fonctionnement. On visera aussi à développer des compétences appliquées à la gestion des écosystèmes et prenant en compte les nécessités de préservation de l’environnement et de la biodiversité.  On ne négligera pas pour autant l’étude fine du fonctionnement des organismes. Il s’agira en outre de développer l’esprit d’initiative et de former l’étudiant à la démarche scientifique, ainsi qu’à la méthodologie et aux outils de la recherche en biologie, y compris à la communication sous toutes ses formes et en langue anglaise.
La 1ère année est fortement mutualisée avec la spécialité BBMB. En deuxième année, deux orientations sont proposées : Gestion des écosystèmes et contraintes (GEC) et Management et conservation des ressources naturelles (Natural Resources Management and Conservation ou RNMC).

%\subsection*{Composition de l'équipe pédagogique}

%% % % % % % % % % % % % % % % % % % % % % % % % % % % % % % % % % % % % % % % % % % % % % % % % % % % % % % % 

%---------------------- % % % Personnalisation des couleurs % % % ----------- ROUGE --------
%---------------------- % % % Personnalisation des couleurs % % % ----------- Vert Licence --------
\definecolor{couleurFonce}{RGB}{108,5,5} % Couleur du Code APOGEE
\definecolor{couleurClaire}{RGB}{147,97,97} % Couleur du fond de la bande
\definecolor{couleurTexte}{RGB}{255,255,255} % Couleur du texte de la bande
%------------------------------------------------------------------------------------------

%------------------------------------------------------------------------------------------

\arrayrulecolor{couleurFonce}% Couleur des lignes séparatrices du tableau
\renewcommand{\arraystretch}{1}% Coeff appliqué à la hauteur des cellules
%\rowcolors[\hline]{ligneDébut}{couleurPaire}{couleurImpaire}% Alternance de couleur (need package xcolor)
\begin{tabular}{|m{2cm}|m{2cm}|m{2cm}|m{1cm}|m{8cm}|}
%\begin{tabular}{c|m{6cm}|cm{0.8cm}|cm{0.8cm}|cm{0.8cm}|cm{0.8cm}|}
%\begin{tabular}{|cm|cm|cm|cm|cm|}
\cline{1-5}

\cellcolor{couleurFonce} \color{white}\bfseries \bfseries Nom & \cellcolor{couleurFonce} \color{white} Prénom & \cellcolor{couleurFonce} \color{white} Grade/Statut & \cellcolor{couleurFonce} \color{white} Section CNU & \cellcolor{couleurFonce} \color{white} Equipe de recherche/Entreprise \\ \cline{1-5}
%----
 \color{black} André & \color{black} Patrice & \color{black} responsable R\&D & \color{black} & \color{black} LVMH \\ \cline{1-5}

 \cellcolor{couleurClaire} \color{couleurTexte} Ardourel  & \cellcolor{couleurClaire} \color{couleurTexte} Maryvonne & \cellcolor{couleurClaire} \color{couleurTexte} MCF/Univ orléans & \cellcolor{couleurClaire} \color{couleurTexte} 64 & \cellcolor{couleurClaire} \color{couleurTexte} INEM \\ \cline{1-5}

 \color{black} André & \color{black} Patrice & \color{black} responsable R\&D & \color{black} & \color{black} LVMH \\ \cline{1-5}

 \cellcolor{couleurClaire} \color{couleurTexte} Ardourel  & \cellcolor{couleurClaire} \color{couleurTexte} Maryvonne & \cellcolor{couleurClaire} \color{couleurTexte} MCF/Univ orléans & \cellcolor{couleurClaire} \color{couleurTexte} 64 & \cellcolor{couleurClaire} \color{couleurTexte} INEM \\ \cline{1-5}

 \color{black} André & \color{black} Patrice & \color{black} responsable R\&D & \color{black} & \color{black} LVMH \\ \cline{1-5}

 \cellcolor{couleurClaire} \color{couleurTexte} Ardourel  & \cellcolor{couleurClaire} \color{couleurTexte} Maryvonne & \cellcolor{couleurClaire} \color{couleurTexte} MCF/Univ orléans & \cellcolor{couleurClaire} \color{couleurTexte} 64 & \cellcolor{couleurClaire} \color{couleurTexte} INEM \\ \cline{1-5}
 \color{black} André & \color{black} Patrice & \color{black} responsable R\&D & \color{black} & \color{black} LVMH \\ \cline{1-5}

 \cellcolor{couleurClaire} \color{couleurTexte} Ardourel  & \cellcolor{couleurClaire} \color{couleurTexte} Maryvonne & \cellcolor{couleurClaire} \color{couleurTexte} MCF/Univ orléans & \cellcolor{couleurClaire} \color{couleurTexte} 64 & \cellcolor{couleurClaire} \color{couleurTexte} INEM \\ \cline{1-5}
 \color{black} André & \color{black} Patrice & \color{black} responsable R\&D & \color{black} & \color{black} LVMH \\ \cline{1-5}

 \cellcolor{couleurClaire} \color{couleurTexte} Ardourel  & \cellcolor{couleurClaire} \color{couleurTexte} Maryvonne & \cellcolor{couleurClaire} \color{couleurTexte} MCF/Univ orléans & \cellcolor{couleurClaire} \color{couleurTexte} 64 & \cellcolor{couleurClaire} \color{couleurTexte} INEM \\ \cline{1-5}
 \color{black} André & \color{black} Patrice & \color{black} responsable R\&D & \color{black} & \color{black} LVMH \\ \cline{1-5}

 \cellcolor{couleurClaire} \color{couleurTexte} Ardourel  & \cellcolor{couleurClaire} \color{couleurTexte} Maryvonne & \cellcolor{couleurClaire} \color{couleurTexte} MCF/Univ orléans & \cellcolor{couleurClaire} \color{couleurTexte} 64 & \cellcolor{couleurClaire} \color{couleurTexte} INEM \\ \cline{1-5}
 \color{black} André & \color{black} Patrice & \color{black} responsable R\&D & \color{black} & \color{black} LVMH \\ \cline{1-5}

 \cellcolor{couleurClaire} \color{couleurTexte} Ardourel  & \cellcolor{couleurClaire} \color{couleurTexte} Maryvonne & \cellcolor{couleurClaire} \color{couleurTexte} MCF/Univ orléans & \cellcolor{couleurClaire} \color{couleurTexte} 64 & \cellcolor{couleurClaire} \color{couleurTexte} INEM \\ \cline{1-5}
 \color{black} André & \color{black} Patrice & \color{black} responsable R\&D & \color{black} & \color{black} LVMH \\ \cline{1-5}

 \cellcolor{couleurClaire} \color{couleurTexte} Ardourel  & \cellcolor{couleurClaire} \color{couleurTexte} Maryvonne & \cellcolor{couleurClaire} \color{couleurTexte} MCF/Univ orléans & \cellcolor{couleurClaire} \color{couleurTexte} 64 & \cellcolor{couleurClaire} \color{couleurTexte} INEM \\ \cline{1-5}
 \color{black} André & \color{black} Patrice & \color{black} responsable R\&D & \color{black} & \color{black} LVMH \\ \cline{1-5}

 \cellcolor{couleurClaire} \color{couleurTexte} Ardourel  & \cellcolor{couleurClaire} \color{couleurTexte} Maryvonne & \cellcolor{couleurClaire} \color{couleurTexte} MCF/Univ orléans & \cellcolor{couleurClaire} \color{couleurTexte} 64 & \cellcolor{couleurClaire} \color{couleurTexte} INEM \\ \cline{1-5}
 \color{black} André & \color{black} Patrice & \color{black} responsable R\&D & \color{black} & \color{black} LVMH \\ \cline{1-5}

 \cellcolor{couleurClaire} \color{couleurTexte} Ardourel  & \cellcolor{couleurClaire} \color{couleurTexte} Maryvonne & \cellcolor{couleurClaire} \color{couleurTexte} MCF/Univ orléans & \cellcolor{couleurClaire} \color{couleurTexte} 64 & \cellcolor{couleurClaire} \color{couleurTexte} INEM \\ \cline{1-5}
 \color{black} André & \color{black} Patrice & \color{black} responsable R\&D & \color{black} & \color{black} LVMH \\ \cline{1-5}

 \cellcolor{couleurClaire} \color{couleurTexte} Ardourel  & \cellcolor{couleurClaire} \color{couleurTexte} Maryvonne & \cellcolor{couleurClaire} \color{couleurTexte} MCF/Univ orléans & \cellcolor{couleurClaire} \color{couleurTexte} 64 & \cellcolor{couleurClaire} \color{couleurTexte} INEM \\ \cline{1-5}
 \color{black} André & \color{black} Patrice & \color{black} responsable R\&D & \color{black} & \color{black} LVMH \\ \cline{1-5}

 \cellcolor{couleurClaire} \color{couleurTexte} Ardourel  & \cellcolor{couleurClaire} \color{couleurTexte} Maryvonne & \cellcolor{couleurClaire} \color{couleurTexte} MCF/Univ orléans & \cellcolor{couleurClaire} \color{couleurTexte} 64 & \cellcolor{couleurClaire} \color{couleurTexte} INEM \\ \cline{1-5}
 \color{black} André & \color{black} Patrice & \color{black} responsable R\&D & \color{black} & \color{black} LVMH \\ \cline{1-5}

 \cellcolor{couleurClaire} \color{couleurTexte} Ardourel  & \cellcolor{couleurClaire} \color{couleurTexte} Maryvonne & \cellcolor{couleurClaire} \color{couleurTexte} MCF/Univ orléans & \cellcolor{couleurClaire} \color{couleurTexte} 64 & \cellcolor{couleurClaire} \color{couleurTexte} INEM \\ \cline{1-5}
 \color{black} André & \color{black} Patrice & \color{black} responsable R\&D & \color{black} & \color{black} LVMH \\ \cline{1-5}

 \cellcolor{couleurClaire} \color{couleurTexte} Ardourel  & \cellcolor{couleurClaire} \color{couleurTexte} Maryvonne & \cellcolor{couleurClaire} \color{couleurTexte} MCF/Univ orléans & \cellcolor{couleurClaire} \color{couleurTexte} 64 & \cellcolor{couleurClaire} \color{couleurTexte} INEM \\ \cline{1-5}
 \color{black} André & \color{black} Patrice & \color{black} responsable R\&D & \color{black} & \color{black} LVMH \\ \cline{1-5}

 \cellcolor{couleurClaire} \color{couleurTexte} Ardourel  & \cellcolor{couleurClaire} \color{couleurTexte} Maryvonne & \cellcolor{couleurClaire} \color{couleurTexte} MCF/Univ orléans & \cellcolor{couleurClaire} \color{couleurTexte} 64 & \cellcolor{couleurClaire} \color{couleurTexte} INEM \\ \cline{1-5}
 \color{black} André & \color{black} Patrice & \color{black} responsable R\&D & \color{black} & \color{black} LVMH \\ \cline{1-5}

 \cellcolor{couleurClaire} \color{couleurTexte} Ardourel  & \cellcolor{couleurClaire} \color{couleurTexte} Maryvonne & \cellcolor{couleurClaire} \color{couleurTexte} MCF/Univ orléans & \cellcolor{couleurClaire} \color{couleurTexte} 64 & \cellcolor{couleurClaire} \color{couleurTexte} INEM \\ \cline{1-5}
 \color{black} André & \color{black} Patrice & \color{black} responsable R\&D & \color{black} & \color{black} LVMH \\ \cline{1-5}

 \cellcolor{couleurClaire} \color{couleurTexte} Ardourel  & \cellcolor{couleurClaire} \color{couleurTexte} Maryvonne & \cellcolor{couleurClaire} \color{couleurTexte} MCF/Univ orléans & \cellcolor{couleurClaire} \color{couleurTexte} 64 & \cellcolor{couleurClaire} \color{couleurTexte} INEM \\ \cline{1-5}
 \color{black} André & \color{black} Patrice & \color{black} responsable R\&D & \color{black} & \color{black} LVMH \\ \cline{1-5}

 \cellcolor{couleurClaire} \color{couleurTexte} Ardourel  & \cellcolor{couleurClaire} \color{couleurTexte} Maryvonne & \cellcolor{couleurClaire} \color{couleurTexte} MCF/Univ orléans & \cellcolor{couleurClaire} \color{couleurTexte} 64 & \cellcolor{couleurClaire} \color{couleurTexte} INEM \\ \cline{1-5}
 \color{black} André & \color{black} Patrice & \color{black} responsable R\&D & \color{black} & \color{black} LVMH \\ \cline{1-5}

 \cellcolor{couleurClaire} \color{couleurTexte} Ardourel  & \cellcolor{couleurClaire} \color{couleurTexte} Maryvonne & \cellcolor{couleurClaire} \color{couleurTexte} MCF/Univ orléans & \cellcolor{couleurClaire} \color{couleurTexte} 64 & \cellcolor{couleurClaire} \color{couleurTexte} INEM \\ \cline{1-5}
 \color{black} André & \color{black} Patrice & \color{black} responsable R\&D & \color{black} & \color{black} LVMH \\ \cline{1-5}

 \cellcolor{couleurClaire} \color{couleurTexte} Ardourel  & \cellcolor{couleurClaire} \color{couleurTexte} Maryvonne & \cellcolor{couleurClaire} \color{couleurTexte} MCF/Univ orléans & \cellcolor{couleurClaire} \color{couleurTexte} 64 & \cellcolor{couleurClaire} \color{couleurTexte} INEM \\ \cline{1-5}
 \color{black} André & \color{black} Patrice & \color{black} responsable R\&D & \color{black} & \color{black} LVMH \\ \cline{1-5}

 \cellcolor{couleurClaire} \color{couleurTexte} Ardourel  & \cellcolor{couleurClaire} \color{couleurTexte} Maryvonne & \cellcolor{couleurClaire} \color{couleurTexte} MCF/Univ orléans & \cellcolor{couleurClaire} \color{couleurTexte} 64 & \cellcolor{couleurClaire} \color{couleurTexte} INEM \\ \cline{1-5}
 \color{black} André & \color{black} Patrice & \color{black} responsable R\&D & \color{black} & \color{black} LVMH \\ \cline{1-5}

 \cellcolor{couleurClaire} \color{couleurTexte} Ardourel  & \cellcolor{couleurClaire} \color{couleurTexte} Maryvonne & \cellcolor{couleurClaire} \color{couleurTexte} MCF/Univ orléans & \cellcolor{couleurClaire} \color{couleurTexte} 64 & \cellcolor{couleurClaire} \color{couleurTexte} INEM \\ \cline{1-5}
 \color{black} André & \color{black} Patrice & \color{black} responsable R\&D & \color{black} & \color{black} LVMH \\ \cline{1-5}

 \cellcolor{couleurClaire} \color{couleurTexte} Ardourel  & \cellcolor{couleurClaire} \color{couleurTexte} Maryvonne & \cellcolor{couleurClaire} \color{couleurTexte} MCF/Univ orléans & \cellcolor{couleurClaire} \color{couleurTexte} 64 & \cellcolor{couleurClaire} \color{couleurTexte} INEM \\ \cline{1-5}
 \color{black} André & \color{black} Patrice & \color{black} responsable R\&D & \color{black} & \color{black} LVMH \\ \cline{1-5}

 \cellcolor{couleurClaire} \color{couleurTexte} Ardourel  & \cellcolor{couleurClaire} \color{couleurTexte} Maryvonne & \cellcolor{couleurClaire} \color{couleurTexte} MCF/Univ orléans & \cellcolor{couleurClaire} \color{couleurTexte} 64 & \cellcolor{couleurClaire} \color{couleurTexte} INEM \\ \cline{1-5}
 \color{black} André & \color{black} Patrice & \color{black} responsable R\&D & \color{black} & \color{black} LVMH \\ \cline{1-5}

 \cellcolor{couleurClaire} \color{couleurTexte} Ardourel  & \cellcolor{couleurClaire} \color{couleurTexte} Maryvonne & \cellcolor{couleurClaire} \color{couleurTexte} MCF/Univ orléans & \cellcolor{couleurClaire} \color{couleurTexte} 64 & \cellcolor{couleurClaire} \color{couleurTexte} INEM \\ \cline{1-5}
 \color{black} André & \color{black} Patrice & \color{black} responsable R\&D & \color{black} & \color{black} LVMH \\ \cline{1-5}

 \cellcolor{couleurClaire} \color{couleurTexte} Ardourel  & \cellcolor{couleurClaire} \color{couleurTexte} Maryvonne & \cellcolor{couleurClaire} \color{couleurTexte} MCF/Univ orléans & \cellcolor{couleurClaire} \color{couleurTexte} 64 & \cellcolor{couleurClaire} \color{couleurTexte} INEM \\ \cline{1-5}

\end{tabular}

% % % % % % % % % % % % % % % % % % % % % % % % % % % % % % % % % % % % % % % % % % % % % % % % % % % % % % % 


\newpage

\subsection*{Le semestre 1}

\subsubsection*{Maquette des enseignements}

% % % % % % % % % % % % % % % % % % % % % % % % % % % % % % % % % % % % % % % % % % % % % % % % % % % % % % % 

%---------------------- % % % Personnalisation des couleurs % % % ----------- ROUGE --------
%---------------------- % % % Personnalisation des couleurs % % % ----------- Vert Licence --------
\definecolor{couleurFonce}{RGB}{108,5,5} % Couleur du Code APOGEE
\definecolor{couleurClaire}{RGB}{147,97,97} % Couleur du fond de la bande
\definecolor{couleurTexte}{RGB}{255,255,255} % Couleur du texte de la bande
%------------------------------------------------------------------------------------------

%------------------------------------------------------------------------------------------

\arrayrulecolor{couleurFonce}% Couleur des lignes séparatrices du tableau
\renewcommand{\arraystretch}{1.5}% Coeff appliqué à la hauteur des cellules
%\rowcolors[\hline]{ligneDébut}{couleurPaire}{couleurImpaire}% Alternance de couleur (need package xcolor)
\begin{tabular}{|m{5cm}|cm{0.75cm}|cm{0.75cm}|cm{0.75cm}|cm{0.75cm}|cm{0.75cm}|}
%\begin{tabular}{c|m{6cm}|cm{0.8cm}|cm{0.8cm}|cm{0.8cm}|cm{0.8cm}|}
%\begin{tabular}{c|m|cm|cm|cm|cm|cm|}
\cline{1-5}

\cellcolor{couleurFonce} \color{white}\bfseries Intitul\'e & \cellcolor{couleurFonce} \color{white}\bfseries ECTS & \cellcolor{couleurFonce} \color{white}\bfseries CM & \cellcolor{couleurFonce} \color{white}\bfseries TD & \cellcolor{couleurFonce} \color{white}\bfseries TP \\ \cline{1-5}
%----
 \color{black} \mbox{Expression du génôme eucaryote} & \color{black} 5 & \color{black} 24 & \color{black} 14 & \color{black} 10 \\ \cline{1-5}

 \cellcolor{couleurClaire} \color{couleurTexte} Anglais scientifique 1  & \cellcolor{couleurClaire} \color{couleurTexte} 3 & \cellcolor{couleurClaire} \color{couleurTexte} & \cellcolor{couleurClaire} \color{couleurTexte} 24 & \cellcolor{couleurClaire} \color{couleurTexte} \\ \cline{1-5}

 \color{black} \mbox{Biologie moléculaire et cellulaire} \mbox{expérimentale} & \color{black} 5 & \color{black} & \color{black} & \color{black} 48 \\ \cline{1-5}

 \cellcolor{couleurClaire} \color{couleurTexte} Biostat 1 : initiation à "R"  & \cellcolor{couleurClaire} \color{couleurTexte} 3 & \cellcolor{couleurClaire} \color{couleurTexte} & \cellcolor{couleurClaire} \color{couleurTexte} 24 & \cellcolor{couleurClaire} \color{couleurTexte} 
\\ \cline{1-5}

 \color{black} Dynamique et fonctionnement des écosystèmes terrestres & \color{black} 5 & \color{black} 31 & \color{black} 9 & \color{black} 8 \\ \cline{1-5}

 \cellcolor{couleurClaire} \color{couleurTexte} Facteurs de la distribution des organismes et 
évolution des espèces & \cellcolor{couleurClaire} \color{couleurTexte} 5 & \cellcolor{couleurClaire} \color{couleurTexte} 24 & \cellcolor{couleurClaire} \color{couleurTexte} 12 & \cellcolor{couleurClaire} 12 
\\ \cline{1-5}

 \color{black} Ecologie du paysage & \color{black} 4 & \color{black} 18 & \color{black} 4 & \color{black} 4 \\ \cline{1-5}


\end{tabular}

% % % % % % % % % % % % % % % % % % % % % % % % % % % % % % % % % % % % % % % % % % % % % % % % % % % % % % % 


\subsection*{Détail des enseignements}


\end{spacing}

\end{document}
