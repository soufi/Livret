\documentclass[10pt, a5paper]{report}

\usepackage[T1]{fontenc}%
\usepackage[utf8]{inputenc}% encodage utf8
\usepackage[francais]{babel}% texte français
\usepackage[top=10mm, bottom=20mm, foot=5mm, left=7mm, offset=7mm, textwidth=350pt]{geometry}
\usepackage{url}
\usepackage{init-preambule}
\usepackage[final]{pdfpages}

% Définir les couleur du document
%---------------------- % % % Personnalisation des couleurs % % % ----------- Vert Licence --------
\definecolor{couleurFonce}{RGB}{18,92,40} % Couleur du Code APOGEE
\definecolor{couleurClaire}{RGB}{28,161,68} % Couleur du fond de la bande
\definecolor{couleurTexte}{RGB}{255,255,255} % Couleur du texte de la bande
%------------------------------------------------------------------------------------------

\begin{document}

\begin{spacing}{1.5}

\chapter*{Licence Biologie-Biochimie}

\footnotesize
\section*{Objectifs}

	\begin{wrapfigure}{r}{0.35\textwidth}
          \vspace{-20pt}
            \begin{center}
                \begin{tikzpicture}
                    \node [rectangle, draw=couleurFonce, thick, drop shadow, fill=couleurBox, inner sep=10pt, inner ysep=10pt] (box) {
                    \begin{minipage}{0.35\textwidth}%
                        \begin{spacing}{1}
                        \begin{tabular}[t]{@{}m{10mm}@{~~}m{30mm}@{}}
                          % \includegraphics[scale=0.6]{img/photos/avatar.png} & \normalsize{\textbf{Valérie}\newline \textbf{ALTEMAYER}}\newline \footnotesize{}\\
		\multicolumn{2}{l}{\textbf{Valérie Altemayer}}\\
                \multicolumn{2}{c}{{\scriptsize \textit{valerie.altemayer@univ-orleans.fr}}} \\
                          \multicolumn{2}{c}{\includegraphics[scale=0.7]{img/telephone.png}{\scriptsize (Secrétariat)+33238417097}}
                        \end{tabular}
                        \end{spacing}
                    \end{minipage}
                    };
                    \node[fancytitle, right=5pt, rounded corners, inner xsep=10pt] at (box.north west) {\normalsize{Directrice des études}};
                \end{tikzpicture}
            \end{center}
            \vspace{-20pt}
      \end{wrapfigure}





\letterine{D}ans le cadre de l’harmonisation européenne des cursus universitaires (réforme LMD Licence-Master-Doctorat), les mentions de  Licence sont constituées de 6 semestres.

\subsection*{Parcours et objectifs}

\letterine{L}a \textbf{licence mention « Biologie – Biochimie »} a pour objectifs de : 

- former un public étudiant à toutes les disciplines de la biologie : biologie générale animale et végétale, biologie cellulaire et moléculaire, génétique formelle et des populations, physiologie animale et physiologie végétale, biotechnologies, écologie, biologie des organismes et des populations.

- lui permettre ainsi d’acquérir des connaissances fondamentales et pratiques de niveau Bac+3 dans tous les secteurs techniques et scientifiques des Sciences de la Vie, ces connaissances étant indispensables à une poursuite d’études vers les Masters Recherches et Professionnels ou à l’intégration dans le monde professionnel dans ce domaine.

- préparer au concours du CAPES Sciences de la Vie et de la Terre, via une insertion dans un Master de préparation au concours du CAPES (exemple le Master MED-SVT d’Orléans)

- préparer au concours pour le Professorat des écoles, via une insertion dans un Master IUFM (exemple le Master MEEFA IUFM d’Orléans) mettre à disposition les savoirs et les connaissances existants à l’Université d’Orléans dans le domaine des Sciences biologiques.
\newline

\letterine{L}a première année de la licence mention « Biologie – Biochimie » est organisée en deux semestres. Le premier est un semestre d'enseignement pluridisciplinaire, commun à tous les étudiants. Au second semestre, deux parcours peuvent être choisis par les étudiants : 
\begin{itemize}
\item Le parcours \textbf{général}, qui permet d'acquérir les bases d'une formation généraliste en sciences biologiques. Ce parcours permet par la suite d'accéder aux différents parcours de la licence Biologie-Biochimie : 
	\subitem Le parcours « \textbf{Biologie des Organismes, des Populations et des Ecosystèmes} » (BOPE) s’intègre dans le contexte actuel de l’essor des disciplines de l’écologie de la biodiversité et de l’étude de l’environnement. Ce parcours permet aux étudiants d’approfondir leurs connaissances sur les inter-relations des êtres vivants, animaux ou végétaux, ainsi que les liens avec leur environnement. Les mécanismes adaptatifs, évolutifs, physiologiques et génétiques seront abordés au niveau des organismes et des populations. Pour ce faire, au-delà des connaissances théoriques, des études sur le terrain et en laboratoire, intégrant les outils modernes de la biologie moléculaire sont  proposées.
	\subitem Le parcours « \textbf{Biologie Moléculaire et cellulaire} » (BMC) est destiné aux étudiants voulant aborder le monde vivant, par ses aspects physiologiques, cellulaires et moléculaires. Ce parcours permettra aux étudiants d’acquérir des connaissances en biologie moléculaire, en biochimie structurale, en génétique, en biochimie, en biophysique en physiologie humaine et animale et en immunologie. Les aspects pratiques permettront aux étudiants d’acquérir une initiation au travail en laboratoire dans ces différents domaines.  Un choix d’option de chimie en semestre 6 peut aussi permettre à l’étudiant d’acquérir des connaissances spécifiques à la chimie bioorganique.
	\subitem Le parcours « \textbf{Biologie et Biochimie du Végétal} » (BBV) est spécialement conçu pour les étudiants intéressés par le monde végétal et souhaitant se former de façon large aux disciplines de la physiologie végétale, de la biologie moléculaire et des biotechnologies végétales, ainsi qu’à l’écologie.  Il recouvre ainsi partiellement les parcours BOPE et BMC. Ce parcours qui a été ouvert aux étudiants durant le contrat précédent a montré qu’il correspondait parfaitement aux attentes des étudiants « végétalistes ». Sans surcoût, il permet par le jeu de la mutualisation des unités de biologie végétale et de biochimie d’offrir une formation originale dans ce domaine.
\item Le parcours \textbf{enseignement}, contenant des enseignements adaptés aux étudiants désireux de poursuivre une formation pour préparer les concours du CAPES Sciences de la vie et de la terre (Parcours \textbf{BGST}) ou du professorat des écoles (Parcours \textbf{PLURI}).
\end{itemize}

\subsection*{Conditions d'admission en première année de licence}

Baccalauréats scientifiques ou équivalent.

\subsection*{Orientation et réorientation}

\letterine{L}e choix du parcours de licence s'effectue l’issue du premier semestre et sera réalisé en concertation avec l'enseignant référent et le directeur des études.

\subsection*{Glossaire}

\subsubsection*{Unités d’enseignement :} 

Ensemble d’enseignements comprenant des cours, TD, TP ou autres travaux personnels. Chaque unité est affectée d’ECTS et fait l’objet d’un contrôle de connaissances.
\begin{itemize}
\item[\textit{Unités d’ossature}] : unités d’enseignement obligatoires correspondant à la formation de l’étudiant dans la licence  pour le parcours choisi.
\item[\textit{Unités de choix}] : unités à choisir dans une sélection d’unités proposées pour personnaliser son parcours.
\item[\textit{Unités libres}] : UE permettant à l’étudiant de compléter sa formation (culture générale, méthodologie universitaire, éléments de professionnalisation, stages, compléments de langue, utilisation des ressources documentaires…). Unités à choisir dans le livret des UE libres de l’ensemble des composantes de l’Université.
\end{itemize}

\subsubsection*{Crédits ECTS (European Credits Transfer System = Système Européen de Crédits de Transfert)}

\begin{itemize}
\item A chaque unité d’enseignement (UE) est affectée une valeur en crédits qui correspond au volume du horaire de l’unité. Les crédits sont attribués quand l’unité est validée (note > ou = à 10/20).
\item Chaque semestre validé correspond à 30 crédits. Une licence correspond donc à 180 crédits (6 semestres).
\item Ces crédits représentent une monnaie d’échange et sont : 
\begin{itemize}
\item Transférables dans toute autre université européenne ;
\item Capitalisables c’est-à-dire définitivement acquis quelle que soit la durée du parcours de l’étudiant. 
\end{itemize}
\end{itemize}

\subsubsection*{Grade :}

Les grades universitaires sanctionnent les divers niveaux de l’enseignement supérieur. Depuis le décret du 8 avril 2002 ce sont : le baccalauréat, la licence, le master et le doctorat.

\subsubsection*{Diplôme :}

A chaque grade correspond un titre ou un diplôme. Sont conservées les possibilités de délivrer deux diplômes intermédiaires, ne correspondant pas à un des grades précédents :
\begin{itemize}
\item celui de maîtrise (correspondant aux 60 crédits ECTS des semestres 1 et 2 de master) – (article 9 de l’arrêté du 25 avril 2002 relatif au diplôme de master),
\item celui de DEUG (correspondant aux 120 crédits ECTS des semestres 1, 2, 3 et 4 de licence) – (article 2 de l’arrêté du 23 avril 2002 relatif aux études universitaires conduisant au grade de licence).
\end{itemize}

\textbf{Ces deux derniers diplômes sont délivrés dans la mention retenue sans indication de parcours ou de spécialité, mais accompagnés d’un « supplément au diplôme ».}

\subsubsection*{Supplément au diplôme} 

(décret du 8 avril 2002 portant application au système français d’enseignement supérieur de la construction de l’Espace européen de l’enseignement supérieur. Art. 2). C’est une annexe descriptive au diplôme,  destinée à assurer la lisibilité des connaissances et aptitudes acquises dans le cadre de la mobilité internationale.

\subsubsection*{Equipe de formation :} Chaque mention de licence ou master est pilotée par une équipe de formation qui veille à l’adéquation de l’organisation des études avec les objectifs fixés, organise l’évaluation générale des formations et élabore  un bilan annuel de la formation.

\subsubsection*{Enseignant-référent :} Il est en charge du suivi individualisé d’une dizaine d’étudiants. Il conseille l’étudiant dans le choix des UE de sensibilisation ou d’ouverture en accord avec son Projet Personnel et Professionnel.

\subsubsection*{Projet Personnel et Professionnel :} Le Projet Personnel et Professionnel (SOL1PP01 ou SSL1PP01) est réalisé lors du semestre 1. Il a pour but d’aider l’étudiant à définir et/ou préciser un projet en termes de vie professionnelle. Pour cela, il doit s’informer sur les formations proposées et confronter son projet aux réalités professionnelles… Cette démarche active de son orientation future permet à l’étudiant de faire des choix pertinents tout au long de ses études.

\subsubsection*{Directeur des études :} en contact direct avec les étudiants et la scolarité de l’UFR, il anime l’équipe pédagogique.
Jury : Constitué pour chaque semestre et présidé par le Directeur des études, il a en charge le recueil des notes, l’établissement des moyennes et la validation du semestre.

\subsubsection*{Compensations :}

\begin{itemize}
\item Pour chaque semestre, si la note globale moyenne est supérieure ou égale à 10/20, le semestre est validé et lui sont associés 30 ECTS.
\item La compensation annuelle s'organise dès la 1ère session entre deux semestres immédiatement consécutifs d'une même année universitaire : soit le semestre 1 avec le semestre 2.
\end{itemize}

\subsubsection*{Notes éliminatoires :}

Pas de notes éliminatoire en licence

\newpage
\section*{Semestre 1}
\subsection*{Maquette des enseignements}

% % % % % % % % % % % % % % % % % % % % % % % % % % % % % % % % % % % % % % % % % % % % % % % % % % % % % % % 

%---------------------- % % % Personnalisation des couleurs % % % ----------- ROUGE --------
%---------------------- % % % Personnalisation des couleurs % % % ----------- Vert Licence --------
\definecolor{couleurFonce}{RGB}{18,92,40} % Couleur du Code APOGEE
\definecolor{couleurClaire}{RGB}{28,161,68} % Couleur du fond de la bande
\definecolor{couleurTexte}{RGB}{255,255,255} % Couleur du texte de la bande
%------------------------------------------------------------------------------------------

%------------------------------------------------------------------------------------------

\arrayrulecolor{couleurFonce}% Couleur des lignes séparatrices du tableau
\renewcommand{\arraystretch}{1.5}% Coeff appliqué à la hauteur des cellules
%\rowcolors[\hline]{ligneDébut}{couleurPaire}{couleurImpaire}% Alternance de couleur (need package xcolor)
%\begin{tabular}{c|m{6cm}|cm{1cm}|cm{1cm}|cm{1cm}|cm{1cm}|}
\begin{tabular}{c|m{4.5cm}|cm{0.75cm}|cm{0.75cm}|cm{0.75cm}|cm{0.75cm}|cm{0.75cm}|}
%\begin{tabular}{c|m|cm|cm|cm|cm|cm|}
\cline{2-7}

&
\cellcolor{couleurFonce} \color{white}\bfseries Intitul\'e & \cellcolor{couleurFonce} \color{white}\bfseries ECTS & \cellcolor{couleurFonce} \color{white}\bfseries CM & \cellcolor{couleurFonce} \color{white}\bfseries TD & \cellcolor{couleurFonce} \color{white}\bfseries CTP & \cellcolor{couleurFonce} \color{white}\bfseries TP\\ \cline{2-7}

\hline \multirow{8}{*}{\rotatebox{90}{\color{couleurFonce}\bfseries Ossature - 30 ECTS}}
 & \color{black} \mbox{Panorama} \mbox{du} \mbox{monde} \mbox{animal} \mbox{et} \mbox{végétal}  & \color{black} 5 & \color{black} 30 & \color{black} 3 & & \color{black} 15\\ \cline{2-7}
 & \cellcolor{couleurClaire} \color{couleurTexte} \mbox{Atomistique et thermodynamique}  & \cellcolor{couleurClaire} \color{couleurTexte} 5 & \cellcolor{couleurClaire} \color{couleurTexte} 17 & \cellcolor{couleurClaire} \color{couleurTexte} 16 & \cellcolor{couleurClaire} & \cellcolor{couleurClaire} \color{couleurTexte} 15
\\ \cline{2-7}
 & \color{black} \mbox{Molécules du vivant} & \color{black} 5 & \color{black} 28 & \color{black} 10 & & \color{black} 10
\\ \cline{2-7}
 & \cellcolor{couleurClaire} \color{couleurTexte} \mbox{Anglais 1}  & \cellcolor{couleurClaire} \color{couleurTexte} 3 & \cellcolor{couleurClaire} \color{couleurTexte}  & \cellcolor{couleurClaire} \color{couleurTexte} 24 & \cellcolor{couleurClaire} \color{couleurTexte} & \cellcolor{couleurClaire}  
\\ \cline{2-7}
 & \color{black} \mbox{Analyse des données en biosciences}  & \color{black} 2 & \color{black} 6 & \color{black} 6 & \color{black} &  
\\ \cline{2-7}
 & \cellcolor{couleurClaire} \color{couleurTexte} \mbox{Introduction à la biologie cellulaire} & \cellcolor{couleurClaire} \color{couleurTexte} 4 & \cellcolor{couleurClaire} \color{couleurTexte} 18 & \cellcolor{couleurClaire} \color{couleurTexte} 6 & \cellcolor{couleurClaire} & \cellcolor{couleurClaire} \color{couleurTexte} 12 
\\ \cline{2-7}
 & \color{black} \mbox{Préparation au C2i} & \color{black} 3 & \color{black}  & \color{black}  & \color{black} 24 & 
\\ \cline{2-7}
 & \cellcolor{couleurClaire} \color{couleurTexte} \mbox{Projet} \mbox{personnel} \mbox{et} \mbox{professionnel} \mbox{et} \mbox{techniques} \mbox{de} \mbox{communications} & \cellcolor{couleurClaire} \color{couleurTexte} 3 & \cellcolor{couleurClaire} \color{couleurTexte} 2 & \cellcolor{couleurClaire} \color{couleurTexte} 14 & \cellcolor{couleurClaire} & \cellcolor{couleurClaire} \color{couleurTexte} \\ \cline{1-7} 
\end{tabular}

% % % % % % % % % % % % % % % % % % % % % % % % % % % % % % % % % % % % % % % % % % % % % % % % % % % % % % % 


\subsection*{Détail des enseignements}

\end{spacing}

\end{document}
