\documentclass[10pt, a5paper]{report}

\usepackage[T1]{fontenc}%
\usepackage[utf8]{inputenc}% encodage utf8
\usepackage[francais]{babel}% texte français
\usepackage[final]{pdfpages}
\usepackage{modules-livret}% style du livret
\usepackage{url}
%\usepackage{init-preambule}
\pagestyle{empty}

% % % % % % % % % % % % % % % % % % % % % % % % % % % % % % % % % % % % % % % % % % % % % % % % % % % % % % % 
\begin{document}

%---------------------- % % % Personnalisation des couleurs % % % ----------- Vert Licence --------
\definecolor{couleurFonce}{RGB}{18,92,40} % Couleur du Code APOGEE
\definecolor{couleurClaire}{RGB}{28,161,68} % Couleur du fond de la bande
\definecolor{couleurTexte}{RGB}{255,255,255} % Couleur du texte de la bande
%------------------------------------------------------------------------------------------


%==========================================================================================
% Semestre 1
%==========================================================================================
\module[codeApogee={SOL1BO01 SSL1BO01},
titre={Panorama du monde animal et végétal}, 
COURS={30}, 
TD={3}, 
TP={15}, 
CTD={},
CTP={}, 
TOTAL={48}, 
SEMESTRE={Semestre 1}, 
COEFF={5}, 
ECTS={5}, 
MethodeEval={Ecrit},
ModalitesCCSemestreUn={RNE : CCI(3) 4h ; RSE : CT 2h},
ModalitesCCSemestreDeux={RNE et RSE : CT 2h},
CalculNFSessionUne={Ecrit 100\%},
CalculNFSessionDeux={Ecrit 100\%},
NoteEliminatoire={}, 
nomPremierResp={François Leutier}, 
emailPremierResp={francois.lieutier@univ-orleans.fr}, 
nomSecondResp={}, 
emailSecondResp={}, 
langue={Français}, 
nbPrerequis={0}, 
descriptionCourte={true}, 
descriptionLongue={true}, 
objectifs={true}, 
ressources={true}, 
bibliographie={false}] 
% ******* Texte introductif
{
Tronc commun 
} 
% ******* Contenu détaillé
{
Origine de la vie et grandes étapes de sa complexification .Eubactéries, archées, eucaryotes ; les diverses lignées unicellulaires et pluricellulaires ; les plans d’organisation du monde animal (feuillets embryonnaires, coelome, ..) et caractères généraux des grands embranchements. Plans d’organisation du monde végétal, les étapes de l’acquisition des différents organes des algues aux angiospermes; particularités des mycètes.
} 
% ******* Pré-requis
{} 
% ******* Objectifs
{\begin{itemize} 
  \ObjItem Sensibiliser l’étudiant à la diversité du monde vivant et lui faire connaître son organisation.
\end{itemize} 
} 
% ******* Ressources pédagogiques
{} 
% ******* Bibliographie éventuelle
{Biblio}
 
\vfill

%==========================================================================================
\module[codeApogee={SOL1CH02 SSL1CH02},
titre={Atomistique et thermodynamique}, 
COURS={17}, 
TD={16}, 
TP={15}, 
CTD={},
CTP={}, 
TOTAL={48}, 
SEMESTRE={Semestre 1}, 
COEFF={5}, 
ECTS={5}, 
MethodeEval={Ecrit}, 
ModalitesCCSemestreUn={RNE : CCI (E(3) + TP(5)) 5h ; RSE : CT (E+TP) 3h}, 
ModalitesCCSemestreDeux={RNE et RSE : CT (E+TP) 3h}, 
CalculNFSessionUne={Ecrit : 67 \% ; TP : 33 \%}, 
CalculNFSessionDeux={Ecrit : 67 \% ; TP : 33 \%}, 
NoteEliminatoire={}, 
nomPremierResp={Fabienne Méducin}, 
emailPremierResp={fabienne.meducin@univ-orleans.fr}, 
nomSecondResp={}, 
emailSecondResp={}, 
langue={Français}, 
nbPrerequis={0}, 
descriptionCourte={true}, 
descriptionLongue={true}, 
objectifs={true}, 
ressources={true}, 
bibliographie={false}] 
% ******* Texte introductif
{
Tronc commun 
} 
% ******* Contenu détaillé
{
\begin{itemize}
\item Structure de l'atome (le noyau, les électrons : introduction des orbitales atomiques), formation des liaisons (introduction des orbitales moléculaires)
\item Radioactivité, quelques propriétés moléculaires (moment dipolaire, énergies de liaison, ...)
\item Gaz parfaits, premier et deuxième principes de thermodynamique.
\end{itemize}
} 
% ******* Pré-requis
{} 
% ******* Objectifs
{\begin{itemize} 
  \ObjItem Bases de chimie générale.
\end{itemize} 
} 
% ******* Ressources pédagogiques
{} 
% ******* Bibliographie éventuelle
{Biblio}
 
\vfill

%==========================================================================================
\module[codeApogee={SOL1BH01 SSL1BH01},
titre={Molécules du vivant}, 
COURS={28}, 
TD={10}, 
TP={10}, 
CTD={},
CTP={}, 
TOTAL={48}, 
SEMESTRE={Semestre 1}, 
COEFF={5}, 
ECTS={5}, 
MethodeEval={Ecrit}, 
ModalitesCCSemestreUn={RNE : CCI (E(3)+TP) 4h ; RSE : CT (E+TP) 4h}, 
ModalitesCCSemestreDeux={RNE et RSE : CT (E+TP) 2h30}, 
CalculNFSessionUne={Ecrit : 75 \% ; TP : 25 \%}, 
CalculNFSessionDeux={Ecrit : 75 \% ; TP : 25 \%}, 
NoteEliminatoire={}, 
nomPremierResp={Eric Hébert}, 
emailPremierResp={eric.hebert@univ-orleans.fr}, 
nomSecondResp={}, 
emailSecondResp={}, 
langue={Français}, 
nbPrerequis={0}, 
descriptionCourte={true}, 
descriptionLongue={true}, 
objectifs={true}, 
ressources={true}, 
bibliographie={false}] 
% ******* Texte introductif
{
Tronc commun 
} 
% ******* Contenu détaillé
{
\begin{itemize}
\item Protéines : acides aminés - propriétés physico-chimiques. Structure primaire des protéines. Structure secondaire : hélice $\alpha$ , feuillets $\beta$.
\item Lipides : acides gras, triglycérides, glycéro-phospholipides, stéroïdes. Membranes et surface cellulaire, structure, composition des protéines membranaires.
\item Oses : nomenclature, mono et diholosides, polyosides de réserve et de structure. 
\item Acides nucléiques : structure primaire : bases puriques et pyrimidiques.  Nucléosides, nucléotides. Structure secondaire : liaisons faibles, double brin, double hélice, séquençage chimique de l’ADN. 
\item Enzymologie : rappels de cinétique chimique ; structure et mécanisme d’action des enzymes ; régulation de l’activité enzymatique ; 
\item Introduction au métabolisme : glycolyse et cycle de Krebs ; oxydations phosphorylantes.
\end{itemize}
} 
% ******* Pré-requis
{} 
% ******* Objectifs
{\begin{itemize} 
  \ObjItem Bases moléculaires de la biologie.
\end{itemize} 
} 
% ******* Ressources pédagogiques
{} 
% ******* Bibliographie éventuelle
{Biblio}
 
\vfill

%==========================================================================================
\module[codeApogee={SOL1AG11 SSL1AG11},
titre={Anglais 1}, 
COURS={}, 
TD={24}, 
TP={}, 
CTD={},
CTP={}, 
TOTAL={24}, 
SEMESTRE={Semestre 1}, 
COEFF={3}, 
ECTS={3}, 
MethodeEval={Ecrit et/ou oral}, 
ModalitesCCSemestreUn={RNE : CCI(5) 2h ; RSE : CT 1h}, 
ModalitesCCSemestreDeux={RNE et RSE : CT 1h30}, 
CalculNFSessionUne={CCI ou CT 100\%},
CalculNFSessionDeux={CT 100\%},
NoteEliminatoire={}, 
nomPremierResp={Murielle Pasquet}, 
emailPremierResp={murielle.pasquet@univ-orleans.fr}, 
nomSecondResp={}, 
emailSecondResp={}, 
langue={Français/Anglais}, 
nbPrerequis={1}, 
descriptionCourte={true}, 
descriptionLongue={true}, 
objectifs={true}, 
ressources={true}, 
bibliographie={false}] 
% ******* Texte introductif
{
Tronc commun 
} 
% ******* Contenu détaillé
{
Travail de compréhension et d’expression orale et écrite à partir de documents authentiques simples et / ou courts centrés sur le monde universitaire anglo-saxon.
} 
% ******* Pré-requis
{Niveau baccalauréat anglais LV1 ou LV2 ou environ 400 heures de formation} 
% ******* Objectifs
{\begin{itemize} 
  \ObjItem Etre à même de préparer un projet de séjour d’études universitaires en pays anglophone dans une langue écrite et orale simple et suffisamment claire.
\end{itemize} 
} 
% ******* Ressources pédagogiques
{} 
% ******* Bibliographie éventuelle
{Biblio}
 
\vfill

%==========================================================================================
\module[codeApogee={SOL1BO03 SSL1BO03},
titre={Analyse des données en biosciences}, 
COURS={6}, 
TD={6}, 
TP={}, 
CTD={},
CTP={}, 
TOTAL={12}, 
SEMESTRE={Semestre 1}, 
COEFF={2}, 
ECTS={2}, 
MethodeEval={Ecrit}, 
ModalitesCCSemestreUn={RNE et RSE : CCI(3) 1h30}, 
ModalitesCCSemestreDeux={RNE et RSE : CT 1h}, 
CalculNFSessionUne={Ecrit 100\%},
CalculNFSessionDeux={Ecrit 100\%},
NoteEliminatoire={}, 
nomPremierResp={Aurélien Sallé}, 
emailPremierResp={aurelien.salle@univ-orleans.fr}, 
nomSecondResp={}, 
emailSecondResp={}, 
langue={Français}, 
nbPrerequis={0}, 
descriptionCourte={true}, 
descriptionLongue={true}, 
objectifs={true}, 
ressources={true}, 
bibliographie={false}] 
% ******* Texte introductif
{
Tronc commun 
} 
% ******* Contenu détaillé
{
Représentation des données dans le domaine biologique ; valeurs centrales et de dispersion. Distribution normale et épreuves de normalité. Test de comparaison des moyennes (Student). Applications en écologie : Régression linéaire simple, covariance, corrélation, test de signification du coefficient de corrélation, loi de Poisson, Chi2 d'ajustement à la loi de Poisson. Applications en biochimie-physiologie et génétique : les principes énoncés en cours seront appliqués à des analyses de données issues de travaux dans le domaine bio-santé. Il sera mis l’accent en particulier sur la pertinence d’emploi de tel ou tel test statistique et la présentation graphique des résultats (moyenne, écart type, erreur standard à la moyenne…)
} 
% ******* Pré-requis
{} 
% ******* Objectifs
{\begin{itemize} 
  \ObjItem Eléments statistiques de base pour les applications écologiques et physiologiques en Travaux Dirigés.
\end{itemize} 
} 
% ******* Ressources pédagogiques
{} 
% ******* Bibliographie éventuelle
{Biblio}
 
\vfill

%==========================================================================================
\module[codeApogee={SOL1BO02 SSL1BO02},
titre={Introduction à la biologie cellulaire}, 
COURS={18}, 
TD={6}, 
TP={12}, 
CTD={},
CTP={}, 
TOTAL={36}, 
SEMESTRE={Semestre 1}, 
COEFF={4}, 
ECTS={4}, 
MethodeEval={Ecrit}, 
ModalitesCCSemestreUn={RNE : CCI (E(2)+TP(4-5)) 3h ; RSE : CT (E+TP) 1h30}, 
ModalitesCCSemestreDeux={RNE et RSE : CT 1h}, 
CalculNFSessionUne={Ecrit : 75 \% ; TP : 25 \%}, 
CalculNFSessionDeux={Ecrit 100\%},
NoteEliminatoire={}, 
nomPremierResp={Valérie Altemayer}, 
emailPremierResp={valerie.altemayer@univ-orleans.fr}, 
nomSecondResp={Eric Duverger}, 
emailSecondResp={eric.duverger@univ-orleans.fr}, 
langue={Français}, 
nbPrerequis={0}, 
descriptionCourte={true}, 
descriptionLongue={true}, 
objectifs={true}, 
ressources={true}, 
bibliographie={false}] 
% ******* Texte introductif
{
Tronc commun 
} 
% ******* Contenu détaillé
{
\begin{itemize}
\item Acaryotes, Procaryotes et Eucaryotes : Définition, classification, multiplication. 
\item Description de la cellule : membrane plasmique, parois, système membranaire interne, organites semi-autonome, cytosquelette. 
\item Différenciation cellulaire : types cellulaires et structure des cellules spécialisées. 
\item Particularité de la cellule végétale : plastes, paroi, vacuoles, pigments… 
\item TP : Apprentissage de quelques techniques d’observation de la cellule et illustration du cours. 
\end{itemize}
} 
% ******* Pré-requis
{} 
% ******* Objectifs
{\begin{itemize} 
  \ObjItem Sensibiliser l’étudiant à la diversité du monde vivant et lui faire connaître son organisation.
\end{itemize} 
} 
% ******* Ressources pédagogiques
{} 
% ******* Bibliographie éventuelle
{Biblio}
 
\vfill

%==========================================================================================
\module[codeApogee={SOL1II01 SSL1II01},
titre={Préparation au C2i (Certificat informatique et internet)}, 
COURS={}, 
TD={}, 
TP={}, 
CTD={},
CTP={24}, 
TOTAL={24}, 
SEMESTRE={Semestre 1}, 
COEFF={3}, 
ECTS={3}, 
MethodeEval={Ecrit}, 
ModalitesCCSemestreUn={RNE : CCI(3) 5h ; RSE : CT 2h}, 
ModalitesCCSemestreDeux={RNE et RSE : CT 2h}, 
CalculNFSessionUne={CCI ou CT 100\%},
CalculNFSessionDeux={Ecrit 100\%},
NoteEliminatoire={}, 
nomPremierResp={Laure Kahlem}, 
emailPremierResp={laure.kahlem@univ-orleans.fr}, 
nomSecondResp={}, 
emailSecondResp={}, 
langue={Français}, 
nbPrerequis={0}, 
descriptionCourte={true}, 
descriptionLongue={true}, 
objectifs={true}, 
ressources={true}, 
bibliographie={false}] 
% ******* Texte introductif
{
Tronc commun 
} 
% ******* Contenu détaillé
{
Cette UE a pour objectif de développer les compétences de base nécessaires à l’usage des Technologies de l’Information et de la Communication telles que définies dans le référentiel national du C2i niveau 1 et de préparer les étudiants à la certification. } 
% ******* Pré-requis
{} 
% ******* Objectifs
{} 
% ******* Ressources pédagogiques
{} 
% ******* Bibliographie éventuelle
{Biblio}
 
\vfill

%==========================================================================================
\module[codeApogee={SOL1PP01 SSL1PP01},
titre={Projet personnel et professionnel - techniques de communications}, 
COURS={2}, 
TD={14}, 
TP={}, 
CTD={},
CTP={}, 
TOTAL={16}, 
SEMESTRE={Semestre 1}, 
COEFF={3}, 
ECTS={3}, 
MethodeEval={Ecrit et oral}, 
ModalitesCCSemestreUn={RNE et RSE : CCI(2) 1h}, 
ModalitesCCSemestreDeux={RNE et RSE : CT 1h}, 
CalculNFSessionUne={Ecrit 50 \% ; Oral 50 \%}, 
CalculNFSessionDeux={Ecrit 50 \% ; Oral 50 \%}, 
NoteEliminatoire={}, 
nomPremierResp={Olivier Richard}, 
emailPremierResp={olivier.richard@univ-orleans.fr}, 
nomSecondResp={}, 
emailSecondResp={}, 
langue={Français}, 
nbPrerequis={0}, 
descriptionCourte={true}, 
descriptionLongue={true}, 
objectifs={true}, 
ressources={true}, 
bibliographie={false}] 
% ******* Texte introductif
{
Tronc commun 
} 
% ******* Contenu détaillé
{
Après une présentation générale des objectifs de l’UE et des modalités de recherche documentaire au cours d’un premier cours magistral, l’élaboration d’une fiche de projet individuel sera réalisée par l’étudiant. A l’issue du dépouillement de ces fiches, des groupes de travail (3-4 étudiants) seront organisés pour les séances de TD suivantes. Ces séances seront destinées à préparer une présentation sous forme de poster et de diaporama informatisés soutenu oralement lors de la dernière séance.Des séances spécifiques seront consacrées à la prise de parole en public, la capacité à s’exprimer avec/sans notes, aux techniques d’interview, d’entretien et à la rédaction écrite. La réalisation du diaporama informatique sera facilitée par le travail en deux séances de TP (initiation au logiciel, réalisation du document). La présentation finale sera construite autour d’une recherche de documents sur le métier ou l’activité choisie en début de semestre, et d’une rencontre avec un professionnel. Le travail personnel sera guidé par un enseignant lors des séances de TD. Le SUIO sera présenté et l’étudiant devra l’utiliser dans sa démarche personnelle. Un rapport individuel issu du travail produit pendant le module sera remis pour évaluation lors de la dernière séance de TD} 
% ******* Pré-requis
{} 
% ******* Objectifs
{\begin{itemize} 
  \ObjItem Recherche d’informations sur les métiers (sites internet appropriés, nécessité de mise à jour pour ce type d’information). 
\ObjItem Prise de parole lors d’interviews et face à un public. 
\ObjItem Réalisation d’une production écrite sous forme de poster.
\end{itemize} 
} 
% ******* Ressources pédagogiques
{} 
% ******* Bibliographie éventuelle
{Biblio}
 
\vfill


%==========================================================================================

%-------------------Semestre 2

%==========================================================================================
\module[codeApogee={SOL2BO01 SSL2BO01},
titre={Organisation et fonctionnement de la cellule eucaryote}, 
COURS={20}, 
TD={4}, 
TP={}, 
CTD={},
CTP={}, 
TOTAL={24}, 
SEMESTRE={Semestre 2}, 
COEFF={3}, 
ECTS={3}, 
MethodeEval={Ecrit}, 
ModalitesCCSemestreUn={RNE : CCI(2) 2h ; RSE : CT 2h}, 
ModalitesCCSemestreDeux={RNE et RSE : CT 2h}, 
CalculNFSessionUne={Ecrit 100\%},
CalculNFSessionDeux={Ecrit 100\%},
NoteEliminatoire={}, 
nomPremierResp={Jacques Pichon}, 
emailPremierResp={jacques.pichon@univ-orleans.fr}, 
nomSecondResp={Eric Duverger}, 
emailSecondResp={Eric.duverger@univ-orleans.fr}, 
langue={Français}, 
nbPrerequis={1}, 
descriptionCourte={true}, 
descriptionLongue={true}, 
objectifs={true}, 
ressources={true}, 
bibliographie={false}] 
% ******* Texte introductif
{
Ossature - tous parcours 
} 
% ******* Contenu détaillé
{
La membrane plasmique : structure et transport des molécules. Réticulum endoplasmique. Appareil de Golgi. Lysosome….  Le noyau : Introduction à l’expression des gènes et le cycle cellulaire.} 
% ******* Pré-requis
{\begin{itemize}
\item Molécules du vivant
\item Introduction à la biologie cellulaire
\end{itemize}} 
% ******* Objectifs
{\begin{itemize} 
  \ObjItem Initiation au fonctionnement de la cellule eucaryote.
\end{itemize} 
} 
% ******* Ressources pédagogiques
{} 
% ******* Bibliographie éventuelle
{Biblio}
 
\vfill
%==========================================================================================
\module[codeApogee={SOL2CH04 SSL2CH04},
titre={Chimie en solution 1}, 
COURS={8}, 
TD={7}, 
TP={9}, 
CTD={},
CTP={}, 
TOTAL={24}, 
SEMESTRE={Semestre 2}, 
COEFF={3}, 
ECTS={3}, 
MethodeEval={Ecrit}, 
ModalitesCCSemestreUn={RNE : CCI ((E(2-3) + TP(5)) 2h ; RSE : CT (E + TP) 2h}, 
ModalitesCCSemestreDeux={RNE et RSE : CT (E + TP) 3h}, 
CalculNFSessionUne={Ecrit 67 \% ; TP 33 \%}, 
CalculNFSessionDeux={Ecrit 67 \% ; TP 33 \%}, 
NoteEliminatoire={}, 
nomPremierResp={Valérie Bertagna}, 
emailPremierResp={valerie.bertagna@univ-orleans.fr}, 
nomSecondResp={}, 
emailSecondResp={}, 
langue={Français}, 
nbPrerequis={1}, 
descriptionCourte={true}, 
descriptionLongue={true}, 
objectifs={true}, 
ressources={true}, 
bibliographie={false}] 
% ******* Texte introductif
{
Ossature - tous parcours 
} 
% ******* Contenu détaillé
{
Equilibres, pH d'une solution aqueuse, Oxydo-réduction. Ces chapitres seront abordés avec le minimum de notions nécessaires
} 
% ******* Pré-requis
{Atomistique et thermodynamique} 
% ******* Objectifs
{\begin{itemize} 
  \ObjItem Aborder les notions de base de la chimie en solution, avec applications à la biologie : les équilibres chimiques, pH de solutions aqueuses, dosages acides-bases, précipitation, complexation.
\end{itemize} 
} 
% ******* Ressources pédagogiques
{} 
% ******* Bibliographie éventuelle
{Biblio}
 
\vfill
%==========================================================================================
\module[codeApogee={SOL2BO02 SSL2BO02},
titre={Principe de la génétique formelle}, 
COURS={12}, 
TD={16}, 
TP={8}, 
CTD={},
CTP={}, 
TOTAL={36}, 
SEMESTRE={Semestre 2}, 
COEFF={4}, 
ECTS={4}, 
MethodeEval={Ecrit}, 
ModalitesCCSemestreUn={RNE : CT (E) 2h / CC (TP) ; RSE : CT (E+TP) 3h}, 
ModalitesCCSemestreDeux={RNE et RSE : CT (E+TP) 1h}, 
CalculNFSessionUne={Ecrit 75 \% ; TP 25 \%}, 
CalculNFSessionDeux={Ecrit 75 \% ; TP 25 \%}, 
NoteEliminatoire={}, 
nomPremierResp={Catherine Mura}, 
emailPremierResp={catherine.mura@univ-orleans.fr}, 
nomSecondResp={}, 
emailSecondResp={}, 
langue={Français}, 
nbPrerequis={0}, 
descriptionCourte={true}, 
descriptionLongue={true}, 
objectifs={true}, 
ressources={true}, 
bibliographie={false}] 
% ******* Texte introductif
{
Ossature - tous parcours 
} 
% ******* Contenu détaillé
{
Le modèle mendélien mono et dihybridisme.
Notions fondamentales en génétique formelle : Le gène, Génotype/Phénotype, dominance récessivité, mutation, allèles, allèles multiples, allèles létaux, complémentation, pléiotropie, Interaction entre les gènes
Théorie Chromosomique de l’hérédité. Distance génétique, cartographie des loci.
Génétique humaine. L’analyse des pedigrees.
Génétique formelle des phages et des bactéries.
TP : Croisements Drosophiles, Polymorphisme de l’amylase} 
% ******* Pré-requis
{} 
% ******* Objectifs
{\begin{itemize} 
  \ObjItem Acquérir les bases de la génétique formelle permettant d’interpréter la ségrégation des caractères héréditaires.
\end{itemize} 
} 
% ******* Ressources pédagogiques
{} 
% ******* Bibliographie éventuelle
{Biblio}
 
\vfill
%==========================================================================================
\module[codeApogee={SOL2BO03 SSL2BO03},
titre={Bases de la génétique des populations}, 
COURS={6}, 
TD={6}, 
TP={}, 
CTD={},
CTP={}, 
TOTAL={12}, 
SEMESTRE={Semestre 2}, 
COEFF={2}, 
ECTS={2}, 
MethodeEval={Ecrit}, 
ModalitesCCSemestreUn={RNE et RSE : CT 2h}, 
ModalitesCCSemestreDeux={RNE et RSE : CT 2h}, 
CalculNFSessionUne={Ecrit 100\%},
CalculNFSessionDeux={Ecrit 100\%},
NoteEliminatoire={}, 
nomPremierResp={Catherine Mura}, 
emailPremierResp={catherine.mura@univ-orleans.fr}, 
nomSecondResp={}, 
emailSecondResp={}, 
langue={Français}, 
nbPrerequis={1}, 
descriptionCourte={true}, 
descriptionLongue={true}, 
objectifs={true}, 
ressources={true}, 
bibliographie={false}] 
% ******* Texte introductif
{
Ossature - tous parcours 
} 
% ******* Contenu détaillé
{
Notions de génétique des populations. Le modèle de Hardy Weinberg. La sélection. Les mutations. La migration. Le hasard et la dérive. 
TP : Modélisation dérive génétique
} 
% ******* Pré-requis
{Principe de la génétique formelle} 
% ******* Objectifs
{\begin{itemize} 
  \ObjItem Acquérir les bases de la génétique des populations
\end{itemize} 
} 
% ******* Ressources pédagogiques
{} 
% ******* Bibliographie éventuelle
{Biblio}
 
\vfill%==========================================================================================
\module[codeApogee={SOL2AG12 SSL2AG12},
titre={Anglais 2}, 
COURS={0}, 
TD={24}, 
TP={}, 
CTD={},
CTP={}, 
TOTAL={24}, 
SEMESTRE={Semestre 2}, 
COEFF={3}, 
ECTS={3}, 
MethodeEval={Ecrit et/ou oral}, 
ModalitesCCSemestreUn={RNE : CCI(4-7) 2h ; RSE : CT 1h}, 
ModalitesCCSemestreDeux={RNE et RSE : CT 1h30}, 
CalculNFSessionUne={CCI ou CT 100\%},
CalculNFSessionDeux={CT 100\%},
NoteEliminatoire={}, 
nomPremierResp={Murielle Pasquet}, 
emailPremierResp={murielle.pasquet@univ-orleans.fr}, 
nomSecondResp={}, 
emailSecondResp={}, 
langue={Français/Anglais}, 
nbPrerequis={1}, 
descriptionCourte={true}, 
descriptionLongue={true}, 
objectifs={true}, 
ressources={true}, 
bibliographie={false}] 
% ******* Texte introductif
{
Ossature - tous parcours 
} 
% ******* Contenu détaillé
{
Travail de compréhension et d’expression orale et écrite à partir de documents authentiques simples et / ou courts centrés sur le monde universitaire anglo-saxon.} 
% ******* Pré-requis
{Avoir suivi Anglais 1 ou environ  400 heures de formation équivalente.} 
% ******* Objectifs
{\begin{itemize} 
  \ObjItem Comprendre et s’exprimer de manière plus autonome dans des situations de séjour d’études universitaires en pays anglophone (niveau européen : B1).
\end{itemize} 
} 
% ******* Ressources pédagogiques
{} 
% ******* Bibliographie éventuelle
{Biblio}
 
\vfill%==========================================================================================
\module[codeApogee={SOL2BO04 SSL2BO04},
titre={Méthodologie, analyse et recherche documentaire}, 
COURS={10}, 
TD={14}, 
TP={}, 
CTD={},
CTP={}, 
TOTAL={24}, 
SEMESTRE={Semestre 2}, 
COEFF={3}, 
ECTS={3}, 
MethodeEval={Ecrit et oral}, 
ModalitesCCSemestreUn={RNE : CC(2)/CT 2h ; RSE : CT 1h}, 
ModalitesCCSemestreDeux={RNE et RSE : CT 1h}, 
CalculNFSessionUne={Ecrit 100\%},
CalculNFSessionDeux={Ecrit 100\%},
NoteEliminatoire={}, 
nomPremierResp={Jean-Pierre Gomez}, 
emailPremierResp={jean-pierre.gomez@univ-orleans.fr}, 
nomSecondResp={}, 
emailSecondResp={}, 
langue={Français}, 
nbPrerequis={0}, 
descriptionCourte={true}, 
descriptionLongue={true}, 
objectifs={true}, 
ressources={true}, 
bibliographie={false}] 
% ******* Texte introductif
{
Parcours général et enseignement PLURI
} 
% ******* Contenu détaillé
{
Des cours-conférences sont destinés à présenter des approches biochimiques, physiologiques et génétiques de pathologies humaines d’actualité. Par une présentation démarrant des fondements les plus simples accessibles par tous, et suivis des découvertes les plus récentes, l’étudiant comprendra en quoi la recherche peut apporter des solutions à plus ou moins brève échéance.
Les thématiques de ces cours pourront évoluer chaque année en fonction de l’actualité biomédicale : des sujets comme l’obésité, l’hypertension artérielle et ses conséquences cardio-vasculaires, la maladie d’Alzheimer, la mucoviscidose pourront par exemple être abordées.   
Les thématiques proposées en cours-conférences seront ensuite reprises en TD avec un travail sur documents scientifiques afin de compléter et conforter les notions acquises en cours} 
% ******* Pré-requis
{Atomistique et thermodynamique} 
% ******* Objectifs
{\begin{itemize} 
  \ObjItem Ce module se veut comme une première approche du monde la recherche dans le domaine de la biologie de la santé. Il doit permettre à l’étudiant d’appréhender de façon concrète les découvertes les plus récentes et d’acquérir la démarche et la méthodologie nécessaire dans la recherche d’information scientifique. 
\end{itemize} 
} 
% ******* Ressources pédagogiques
{} 
% ******* Bibliographie éventuelle
{Biblio}
 
\vfill%==========================================================================================
\module[codeApogee={SOL2CH06 SSL2CH06},
titre={Bases de la chimie organique}, 
COURS={12}, 
TD={12}, 
TP={}, 
CTD={},
CTP={}, 
TOTAL={24}, 
SEMESTRE={Semestre 2}, 
COEFF={3}, 
ECTS={3}, 
MethodeEval={Ecrit}, 
ModalitesCCSemestreUn={RNE : CCI (2) 3h ; RSE : CT 2h}, 
ModalitesCCSemestreDeux={RNE et RSE : CT 2h}, 
CalculNFSessionUne={CCI ou CT 100\%},
CalculNFSessionDeux={Ecrit 100\%},
NoteEliminatoire={}, 
nomPremierResp={Arnaud Tatibouet}, 
emailPremierResp={arnaud.tatibouet@univ-orleans.fr}, 
nomSecondResp={}, 
emailSecondResp={}, 
langue={Français}, 
nbPrerequis={1}, 
descriptionCourte={true}, 
descriptionLongue={true}, 
objectifs={true}, 
ressources={true}, 
bibliographie={false}] 
% ******* Texte introductif
{
Parcours général
} 
% ******* Contenu détaillé
{
Initiation à la chimie organique : Etude des grandes réactivités (Substitution, Elimination, Addition) associées aux fonctions essentielles du monde du vivant, les insaturations (alcène et alcynes) les fonctions alcool, carbonyle, pour une meilleure compréhension des phénomènes biochimiques. Présentation des réactivités, mécanismes et des intermédiaires réactionnels.
} 
% ******* Pré-requis
{\begin{itemize}
\item Atomistique et Thermodynamique
\item Chimie en Solution 1
\end{itemize}
} 
% ******* Objectifs
{\begin{itemize} 
  \ObjItem Connaissances fondamentales de la chimie bio-organique réactionnelle.
\end{itemize} 
} 
% ******* Ressources pédagogiques
{} 
% ******* Bibliographie éventuelle
{Biblio}
 
\vfill
%==========================================================================================
\module[codeApogee={SOL2BO05 SSL2BO05},
titre={Ecologie générale : environnement et fonctionnement de la biosphère}, 
COURS={12}, 
TD={6}, 
TP={6}, 
CTD={},
CTP={}, 
TOTAL={24}, 
SEMESTRE={Semestre 2}, 
COEFF={3}, 
ECTS={3}, 
MethodeEval={Ecrit}, 
ModalitesCCSemestreUn={RNE et RSE : CT(CM+TD) 1h30 / CC(TP)}, 
ModalitesCCSemestreDeux={RNE et RSE : CT(CM+TD+TP) 2h}, 
CalculNFSessionUne={CT 67 \% ; CC 33 \%}, 
CalculNFSessionDeux={CM 33 \% ; TD 33 \% ; TP 33 \%}, 
NoteEliminatoire={}, 
nomPremierResp={François Lieutier}, 
emailPremierResp={francois.lieutier@univ-orleans.fr}, 
nomSecondResp={}, 
emailSecondResp={}, 
langue={Français}, 
nbPrerequis={0}, 
descriptionCourte={true}, 
descriptionLongue={true}, 
objectifs={true}, 
ressources={true}, 
bibliographie={false}] 
% ******* Texte introductif
{
Parcours général et enseignement
} 
% ******* Contenu détaillé
{
Quelques problèmes environnementaux et notions d'interdépendance; organisation générale de la biosphère (bilan énergétique, circulation atmosphérique, organisation spatiale); organisation fonctionnelle (chaînes et réseaux trophiques, perturbations d'origine humaine); circulation de la matière et cycles biogéochimiques (eau, carbone, azote, oxygène), et relations avec les activités humaines; flux d'énergie (bilans énergétiques, productivité, efficacité des écosystèmes naturels et anthropisés); organisation des biocénoses, biogéographie et fonctionnement des grands biomes terrestres; notions de biodiversité.} 
% ******* Pré-requis
{} 
% ******* Objectifs
{\begin{itemize} 
  \ObjItem Montrer comment la connaissance du fonctionnement de la biosphère, en liaison avec les activités humaines, peut permettre de comprendre les problèmes environnementaux actuels, d'envisager des solutions et d'appréhender les problèmes futurs.
\end{itemize} 
} 
% ******* Ressources pédagogiques
{} 
% ******* Bibliographie éventuelle
{Biblio}
 
\vfill
%==========================================================================================
\module[codeApogee={SOL2BO06 SSL2BO06},
titre={Parasitisme et grandes endémies}, 
COURS={18}, 
TD={}, 
TP={6}, 
CTD={},
CTP={}, 
TOTAL={24}, 
SEMESTRE={Semestre 2}, 
COEFF={3}, 
ECTS={3}, 
MethodeEval={Ecrit}, 
ModalitesCCSemestreUn={RNE et RSE : CT 1h}, 
ModalitesCCSemestreDeux={RNE et RSE : CT 1h}, 
CalculNFSessionUne={Ecrit 100\%},
CalculNFSessionDeux={Ecrit 100\%},
NoteEliminatoire={}, 
nomPremierResp={Valérie Altemayer}, 
emailPremierResp={valerie.altemayer@univ-orleans.fr}, 
nomSecondResp={}, 
emailSecondResp={}, 
langue={Français}, 
nbPrerequis={0}, 
descriptionCourte={true}, 
descriptionLongue={true}, 
objectifs={true}, 
ressources={true}, 
bibliographie={false}] 
% ******* Texte introductif
{
Parcours général et enseignement PLURI
} 
% ******* Contenu détaillé
{
Les grandes parasitoses mondiales. Les parasites animaux d’intérêts médicaux Morphologie et cycle parasitaire. Répartition géographiques et coûts des principales parasitoses humaines. Traitements.  Notion d’épidémiologie.} 
% ******* Pré-requis
{} 
% ******* Objectifs
{\begin{itemize} 
  \ObjItem Connaissance des principaux parasites de l’Homme et des animaux domestiques et notion d’écologie parasitaire.
\end{itemize} 
} 
% ******* Ressources pédagogiques
{} 
% ******* Bibliographie éventuelle
{Biblio}
 
\vfill
%==========================================================================================
\module[codeApogee={SOL2BO07 SSL2BO07},
titre={Algues et mycètes}, 
COURS={10}, 
TD={4}, 
TP={10}, 
CTD={},
CTP={}, 
TOTAL={24}, 
SEMESTRE={Semestre 2}, 
COEFF={3}, 
ECTS={3}, 
MethodeEval={Ecrit}, 
ModalitesCCSemestreUn={RNE et RSE : CT 1h}, 
ModalitesCCSemestreDeux={RNE et RSE : CT 1h}, 
CalculNFSessionUne={Ecrit 100\%},
CalculNFSessionDeux={Ecrit 100\%},
NoteEliminatoire={}, 
nomPremierResp={Christiane Depierreux}, 
emailPremierResp={christiane.depierreux@univ-orleans.fr}, 
nomSecondResp={}, 
emailSecondResp={}, 
langue={Français}, 
nbPrerequis={0}, 
descriptionCourte={true}, 
descriptionLongue={true}, 
objectifs={true}, 
ressources={true}, 
bibliographie={false}] 
% ******* Texte introductif
{
Parcours général et enseignement PLURI
} 
% ******* Contenu détaillé
{
Biologie et reproduction des algues et des champignons, initiation à la reconnaissance des champignons. Intérêts économiques, pharmaceutiques et biotechnologiques des algues et des champignons } 
% ******* Pré-requis
{} 
% ******* Objectifs
{\begin{itemize} 
  \ObjItem Intérêts économiques et biologie des algues et des champignons.
\end{itemize} 
} 
% ******* Ressources pédagogiques
{} 
% ******* Bibliographie éventuelle
{Biblio}
 
\vfill
%==========================================================================================
\module[codeApogee={SOL2CH14 SSL2CH14},
titre={Chimie en solution 2}, 
COURS={10}, 
TD={11}, 
TP={3}, 
CTD={},
CTP={}, 
TOTAL={24}, 
SEMESTRE={Semestre 2}, 
COEFF={3}, 
ECTS={3}, 
MethodeEval={Ecrit}, 
ModalitesCCSemestreUn={RNE : CCI (E(2)+TP(5)) 4h ; RSE : CT (E+TP) 3h}, 
ModalitesCCSemestreDeux={RNE et RSE : CT (E+TP) 3h}, 
CalculNFSessionUne={Ecrit 67 \% ; TP 33 \%}, 
CalculNFSessionDeux={Ecrit 67 \% ; TP 33 \%}, 
NoteEliminatoire={}, 
nomPremierResp={Valérie Bertagna}, 
emailPremierResp={valerie.bertagna@univ-orleans.fr}, 
nomSecondResp={}, 
emailSecondResp={}, 
langue={Français}, 
nbPrerequis={1}, 
descriptionCourte={true}, 
descriptionLongue={true}, 
objectifs={true}, 
ressources={true}, 
bibliographie={false}] 
% ******* Texte introductif
{
Parcours général
} 
% ******* Contenu détaillé
{
Approfondissement des notions de pH, et d'oxydo-réduction (notions non abordées en chimie en solution 1)), conductivité solubilité  et cinétique. Applications à la biologie.} 
% ******* Pré-requis
{Chimie en Solution 1
} 
% ******* Objectifs
{\begin{itemize} 
  \ObjItem Aborder les notions de base de la chimie en solution, avec applications à la biologie : équilibres d’oxydo-réduction, cinétique chimique.
\end{itemize} 
} 
% ******* Ressources pédagogiques
{} 
% ******* Bibliographie éventuelle
{Biblio}
 
\vfill
%==========================================================================================
\module[codeApogee={???},
titre={UEL IUFM - Histoire Géo ou EPS à l'école}, 
COURS={20}, 
TD={}, 
TP={}, 
CTD={},
CTP={}, 
TOTAL={20}, 
SEMESTRE={Semestre 2}, 
COEFF={3}, 
ECTS={3}, 
MethodeEval={Ecrit}, 
ModalitesCCSemestreUn={RNE et RSE : CT 1h30}, 
ModalitesCCSemestreDeux={RNE et RSE : CT 1h30}, 
CalculNFSessionUne={Ecrit 100\%},
CalculNFSessionDeux={Ecrit 100\%},
NoteEliminatoire={}, 
nomPremierResp={Pierre-Olivier Hochard}, 
emailPremierResp={pierre-olivier.hochard@univ-orleans.fr}, 
nomSecondResp={Jocelyne Jamet}, 
emailSecondResp={jocelyne.jamet@univ-orleans.fr}, 
langue={Français}, 
nbPrerequis={0}, 
descriptionCourte={true}, 
descriptionLongue={true}, 
objectifs={true}, 
ressources={true}, 
bibliographie={false}] 
% ******* Texte introductif
{
Parcours enseignement PLURI
} 
% ******* Contenu détaillé
{???} 
% ******* Pré-requis
{} 
% ******* Objectifs
{\begin{itemize} 
  \ObjItem ???
\end{itemize} 
} 
% ******* Ressources pédagogiques
{} 
% ******* Bibliographie éventuelle
{Biblio}
 
\vfill
%==========================================================================================
\module[codeApogee={SOL2ST01 SSL2ST01},
titre={Cours Paléoenvironnement, stratigraphie et paléontologie}, 
COURS={18}, 
TD={6}, 
TP={24}, 
CTD={},
CTP={}, 
TOTAL={48}, 
SEMESTRE={Semestre 2}, 
COEFF={6}, 
ECTS={6}, 
MethodeEval={Ecrit}, 
ModalitesCCSemestreUn={RNE : CCI (E(2)+TP(2)) 5h ; RSE : CT 3h}, 
ModalitesCCSemestreDeux={RNE et RSE : CT (E+TP) 3h}, 
CalculNFSessionUne={Ecrit 67 \% ; TP 33 \%}, 
CalculNFSessionDeux={Ecrit 67 \% ; TP 33 \%}, 
NoteEliminatoire={}, 
nomPremierResp={Christian Di Giovanni}, 
emailPremierResp={christian.di-giovanni@univ-orleans.fr}, 
nomSecondResp={}, 
emailSecondResp={}, 
langue={Français}, 
nbPrerequis={0}, 
descriptionCourte={true}, 
descriptionLongue={true}, 
objectifs={true}, 
ressources={true}, 
bibliographie={false}] 
% ******* Texte introductif
{
Parcours enseignement BGST
} 
% ******* Contenu détaillé
{
La caractérisation du paysage (relief terrestre) - L'enregistrement du temps (grands principes de stratigraphie, lithostratigraphie, biostratigraphie, chimiostratigraphie) - Histoire des climats et environnements (paléoclimats, paléoenvironnements, marqueurs paléoclimatiques et paléoenvironnementaux) - Histoire de la vie (grandes étapes de la vie et crises biologiques)} 
% ******* Pré-requis
{} 
% ******* Objectifs
{\begin{itemize} 
  \ObjItem Notion de stratigraphie, échelle biostratigraphique, variabilité climatique naturelle, histoire de la vie.
\end{itemize} 
} 
% ******* Ressources pédagogiques
{} 
% ******* Bibliographie éventuelle
{Biblio}
 
\vfill
%==========================================================================================
\module[codeApogee={SOL2ST02 SSL2ST02},
titre={Cours Minéralogie}, 
COURS={18}, 
TD={6}, 
TP={24}, 
CTD={},
CTP={}, 
TOTAL={48}, 
SEMESTRE={Semestre 2}, 
COEFF={6}, 
ECTS={6}, 
MethodeEval={Ecrit}, 
ModalitesCCSemestreUn={RNE et RSE : CCI (E(2)+TP(2)) 8h}, 
ModalitesCCSemestreDeux={RNE et RSE : CT (E+TP) 4h}, 
CalculNFSessionUne={Ecrit 67 \% ; TP 33 \%}, 
CalculNFSessionDeux={Ecrit 67 \% ; TP 33 \%}, 
NoteEliminatoire={}, 
nomPremierResp={Nicole Lebreton}, 
emailPremierResp={nicole.lebreton@univ-orleans.fr}, 
nomSecondResp={}, 
emailSecondResp={}, 
langue={Français}, 
nbPrerequis={1}, 
descriptionCourte={true}, 
descriptionLongue={true}, 
objectifs={true}, 
ressources={true}, 
bibliographie={false}] 
% ******* Texte introductif
{
Parcours enseignement BGST
} 
% ******* Contenu détaillé
{
\begin{itemize}
\item[CM :] Définition d’un minéral – Bases élémentaires de cristallographie et cristallochimie – Notions de symétrie cristalline – Présentation des critères de détermination macroscopique des minéraux – Notions d’optique cristalline pour la détermination des minéraux à l’aide du microscope polarisant – La classification des minéraux et leur répartition dans le globe – Présentation des grandes familles de minéraux (cristallochimie, gisements, applications industrielles).
\item[TD :] La symétrie des mailles élémentaires – Calcul de formules structurales.
\end{itemize}
}
% ******* Pré-requis
{Connaissances de base en chimie.} 
% ******* Objectifs
{\begin{itemize} 
  \ObjItem Connaissances du monde minéral dans sa diversité et ses principes – Apprentissage des méthodes d’identification des minéraux.
\end{itemize} 
} 
% ******* Ressources pédagogiques
{} 
% ******* Bibliographie éventuelle
{Biblio}
 
\vfill
\end{document}
