\documentclass[10pt, a5paper]{report}

\usepackage[T1]{fontenc}%
\usepackage[utf8]{inputenc}% encodage utf8
\usepackage[francais]{babel}% texte français
\usepackage[top=10mm, bottom=20mm, foot=5mm, left=7mm, offset=7mm, textwidth=350pt]{geometry}
\usepackage{url}
\usepackage{init-preambule}
\usepackage[final]{pdfpages}

% Définir les couleur du document
%---------------------- % % % Personnalisation des couleurs % % % ----------- Vert Licence --------
\definecolor{couleurFonce}{RGB}{18,92,40} % Couleur du Code APOGEE
\definecolor{couleurClaire}{RGB}{28,161,68} % Couleur du fond de la bande
\definecolor{couleurTexte}{RGB}{255,255,255} % Couleur du texte de la bande
%------------------------------------------------------------------------------------------


\begin{document}
\begin{spacing}{1.5}

\chapter*{Licence Biologie-Biochimie}

\footnotesize
\section*{Objectifs}

	\begin{wrapfigure}{r}{0.35\textwidth}
          \vspace{-20pt}
            \begin{center}
                \begin{tikzpicture}
                    \node [rectangle, draw=couleurFonce, thick, drop shadow, fill=couleurBox, inner sep=10pt, inner ysep=10pt] (box) {
                    \begin{minipage}{0.35\textwidth}%
                        \begin{spacing}{1}
                        \begin{tabular}[t]{@{}m{10mm}@{~~}m{30mm}@{}}
                          % \includegraphics[scale=0.6]{img/photos/avatar.png} & \normalsize{\textbf{Valérie}\newline \textbf{ALTEMAYER}}\newline \footnotesize{}\\
		\multicolumn{2}{l}{\textbf{Valérie Altemayer}}\\
                \multicolumn{2}{c}{{\scriptsize \textit{valerie.altemayer@univ-orleans.fr}}} \\
                          \multicolumn{2}{c}{\includegraphics[scale=0.7]{img/telephone.png}{\scriptsize (Secrétariat)+33238417097}}
                        \end{tabular}
                        \end{spacing}
                    \end{minipage}
                    };
                    \node[fancytitle, right=5pt, rounded corners, inner xsep=10pt] at (box.north west) {\normalsize{Directrice des études}};
                \end{tikzpicture}
            \end{center}
            \vspace{-20pt}
      \end{wrapfigure}





\letterine{D}ans le cadre de l’harmonisation européenne des cursus universitaires (réforme LMD Licence-Master-Doctorat), les mentions de  Licence sont constituées de 6 semestres.

Le Tronc SVT – Sciences de la Vie et de la Terre, est organisé en 2 semestres dont le premier, largement pluridisciplinaire, est intégralement commun à trois mentions de licence pour permettre aux étudiants de se déterminer au second semestre, après un aperçu des différents aspects des Sciences de la Vie et de la Terre.

Les 2 mentions de licence  issues du tronc SVT sont :
-    la licence Mention  Biologie - Biochimie
-    la licence Mention  Terre et Environnement

La licence mention « Biologie – Biochimie »  a pour  objectif de :
     - former un public étudiant à toutes les disciplines de la Biologie : biologie générale animale et végétale, biologie cellulaire et moléculaire, génétique formelle et des populations, physiologie animale et physiologie végétale, biotechnologies, écologie, biologie des organismes et des populations. 
     -lui permettre ainsi d’acquérir des connaissances fondamentales et pratiques de niveau Bac+3 dans tous les secteurs techniques et scientifiques des Sciences de la Vie, ces connaissances étant indispensables à une poursuite d’études vers les Masters Recherches et Professionnels ou à l’intégration dans le monde professionnel dans ce domaine.
    - préparer au concours du CAPES Sciences de la Vie et de la Terre.
    - mettre à disposition les savoirs et les connaissances existants à l’Université d’Orléans dans le domaine des Sciences biologiques.

\section*{Organisation des enseignements}
\subsection*{Semestre 1}

\letterine{L}a première année de la licence Biologie-Biochimie est divisée en deux semestres. Le Semestre 1 est constitué d'un enseignement commun à tous les étudiants. 

% % % % % % % % % % % % % % % % % % % % % % % % % % % % % % % % % % % % % % % % % % % % % % % % % % % % % % % 

%---------------------- % % % Personnalisation des couleurs % % % ----------- ROUGE --------
%---------------------- % % % Personnalisation des couleurs % % % ----------- Vert Licence --------
\definecolor{couleurFonce}{RGB}{18,92,40} % Couleur du Code APOGEE
\definecolor{couleurClaire}{RGB}{28,161,68} % Couleur du fond de la bande
\definecolor{couleurTexte}{RGB}{255,255,255} % Couleur du texte de la bande
%------------------------------------------------------------------------------------------

%------------------------------------------------------------------------------------------

\arrayrulecolor{couleurFonce}% Couleur des lignes séparatrices du tableau
\renewcommand{\arraystretch}{1.5}% Coeff appliqué à la hauteur des cellules
%\rowcolors[\hline]{ligneDébut}{couleurPaire}{couleurImpaire}% Alternance de couleur (need package xcolor)
%\begin{tabular}{c|m{6cm}|cm{1cm}|cm{1cm}|cm{1cm}|cm{1cm}|}
\begin{tabular}{c|m{4.5cm}|cm{0.75cm}|cm{0.75cm}|cm{0.75cm}|cm{0.75cm}|cm{0.75cm}|}
%\begin{tabular}{c|m|cm|cm|cm|cm|cm|}
\cline{2-7}

&
\cellcolor{couleurFonce} \color{white}\bfseries Intitul\'e & \cellcolor{couleurFonce} \color{white}\bfseries ECTS & \cellcolor{couleurFonce} \color{white}\bfseries CM & \cellcolor{couleurFonce} \color{white}\bfseries TD & \cellcolor{couleurFonce} \color{white}\bfseries CTP & \cellcolor{couleurFonce} \color{white}\bfseries TP\\ \cline{2-7}

\hline \multirow{8}{*}{\rotatebox{90}{\color{couleurFonce}\bfseries Ossature - 30 ECTS}}
 & \color{black} \mbox{Panorama} \mbox{du} \mbox{monde} \mbox{animal} \mbox{et} \mbox{végétal}  & \color{black} 5 & \color{black} 30 & \color{black} 3 & & \color{black} 15\\ \cline{2-7}
 & \cellcolor{couleurClaire} \color{couleurTexte} \mbox{Atomistique et thermodynamique}  & \cellcolor{couleurClaire} \color{couleurTexte} 5 & \cellcolor{couleurClaire} \color{couleurTexte} 17 & \cellcolor{couleurClaire} \color{couleurTexte} 16 & \cellcolor{couleurClaire} & \cellcolor{couleurClaire} \color{couleurTexte} 15
\\ \cline{2-7}
 & \color{black} \mbox{Molécules du vivant} & \color{black} 5 & \color{black} 28 & \color{black} 10 & & \color{black} 10
\\ \cline{2-7}
 & \cellcolor{couleurClaire} \color{couleurTexte} \mbox{Anglais 1}  & \cellcolor{couleurClaire} \color{couleurTexte} 3 & \cellcolor{couleurClaire} \color{couleurTexte}  & \cellcolor{couleurClaire} \color{couleurTexte} 24 & \cellcolor{couleurClaire} \color{couleurTexte} & \cellcolor{couleurClaire}  
\\ \cline{2-7}
 & \color{black} \mbox{Analyse des données en biosciences}  & \color{black} 2 & \color{black} 6 & \color{black} 6 & \color{black} &  
\\ \cline{2-7}
 & \cellcolor{couleurClaire} \color{couleurTexte} \mbox{Introduction à la biologie cellulaire} & \cellcolor{couleurClaire} \color{couleurTexte} 4 & \cellcolor{couleurClaire} \color{couleurTexte} 18 & \cellcolor{couleurClaire} \color{couleurTexte} 6 & \cellcolor{couleurClaire} & \cellcolor{couleurClaire} \color{couleurTexte} 12 
\\ \cline{2-7}
 & \color{black} \mbox{Préparation au C2i} & \color{black} 3 & \color{black}  & \color{black}  & \color{black} 24 & 
\\ \cline{2-7}
 & \cellcolor{couleurClaire} \color{couleurTexte} \mbox{Projet} \mbox{personnel} \mbox{et} \mbox{professionnel} \mbox{et} \mbox{techniques} \mbox{de} \mbox{communications} & \cellcolor{couleurClaire} \color{couleurTexte} 3 & \cellcolor{couleurClaire} \color{couleurTexte} 2 & \cellcolor{couleurClaire} \color{couleurTexte} 14 & \cellcolor{couleurClaire} & \cellcolor{couleurClaire} \color{couleurTexte} \\ \cline{1-7} 
\end{tabular}

% % % % % % % % % % % % % % % % % % % % % % % % % % % % % % % % % % % % % % % % % % % % % % % % % % % % % % % 


\newpage
\subsection*{Semestre 2}

\letterine{L}e second semestre est divisé en deux parcours (général et enseignement). 

\textbf{Parcours Général :}
\newline

\input{Tableau_S2_gen}

\newpage

\textbf{Parcours Enseignement :}
\newline
% % % % % % % % % % % % % % % % % % % % % % % % % % % % % % % % % % % % % % % % % % % % % % % % % % % % % % % 

%---------------------- % % % Personnalisation des couleurs % % % ----------- ROUGE --------
%---------------------- % % % Personnalisation des couleurs % % % ----------- Vert Licence --------
\definecolor{couleurFonce}{RGB}{18,92,40} % Couleur du Code APOGEE
\definecolor{couleurClaire}{RGB}{28,161,68} % Couleur du fond de la bande
\definecolor{couleurTexte}{RGB}{255,255,255} % Couleur du texte de la bande
%------------------------------------------------------------------------------------------

%------------------------------------------------------------------------------------------

\arrayrulecolor{couleurFonce}% Couleur des lignes séparatrices du tableau
\renewcommand{\arraystretch}{1.5}% Coeff appliqué à la hauteur des cellules
%\rowcolors[\hline]{ligneDébut}{couleurPaire}{couleurImpaire}% Alternance de couleur (need package xcolor)
\begin{tabular}{c|m{6cm}|cm{0.75cm}|cm{0.75cm}|cm{0.75cm}|cm{0.75cm}|}
%\begin{tabular}{c|m{6cm}|cm{0.8cm}|cm{0.8cm}|cm{0.8cm}|cm{0.8cm}|}
%\begin{tabular}{c|m|cm|cm|cm|cm|cm|}
\cline{2-6}

&
\cellcolor{couleurFonce} \color{white}\bfseries Intitul\'e & \cellcolor{couleurFonce} \color{white}\bfseries ECTS & \cellcolor{couleurFonce} \color{white}\bfseries CM & \cellcolor{couleurFonce} \color{white}\bfseries TD & \cellcolor{couleurFonce} \color{white}\bfseries TP \\ \cline{2-6}
%----
\cline{1-6} \multirow{6}{*}{\rotatebox{90}{\color{couleurFonce}\bfseries Ossature}}
\multirow{6}{*}{\rotatebox{90}{\color{couleurFonce}\bfseries 18 ECTS}}
 & \color{black} \mbox{Organisation} \mbox{et} \mbox{fonctionnement} \mbox{de} \mbox{la} \mbox{cellule} \mbox{eucaryote} & \color{black} 3 & \color{black} 20 & \color{black} 4 & \\ \cline{2-6}
 & \cellcolor{couleurClaire} \color{couleurTexte} \mbox{Chimie en solution 1}  & \cellcolor{couleurClaire} \color{couleurTexte} 3 & \cellcolor{couleurClaire} \color{couleurTexte} 8 & \cellcolor{couleurClaire} \color{couleurTexte} 7 & \cellcolor{couleurClaire} \color{couleurTexte} 9
\\ \cline{2-6}
 & \color{black} \mbox{Principe de la génétique formelle} & \color{black} 4 & \color{black} 12 & \color{black} 16 & \color{black} 8
\\ \cline{2-6}
 & \cellcolor{couleurClaire} \color{couleurTexte} \mbox{Bases} \mbox{de} \mbox{la} \mbox{génétique} \mbox{des} \mbox{populations} & \cellcolor{couleurClaire} \color{couleurTexte} 2 & \cellcolor{couleurClaire} \color{couleurTexte} 6 & \cellcolor{couleurClaire} \color{couleurTexte} 6 & \cellcolor{couleurClaire}  
\\ \cline{2-6}
 & \color{black} \mbox{Anglais 1}  & \color{black} 3 & \color{black} & \color{black} 24 & \color{black} &  
\\ 
& \cellcolor{couleurClaire} \color{couleurTexte} \mbox{Ecologie générale : environnement et} \mbox{fonctionnement de la biosphère} & \cellcolor{couleurClaire} \color{couleurTexte} 3 & \cellcolor{couleurClaire} \color{couleurTexte} 12 & \cellcolor{couleurClaire} \color{couleurTexte} 6 &\cellcolor{couleurClaire} \color{couleurTexte} 6 \\ 
\cline{1-6} 
%------
\multirow{4}{*}{\rotatebox{90}{\color{couleurFonce}\bfseries Ossature}}
\multirow{4}{*}{\rotatebox{90}{\color{couleurFonce}\bfseries Pluri}}
\multirow{4}{*}{\rotatebox{90}{\color{couleurFonce}\bfseries 12 ECTS}}
& \color{black} \mbox{UEL IUFM} & \color{black} 3 & \color{black} 20 & \color{black} & \color{black} & 
\\ \cline{2-6}
& \cellcolor{couleurClaire} \color{couleurTexte} Méthodologie, analyse et recherche documentaire & \cellcolor{couleurClaire} \color{couleurTexte} 3 & \cellcolor{couleurClaire} \color{couleurTexte} 10 & \cellcolor{couleurClaire} \color{couleurTexte} 14 &\cellcolor{couleurClaire} \color{couleurTexte} \\ 
\cline{2-6} 
& \color{black} \mbox{Parasitime et grandes endémies} & \color{black} 3 & \color{black} 18 & \color{black} & \color{black} 6 & 
\\ 
\cline{2-6}
 & \cellcolor{couleurClaire} \color{couleurTexte} \mbox{Algues et mycètes} & \cellcolor{couleurClaire} \color{couleurTexte} 3 & \cellcolor{couleurClaire} \color{couleurTexte} 10 & \cellcolor{couleurClaire} \color{couleurTexte} 4 &\cellcolor{couleurClaire} \color{couleurTexte} 10 \\ 
\cline{1-6} 
%------
\multirow{2}{*}{\rotatebox{90}{\color{couleurFonce}\bfseries Ossature}}
\multirow{2}{*}{\rotatebox{90}{\color{couleurFonce}\bfseries BGST}}
 \multirow{2}{*}{\rotatebox{90}{\color{couleurFonce}\bfseries 12 ECTS}}
& \color{black} \mbox{Paléontologie} & \color{black} 6 & \color{black} 18 & \color{black} 6 & \color{black} 24 & 
\\ 
\cline{2-6}
 & \cellcolor{couleurClaire} \color{couleurTexte} \mbox{Minéralogie} & \cellcolor{couleurClaire} \color{couleurTexte} 6 & \cellcolor{couleurClaire} \color{couleurTexte} 18 & \cellcolor{couleurClaire} \color{couleurTexte} 6 &\cellcolor{couleurClaire} \color{couleurTexte} 24 \\ 
\cline{1-6} 
\end{tabular}

% % % % % % % % % % % % % % % % % % % % % % % % % % % % % % % % % % % % % % % % % % % % % % % % % % % % % % % 


\section*{Détail des enseignements}



\end{spacing}

\end{document}
