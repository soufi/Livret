\documentclass[10pt, a5paper]{report}

\usepackage[T1]{fontenc}%
\usepackage[utf8]{inputenc}% encodage utf8
\usepackage[francais]{babel}% texte français
\usepackage[final]{pdfpages}
\usepackage{modules-livret}% style du livret
\usepackage{url}
%\usepackage{init-preambule}
\pagestyle{empty}

% % % % % % % % % % % % % % % % % % % % % % % % % % % % % % % % % % % % % % % % % % % % % % % % % % % % % % % 
\begin{document}

%---------------------- % % % Personnalisation des couleurs % % % ----------- Vert Licence --------
\definecolor{couleurFonce}{RGB}{18,92,40} % Couleur du Code APOGEE
\definecolor{couleurClaire}{RGB}{28,161,68} % Couleur du fond de la bande
\definecolor{couleurTexte}{RGB}{255,255,255} % Couleur du texte de la bande
%------------------------------------------------------------------------------------------


%==========================================================================================
% Semestre 5
%==========================================================================================
\module[codeApogee={SOL5AG35},
titre={Anglais 5}, 
COURS={}, 
TD={24}, 
TP={}, 
CTD={},
CTP={}, 
TOTAL={24}, 
SEMESTRE={Semestre 5}, 
COEFF={3}, 
ECTS={3}, 
MethodeEval={Ecrit},
ModalitesCCSemestreUn={RNE : CC 3h ; RSE : CT 1h30},
ModalitesCCSemestreDeux={RNE et RSE : CT 1h30},
CalculNFSessionUne={100\%},
CalculNFSessionDeux={100\%},
NoteEliminatoire={}, 
nomPremierResp={Hervé Perreau}, 
emailPremierResp={herve.perreau@univ-orleans.fr}, 
nomSecondResp={}, 
emailSecondResp={}, 
langue={Français/Anglais}, 
nbPrerequis={1}, 
descriptionCourte={true}, 
descriptionLongue={true}, 
objectifs={true}, 
ressources={true}, 
bibliographie={false}] 
% ******* Texte introductif
{
Parcours BOPE / BMC/ BBV / BGST / PLURI
} 
% ******* Contenu détaillé
{
Travail de compréhension et d’expression orale à partir de documents authentiques longs et/ou complexes, portant sur des innovations technologiques, des découvertes ou avancées scientifiques.
} 
% ******* Pré-requis
{Avoir suivi Anglais 3 + 4 ou environ 500 heures de formation équivalente.
} 
% ******* Objectifs
{\begin{itemize} 
  \ObjItem Comprendre l’information exprimée dans des messages complexes sur le domaine des Sciences et Technologies et s’exprimer sur ce même domaine à l’écrit dans un registre de langue approprié.
\end{itemize} 
} 
% ******* Ressources pédagogiques
{} 
% ******* Bibliographie éventuelle
{Biblio}
 
\vfill
\module[codeApogee={SOL5BO04},
titre={Lois de probabilités et estimations de paramètres usuels}, 
COURS={}, 
TD={}, 
TP={}, 
CTD={24},
CTP={}, 
TOTAL={24}, 
SEMESTRE={Semestre 5}, 
COEFF={3}, 
ECTS={3}, 
MethodeEval={Ecrit/Oral},
ModalitesCCSemestreUn={RNE et RSE : CT (Ecrit) 1h},
ModalitesCCSemestreDeux={RNE et RSE : CT (Oral) 15 min},
CalculNFSessionUne={100\% CT},
CalculNFSessionDeux={100\% CT},
NoteEliminatoire={}, 
nomPremierResp={Franck Brignolas}, 
emailPremierResp={franck.brignolas@univ-orleans.fr}, 
nomSecondResp={}, 
emailSecondResp={}, 
langue={Français}, 
nbPrerequis={0}, 
descriptionCourte={true}, 
descriptionLongue={true}, 
objectifs={true}, 
ressources={true}, 
bibliographie={false}] 
% ******* Texte introductif
{
Parcours BOPE / BMC / BBV
} 
% ******* Contenu détaillé
{
Notion de variable aléatoire (VA) ; VA qualitatives, VA quantitatives discrètes et continues ; principales lois de probabilité et leur utilisation en biologie ; estimation de paramètres et intervalles de confiance.
} 
% ******* Pré-requis
{
} 
% ******* Objectifs
{\begin{itemize} 
  \ObjItem Acquisition de connaissances de bases en statistique.
\end{itemize} 
} 
% ******* Ressources pédagogiques
{} 
% ******* Bibliographie éventuelle
{Biblio}
 
\vfill

\module[codeApogee={SOL5BO10},
titre={Physiologie des grandes fonctions}, 
COURS={20}, 
TD={2}, 
TP={14}, 
CTD={},
CTP={}, 
TOTAL={36}, 
SEMESTRE={Semestre 5}, 
COEFF={4}, 
ECTS={4}, 
MethodeEval={Ecrit},
ModalitesCCSemestreUn={RNE : E(CT) 2h / TP(CC+CT) 1h ; RSE : E(CT) 2h / TP(CT) 1h},
ModalitesCCSemestreDeux={RNE et RSE : E(CT) 2h / TP(CT) 1h},
CalculNFSessionUne={50\% E + 50 \% TP},
CalculNFSessionDeux={50\% E + 50 \% TP},
NoteEliminatoire={}, 
nomPremierResp={Olivier Richard}, 
emailPremierResp={olivier.richard@univ-orleans.fr}, 
nomSecondResp={}, 
emailSecondResp={}, 
langue={Français}, 
nbPrerequis={1}, 
descriptionCourte={true}, 
descriptionLongue={true}, 
objectifs={true}, 
ressources={true}, 
bibliographie={false}] 
% ******* Texte introductif
{
} 
% ******* Contenu détaillé
{
Ce module aborde la plupart des fonctions de l’organisme animal par une approche physiologique et pathologique (notions). Bases fondamentales de la régulation du milieu intérieur. Physiologies cardiaque, vasculaire, respiratoire, digestive et rénale. Ces fonctions exposées en cours seront illustrées au travers d’expérimentations assistées par ordinateur utilisant le modèle animal, humain et la simulation informatique. Les aspects réglementaires et législatifs sur l’expérimentation animale, les techniques d’investigation et d’anesthésie seront également abordés.
} 
% ******* Pré-requis
{Physiologie des cellules excitables.
} 
% ******* Objectifs
{\begin{itemize} 
  \ObjItem Connaissances fondamentales de physiologie. Méthodologie de rédaction (TD). Les enseignements pratiques en lien avec les notions théoriques de physiologie initient l’étudiant à l’expérimentation sur animal vivant en lui apportant les notions législatives et éthiques liées à ce travail. 
\end{itemize} 
} 
% ******* Ressources pédagogiques
{} 
% ******* Bibliographie éventuelle
{Biblio}
 
\vfill
\module[codeApogee={SOL5BH01},
titre={Régulation de l'expression des gènes}, 
COURS={24}, 
TD={12}, 
TP={12}, 
CTD={},
CTP={}, 
TOTAL={48}, 
SEMESTRE={Semestre 5}, 
COEFF={5}, 
ECTS={5}, 
MethodeEval={Ecrit},
ModalitesCCSemestreUn={RNE : E(CT) 2h / TP(CC) ; RSE : E(CT) 2h / TP(CT) 1h},
ModalitesCCSemestreDeux={RNE et RSE : E(CT) 2h / TP(CT) 1h},
CalculNFSessionUne={75\% E + 25\% TP},
CalculNFSessionDeux={75\% E + 25\% TP},
NoteEliminatoire={}, 
nomPremierResp={Alain Legrand}, 
emailPremierResp={alain.legrand@univ-orleans.fr}, 
nomSecondResp={}, 
emailSecondResp={}, 
langue={Français}, 
nbPrerequis={1}, 
descriptionCourte={true}, 
descriptionLongue={true}, 
objectifs={true}, 
ressources={true}, 
bibliographie={false}] 
% ******* Texte introductif
{
Parcours BMC / BBV
} 
% ******* Contenu détaillé
{
Mécanismes généraux de la régulation de la transcription et de la traduction chez les procaryotes. Notions de signaux exogènes et endogènes de la régulation. Étude détaillée de quelques systèmes de régulation chez les procaryotes : approches biologiques, physiologiques, génétiques et moléculaires. Mécanismes de régulation chez les eucaryotes : régulation transcriptionnelle : séquences régulatrices et facteurs de transcription. Complexes d’initiation de la transcription. Structure de la chromatine et expression génique. Régulation post-transcriptionnelle : modifications des ARNm (coiffe, épissage, polyadénylation), durée de vie des ARNm, régulation par les petits ARN. Régulation de l’initiation de la traduction. Régulation post-traductionnelle : modifications et durée de vie des protéines.
} 
% ******* Pré-requis
{Bases fondamentales de Biologie Moléculaire
} 
% ******* Objectifs
{\begin{itemize} 
  \ObjItem Comprendre les mécanismes de base qui gouvernent l'activité des gènes.
\end{itemize} 
} 
% ******* Ressources pédagogiques
{} 
% ******* Bibliographie éventuelle
{Biblio}
 
\vfill
\module[codeApogee={SOL5BH02},
titre={Analyse spectroscopique des biomolécules}, 
COURS={26}, 
TD={10}, 
TP={}, 
CTD={},
CTP={}, 
TOTAL={36}, 
SEMESTRE={Semestre 5}, 
COEFF={4}, 
ECTS={4}, 
MethodeEval={Ecrit},
ModalitesCCSemestreUn={RNE et RSE : CT 2h},
ModalitesCCSemestreDeux={RNE et RSE : CT 2h},
CalculNFSessionUne={100\% CT},
CalculNFSessionDeux={100\% CT},
NoteEliminatoire={}, 
nomPremierResp={Daniel Auguin}, 
emailPremierResp={daniel.auguin@univ-orleans.fr}, 
nomSecondResp={}, 
emailSecondResp={}, 
langue={Français}, 
nbPrerequis={1}, 
descriptionCourte={true}, 
descriptionLongue={true}, 
objectifs={true}, 
ressources={true}, 
bibliographie={false}] 
% ******* Texte introductif
{
Parcours BMC / BBV
} 
% ******* Contenu détaillé
{
\begin{itemize}
\item[Spectroscopie optique :] Notions générales ; spectroscopie UV-visible, d’absorption, de fluorescence, de phosphorescence ; spectroscopie infra rouge. Initiation au dichroïsme circulaire.
\item[RMN des bio-molécules :] Principes généraux : propriétés magnétiques des noyaux atomiques ; appareillage et séquenceurs d’impulsion ; le signal RMN ; les Paramètres de la RMN : déplacement chimique, couplage scalaire, nOe, temps de relaxation. Initiation aux spectres 1D/2D/3D. Résolution des structures de protéines.
\item[Spectrométrie de masse :] Domaines d’utilisation. Description de base d’un spectromètre de masse : sources d’ions (EI, CI, FAB, ESI et MALDI) ; analyseurs (quadripôle, magnétique, temps de vol, trappe ionique, résonance cyclotronique) et
détecteurs. Spectrométrie de masse en tandem, séquençage des peptides, analyse du protéome.
\end{itemize}
} 
% ******* Pré-requis
{Ce module d’initiation aux méthodes de la biophysique pour l’étude du vivant requiert de préférence des notions sur les structures et propriétés des biomolécules telles qu’elles sont abordées en L1 et L2.
} 
% ******* Objectifs
{\begin{itemize} 
  \ObjItem Fournir les bases des spectroscopies utilisées couramment aujourd’hui dans l’analyse des molécules et macromolécules biologiques.
\end{itemize} 
} 
% ******* Ressources pédagogiques
{} 
% ******* Bibliographie éventuelle
{Biblio}
 
\vfill
\module[codeApogee={SOL5IP01},
titre={Insertion professionnelle}, 
COURS={10}, 
TD={9}, 
TP={}, 
CTD={},
CTP={}, 
TOTAL={19}, 
SEMESTRE={Semestre 5}, 
COEFF={1}, 
ECTS={1}, 
MethodeEval={Ecrit / Oral},
ModalitesCCSemestreUn={RNE et RSE : CT(Ecrit) 1h30},
ModalitesCCSemestreDeux={RNE et RSE : CT(Ecrit) 1h / CT (Oral) 15 min },
CalculNFSessionUne={100\% CT},
CalculNFSessionDeux={50\% Ecrit + 50\% Oral},
NoteEliminatoire={}, 
nomPremierResp={Olivier Richard}, 
emailPremierResp={olivier.richard@univ-orleans.fr}, 
nomSecondResp={}, 
emailSecondResp={}, 
langue={Français}, 
nbPrerequis={0}, 
descriptionCourte={true}, 
descriptionLongue={true}, 
objectifs={true}, 
ressources={true}, 
bibliographie={false}] 
% ******* Texte introductif
{
Parcours BOPE / BMC / BBV
} 
% ******* Contenu détaillé
{
\begin{itemize}
\item (4h par des intervenants du domaine) Découverte de l’entreprise privée (dans le domaine des Sciences de la Vie) : rôle économique, organisation, fonctionnement, types de métiers, modes de recrutement, droit du travail (4h par des intervenants du domaine) Découverte de l’entreprise publique (dans le domaine des Sciences de la vie : CNRS/INRA/Université etc ...) : statuts, buts, organisation hiérarchiques et carrières, modes de financement, modes de recrutement.
\item Travail en groupe (TD groupe de 20) Initiation à la rédaction d’un CV, d’une lettre de motivation dans le cadre d’une demande de stage ou d’inscription en Master. Première approche de la situation de l’entretien de recrutement (niveau L, objectif le stage ou l’insertion en M1). Sensibilisation aux moyens pour rechercher l’information sur les stages et les emplois (secteur privé). Utilisation des ressources en ligne sur les métiers de la fonction publique.
\end{itemize}
} 
% ******* Pré-requis
{} 
% ******* Objectifs
{\begin{itemize} 
  \ObjItem Rendre actif la démarche étudiante pour une insertion après la Licence (emploi, formation complémentaire, Master). Connaissance du tissu économique dans le domaine des Sciences de la Vie. Réflexion de l’étudiant sur son projet personnel. Elaboration d’un CV niveau L, d’une demande de stage niveau L.
\end{itemize} 
} 
% ******* Ressources pédagogiques
{} 
% ******* Bibliographie éventuelle
{Biblio}

 
\vfill
\module[codeApogee={SOL5BH03},
titre={Immunologie générale}, 
COURS={18}, 
TD={6}, 
TP={}, 
CTD={},
CTP={}, 
TOTAL={24}, 
SEMESTRE={Semestre 5}, 
COEFF={3}, 
ECTS={3}, 
MethodeEval={Ecrit},
ModalitesCCSemestreUn={RNE et RSE : CT 2h},
ModalitesCCSemestreDeux={RNE et RSE : CT 2h},
CalculNFSessionUne={100\% CT},
CalculNFSessionDeux={100\% CT},
NoteEliminatoire={}, 
nomPremierResp={François Erard}, 
emailPremierResp={francois.erard@univ-orleans.fr}, 
nomSecondResp={}, 
emailSecondResp={}, 
langue={Français}, 
nbPrerequis={0}, 
descriptionCourte={true}, 
descriptionLongue={true}, 
objectifs={false}, 
ressources={true}, 
bibliographie={false}] 
% ******* Texte introductif
{
} 
% ******* Contenu détaillé
{
Concepts fondamentaux : les composants du système immunitaire, principes de l’immunité innée et acquise, l’immunité à médiation cellulaire T, la réponse immune humorale. Molécules membranaires et médiateurs solubles de signalisation du système immunitaire, cytokines, et récepteurs.
Interactions moléculaires – la reconnaissance de l’antigène : l’origine génétique de la diversité dans la réponse immune. Développement des lymphocytes, structure de leurs récepteurs. Anticorps et superfamille des immunoglobulines : Isotypes, allotypes, idiotypes ; Interactions antigènes/anticorps, épitopes, paratopes, haptènes. Cellules dendritiques et lymphocytes T : Présentation d’antigènes; Complexes Majeurs d’Histocompatibilité ; Tolérance immunologique} 
% ******* Pré-requis
{
} 
% ******* Objectifs
{
} 
% ******* Ressources pédagogiques
{} 
% ******* Bibliographie éventuelle
{Biblio}
 
\vfill
\module[codeApogee={SOL5BH04},
titre={Outils de caractérisation des biomolécules}, 
COURS={30}, 
TD={}, 
TP={6}, 
CTD={},
CTP={}, 
TOTAL={36}, 
SEMESTRE={Semestre 5}, 
COEFF={4}, 
ECTS={4}, 
MethodeEval={Ecrit/oral},
ModalitesCCSemestreUn={RNE et RSE : E(CT) 2h / TP(CC)},
ModalitesCCSemestreDeux={RNE et RSE : E(CT) 2h / TP(CT oral) 15min},
CalculNFSessionUne={66\% E + 33\% TP},
CalculNFSessionDeux={66\% E + 33\% TP},
NoteEliminatoire={}, 
nomPremierResp={Stéphane Petoud}, 
emailPremierResp={stephane.petoud@univ-orleans.fr}, 
nomSecondResp={}, 
emailSecondResp={}, 
langue={Français}, 
nbPrerequis={1}, 
descriptionCourte={true}, 
descriptionLongue={true}, 
objectifs={true}, 
ressources={true}, 
bibliographie={false}] 
% ******* Texte introductif
{
} 
% ******* Contenu détaillé
{
Analyse des formes du vivant en utilisant la « théorie des groupes ». Notions sur le repliement des protéines et sur les méthodes qui permettent de l’appréhender.
} 
% ******* Pré-requis
{Ce module optionnel requiert un intérêt pour la structure des biomolécules.
} 
% ******* Objectifs
{\begin{itemize} 
  \ObjItem Faire acquérir aux étudiants une compréhension des techniques physiques utilisées en biochimie. Ouverture vers un master de biophysique.
\end{itemize} 
} 
% ******* Ressources pédagogiques
{} 
% ******* Bibliographie éventuelle
{Biblio}
 
\vfill
\module[codeApogee={SOL5BO12},
titre={Les outils de la génétique}, 
COURS={12}, 
TD={8}, 
TP={16}, 
CTD={},
CTP={}, 
TOTAL={36}, 
SEMESTRE={Semestre 5}, 
COEFF={4}, 
ECTS={4}, 
MethodeEval={Ecrit/Oral},
ModalitesCCSemestreUn={RNE : E(CT) 2h / Oral (CC) / TP(CC) ; RSE : E(CT) 2h / Oral (CT) / TP(CT)},
ModalitesCCSemestreDeux={RNE et RSE : E(CT) 2h / TP(CT) 1h},
CalculNFSessionUne={50\% E + 25\% Oral + 25 \% TP},
CalculNFSessionDeux={66\% E + 33\% TP},
NoteEliminatoire={}, 
nomPremierResp={Martine Decoville}, 
emailPremierResp={martine.decoville@univ-orleans.fr}, 
nomSecondResp={}, 
emailSecondResp={}, 
langue={Français}, 
nbPrerequis={0}, 
descriptionCourte={true}, 
descriptionLongue={true}, 
objectifs={true}, 
ressources={true}, 
bibliographie={false}] 
% ******* Texte introductif
{
} 
% ******* Contenu détaillé
{
Mutagenèse chez les organismes diploïdes. Transposon et mutagenèses. Transgénèse et invalidation de gène. Surexpression et expression ectopique de gènes (UAS/Gal4). La recombinaison mitotique. Obtention de clones cellulaires. Les marqueurs utilisés en génétique humaine : RFLP, VNTR, microsatellites, SNP. Génétique humaine : Lod score. Analyse des haplotypes.
TP : Utilisation de l’outil UAS/Gal4, clones mitotiques chez la drosophile, applications à des maladies. Utilisation des microsatellites et police scientifique.} 
% ******* Pré-requis
{
} 
% ******* Objectifs
{\begin{itemize} 
  \ObjItem acquérir la maitrise des outils génétiques utilisés couramment pour répondre à des problèmes biologiques comme l’étude de pathologies, du développement ou de divers processus biologiques.
\end{itemize} 
} 
% ******* Ressources pédagogiques
{} 
% ******* Bibliographie éventuelle
{Biblio}
 
\vfill\module[codeApogee={SOL5BH05},
titre={Organisation du génôme eucaryote}, 
COURS={14}, 
TD={10}, 
TP={}, 
CTD={},
CTP={}, 
TOTAL={24}, 
SEMESTRE={Semestre 5}, 
COEFF={3}, 
ECTS={3}, 
MethodeEval={Ecrit},
ModalitesCCSemestreUn={RNE : E(CT) 2h / TP(CC) ; RSE : E(CT) 2h / TP(CT) 1h},
ModalitesCCSemestreDeux={RNE et RSE : E(CT) 2h / TP(CT) 1h},
CalculNFSessionUne={66\% E + 33\% TP},
CalculNFSessionDeux={66\% E + 33\% TP},
NoteEliminatoire={}, 
nomPremierResp={Martine Guérin}, 
emailPremierResp={martine.guerin@univ-orleans.fr}, 
nomSecondResp={}, 
emailSecondResp={}, 
langue={Français}, 
nbPrerequis={1}, 
descriptionCourte={true}, 
descriptionLongue={true}, 
objectifs={true}, 
ressources={true}, 
bibliographie={false}] 
% ******* Texte introductif
{
} 
% ******* Contenu détaillé
{
Le génome eucaryote : Structure, organisation et expression. Les génomes mitochondriaux. Etude comparative de différents génomes. Les transposons : éléments P, SINE, LINE et virus. Importance de la transposition dans la structure et l’évolution des génomes. Structure du chromosome : Le chromosome métaphasique. Eu et Hétérochromatine. Les accidents chromosomiques, leurs rôles dans l’évolution.
} 
% ******* Pré-requis
{Bases de génétique et de biologie moléculaire.
} 
% ******* Objectifs
{\begin{itemize} 
  \ObjItem Approfondir les acquis de la génétique moléculaire.
\end{itemize} 
} 
% ******* Ressources pédagogiques
{} 
% ******* Bibliographie éventuelle
{Biblio}
 
\vfill
\module[codeApogee={SOL5BO05},
titre={Bases anatomiques des grandes fonctions animales}, 
COURS={14}, 
TD={}, 
TP={10}, 
CTD={},
CTP={}, 
TOTAL={24}, 
SEMESTRE={Semestre 5}, 
COEFF={3}, 
ECTS={3}, 
MethodeEval={Ecrit/Oral},
ModalitesCCSemestreUn={RNE et RSE : CT (Ecrit) 1h},
ModalitesCCSemestreDeux={RNE et RSE : CT (Oral) 1h},
CalculNFSessionUne={100\%},
CalculNFSessionDeux={100\%},
NoteEliminatoire={}, 
nomPremierResp={Valérie Altemayer}, 
emailPremierResp={valerie.altemayer@univ-orleans.fr}, 
nomSecondResp={}, 
emailSecondResp={}, 
langue={Français}, 
nbPrerequis={1}, 
descriptionCourte={true}, 
descriptionLongue={true}, 
objectifs={true}, 
ressources={true}, 
bibliographie={false}] 
% ******* Texte introductif
{
Parcours BOPE/ BMC / BGST / PLURI
} 
% ******* Contenu détaillé
{
\begin{itemize}
\item Cours : Organisation des tissus : tissus épithéliaux – conjonctifs – musculaires. Description anatomique et histologique des appareils : circulatoire – respiratoire – digestif excréteur et reproducteur. 
\item TP illustration du cours à partir de préparations et de coupes histologiques – Etude de l’organisation de tous les appareils présentés en cours chez une souris. Les TP pourront être enseignés en anglais.
\end{itemize}
} 
% ******* Pré-requis
{Avoir de bonnes bases en biologie cellulaire.
} 
% ******* Objectifs
{\begin{itemize} 
  \ObjItem Acquisition de l’anatomie et de l’histologie des grandes fonctions avant d’appréhender la physiologie.
\end{itemize} 
} 
% ******* Ressources pédagogiques
{} 
% ******* Bibliographie éventuelle
{Biblio}

\vfill
 
\module[codeApogee={UEL},
titre={Maths prépa concours : techniques de calcul en mathématiques}, 
COURS={}, 
TD={20}, 
TP={}, 
CTD={},
CTP={}, 
TOTAL={20}, 
SEMESTRE={Semestre 5}, 
COEFF={3}, 
ECTS={3}, 
%MethodeEval={???}, 
ModalitesCCSemestreUn={Cf. modalités de contrôle de connaissances des UE Libres}, 
ModalitesCCSemestreDeux={Cf. modalités de contrôle de connaissances des UE Libres}, 
%CalculNFSessionUne={Examen 67 \% ; TP 33 \%}, 
%CalculNFSessionDeux={Examen 67 \% ; TP 33 \%}, 
NoteEliminatoire={}, 
nomPremierResp={Emmanuel Cepa}, 
emailPremierResp={emmanuel.cepa@univ-orleans.fr}, 
nomSecondResp={}, 
emailSecondResp={}, 
langue={Français}, 
nbPrerequis={1}, 
descriptionCourte={true}, 
descriptionLongue={true}, 
objectifs={true}, 
ressources={false}, 
bibliographie={false}] 
% ******* Texte introductif
{
UE pouvant être prise aussi en semestre 3
} 
% ******* Contenu détaillé
{
Les thèmes suivants seront abordés. Les notions minimales de cours seront données afin de privilégier les exemples.
\begin{itemize}
\item Fonctions numériques de la variable réelle / Intégrales / Equations différentielles
\item Suites / Algèbre Linéaire /Probabilités
\end{itemize}
}
% ******* Pré-requis
{Maths niveau BAC S, forte motivation.} 
% ******* Objectifs
{\begin{itemize} 
  \ObjItem Permettre à des étudiants d’acquérir des connaissances utiles pour passer les concours d’accès aux écoles nationales vétérinaires (concours ENV- voie B), les concours d’ingénieurs agro (concours ENSA voie B) mais également certains concours administratifs.
\end{itemize} 
} 
% ******* Ressources pédagogiques
{} 
% ******* Bibliographie éventuelle
{Biblio}
\vfill
%==========================================================================================
%===================================================================================
\end{document}
