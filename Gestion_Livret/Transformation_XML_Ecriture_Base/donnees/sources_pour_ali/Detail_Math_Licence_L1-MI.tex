\documentclass[10pt, a5paper]{report}

\usepackage[T1]{fontenc}%
\usepackage[utf8]{inputenc}% encodage utf8
\usepackage[francais]{babel}% texte français
\usepackage[final]{pdfpages}
\usepackage{modules-livret}% style du livret
\usepackage{url,amsmath,amssymb,url}
%\usepackage{init-preambule}
\pagestyle{empty}

% % % % % % % % % % % % % % % % % % % % % % % % % % % % % % % % % % % % % % % % % % % % % % % % % % % % % % % 
\begin{document}

\input{Extra/couleurLicenceMI.tex}

%==========================================================================================
% Semestre 1
%==========================================================================================

\module[codeApogee={1MT02},
titre={Introduction au raisonnement math\'ematique}, 
COURS={}, 
TD={}, 
TP={}, 
CTD={60}, 
TOTAL={60}, 
SEMESTRE={Semestre 1}, 
COEFF={6}, 
ECTS={6}, 
MethodeEval={Contrôle continu intégral},
ModalitesCCSemestreUn={CC et CT}, 
ModalitesCCSemestreDeux={CT}, 
%CalculNFSessionUne={$\frac{(CC+2*CT)}{3}$}, 
%CalculNFSessionDeux={CT}, 
NoteEliminatoire={}, 
nomPremierResp={Cécile Louchet}, 
emailPremierResp={cecile.louchet@univ-orleans.fr}, 
nomSecondResp={Alexandre Tessier},
emailSecondResp={alexandre.tessier@univ-orleans.fr }, 
langue={Français}, 
nbPrerequis={}, 
descriptionCourte={true}, 
descriptionLongue={true}, 
objectifs={true}, 
ressources={false}, 
bibliographie={false}]
% ******* Texte introductif
{
Unité obligatoire. 
} 
% ******* Contenu détaillé
{
\begin{itemize} 
  \item Logique \'el\'ementaire, implication, quantificateurs,
  \item Les différents types de démonstrations en mathématiques (implication directe, récurrence, preuve par l'absurde ...),
  \item Manipulations ensemblistes, application d'un ensemble vers un autre,
  \item Applications injectives, surjectives, bijectives,
  \item Nombres complexes,
  \item Systèmes lin\'eaires, familles libres, matrices dans $\mathbb{R}^2$ et $\mathbb{R}^3$.
\end{itemize} 
} 
{Mathématiques de Terminale S.} 
{Savoir mettre en \oe uvre un raisonnement mathématique de base.}
{}
{Biblio}

\vfill

%==========================================================================================

\module[codeApogee={1MT03},
titre={Suites et fonctions réelles}, 
COURS={}, 
TD={}, 
TP={}, 
CTD={60}, 
TOTAL={60}, 
SEMESTRE={Semestre 1}, 
COEFF={6}, 
ECTS={6}, 
MethodeEval={Contrôle continu intégral}, 
ModalitesCCSemestreUn={CC et CT}, 
ModalitesCCSemestreDeux={CT},
%CalculNFSessionUne={$\frac{(CC+2*CT)}{3}$}, 
%CalculNFSessionDeux={CT},
nomPremierResp={Jean-Philippe Anker},
emailPremierResp={jean-philippe.anker@univ-orleans.fr}, 
nomSecondResp={Alexandre Tessier}, 
emailSecondResp={alexandre.tessier@univ-orleans.fr}, 
langue={Français}, 
nbPrerequis={}, 
descriptionCourte={true}, 
descriptionLongue={true}, 
objectifs={true}, 
ressources={false}, 
bibliographie={false}]
% ******* Texte introductif
{
Unité obligatoire. 
}
% ******* Contenu détaillé
{
\begin{itemize} 
  \item Structures d'ordre, majorants-minorants, bornes supérieures et inférieures,
  \item Suites numériques: étude, convergence, suites récurrentes,
  \item Fonctions numériques: limite, continuité,
  \item Dérivabilité, fonctions usuelles, étude de fonctions,
  \item Fonctions réciproques.
\end{itemize}
}
% ******* Pré-requis
{Mathématiques de Terminale S.}
% ******* Objectifs
{Ce module d'analyse traite des suites et fonctions réelles. Les notions de limite, de continuité, de dérivabilité sont proprement établies et permettent alors l'étude précise de suites et de fonctions.}
% ******* Ressources pédagogiques
{}
% ******* Biblio
{}
 
\vfill

%==========================================================================================

\module[codeApogee={1IF02},
titre={Algorithmique et programmation 1}, 
COURS={}, 
TD={}, 
TP={15}, 
CTD={45}, 
TOTAL={60}, 
SEMESTRE={Semestre 1}, 
COEFF={6}, 
ECTS={6}, 
MethodeEval={Contrôle continu intégral}, 
ModalitesCCSemestreUn={CC et CT}, 
ModalitesCCSemestreDeux={CT}, 
%CalculNFSessionUne={$\frac{(CC+2*CT)}{3}$}, 
%CalculNFSessionDeux={CT}, 
NoteEliminatoire={}, 
nomPremierResp={Alexandre Tessier},
emailPremierResp={alexandre.tessier@univ-orleans.fr}, 
langue={Français}, 
nbPrerequis={0}, 
descriptionCourte={true},
objectifs={true}, 
ressources={false}, 
bibliographie={false}]
% ******* Texte introductif
{
Unité obligatoire. 
} 
% ******* Contenu détaillé
{Algorithmique élémentaire : expressions, variables, instructions, séquences, conditionnelles, boucles, tableaux, preuves, invariants, traduction dans le langage Java.}
% ******* Prérequis
{}
% ******* Objectifs
{Maîtriser les concepts élémentaires de l'algorithmique et être capable de les traduire dans le langage C.}
{Ressources}
{Biblio}
 
\vfill

%==========================================================================================

\module[codeApogee={1MT04}, 
titre={Arithmétique dans Z}, 
COURS={}, 
TD={}, 
TP={}, 
CTD={24}, 
TOTAL={24}, 
SEMESTRE={Semestre 1}, 
COEFF={3}, 
ECTS={3}, 
MethodeEval={Contrôle continu intégral}, 
ModalitesCCSemestreUn={CC et CT}, 
ModalitesCCSemestreDeux={CT}, 
%CalculNFSessionUne={$\frac{(CC+2*CT)}{3}$}, 
%CalculNFSessionDeux={CT}, 
NoteEliminatoire={}, 
nomPremierResp={Emmanuel Cépa},
emailPremierResp={emmanuel.cepa@univ-orleans.fr}, 
nomSecondResp={Alexandre Tessier}, 
emailSecondResp={alexandre.tessier@univ-orleans.fr}, 
langue={Français}, 
nbPrerequis={}, 
descriptionCourte={true},
objectifs={true}, 
ressources={false}, 
bibliographie={false}] 
{
Unité obligatoire. 
} 
{
\begin{itemize} 
  \item Divisibilité,
  \item PGCD-PPCM,
  \item théorèmes de Bézout et de Gauss,
  \item décomposition en produits de facteurs premiers,
  \item congruences.
\end{itemize}
}
{Mathématiques de Terminale S.}
{Grâce aux exemples d'arithmétique élémentaire, découvrir l'importance de quelques structures algébriques.}
{Ressources} 
{Biblio} 
 
\vfill

%==========================================================================================

\module[codeApogee={1IF04}, 
titre={Atelier de l'informaticien}, 
COURS={}, 
TD={}, 
TP={}, 
CTD={24}, 
TOTAL={24}, 
SEMESTRE={Semestre 1}, 
COEFF={3}, 
ECTS={3}, 
MethodeEval={Contrôle continu intégral}, 
ModalitesCCSemestreUn={CC et CT}, 
ModalitesCCSemestreDeux={CT}, 
%CalculNFSessionUne={$\frac{(CC+2*CT)}{3}$}, 
%CalculNFSessionDeux={CT}, 
NoteEliminatoire={}, 
nomPremierResp={Pierre Réty},
emailPremierResp={pierre.rety@univ-orleans.fr}, 
nomSecondResp={Alexandre Tessier}, 
emailSecondResp={alexandre.tessier@univ-orleans.fr}, 
langue={Français}, 
nbPrerequis={0}, 
descriptionCourte={true},
objectifs={true}, 
ressources={false}, 
bibliographie={false}]
{
Unité obligatoire. 
}
{
Présenter le système sous l'angle de l'utilisateur. Configuration de l'environnement de travail de l'informaticien. Notions sur le matériel, les systèmes d'exploitation, et les réseaux. Installation, administration, et utilisation d'un système Linux sur un PC.
}
{}
{
Être autonome dans la manipulation du système.
}
{Ressources} 
{Biblio}
 
\vfill

%==========================================================================================

\module[codeApogee={1AG11}, 
titre={Anglais 1}, 
COURS={}, 
TD={24}, 
TP={}, 
CTD={}, 
TOTAL={24}, 
SEMESTRE={Semestre 1}, 
COEFF={3}, 
ECTS={3}, 
%MethodeEval={Contrôle continu intégral}, 
ModalitesCCSemestreUn={Rapport et soutenance de projet}, 
ModalitesCCSemestreDeux={Pas de 2nde session}, 
%CalculNFSessionUne={$\frac{(CC+2*CT)}{3}$}, 
%CalculNFSessionDeux={CT}, 
NoteEliminatoire={}, 
nomPremierResp={Murielle Pasquet}, 
emailPremierResp={murielle.pasquet@univ-orleans.fr}, 
nomSecondResp={Alexandre Tessier}, 
emailSecondResp={alexandre.tessier@univ-orleans.fr}, 
langue={Français}, 
nbPrerequis={}, 
descriptionCourte={true},
objectifs={true}, 
ressources={false}, 
bibliographie={false}]
{
Unité obligatoire.
}
{
Travail de compréhension et d'expression orale et écrite, à partir de documents authentiques simples, ou courts, centrés sur le monde universitaire anglo-saxon.
}
{Niveau anglais baccalauréat LV1 ou LV2 ou équivalent.}
{
Être à même de préparer un projet de séjour d'études universitaires en pays anglophone dans une langue écrite et orale simple.
}
{Ressources} 
{Biblio} 
 
\vfill

%==========================================================================================

\module[codeApogee={1II01}, 
titre={Préparation au C2I}, 
COURS={}, 
TD={}, 
TP={}, 
CTD={24}, 
TOTAL={24}, 
SEMESTRE={Semestre 1}, 
COEFF={3}, 
ECTS={3}, 
MethodeEval={Contrôle continu intégral}, 
ModalitesCCSemestreUn={CC et CT}, 
ModalitesCCSemestreDeux={CT}, 
%CalculNFSessionUne={$\frac{(CC+2*CT)}{3}$}, 
%CalculNFSessionDeux={CT}, 
NoteEliminatoire={}, 
nomPremierResp={Laure Kahlem}, 
emailPremierResp={laure.kahlem@univ-orleans.fr}, 
nomSecondResp={Alexandre Tessier}, 
emailSecondResp={alexandre.tessier@univ-orleans.fr}, 
langue={Français}, 
nbPrerequis={0}, 
descriptionCourte={true},
objectifs={true}, 
ressources={false}, 
bibliographie={false}]
{
Unité obligatoire. 
} 
{
Préparation aux domaines de compétences du réferentiel national suivant:
\begin{description}
\item[D1] Travailler dans un environnement numérique évolutif,
\item[D2] Être responsable à l'aire du numérique,
\item[D3] Produire, traiter, exploiter et diffuser des documents numériques,
\item[D5] Travailler en réseau, communiquer et collaborer.
\end{description}
} 
{}
{
Maîtriser les compétences D1, D2, D3, et D5 du référentiel national C2I1.
}
{}
{}

\vfill
 
%==========================================================================================
% Semestre 2
%==========================================================================================

\module[codeApogee={2MT06}, 
titre={Algèbre 1}, 
COURS={}, 
TD={}, 
TP={}, 
CTD={60}, 
TOTAL={60}, 
SEMESTRE={Semestre 2}, 
COEFF={6}, 
ECTS={6}, 
MethodeEval={Contrôle continu et terminal}, 
ModalitesCCSemestreUn={CC et CT}, 
ModalitesCCSemestreDeux={CT}, 
%CalculNFSessionUne={$\frac{(CC+2*CT)}{3}$}, 
%CalculNFSessionDeux={CT}, 
NoteEliminatoire={}, 
nomPremierResp={Patrick Maheux}, 
emailPremierResp={patrick.maheux@univ-orleans.fr}, 
nomSecondResp={Guillaume Havard}, 
emailSecondResp={guillaume.havard@univ-orleans.fr}, 
langue={Français}, 
nbPrerequis={}, 
descriptionCourte={true}, 
descriptionLongue={true}, 
objectifs={true}, 
ressources={false}, 
bibliographie={false}] 
{
Unité obligatoire. 
} 
{\begin{itemize} 
  \item Arithmétique des polynômes, décomposition des fractions rationnelles,
  \item Espaces et sous-espaces vectoriels,
  \item Bases en dimension finie, théorie de la dimension,
  \item Applications linéaires,
  \item Matrices, calcul matriciel,
  \item Déterminant.
\end{itemize} 
}
{Avoir suivi Introduction au raisonnement mathématique au semestre 1.}
{Se familiariser avec les polynômes. Apprendre l'algèbre linéaire et manipuler des matrices.}
{Ressources} 
{Biblio} 
 
\vfill

%==========================================================================================

\module[codeApogee={2MT05}, 
titre={Analyse 1}, 
COURS={}, 
TD={}, 
TP={}, 
CTD={60}, 
TOTAL={60}, 
SEMESTRE={Semestre 2}, 
COEFF={6}, 
ECTS={6}, 
MethodeEval={Contrôle continu et terminal}, 
ModalitesCCSemestreUn={CC et CT}, 
ModalitesCCSemestreDeux={CT}, 
%CalculNFSessionUne={$\frac{(CC+2*CT)}{3}$}, 
%CalculNFSessionDeux={CT}, 
NoteEliminatoire={}, 
nomPremierResp={Guillaume Havard}, 
emailPremierResp={guillaume.havard@univ-orleans.fr}, 
langue={Français}, 
nbPrerequis={}, 
descriptionCourte={true}, 
descriptionLongue={true}, 
objectifs={true}, 
ressources={false}, 
bibliographie={false}] 
{
Unité obligatoire. 
} 
{
\begin{itemize} 
  \item Continuité uniforme,
  \item Dérivation, fonctions dérivables sur un intervalle, dérivée d'une fonction réciproque,
  \item Théorème de Taylor, développements limités,
  \item Introduction à l'intégrale de Riemann,
  \item Calcul des primitives.
\end{itemize}
}
{Avoir suivi Suites et fonctions réelles au semestre 1.}
{S'initier aux méthodes plus fines d'analyse des fonctions réelles.}
{Ressources} 
{Biblio} 
 
\vfill

%==========================================================================================

\module[codeApogee={2IF01}, 
titre={Algorithmique et programmation 2}, 
COURS={}, 
TD={}, 
TP={}, 
CTD={60}, 
TOTAL={60}, 
SEMESTRE={Semestre 2}, 
COEFF={6}, 
ECTS={6}, 
MethodeEval={Contrôle continu et terminal}, 
ModalitesCCSemestreUn={CC et CT}, 
ModalitesCCSemestreDeux={CT}, 
%CalculNFSessionUne={$\frac{(CC+2*CT)}{3}$}, 
%CalculNFSessionDeux={CT}, 
NoteEliminatoire={}, 
nomPremierResp={Wadoud Bousdira}, 
emailPremierResp={wadoud.bousdira@univ-orleans.fr}, 
nomSecondResp={Guillaume Havard}, 
emailSecondResp={guillaume.havard@univ-orleans.fr}, 
langue={Français}, 
nbPrerequis={}, 
descriptionCourte={true}, 
descriptionLongue={true}, 
objectifs={true}, 
ressources={false}, 
bibliographie={false}] 
{
Unité optionnelle. Option 1 (parmi 3) 
} 
{Algorithmique élémentaire : récursivité, objets, structures de données chaînées (listes, files, piles), notions élémentaires (allocation dynamique, chaînage des données), traduction
dans un langage de programmation orienté objets.
}
{Avoir suivi Algorithmique et programmation 1}
{Assimiler la programmation récursive d'une part, et d'autre part, la définition et l'utilisation de structures de données récursives.} 
{Ressources} 
{Biblio} 
 
\vfill

%==========================================================================================

\module[codeApogee={2IF02}, 
titre={Outil mathématique pour l'informatique}, 
COURS={}, 
TD={}, 
TP={}, 
CTD={48}, 
TOTAL={48}, 
SEMESTRE={Semestre 2}, 
COEFF={4}, 
ECTS={4}, 
%MethodeEval={}, 
ModalitesCCSemestreUn={}, 
ModalitesCCSemestreDeux={}, 
%CalculNFSessionUne={$\frac{(CC+2*CT)}{3}$}, 
%CalculNFSessionDeux={CT}, 
NoteEliminatoire={}, 
nomPremierResp={Pierre Réty}, 
emailPremierResp={pierre.rety@univ-orleans.fr}, 
nomSecondResp={Guillaume Havard}, 
emailSecondResp={guillaume.havard@univ-orleans.fr}, 
langue={Français}, 
nbPrerequis={}, 
descriptionCourte={true}, 
descriptionLongue={true}, 
objectifs={true}, 
ressources={false}, 
bibliographie={false}] 
{
Unité optionnelle : option 2 (parmi 2).
} 
{Logique des propositions et des prédicats. \'Etude des procédés de base des démonstrations mathématiques, sur des notions ensemblistes. Relations binaires, fermeture transitive, relations d'équivalences, relations d'ordre partiel. Récurrence forte sur la longueur des mots d'un langage. Algèbre de Boole. Circuits.}
{Les notions ensemblistes (unité Introduction au raisonnement mathématique du semestre 1).}
{
\begin{itemize}
\item Comprendre et savoir écrire des démonstrations de mathématiques sur les ensembles et les relations binaires. 
\item Comprendre les relations d'équivalences et les relations d'ordre partiel.
\item Être initié aux récurrences non-élémentaires, afin de pouvoir travailler sur l'induction en 2ème année.
\item Être initié aux circuits booléens.
\end{itemize}
} 
{Ressources} 
{Biblio} 
 
\vfill

%==========================================================================================

\module[codeApogee={2AG12}, 
titre={Anglais 2}, 
COURS={}, 
TD={24}, 
TP={}, 
CTD={}, 
TOTAL={24}, 
SEMESTRE={Semestre 2}, 
COEFF={3}, 
ECTS={3}, 
MethodeEval={Contrôle continu et terminal}, 
ModalitesCCSemestreUn={CC et CT}, 
ModalitesCCSemestreDeux={CT}, 
%CalculNFSessionUne={$\frac{(CC+2*CT)}{3}$}, 
%CalculNFSessionDeux={CT}, 
NoteEliminatoire={}, 
nomPremierResp={Sylvain Gendron}, 
emailPremierResp={sylvain.gendron@univ-orleans.fr}, 
nomSecondResp={Guillaume Havard}, 
emailSecondResp={guillaume.havard@univ-orleans.fr}, 
langue={Français}, 
nbPrerequis={}, 
descriptionCourte={true}, 
descriptionLongue={true}, 
objectifs={true}, 
ressources={false}, 
bibliographie={false}] 
{
Unité obligatoire
} 
{Travail de compréhension et d'expression orale et écrite à partir de documents authentiques simples et/ou cours centrés sur le monde universitaire anglo-saxon. Supports : vidéo, audio, articles de presse.} 
{Avoir suivi Anglais 1 ou environ 400 heures de formation équivalente}
{Comprendre et s'exprimer de manière plus autonome dans des situations de séjour d'études universitaires en pays anglophone (niveau européen : B1)}
{Ressources} 
{Biblio} 
 
\vfill

%==========================================================================================

\module[codeApogee={2PP02}, 
titre={Projet personnel et professionnel}, 
COURS={2}, 
TD={10},
TP={}, 
CTD={}, 
TOTAL={12}, 
SEMESTRE={Semestre 2}, 
COEFF={2}, 
ECTS={2}, 
MethodeEval={Contrôle continu et terminal}, 
ModalitesCCSemestreUn={CC et CT}, 
ModalitesCCSemestreDeux={CT}, 
%CalculNFSessionUne={$\frac{(CC+2*CT)}{3}$}, 
%CalculNFSessionDeux={CT}, 
NoteEliminatoire={}, 
nomPremierResp={Guillaume Havard}, 
emailPremierResp={guillaume.havard@univ-orleans.fr}, 
nomSecondResp={}, 
emailSecondResp={}, 
langue={Français}, 
nbPrerequis={0}, 
descriptionCourte={true}, 
descriptionLongue={true}, 
objectifs={true}, 
ressources={false}, 
bibliographie={false}] 
{
Unité optionnelle. Option 1 (parmi 3).
} 
{Cours : 
\begin{itemize}
\item présentation des objectifs,
\item modalités de recherche documentaire,
\item présentation du SUIO,
\item élaboration d'une fiche de projet individuel.
\end{itemize}
TD : 
\begin{itemize}
\item recherche massive de documents sur le métier ou l'activité choisie,
\item préparation d'une rencontre avec un professionnel correspondant au projet,
\item préparation du rapport écrit, du poster et de la soutenance.
\end{itemize}
} 
{}
{Initiation à la recherche documentaire, au travail en groupes, à la présentation orale et à la présentation d'un poster}
{Ressources} 
{Biblio} 

\vfill

%==========================================================================================

\module[codeApogee={2UL04}, 
titre={Unité d'enseignement libre}, 
COURS={}, 
TD={20}, 
TP={}, 
CTD={}, 
TOTAL={20}, 
SEMESTRE={Semestre 2}, 
COEFF={3}, 
ECTS={3}, 
MethodeEval={Contrôle continu et terminal}, 
ModalitesCCSemestreUn={CC et CT}, 
ModalitesCCSemestreDeux={CT}, 
%CalculNFSessionUne={$\frac{(CC+2*CT)}{3}$}, 
%CalculNFSessionDeux={CT}, 
NoteEliminatoire={}, 
nomPremierResp={Guillaume Havard},
emailPremierResp={guillaume.havard@univ-orleans.fr}, 
nomSecondResp={}, 
emailSecondResp={}, 
langue={Français}, 
nbPrerequis={0}, 
descriptionCourte={true}, 
descriptionLongue={true}, 
objectifs={true}, 
ressources={true}, 
bibliographie={false}] 
{
Unité obligatoire
} 
{L'unité d'ouverture est à choisir, en début du semestre, parmi la centaine d'enseignements dédiés à cet usage et offerts par toutes les composantes de l'université (Sciences, Droit-
\'Economie-Gestion, Sport). Voici quelques exemples d'unités d'ouverture :
\begin{itemize}
\item Sport.
\item Traitement de signal et d'image.
\item Droit de l'informatique.
\item Problèmes économiques contemporains.
\item Histoire du cinéma, histoire des arts.
\item Enseigner : posture et identité professionnelles.
\item Lecture critique du réchauffement climatique.
\item Maîtriser son expression ; les enjeux de la communication orale : le corps, l'espace, la voix.
\end{itemize}
} 
{}
{Comprendre comment ce qu'on apprend dans le cadre d'un diplôme déjà très spécialisé ; s'insérer dans le large champ des connaissances et des savoirs auxquels on sera confronté dans son expérience professionnelle ou personnelle.}
{La page du site de l'université dédiée aux unités d'ouverture : http://www.univ-orleans.fr/scolarite/inscriptions/?page=2} 
{Biblio} 
 
\vfill


%==========================================================================================
% Semestre 3
%==========================================================================================

\module[codeApogee={3MT08},
titre={Algèbre 2}, 
COURS={24}, 
TD={36}, 
TP={}, 
CTD={}, 
TOTAL={60}, 
SEMESTRE={Semestre 3}, 
COEFF={6}, 
ECTS={6}, 
MethodeEval={Contrôle continu et terminal}, 
ModalitesCCSemestreUn={CC et CT}, 
ModalitesCCSemestreDeux={CT}, 
%CalculNFSessionUne={$\frac{(CC+2*CT)}{3}$}, 
%CalculNFSessionDeux={CT}, 
NoteEliminatoire={}, 
nomPremierResp={Vittoria Pierfelice},
emailPremierResp={vittoria.pierfelice@univ-orleans.fr},
nomSecondResp={},
emailSecondResp={}, 
langue={Français},
nbPrerequis={}, 
objectifs={true}, 
ressources={false}, 
bibliographie={false}] 
{
Unité obligatoire. 
}
{
\begin{itemize} 
  \item Valeurs propres, vecteurs propres,
  \item Polynôme caractéristique,
  \item Diagonalisation - Trigonalisation,
  \item Polynômes d'endomorphismes,
  \item Polynômes annulateurs,
  \item Anneaux de polynômes [optionnel].
\end{itemize}
}
{Avoir suivi Algèbre 1 au semestre 2.}
{Se former à l'étude spectrale de matrices.}
{Ressources}
{Biblio}
 
\vfill

%==========================================================================================

\module[codeApogee={3MT09},
titre={Analyse 2}, 
COURS={24}, 
TD={36}, 
TP={}, 
CTD={}, 
TOTAL={60}, 
SEMESTRE={Semestre 3}, 
COEFF={6}, 
ECTS={6}, 
MethodeEval={Contrôle continu et terminal}, 
ModalitesCCSemestreUn={CC et CT}, 
ModalitesCCSemestreDeux={CT}, 
%CalculNFSessionUne={$\frac{(CC+2*CT)}{3}$}, 
%CalculNFSessionDeux={CT}, 
NoteEliminatoire={}, 
nomPremierResp={Vittoria Pierfelice}, 
emailPremierResp={vittoria.pierfelice@univ-orleans.fr}, 
nomSecondResp={},
emailSecondResp={},
langue={Français}, 
nbPrerequis={}, 
objectifs={true},
ressources={false}, 
bibliographie={false}] 
{
Unité obligatoire. 
} 
{
\begin{itemize} 
  \item Suites de Cauchy - Suites extraites - Théorème de Bolzano-Weierstrass,
  \item Séries numériques,
  \item Intégrales généralisées.
\end{itemize}
}
{Avoir suivi Analyse 1 au semestre 2.}
{Approfondir les notions sur les suites, et s'initier aux intégrales généralisées}
{Ressources} 
{Biblio} 
 
\vfill

%==========================================================================================

\module[codeApogee={3MT10},
titre={Mathématiques numériques}, 
COURS={}, 
TD={}, 
TP={}, 
CTD={36}, 
TOTAL={36}, 
SEMESTRE={Semestre 3}, 
COEFF={4}, 
ECTS={4}, 
MethodeEval={Contrôle continu et terminal}, 
ModalitesCCSemestreUn={CC et CT}, 
ModalitesCCSemestreDeux={CT}, 
%CalculNFSessionUne={$\frac{(CC+2*CT)}{3}$}, 
%CalculNFSessionDeux={CT}, 
NoteEliminatoire={}, 
nomPremierResp={Carine Lucas},
emailPremierResp={carine.lucas@univ-orleans.fr}, 
nomSecondResp={Vittoria Pierfelice},
emailSecondResp={vittoria.pierfelice@univ-orleans.fr}, 
langue={Français}, 
nbPrerequis={}, 
objectifs={true},
ressources={false}, 
bibliographie={false}] 
{
Unité obligatoire. 
} 
{
\begin{itemize} 
  \item Introduction à Scilab,
  \item Manipulation de données matricielles,
  \item Construction de vecteurs, matrices, extraction de sous-matrices,
  \item Représentation graphique 2D,
  \item Suites récurrentes,
  \item Systèmes linéaires - Pivot de Gauss,
  \item Générateurs aléatoires, lois uniformes, simulation de lois discrètes,
  \item Introduction aux équations différentielles,
  \item Quelques problèmes de mathématiques appliquées.
\end{itemize}
}
{Avoir suivi les modules de mathématiques de première année.}
{Se familiariser avec le logiciel de calcul matriciel Scilab.}
{Ressources} 
{Biblio} 
 
\vfill

%==========================================================================================

\module[codeApogee={3AG23}, 
titre={Anglais 3}, 
COURS={}, 
TD={24}, 
TP={}, 
CTD={}, 
TOTAL={24}, 
SEMESTRE={Semestre 3}, 
COEFF={3}, 
ECTS={3}, 
MethodeEval={ContrÃŽle continu et terminal}, 
ModalitesCCSemestreUn={CC et CT}, 
ModalitesCCSemestreDeux={CT}, 
%CalculNFSessionUne={$\frac{(CC+2*CT)}{3}$}, 
%CalculNFSessionDeux={CT}, 
NoteEliminatoire={}, 
nomPremierResp={Sylvain Gendron}, 
emailPremierResp={sylvain.gendron@univ-orleans.fr},
nomSecondResp={Vittoria Pierfelice},
emailSecondResp={vittoria.pierfelice@univ-orleans.fr}, 
langue={Français}, 
nbPrerequis={}, 
objectifs={true},
ressources={false}, 
bibliographie={false}] 
{
Unité obligatoire. 
}
{
Travail de compréhension et d'expression à partir de documents authentiques simples et/ou courts portant sur des innovations technologiques, des découvertes et avancées scientifiques.
}
{Avoir suivi Anglais 1 \& 2 ou environ 450 heures de formation équivalente.}
{Découvrir les bases de l'anglais scientifique et les utiliser à l'écrit et à l'oral.}
{Ressources} 
{Biblio} 
 
\vfill

%==========================================================================================

\module[codeApogee={3IF02}, 
titre={Algorithmique et programmation 3}, 
COURS={24}, 
TD={36}, 
TP={}, 
CTD={}, 
TOTAL={60}, 
SEMESTRE={Semestre 3}, 
COEFF={6}, 
ECTS={6}, 
MethodeEval={Contrôle continu et terminal}, 
ModalitesCCSemestreUn={CC et CT}, 
ModalitesCCSemestreDeux={CT}, 
%CalculNFSessionUne={$\frac{(CC+2*CT)}{3}$}, 
%CalculNFSessionDeux={CT}, 
NoteEliminatoire={}, 
nomPremierResp={Mathieu Liedloff}, 
emailPremierResp={mathieu.liedloff@univ-orleans.fr}, 
nomSecondResp={Vittoria Pierfelice},
emailSecondResp={vittoria.pierfelice@univ-orleans.fr}, 
langue={Français}, 
nbPrerequis={}, 
objectifs={true},
ressources={false}, 
bibliographie={false}]
{
Unité optionnelle. Option 2 (1 parmi 3).
}
{Présentation de l'approche objet (valeurs + message), bases de conception/analyse objet.
Notions de classes, méthodes, attributs, encapsulation, héritage (simple), interface, classes internes, exceptions... Mise en \oe uvre des interfaces graphiques et de la programmation événementielle.}
{Programmation impérative, algorithmes et structures de données (algorithmique et programmation 1 et 2).}
{Maîtrise des bases de la conception et de la programmation objet.}
{Ressources} 
{Biblio} 
 
\vfill

%==========================================================================================

\module[codeApogee={3MT12}, 
titre={Géométrie du plan et de l'espace}, 
COURS={24}, 
TD={36}, 
TP={}, 
CTD={}, 
TOTAL={60}, 
SEMESTRE={Semestre 3}, 
COEFF={6}, 
ECTS={6}, 
MethodeEval={Contrôle continu et terminal}, 
ModalitesCCSemestreUn={CC et CT}, 
ModalitesCCSemestreDeux={CT}, 
%CalculNFSessionUne={$\frac{(CC+2*CT)}{3}$}, 
%CalculNFSessionDeux={CT}, 
NoteEliminatoire={}, 
nomPremierResp={Jean Renault}, 
emailPremierResp={jean.renault@univ-orleans.fr}, 
nomSecondResp={Vittoria Pierfelice},
emailSecondResp={vittoria.pierfelice@univ-orleans.fr}, 
langue={Français}, 
nbPrerequis={0}, 
objectifs={true},
ressources={false}, 
bibliographie={false}]
{
Unité optionnelle. Option 2 (1 parmi 3).
}
{\begin{itemize}
\item Relations d'équivalence et quotients,
\item actions de groupes,
\item notions de géométrie affine : points et vecteurs,
\item repères, barycentres,
\item droites et plans dans l'espace,
\item applications affines,
\item géométrie dans $\mathbb{C}$.
\end{itemize}
}
{}
{}
{Ressources} 
{Biblio} 
 
\vfill

%==========================================================================================

\module[codeApogee={3PY01}, 
titre={Champs électrostatiques}, 
COURS={19}, 
TD={20}, 
TP={9}, 
CTD={}, 
TOTAL={48}, 
SEMESTRE={Semestre 3}, 
COEFF={6}, 
ECTS={6}, 
MethodeEval={Contrôle continu et terminal}, 
ModalitesCCSemestreUn={}, 
ModalitesCCSemestreDeux={}, 
%CalculNFSessionUne={$\frac{(CC+2*CT)}{3}$}, 
%CalculNFSessionDeux={CT}, 
NoteEliminatoire={}, 
nomPremierResp={Gerald Kneller},
emailPremierResp={gerald.kneller@univ-orleans.fr}, 
nomSecondResp={Vittoria Pierfelice},
emailSecondResp={vittoria.pierfelice@univ-orleans.fr}, 
langue={Français},
nbPrerequis={0}, 
objectifs={true},
ressources={false}, 
bibliographie={false}] 
{
Unité optionnelle. Option 1 (1 parmi 3).
}
{}
{}
{}
{Ressources} 
{Biblio}
 
\vfill

%==========================================================================================

\module[codeApogee={3IF03}, 
titre={Bases de données et internet}, 
COURS={12}, 
TD={}, 
TP={24}, 
CTD={}, 
TOTAL={36}, 
SEMESTRE={Semestre 3}, 
COEFF={5}, 
ECTS={5}, 
MethodeEval={Contrôle continu et terminal}, 
ModalitesCCSemestreUn={CC et CT}, 
ModalitesCCSemestreDeux={CT}, 
%CalculNFSessionUne={$\frac{(CC+2*CT)}{3}$}, 
%CalculNFSessionDeux={CT}, 
NoteEliminatoire={}, 
nomPremierResp={Jérôme Durand-Lose},
emailPremierResp={jerome.durand-lose@univ-orleans.fr}, 
nomSecondResp={Vittoria Pierfelice},
emailSecondResp={vittoria.pierfelice@univ-orleans.fr}, 
langue={Français}, 
nbPrerequis={}, 
ressources={false}, 
bibliographie={false}] 
{
Unité optionnelle. Option 2 (1 parmi 3). 
} 
{Architecture LAMP (Linux, Apache, MySQL, PHP). Modélisation d'une base de donnée : modélisation conceptuelle (entité-association) ; modélisation logique (relationnelle). Manipulation de données avec SQL. Structuration de pages web statiques et dynamiques. Réalisation d'une application web dynamique (type PHP / MySQL).}
{Maîtrise des bases de l'algorithmique et de la programmation (pour la réalisation de l'application web).}
{Être à même de concevoir et réaliser une application web dynamique utilisant une base de données relationnelles.} 
{Ressources} 
{Biblio} 
 
\vfill
%========================================================================================== 

\module[codeApogee={3MT13}, 
titre={Calculus \& calcul formel}, 
COURS={}, 
TD={}, 
TP={}, 
CTD={48}, 
TOTAL={48}, 
SEMESTRE={Semestre 3}, 
COEFF={5}, 
ECTS={5}, 
MethodeEval={Contrôle continu et terminal}, 
ModalitesCCSemestreUn={CC et CT}, 
ModalitesCCSemestreDeux={CT}, 
%CalculNFSessionUne={$\frac{(CC+2*CT)}{3}$}, 
%CalculNFSessionDeux={CT}, 
NoteEliminatoire={}, 
nomPremierResp={Carine Lucas}, 
emailPremierResp={carine.lucas@univ-orleans.fr}, 
nomSecondResp={Vittoria Pierfelice},
emailSecondResp={vittoria.pierfelice@univ-orleans.fr}, 
langue={Français},
nbPrerequis={0}, 
objectifs={true},
ressources={false}, 
bibliographie={false}] 
{
Unité optionnelle. Option 2 (1 parmi 3). 
} 
{
\begin{itemize}
\item La partie "calculus" permet de passer en revue les principaux points calculatoires qui doivent être maîtrisés pour la suite de la licence. Elle traite des fonctions usuelles, suites, dérivation, intégration, calculs de primitives, équations différentielles et d'algèbre (matrices, déterminants).
\item La partie "calcul formel" est faite sur le logiciel "Sage". Après une prise en main du logiciel et de son mode de programmation, plusieurs exemples sont traités, comme les graphiques, les suites récurrentes, les résolutions exactes de systèmes linéaires et des équations différentielles.
\end{itemize}
}
{}
{Le but de ce module est d'aider les étudiants à mieux appréhender les calculs.} 
{Ressources} 
{Biblio}  
 
\vfill

%==========================================================================================

\module[codeApogee={3PY05}, 
titre={Mécanique des solides et vibration}, 
COURS={}, 
TD={}, 
TP={}, 
CTD={48}, 
TOTAL={48}, 
SEMESTRE={Semestre 3}, 
COEFF={6}, 
ECTS={6}, 
MethodeEval={Contrôle continu et terminal}, 
ModalitesCCSemestreUn={CC et CT}, 
ModalitesCCSemestreDeux={CT}, 
%CalculNFSessionUne={$\frac{(CC+2*CT)}{3}$}, 
%CalculNFSessionDeux={CT}, 
NoteEliminatoire={}, 
nomPremierResp={Isabelle Rannou}, 
emailPremierResp={isabelle.rannou@univ-orleans.fr}, 
nomSecondResp={Vittoria Pierfelice},
emailSecondResp={vittoria.pierfelice@univ-orleans.fr}, 
langue={Français},
nbPrerequis={}, 
objectifs={true},
ressources={false}, 
bibliographie={false}] 
{
Unité optionnelle. Option 2 (1 parmi 3). 
} 
{
Cinématique : 
\begin{itemize}
\item translation et rotation d'un solide ; 
\item géométrie de masses ; 
\item éléments cinétiques d'un système de points mécaniques et d'un solide indéformable ; 
\item théorèmes généraux : étude dynamique, étude énergétique ; 
\item contact entre solides et lois de frottement.
\end{itemize}
}
{Mécanique du point; Mathématiques pour sciences physiques 1 \& 2}
{Pouvoir mettre en équation, analyser théoriquement et prédire l'évolution d'un solide indéformable simple et d'un système mécanique dans leurs mouvements.} 
{Ressources} 
{Biblio} 

\vfill

%==========================================================================================
% Semestre 4
%==========================================================================================

\module[codeApogee={4AG24},
titre={Anglais 4}, 
COURS={}, 
TD={24}, 
TP={}, 
CTD={}, 
TOTAL={24}, 
SEMESTRE={Semestre 4}, 
COEFF={3}, 
ECTS={3}, 
MethodeEval={Contrôle continu et terminal}, 
ModalitesCCSemestreUn={CC et CT}, 
ModalitesCCSemestreDeux={CT}, 
%CalculNFSessionUne={$\frac{(CC+2*CT)}{3}$}, 
%CalculNFSessionDeux={CT}, 
NoteEliminatoire={}, 
nomPremierResp={Michèle Cimolino}, 
emailPremierResp={michele.cimolino@univ-orleans.fr}, 
nomSecondResp={Noureddine El Jaouhari}, 
emailSecondResp={noureddine.el-jaouhari@univ-orleans.fr }, 
langue={Français}, 
nbPrerequis={}, 
descriptionCourte={true}, 
descriptionLongue={true}, 
objectifs={true}, 
ressources={false}, 
bibliographie={false}] 
{
Unité obligatoire. 
} 
{Travail de compréhension et d'expression à partir de documents authentiques simples et/ou courts portant sur des innovations technologiques, des découvertes et avancées scientifiques. Supports : vidéo, audio, articles de presse.}
{Avoir suivi Anglais 3 ou environ 450 heures de formation équivalente.}
{Analyser dans une langue simple et cohérente les rapports entre science et société à l'écrit et à l'oral (niveau européen : B1+)}
{Ressources}
{Biblio}

\vfill

%==========================================================================================

\module[codeApogee={4MT05},
titre={Suites et séries de fonctions}, 
COURS={24}, 
TD={24}, 
TP={}, 
CTD={}, 
TOTAL={48}, 
SEMESTRE={Semestre 4}, 
COEFF={5}, 
ECTS={5}, 
MethodeEval={Contrôle continu et terminal}, 
ModalitesCCSemestreUn={CC et CT}, 
ModalitesCCSemestreDeux={CT}, 
%CalculNFSessionUne={$\frac{(CC+2*CT)}{3}$}, 
%CalculNFSessionDeux={CT}, 
NoteEliminatoire={}, 
nomPremierResp={Noureddine El Jaouhari},
emailPremierResp={noureddine.el-jaouhari@univ-orleans.fr},
nomSecondResp={},
emailSecondResp={},
langue={Français}, 
nbPrerequis={}, 
objectifs={true},
ressources={false}, 
bibliographie={false}] 
{
Unité obligatoire. 
} 
{
\begin{itemize} 
  \item Suites et séries de fonctions,
  \item Séries entières, séries de Fourier,
  \item Intégrales dépendant d'un paramètre.
\end{itemize}
}
{Avoir suivi Analyse 2 au semestre 3.}
{Apprendre à manipuler des suites de fonctions et des intégrales dépendant d'un paramètre.}
{Ressources} 
{Biblio} 
 
\vfill

%==========================================================================================

\module[codeApogee={4MT09},
titre={Programmation fonctionnelle}, 
COURS={24}, 
TD={36}, 
TP={}, 
CTD={}, 
TOTAL={60}, 
SEMESTRE={Semestre 4}, 
COEFF={6}, 
ECTS={6}, 
MethodeEval={Contrôle continu et terminal}, 
ModalitesCCSemestreUn={CC et CT}, 
ModalitesCCSemestreDeux={CT}, 
%CalculNFSessionUne={$\frac{(CC+2*CT)}{3}$}, 
%CalculNFSessionDeux={CT}, 
NoteEliminatoire={}, 
nomPremierResp={Frédéric Dabrowski}, 
emailPremierResp={frederic.dabrowski@univ-orleans.fr}, 
nomSecondResp={Noureddine El Jaouhari}, 
emailSecondResp={noureddine.el-jaouhari@univ-orleans.fr }, 
langue={Français}, 
nbPrerequis={}, 
objectifs={true},
ressources={false}, 
bibliographie={false}]
{Unité obligatoire}
{Présentation générale du langage fonctionnel utilisé. Expressions, valeurs et types de base. Définitions locales, liaisons et environnements. Expressions et valeurs fonctionnelles à une variable. Définitions globales, entrées-sorties, compilation en ligne de commande. Fonctions d'ordre supérieur. Filtrage, tuples. Polymorphisme et inférence de type. Fonctions récursives. Listes. Types composés : type enregistrement, type somme (polymorphes récursifs). Structures de données et algorithmes : tris, arbres binaires, arbres binaires de recherche, arbres équilibrés.}
{Mathématiques élémentaires dont preuve par récurrence. Utilisation élémentaire d'un environnement Unix.}
{Prise en main d'un  des langages de  programmation fonctionnelle et des notions de programmation associée. Développement d'une application autonome complète.} 
{Ressources} 
{Biblio} 
 
\vfill

%==========================================================================================

\module[codeApogee={4MT09},
titre={Analyse des données}, 
COURS={}, 
TD={}, 
TP={}, 
CTD={24}, 
TOTAL={24}, 
SEMESTRE={Semestre 4}, 
COEFF={3}, 
ECTS={3}, 
MethodeEval={Contrôle continu et terminal}, 
ModalitesCCSemestreUn={CC et CT}, 
ModalitesCCSemestreDeux={CT}, 
%CalculNFSessionUne={$\frac{(CC+2*CT)}{3}$}, 
%CalculNFSessionDeux={CT}, 
NoteEliminatoire={}, 
nomPremierResp={Sophie Jacquot}, 
emailPremierResp={sophie.jacquot@univ-orleans.fr}, 
nomSecondResp={Noureddine El Jaouhari}, 
emailSecondResp={noureddine.el-jaouhari@univ-orleans.fr}, 
langue={Français}, 
nbPrerequis={0}, 
objectifs={true},
ressources={false}, 
bibliographie={false}] 
{ 
Unité obligatoire. 
} 
{
\begin{itemize} 
  \item Initiation aux méthodes élémentaires d'analyse multidimensionnelle. En quantitatif, Analyse en Composantes Principales (ACP) ; en qualitatif, Analyse Factorielle des Correspondances (AFC) ;
  \item Applications à des jeux de données exemples (avec le logiciel R).
\end{itemize}
}
{}
{}
{Ressources} 
{Biblio} 

\vfill
%==========================================================================================

\module[codeApogee={},
titre={Unité d'enseignement libre}, 
COURS={}, 
TD={20}, 
TP={}, 
CTD={}, 
TOTAL={20}, 
SEMESTRE={Semestre 4}, 
COEFF={3}, 
ECTS={3}, 
MethodeEval={Contrôle continu et terminal}, 
ModalitesCCSemestreUn={CC et CT}, 
ModalitesCCSemestreDeux={CT}, 
%CalculNFSessionUne={$\frac{(CC+2*CT)}{3}$}, 
%CalculNFSessionDeux={CT}, 
NoteEliminatoire={}, 
nomPremierResp={Noureddine El Jaouhari},
emailPremierResp={noureddine.el-jaouhari@univ-orleans.fr},
nomSecondResp={},
emailSecondResp={},
langue={Français}, 
nbPrerequis={0}, 
objectifs={true},
ressources={true}, 
bibliographie={false}] 
{ 
Unité obligatoire. 
} 
{L'unité d'ouverture est à choisir, en début du semestre, parmi la centaine d'enseignements dédiés à cet usage et offerts par toutes les composantes de l'université (Sciences, Droit-
\'Economie-Gestion, Sport). Voici quelques exemples d'unités d'ouverture :
\begin{itemize}
\item Sport.
\item Traitement de signal et d'image.
\item Droit de l'informatique.
\item Problèmes économiques contemporains.
\item Histoire du cinéma, histoire des arts.
\item Enseigner : posture et identité professionnelles.
\item Lecture critique du réchauffement climatique.
\item Maîtriser son expression ; les enjeux de la communication orale : le corps, l'espace, la voix.
\end{itemize}
} 
{}
{Comprendre comment ce qu'on apprend dans le cadre d'un diplôme déjà très spécialisé ; s'insérer dans le large champ des connaissances et des savoirs auxquels on sera confronté dans son expérience professionnelle ou personnelle.}
{La page du site de l'université dédiée aux unités d'ouverture : http://www.univ-orleans.fr/scolarite/inscriptions/?page=2} 
{Biblio} 

\vfill

%==========================================================================================

\module[codeApogee={4MT08}, 
titre={Probabilités discrètes}, 
COURS={18}, 
TD={36}, 
TP={}, 
CTD={}, 
TOTAL={54}, 
SEMESTRE={Semestre 4}, 
COEFF={5}, 
ECTS={5}, 
MethodeEval={Contrôle continu et terminal}, 
ModalitesCCSemestreUn={CC et CT}, 
ModalitesCCSemestreDeux={CT}, 
%CalculNFSessionUne={$\frac{(CC+2*CT)}{3}$}, 
%CalculNFSessionDeux={CT}, 
NoteEliminatoire={}, 
nomPremierResp={Pierre Debs}, 
emailPremierResp={pierre.debs@univ-orleans.fr}, 
nomSecondResp={Noureddine El Jaouhari}, 
emailSecondResp={noureddine.el-jaouhari@univ-orleans.fr}, 
langue={Français}, 
nbPrerequis={},  
objectifs={true}, 
ressources={false}, 
bibliographie={false}]
{
Unité obligatoire. 
} 
{
\begin{itemize} 
\item Espace des possibles - Modélisation de phénomènes aléatoires ;
\item Notions de dénombrement ;
\item Calculs des probabilités : union disjointe, formule des probabilités totales, formule du crible
\item Probabilités conditionnelles, indépendance, formule de Bayes, Variables aléatoires discrètes - Lois usuelles - Moments ;
\item Sensibilisation à la loi des grands nombres.
\end{itemize} 
}
{Avoir suivi les modules de mathématiques du semestre 3.}
{S'initier aux probabilités discrètes.}
{Ressources} 
{Biblio} 
 
\vfill

%==========================================================================================

\module[codeApogee={4IF04},
titre={Algorithmique et combinatoire des structures discrètes}, 
COURS={24}, 
TD={36}, 
TP={}, 
CTD={}, 
TOTAL={60}, 
SEMESTRE={Semestre 4}, 
COEFF={5}, 
ECTS={5}, 
MethodeEval={Contrôle continu et terminal}, 
ModalitesCCSemestreUn={CC et CT}, 
ModalitesCCSemestreDeux={CT}, 
%CalculNFSessionUne={$\frac{(CC+2*CT)}{3}$}, 
%CalculNFSessionDeux={CT}, 
NoteEliminatoire={}, 
nomPremierResp={Ioan Todinca}, 
emailPremierResp={ioan.todinca@univ-orleans.fr}, 
nomSecondResp={Noureddine El Jaouhari}, 
emailSecondResp={noureddine.el-jaouhari@univ-orleans.fr}, 
langue={Français}, 
nbPrerequis={}, 
objectifs={true},
ressources={false}, 
bibliographie={false}] 
% ******* Texte introductif
{
Unité obligatoire. 
} 
{
\begin{itemize}
\item Dénombrement. Relation d'ordre partiel : calcul de la fermeture transitive, tri topologique.
\item Graphes : parcours, plus court chemin, arbres recouvrants de poids minimum, flot.
\end{itemize}}
{Algorithmique et programmation élémentaires.}
{Modélisation et résolution de problèmes à l'aide de structures discrètes.}
{Ressources}
{Biblio}
 
\vfill

%==========================================================================================
% Semestre 5
%==========================================================================================

\module[codeApogee={5AG35},
titre={Anglais 5}, 
COURS={}, 
TD={24}, 
TP={}, 
CTD={}, 
TOTAL={24}, 
SEMESTRE={Semestre 5}, 
COEFF={3}, 
ECTS={3}, 
MethodeEval={Contrôle continu et terminal}, 
ModalitesCCSemestreUn={CC et CT}, 
ModalitesCCSemestreDeux={CT}, 
%CalculNFSessionUne={$\frac{(CC+2*CT)}{3}$}, 
%CalculNFSessionDeux={CT}, 
NoteEliminatoire={}, 
nomPremierResp={Hervé Perreau}, 
emailPremierResp={herve.perreau@univ-orleans.fr}, 
nomSecondResp={Nawfal Elhage Hassan}, 
emailSecondResp={nawfal.elhage\_hassan@univ-orleans.fr}, 
langue={Français}, 
nbPrerequis={}, 
objectifs={true},
ressources={false}, 
bibliographie={false}] 
{ 
Unité obligatoire. 
} 
{
Travail de compréhension et d'expression à partir de documents authentiques longs et/ou complexes portant sur des innovations technologiques, des découvertes et avancées scientifiques.
}
{Avoir suivi anglais 3 \& 4 ou environ 500 heures de formation équivalente.}
{Comprendre l'information exprimée dans des messages complexes sur le domaine des sciences et technologies et s'exprimer sur ce même domaine à l'écrit dans un registre de langue approprié.}
{Ressources} 
{Biblio} 
 
\vfill

%==========================================================================================

\module[codeApogee={5IF02},
titre={Insertion professionnelle}, 
COURS={}, 
TD={}, 
TP={}, 
CTD={12}, 
TOTAL={12}, 
SEMESTRE={Semestre 5}, 
COEFF={3}, 
ECTS={3}, 
MethodeEval={Contrôle continu et terminal}, 
ModalitesCCSemestreUn={CC et CT}, 
ModalitesCCSemestreDeux={Pas de seconde session}, 
%CalculNFSessionUne={$\frac{(CC+2*CT)}{3}$}, 
%CalculNFSessionDeux={CT}, 
NoteEliminatoire={}, 
nomPremierResp={Pôle Avenir}, 
emailPremierResp={poleavenir.ove@univ-orleans.fr}, 
nomSecondResp={Nawfal Elhage Hassan}, 
emailSecondResp={nawfal.elhage\_hassan@univ-orleans.fr}, 
langue={Français}, 
nbPrerequis={0},
ressources={false}, 
bibliographie={false}] 
{
Unité obligatoire. 
} 
{Préparation de CV, de lettres de motivation, gestion de carrière.}
{}
{}
{Ressources} 
{Biblio} 
 
\vfill

%==========================================================================================

\module[codeApogee={5MT05},
titre={Analyse numérique matricielle}, 
COURS={}, 
TD={}, 
TP={}, 
CTD={24}, 
TOTAL={24}, 
SEMESTRE={Semestre 5}, 
COEFF={3}, 
ECTS={3}, 
MethodeEval={Contrôle continu et terminal}, 
ModalitesCCSemestreUn={CC et CT}, 
ModalitesCCSemestreDeux={CT}, 
%CalculNFSessionUne={$\frac{(CC+2*CT)}{3}$}, 
%CalculNFSessionDeux={CT}, 
NoteEliminatoire={}, 
nomPremierResp={Carine Lucas}, 
emailPremierResp={carine.lucas@univ-orleans.fr}, 
nomSecondResp={Nawfal Elhage Hassan}, 
emailSecondResp={nawfal.elhage\_hassan@univ-orleans.fr}, 
langue={Français}, 
nbPrerequis={0},
ressources={false}, 
bibliographie={false}] 
{
Unité obligatoire. 
} 
{
\begin{itemize} 
  \item Inversion de matrices,
  \item Décompositions de matrices,
  \item Moindres carrés,
  \item Mise en \oe uvre avec le logiciel Scilab de calcul matriciel.
\end{itemize}
} 
{}
{}
{Ressources} 
{Biblio} 
 
\vfill

%==========================================================================================
\module[codeApogee={5MT04}, 
titre={Topologie des espaces métriques}, 
COURS={24}, 
TD={36}, 
TP={}, 
CTD={}, 
TOTAL={60}, 
SEMESTRE={Semestre 5}, 
COEFF={6}, 
ECTS={6}, 
MethodeEval={Contrôle continu et terminal}, 
ModalitesCCSemestreUn={CC et CT}, 
ModalitesCCSemestreDeux={CT}, 
%CalculNFSessionUne={$\frac{(CC+2*CT)}{3}$}, 
%CalculNFSessionDeux={CT}, 
NoteEliminatoire={}, 
nomPremierResp={Nawfal Elhage Hassan}, 
emailPremierResp={nawfal.elhage\_hassan@univ-orleans.fr}, 
nomSecondResp={}, 
emailSecondResp={},
langue={Français}, 
nbPrerequis={},
ressources={false}, 
bibliographie={false}]
{
Unité obligatoire. 
} 
{
\begin{itemize} 
  \item Distances, normes
  \item Convergences de suites, continuité
  \item Espaces complets
  \item Espaces compacts
  \item Espaces vectoriels normés, applications linéaires continues
  \item Connexité
\end{itemize}
}
{Avoir suivi les unités de mathématiques des semestres précédents.}
{}
{Ressources} 
{Biblio}

\vfill

%==========================================================================================

\module[codeApogee={5MT07},
titre={Mesure et intégration}, 
COURS={24}, 
TD={36}, 
TP={}, 
CTD={}, 
TOTAL={60}, 
SEMESTRE={Semestre 5}, 
COEFF={6}, 
ECTS={6}, 
MethodeEval={Contrôle continu et terminal}, 
ModalitesCCSemestreUn={CC et CT}, 
ModalitesCCSemestreDeux={CT}, 
%CalculNFSessionUne={$\frac{(CC+2*CT)}{3}$}, 
%CalculNFSessionDeux={CT}, 
NoteEliminatoire={}, 
nomPremierResp={Athanasios Batakis}, 
emailPremierResp={athanasios.batakis@univ-orleans.fr}, 
nomSecondResp={Nawfal Elhage Hassan}, 
emailSecondResp={nawfal.elhage\_hassan@univ-orleans.fr}, 
langue={Français}, 
nbPrerequis={0},
ressources={false}, 
bibliographie={false}]
% ******* Texte introductif
{
Unité obligatoire. 
} 
{
\begin{itemize} 
  \item Tribus, mesures ; mesure de Lebesgue
  \item Intégrale par rapport à une mesure
  \item Théorèmes de convergence monotone, de convergence dominée 
  \item Théorème de Fubini
  \item Convolution
  \item Transformée de Fourier
  \item Espaces $L^p$.
\end{itemize} 
} 
{}
{}
{Ressources}
{Biblio}

\vfill

%==========================================================================================

\module[codeApogee={},
titre={Unité d'enseignement libre}, 
COURS={}, 
TD={}, 
TP={}, 
CTD={20}, 
TOTAL={20}, 
SEMESTRE={Semestre 5}, 
COEFF={3}, 
ECTS={3}, 
MethodeEval={Contrôle continu et terminal}, 
ModalitesCCSemestreUn={CC et CT}, 
ModalitesCCSemestreDeux={CT}, 
%CalculNFSessionUne={$\frac{(CC+2*CT)}{3}$}, 
%CalculNFSessionDeux={CT}, 
NoteEliminatoire={}, 
nomPremierResp={Nawfal Elhage Hassan}, 
emailPremierResp={nawfal.elhage\_hassan@univ-orleans.fr}, 
nomSecondResp={}, 
emailSecondResp={},
langue={Français}, 
nbPrerequis={0}, 
descriptionCourte={true}, 
descriptionLongue={true}, 
objectifs={true}, 
ressources={true}, 
bibliographie={false}] 
% ******* Texte introductif
{
Unité obligatoire. 
} 
{L'unité d'ouverture est à choisir, en début du semestre, parmi la centaine d'enseignements dédiés à cet usage et offerts par toutes les composantes de l'université (Sciences, Droit-
\'Economie-Gestion, Sport). Voici quelques exemples d'unités d'ouverture :
\begin{itemize}
\item Sport.
\item Traitement de signal et d'image.
\item Droit de l'informatique.
\item Problèmes économiques contemporains.
\item Histoire du cinéma, histoire des arts.
\item Enseigner : posture et identité professionnelles.
\item Lecture critique du réchauffement climatique.
\item Maîtriser son expression ; les enjeux de la communication orale : le corps, l'espace, la voix.
\end{itemize}
} 
{}
{Comprendre comment ce qu'on apprend dans le cadre d'un diplôme déjà très spécialisé ; s'insérer dans le large champ des connaissances et des savoirs auxquels on sera confronté dans son expérience professionnelle ou personnelle.}
{La page du site de l'université dédiée aux unités d'ouverture : http://www.univ-orleans.fr/scolarite/inscriptions/?page=2} 
{Biblio} 

\vfill

%==========================================================================================

\module[codeApogee={5IF14},
titre={Analyse des algorithmes}, 
COURS={14}, 
TD={24}, 
TP={}, 
CTD={}, 
TOTAL={38}, 
SEMESTRE={Semestre 5}, 
COEFF={3}, 
ECTS={3}, 
MethodeEval={Contrôle continu et terminal}, 
ModalitesCCSemestreUn={CC et CT}, 
ModalitesCCSemestreDeux={CT}, 
%CalculNFSessionUne={$\frac{(CC+2*CT)}{3}$}, 
%CalculNFSessionDeux={CT}, 
NoteEliminatoire={}, 
nomPremierResp={Mathieu Liedloff},
emailPremierResp={mathieu.liedloff@univ-orleans.fr}, 
nomSecondResp={Nawfal Elhage Hassan}, 
emailSecondResp={nawfal.elhage\_hassan@univ-orleans.fr}, 
langue={Français}, 
nbPrerequis={}, 
descriptionCourte={true}, 
descriptionLongue={true}, 
objectifs={true}, 
ressources={false}, 
bibliographie={false}] 
{
Unité obligatoire. 
} 
{Complexité d'un algorithme. Diviser pour régner. Algorithmes gloutons. Programmation dynamique. Algorithmes de tri ; arbres binaires de recherche.}
{Algorithmique et programmation élémentaire.}
{\begin{itemize}
\item Maîtriser les techniques algorithmiques de base (diviser pour régner, algorithmes gloutons...).
\item Savoir analyser la complexité d'un algorithme.
\end{itemize}}
{Ressources} 
{Biblio} 

\vfill

%========================================================================================== 

\module[codeApogee={5IF15},
titre={Programmation avancée et structures de données},
COURS={18}, 
TD={30}, 
TP={}, 
CTD={}, 
TOTAL={48}, 
SEMESTRE={Semestre 5}, 
COEFF={3}, 
ECTS={3}, 
MethodeEval={Contrôle continu et terminal}, 
ModalitesCCSemestreUn={CC et CT}, 
ModalitesCCSemestreDeux={CT}, 
%CalculNFSessionUne={$\frac{(CC+2*CT)}{3}$}, 
%CalculNFSessionDeux={CT}, 
NoteEliminatoire={}, 
nomPremierResp={Jérôme Durand-Lose}, 
emailPremierResp={jerome.durand-lose@univ-orleans.fr}, 
nomSecondResp={François James}, 
emailSecondResp={francois.james@univ-orleans.fr}, 
langue={Français}, 
nbPrerequis={}, 
descriptionCourte={true}, 
descriptionLongue={true}, 
objectifs={true}, 
ressources={false}, 
bibliographie={false}] 
{
Unité obligatoire. 
} 
{
Introduction au langage ADA. Types non contraints et pointeurs. Unités de compilation, modularité, généricité. Tâches, rendez-vous, type protégés, répartition. Types étiquetés, programmation orientée objet, programmation par classe, héritage, héritage multiple. Interfaçage : autres langages, interface graphique, serveur web,...}
{Maîtrise de l'algorithmique de base (y compris les techniques d'assertion et d'invariant) et des structures statiques. Connaissance des principes de gestion mémoire, de la notion d'état, de l'affectation. Expérience des entrées sorties (non-)bufferisées.}
{\begin{itemize}
\item Combiner plusieurs méthodes de programmation au sein d'un même langage.
\item Intégrer la notion d'abstraction des données et des traitements.
\item Comprendre l'intérêt du typage fort et de l'induction de types.
\item Arbitrer entre des solutions statiques et dynamiques.
\end{itemize}}
{Ressources} 
{Biblio} 

\vfill

%==========================================================================================
% Semestre 6
%==========================================================================================

\module[codeApogee={6AG36}, 
titre={Anglais 6}, 
COURS={}, 
TD={24}, 
TP={}, 
CTD={}, 
TOTAL={24}, 
SEMESTRE={Semestre 6}, 
COEFF={3}, 
ECTS={3}, 
MethodeEval={Contrôle continu et terminal}, 
ModalitesCCSemestreUn={CC et CT}, 
ModalitesCCSemestreDeux={CT}, 
%CalculNFSessionUne={$\frac{(CC+2*CT)}{3}$}, 
%CalculNFSessionDeux={CT}, 
NoteEliminatoire={}, 
nomPremierResp={Sylvain Gendron}, 
emailPremierResp={sylvain.gendron@univ-orleans.fr}, 
nomSecondResp={Pierre Debs}, 
emailSecondResp={pierre.debs@univ-orleans.fr}, 
langue={Français}, 
nbPrerequis={0}, 
descriptionCourte={true}, 
descriptionLongue={true}, 
objectifs={true}, 
ressources={false}, 
bibliographie={false}] 
{
Unité obligatoire. 
} 
{
Travail de compréhension et d'expression à partir de documents authentiques longs et/ou complexes portant sur des innovations technologiques, des découvertes et des avancées scientifiques.
}
{Avoir suivi Anglais 5 ou environ 500 heures de formation équivalente.}
{Comprendre l'information exprimée dans des messages complexes sur le domaine des sciences et technologies, et s'exprimer sur ce même domaine à l'oral avec un degré suffisant de spontanéité et de fluidité (niveau européen B2).}
{Ressources} 
{Biblio}
 
\vfill

%==========================================================================================

\module[codeApogee={6MT04},
titre={Calcul différentiel et optimisation}, 
COURS={24}, 
TD={36}, 
TP={}, 
CTD={}, 
TOTAL={60}, 
SEMESTRE={Semestre 6}, 
COEFF={6}, 
ECTS={6}, 
MethodeEval={Contrôle continu et terminal}, 
ModalitesCCSemestreUn={CC et CT}, 
ModalitesCCSemestreDeux={CT}, 
%CalculNFSessionUne={$\frac{(CC+2*CT)}{3}$}, 
%CalculNFSessionDeux={CT}, 
NoteEliminatoire={}, 
nomPremierResp={Luc Hillairet}, 
emailPremierResp={luc.hillairet@univ-orleans.fr}, 
nomSecondResp={Pierre Debs}, 
emailSecondResp={pierre.debs@univ-orleans.fr}, 
langue={Français}, 
nbPrerequis={0}, 
descriptionCourte={true}, 
descriptionLongue={true}, 
objectifs={true}, 
ressources={false}, 
bibliographie={false}] 
% ******* Texte introductif
{
Unité obligatoire. 
}
{
\begin{itemize} 
  \item Différentielle d'ordre 1,
  \item Inversion locale, théorème des fonctions implicites,
  \item Différentielle d'ordre 2, formule de Taylor,
  \item Extrema liés.
\end{itemize} 
}
{}
{}
{Ressources}
{Biblio}
 
\vfill

%==========================================================================================

\module[codeApogee={6MT06},
titre={\'Equations différentielles ordinaires : théorie et méthodes numériques}, 
COURS={24}, 
TD={24}, 
TP={}, 
CTD={}, 
TOTAL={48}, 
SEMESTRE={Semestre 6}, 
COEFF={5}, 
ECTS={5}, 
MethodeEval={Contrôle continu et terminal}, 
ModalitesCCSemestreUn={CC et CT}, 
ModalitesCCSemestreDeux={CT}, 
%CalculNFSessionUne={$\frac{(CC+2*CT)}{3}$}, 
%CalculNFSessionDeux={CT}, 
NoteEliminatoire={}, 
nomPremierResp={Stéphane Cordier}, 
emailPremierResp={stephane.cordier@univ-orleans.fr}, 
nomSecondResp={Pierre Debs}, 
emailSecondResp={pierre.debs@univ-orleans.fr}, 
langue={Français}, 
nbPrerequis={}, 
descriptionCourte={true}, 
descriptionLongue={true}, 
objectifs={true}, 
ressources={false}, 
bibliographie={false}]
{
Unité obligatoire.
}
{
\begin{itemize} 
  \item Définition et exemples
  \item Notion de solution maximale, lemme de Gronwall, applications
  \item Théorème de Cauchy-Lipschitz, théorème des bouts 
  \item Stabilité  
  \item Méthodes numériques.
\end{itemize}
}
{Unités d'analyse des semestres précédents.}
{}
{Ressources} 
{Biblio} 
 
\vfill

%==========================================================================================

\module[codeApogee={6MT07}, 
titre={Statistiques empiriques}, 
COURS={}, 
TD={}, 
TP={}, 
CTD={24}, 
TOTAL={24}, 
SEMESTRE={Semestre 6}, 
COEFF={2}, 
ECTS={2}, 
MethodeEval={Contrôle continu et terminal}, 
ModalitesCCSemestreUn={CC et CT}, 
ModalitesCCSemestreDeux={CT}, 
%CalculNFSessionUne={$\frac{(CC+2*CT)}{3}$}, 
%CalculNFSessionDeux={CT}, 
NoteEliminatoire={}, 
nomPremierResp={Didier Chauveau}, 
emailPremierResp={didier.chauveau@univ-orleans.fr}, 
nomSecondResp={Pierre Debs}, 
emailSecondResp={pierre.debs@univ-orleans.fr}, 
langue={Français}, 
nbPrerequis={}, 
descriptionCourte={true}, 
descriptionLongue={true}, 
objectifs={true}, 
ressources={false}, 
bibliographie={false}] 
{
Unité obligatoire. 
} 
{
\begin{itemize} 
  \item Notion de modèle statistique et d'estimateur, estimateurs empiriques (moments)
  \item Notions de consistance (LNG). Intervalles de confiance pour la moyenne d'échantillons gaussiens, intervalles de confiance approchés par théorème central limite.
  \item Estimation par maximum de vraisemblance
  \item Notion de tests statistiques  
  \item Application avec le logiciel R.
\end{itemize}
}
{Statistiques descriptives (unité du semestre 2).}
{}
{Ressources} 
{Biblio} 
 
\vfill

%==========================================================================================

\module[codeApogee={6ST01}, 
titre={Projet de fin d'étude}, 
COURS={}, 
TD={}, 
TP={}, 
CTD={4}, 
TOTAL={4}, 
SEMESTRE={Semestre 6}, 
COEFF={8}, 
ECTS={8}, 
MethodeEval={Contrôle continu et terminal}, 
ModalitesCCSemestreUn={CC et CT}, 
ModalitesCCSemestreDeux={CT}, 
%CalculNFSessionUne={$\frac{(CC+2*CT)}{3}$}, 
%CalculNFSessionDeux={CT}, 
NoteEliminatoire={}, 
nomPremierResp={Pierre Debs}, 
emailPremierResp={pierre.debs@univ-orleans.fr}, 
nomSecondResp={},
emailSecondResp={}, 
langue={Français}, 
nbPrerequis={0}, 
descriptionCourte={true}, 
descriptionLongue={true}, 
objectifs={true}, 
ressources={false}, 
bibliographie={false}] 
{
Unité obligatoire. 
} 
{}
{}
{}
{Ressources} 
{Biblio} 
 
\vfill

%==========================================================================================

\module[codeApogee={6MT02}, 
titre={Outils numériques}, 
COURS={}, 
TD={}, 
TP={}, 
CTD={36}, 
TOTAL={36}, 
SEMESTRE={Semestre 6}, 
COEFF={4}, 
ECTS={4}, 
MethodeEval={Contrôle continu et terminal}, 
ModalitesCCSemestreUn={CC et CT}, 
ModalitesCCSemestreDeux={CT}, 
%CalculNFSessionUne={$\frac{(CC+2*CT)}{3}$}, 
%CalculNFSessionDeux={CT}, 
NoteEliminatoire={}, 
nomPremierResp={Cécile Louchet},
emailPremierResp={cecile.louchet@univ-orleans.fr}, 
nomSecondResp={Pierre Debs}, 
emailSecondResp={pierre.debs@univ-orleans.fr}, 
langue={Français}, 
nbPrerequis={}, 
descriptionCourte={true}, 
descriptionLongue={true}, 
objectifs={true}, 
ressources={false}, 
bibliographie={false}] 
{
Unité obligatoire. 
} 
{
\begin{itemize} 
  \item Rappels sur les recherches des zéros, quadratures;
  \item interpolation;
  \item notions d'optimisation numérique : descentes de gradient.
\end{itemize} 
}
{Connaître le langage Scilab ; avoir intégré les notions d'analyse du semestre 5.}
{\'Etudier des méthodes numériques itératives de résolution de problèmes mathématiques (résoudre une équation, calculer une intégrale, représenter une courbe, minimiser une fonction) issus de modèles concrets. Mettre en \oe uvre ces méthodes sous Scilab.}
{Ressources} 
{Biblio}

\vfill

%==========================================================================================

\module[codeApogee={6MT01}, 
titre={Fonctions holomorphes}, 
COURS={}, 
TD={}, 
TP={}, 
CTD={36}, 
TOTAL={36}, 
SEMESTRE={Semestre 6}, 
COEFF={4}, 
ECTS={4}, 
MethodeEval={Contrôle continu et terminal}, 
ModalitesCCSemestreUn={CC et CT}, 
ModalitesCCSemestreDeux={CT}, 
%CalculNFSessionUne={$\frac{(CC+2*CT)}{3}$}, 
%CalculNFSessionDeux={CT}, 
NoteEliminatoire={}, 
nomPremierResp={Michel Zinsmeister},
emailPremierResp={michel.zinsmeister@univ-orleans.fr}, 
nomSecondResp={Pierre Debs}, 
emailSecondResp={pierre.debs@univ-orleans.fr}, 
langue={Français}, 
nbPrerequis={}, 
descriptionCourte={true}, 
descriptionLongue={true}, 
objectifs={true}, 
ressources={false}, 
bibliographie={false}] 
{
Unité obligatoire. 
} 
{
\begin{itemize} 
  \item Définition, exemples
  \item Formule intégrale de Cauchy
  \item Théorèmes de Liouville, d'Alembert, principe du maximum  
  \item Théorème des résidus, application au calcul d'intégrales.
\end{itemize}
}
{}
{S'initier à l'analyse complexe.}
{Ressources} 
{Biblio} 

 
\vfill 

%==========================================================================================

\module[codeApogee={6IF09}, 
titre={Programmation orientée objet}, 
COURS={}, 
TD={12}, 
TP={}, 
CTD={}, 
TOTAL={12}, 
SEMESTRE={Semestre 6}, 
COEFF={2}, 
ECTS={2}, 
MethodeEval={Validation par mini-projet}, 
%ModalitesCCSemestreUn={CC et CT}, 
%ModalitesCCSemestreDeux={CT}, 
%CalculNFSessionUne={$\frac{(CC+2*CT)}{3}$}, 
%CalculNFSessionDeux={CT}, 
NoteEliminatoire={}, 
nomPremierResp={Pierre Debs}, 
emailPremierResp={pierre.debs@univ-orleans.fr}, 
nomSecondResp={}, 
emailSecondResp={},
langue={Français}, 
nbPrerequis={}, 
descriptionCourte={true}, 
descriptionLongue={true}, 
objectifs={true}, 
ressources={false}, 
bibliographie={false}] 
{
Unité obligatoire. 
} 
{Utilisation et administration de systèmes d'exploitation.}
{Module d'initiation} 
{}
{Ressources} 
{Biblio} 
 
\vfill
\end{document}

%%%%%%%%%%%%%%%%%%%%%%%%%%%%%%%%%%%%%%%%%%%%%%%%%%%%%%%%%%%%%%%%%%%%%%%%%%%%%%%%%%%%%%%%%%%%

structure :
\module[]
% ******* Texte introductif
{} 
% ******* Contenu détaillé
{} 
% ******* Pré-requis
{} 
% ******* Objectifs
{}  
% ******* Ressources pédagogiques
{}
% ******* Bibliographie éventuelle
{}

