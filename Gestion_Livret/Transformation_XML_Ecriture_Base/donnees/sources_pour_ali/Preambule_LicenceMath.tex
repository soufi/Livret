\documentclass[10pt, a5paper]{report}

\usepackage[T1]{fontenc}%
\usepackage[utf8]{inputenc}% encodage utf8
\usepackage[francais]{babel}% texte français
\usepackage[top=10mm, bottom=20mm, foot=5mm, left=7mm, offset=7mm, textwidth=350pt]{geometry}
\usepackage{url}
\usepackage{init-preambule}
\usepackage[final]{pdfpages}

% Définir les couleur du document
\input{Extra/couleurLicenceMA.tex}

\begin{document}
\begin{spacing}{1.5}

\chapter*{Licence de Mathématiques}

\footnotesize
\section*{Objectifs}

	\begin{wrapfigure}{r}{0.35\textwidth}
          \vspace{-20pt}
            \begin{center}
                \begin{tikzpicture}
                    \node [rectangle, draw=couleurFonce, thick, drop shadow, fill=couleurBox, inner sep=10pt, inner ysep=10pt] (box) {
                    \begin{minipage}{0.35\textwidth}%
                        \begin{spacing}{1}
                        \begin{tabular}[t]{m{10mm}@{~~}m{30mm}@{}}
                         \multicolumn{2}{c}{\normalsize{\textbf{Fran\c{c}ois} \textbf{JAMES}}} \\
                         \multicolumn{2}{c}{\footnotesize{Professeur d'université}} \\
                         \multicolumn{2}{c}{{\scriptsize \textit{francois.james@univ-orleans.fr}}} \\
                          \multicolumn{2}{c}{\includegraphics[scale=0.7]{img/telephone.png}{\scriptsize +33 (0)2 38 41 72 32}}\\
                          \multicolumn{2}{c}{\includegraphics[scale=0.7]{img/fax.jpg}{\scriptsize +33 (0)2 38 41 72 05}}
                        \end{tabular}
                        \end{spacing}
                    \end{minipage}
                    };
                    \node[fancytitle, right=5pt, rounded corners, inner xsep=10pt] at (box.north west) {\normalsize{Directeur de la Licence}};
                \end{tikzpicture}
            \end{center}
            \vspace{-20pt}
      \end{wrapfigure}





\letterine{L}a licence Mention Mathématiques a pour but de donner aux étudiants titulaires d'un baccalauréat scientifique (Section S) une solide culture de base en mathématiques, étroitement associée à l'informatique et à leurs applications naturelles. Elle propose une formation qui s'ouvre progressivement aux divers aspects des mathématiques et aux applications multiples de cette discpline.

Sans négliger les débouchés naturels que constituent les carrières de l'enseignement et de la recherche, la licence de Mathématiques donne les bases du métier d'ingénieur en mathématiques. Ainsi s'ouvrent des carrières dans l'industrie publique ou privée, où l'on demande toujours plus d'ingénieurs mathématiciens. Un effort tout particulier est fait pour signaler aux étudiants les débouchés intéressants offerts par les mathématiques appliquées, voire les doubles compétences, en Sciences économiques, en Informatique ou en Physique. La licence de Mathématiques d'Orléans est un passeport privilégié pour le master de Mathématiques (par exemple le Master PASSION), mais aussi, en fonction des parcours choisis, certains parcours des masters de Sciences économiques ou de Physique, ou encore le master Mathématiques-Informatique MOCAHP.

La licence Mention Mathématiques propose 5 parcours :
\begin{description}
\item[MA :] Mathématiques et Applications
\item[MI :] Mathématiques et Informatique
\item[MP :] Mathématiques et Physique
\item[MASE :] Mathématiques et Applications aux Sciences \'Economiques
\item[PLURI :] Licence Pluridisciplinaire (à partir de la 3ème année)
\end{description}

%%%%%%%%%%%%%%%%%%%%%%%%%%%%%%%%%%%%%%%%%%%%%%%%%%%%%

\section*{Organisation}

\letterine{L}es trois premiers semestres comportent un important tronc commun (180h, répartis pour l'essentiel entre mathématiques et informatique). Les 4 premiers parcours (MA,MI, MP, MASE) sont donc de difficulté comparable et fournissent un socle commun de connaissances en Mathématiques et Informatique. Au fur et à mesure de l'avancée du cursus, chaque semestre est complété par 100h de cours optionnels qui déterminent le parcours de l'étudiant -- voir les détails des enseignements.

En fin de Licence, l'étudiant(e) dispose donc d'une formation comportant éventuellement une double compétence, en Physique, Informatique ou Sciences économiques. Tous les parcours débouchent naturellement sur le Master de Mathématique d'Orléans, mais le parcours choisi peut ouvrir sur d'autres masters.

Le premier semestre est commun avec la licence d'informatique. Les étudiants choisissent à la fin du semestre la mention Mathématiques ou Informatique. Le parcours MI, qui est commun aux deux mentions de licence, permet cependant de poursuivre un cursus complet mathématiques-informatique.

%%%%%%%%%%%%%%%%%%%%%%%%%%%%%%%%%%%%%%%%%%%%%%%%%%%%%

\section*{Plan réussite licence}

\letterine{U}n effort tout particulier est fait au 1er semestre pour améliorer l'intégration des étudiants dans le système universitaire et réduire ainsi le taux d'échec en 1ère année.

En résumé :
\begin{itemize}
\item Pas de cours magistraux. 
\item Les cours et les TD sont assurés par le même enseignant pour des classes de 30 étudiants maximum. 
\item Contrôle continu régulier. 
\item Les COPS : Contrôles Oraux PersonnaliséS. Le principe est de réunir les étudiants par groupe de trois et de les interroger pendant 45 minutes 4 fois dans le semestre. On donne ainsi l'opportunité à tous les étudiants de rencontrer régulièrement un enseignant afin de faire un bilan sur la qualité et l'avancement de leur travail. Ces rencontres sont sanctionnées par une note dont la moyenne sera prise en compte dans la note de contrôle continu. 
\item Cours de remise à niveau au début du semestre et cours de soutien pour les étudiants en difficulté au cours du semestre, assurés par les enseignants et par des étudiants de Master. 
\item Suivi personnalisé des étudiants.
\end{itemize}

%\subsection*{Enseignements}

%%---------------------- % % % Personnalisation des couleurs % % % ----------- ROUGE --------
\definecolor{couleurFonce}{RGB}{160,0,10} % Couleur du Code APOGEE
\definecolor{couleurClaire}{RGB}{225,135,140} % Couleur du fond de la bande
\definecolor{couleurTexte}{RGB}{255,255,255} % Couleur du texte de la bande
%------------------------------------------------------------------------------------------

\arrayrulecolor{couleurFonce}% Couleur des lignes séparatrices du tableau
\renewcommand{\arraystretch}{1.2}% Coeff appliqué à la hauteur des cellules
%\rowcolors[\hline]{ligneDébut}{couleurPaire}{couleurImpaire}% Alternance de couleur (need package xcolor)
\begin{tabular}{c|m{6cm}|cm{1cm}|cm{1cm}|cm{1cm}|cm{1cm}|}
\cline{2-6}

&
\cellcolor{couleurFonce} \color{white}\bfseries Intitul\'e & \cellcolor{couleurFonce} \color{white}\bfseries ECTS & \cellcolor{couleurFonce} \color{white}\bfseries CM & \cellcolor{couleurFonce} \color{white}\bfseries TD & \cellcolor{couleurFonce} \color{white}\bfseries TP\\ \cline{2-6}

\hline \multirow{6}{*}{\rotatebox{90}{\color{couleurFonce}\bfseries SEMESTRE 3}}
 & \color{black} \mbox{Initiation}  & \color{black} 0 & \color{black} 8 & \color{black}  & \color{black} 8 \\ \cline{2-6}
 & \cellcolor{couleurClaire} \color{couleurTexte} \mbox{Algorithmique}  & \cellcolor{couleurClaire} \color{couleurTexte} 3 & \cellcolor{couleurClaire} \color{couleurTexte} 15 & \cellcolor{couleurClaire} \color{couleurTexte} 15 & \cellcolor{couleurClaire} \color{couleurTexte}  \\ \cline{2-6}
 & \color{black} \mbox{Bases} \mbox{de} \mbox{données}  & \color{black} 6 & \color{black} 20 & \color{black} 25 & \color{black} 25 \\ \cline{2-6}
 & \cellcolor{couleurClaire} \color{couleurTexte} \mbox{Systèmes}  & \cellcolor{couleurClaire} \color{couleurTexte} 3 & \cellcolor{couleurClaire} \color{couleurTexte} 10 & \cellcolor{couleurClaire} \color{couleurTexte} 10 & \cellcolor{couleurClaire} \color{couleurTexte} 15 \\ \cline{2-6}
 & \color{black} \mbox{Réseaux}  & \color{black} 3 & \color{black} 10 & \color{black} 10 & \color{black} 15 \\ \cline{2-6}
 & \cellcolor{couleurClaire} \color{couleurTexte} \mbox{Programmation} \mbox{objet} \mbox{1}  & \cellcolor{couleurClaire} \color{couleurTexte} 6 & \cellcolor{couleurClaire} \color{couleurTexte} 20 & \cellcolor{couleurClaire} \color{couleurTexte} 25 & \cellcolor{couleurClaire} \color{couleurTexte} 25 \\ \cline{2-6}
 & \color{black} \mbox{Projet} \mbox{1}  & \color{black} 3 & \color{black}  & \color{black}  & \color{black}  \\ \cline{2-6}
 & \cellcolor{couleurClaire} \color{couleurTexte} \mbox{Simulation} \mbox{d'entreprise} \mbox{de} \mbox{gestion}  & \cellcolor{couleurClaire} \color{couleurTexte} 3 & \cellcolor{couleurClaire} \color{couleurTexte}  & \cellcolor{couleurClaire} \color{couleurTexte} 24 & \cellcolor{couleurClaire} \color{couleurTexte}  \\ \cline{2-6}
 & \color{black} \mbox{Anglais}  & \color{black} 3 & \color{black}  & \color{black} 20 & \color{black}  \\ \cline{2-6}
\hline \multirow{6}{*}{\rotatebox{90}{\color{couleurFonce}\bfseries SEMESTRE 4}}
 & \color{black} \mbox{Applications} \mbox{internet}  & \color{black} 5 & \color{black} 20 & \color{black} 24 & \color{black} 24 \\ \cline{2-6}
 & \cellcolor{couleurClaire} \color{couleurTexte} \mbox{Génie} \mbox{logiciel}  & \cellcolor{couleurClaire} \color{couleurTexte} 5 & \cellcolor{couleurClaire} \color{couleurTexte} 20 & \cellcolor{couleurClaire} \color{couleurTexte} 24 & \cellcolor{couleurClaire} \color{couleurTexte} 24 \\ \cline{2-6}
 & \color{black} \mbox{Programmation} \mbox{objet} \mbox{2}  & \color{black} 5 & \color{black} 20 & \color{black} 24 & \color{black} 24 \\ \cline{2-6}
 & \cellcolor{couleurClaire} \color{couleurTexte} \mbox{Projet} \mbox{2}  & \cellcolor{couleurClaire} \color{couleurTexte} 5 & \cellcolor{couleurClaire} \color{couleurTexte}  & \cellcolor{couleurClaire} \color{couleurTexte}  & \cellcolor{couleurClaire} \color{couleurTexte}  \\ \cline{2-6}
 & \color{black} \mbox{Stage}  & \color{black} 10 & \color{black} 10 & \color{black}  & \color{black}  \\ \cline{2-6}
\hline
\end{tabular}



\section*{Détail des enseignements}



\end{spacing}

\end{document}
