\documentclass[10pt, a5paper]{report}

\usepackage[T1]{fontenc}%
\usepackage[utf8]{inputenc}% encodage utf8
\usepackage[francais]{babel}% texte français
\usepackage[top=10mm, bottom=20mm, foot=5mm, left=7mm, offset=7mm, textwidth=350pt]{geometry}
%\usepackage{wrapfig}
%\usepackage{graphics}
%\usepackage{setspace}
%\usepackage{tikz}
%\usepackage{array}
%\usepackage[explicit]{titlesec}
\usepackage{init-preambule}
\usepackage{xcolor}

% Définir les couleur du document

%---------------------- % % % Personnalisation des couleurs % % % ----------- BLEU --------
\definecolor{couleurFonce}{RGB}{0,92,133} % Couleur du Code APOGEE
\definecolor{couleurClaire}{RGB}{100,151,186} % Couleur du fond de la bande
\definecolor{couleurTexte}{RGB}{255,255,255} % Couleur du texte de la bande
%------------------------------------------------------------------------------------------



% Définir la couleur du médaillon
\definecolor{couleurBox}{RGB}{255,255,255}

\begin{document}
 \begin{spacing}{1.5}
  \chapter*{Introduction}
  {\footnotesize
  \section*{Parcours offerts}
      \begin{wrapfigure}{r}{0.35\textwidth}
          \vspace{-20pt}
            \begin{center}
                \begin{tikzpicture}
                    \node [rectangle, draw=couleurFonce, thick, drop shadow, fill=couleurBox, inner sep=10pt, inner ysep=10pt] (box) {
                    \begin{minipage}{0.35\textwidth}%
                        \begin{spacing}{1}
                        \begin{tabular}[t]{@{}m{10mm}@{~~}m{30mm}@{}}
                          \includegraphics[scale=0.6]{img/photos/IoanCorneliuTODINCA.jpg} & \normalsize{\textbf{Ioan TODINCA}}\newline \footnotesize{Responsable\linebreak[4]de la Licence} \\ 
                          \multicolumn{2}{c}{{\scriptsize \textit{ioan.todinca@univ-orleans.fr}}} \\
                          \multicolumn{2}{c}{{\scriptsize 02 38 41 72 93}} \\
                          \\
                          \textcolor{couleurFonce}{\rule[10pt]{42mm}{0.5pt}} \\
                          \includegraphics[scale=0.6]{img/photos/CatherineJulieBonnet.jpg} & \normalsize{\textbf{Catherine\linebreak[4]JULIE-BONNET}}\newline \footnotesize{Responsable\linebreak[4]de la MIAGE} \\ 
                          \multicolumn{2}{c}{{\scriptsize \textit{catherine.julie-bonnet@univ-orleans.fr}}} \\
                          \multicolumn{2}{c}{{\scriptsize 02 38 49 47 36}} \\
                        \end{tabular}
                        \end{spacing}
                    \end{minipage}
                    };
                    \node[fancytitle, right=5pt, rounded corners, inner xsep=10pt] at (box.north west) {\normalsize{Responsables Licence}};
                \end{tikzpicture}
          %     \includegraphics{../livretV3/img/flagFR.jpg}
            \end{center}
            \vspace{-20pt}
      \end{wrapfigure}
\letterine{L}a licence d’informatique se compose de deux années de tronc commun (le premier semestre étant commun à la licence de mathématiques) et d’une troisième année de spécialisation STIC (Sciences et Technologies de l’Information et de la Communication) ou MIAGE (Méthodes Informatiques Appliquées à la Gestion des Entreprises).
Le parcours STIC offre une formation solide en informatique théorique et appliquée. 
Le parcours MIAGE comporte, outre des bases solides en techniques et méthodes informatiques, des pôles de compétences en gestion des organisations, droit et techniques de communication. 
La formation MIAGE se décline en formation initiale et continue présentielle (parcours MIAGE) mais également par la voie de la formation à distance aux semestres 5 et 6 (parcours E-MIAGE). Ce dernier parcours en FOAD est réservé aux apprenants géographiquement distants de l’Université d’Orléans, dont la majeure partie constitue un public étranger.

\section*{Débouchés}
\letterine{L}a licence d’informatique débouche sur différents métiers de l’informatique, dans les secteurs public et privé (banque, sociétés services, etc.). Néanmoins, de part la nature du marché de l’informatique d’une part, et de notre offre de formation d’autre part, la très grande majorité des étudiants poursuivent leurs études jusqu’à Bac +5, notamment dans nos masters mention STIC et MIAGE.

\section*{Enseignements}
\textcolor{couleurFonce}{\textbf{\Large{L}}}es enseignements sont articulés autour de bases théoriques (mathématiques, fondements de l’informatique) et de matières appliquées (algorithmique, programmation, bases de données, réseaux, …). Des projets et un stage de fin de licence permettent la mise en perspectives des enseignements reçus et un premier contact avec le monde professionnel.
Les enseignements sont en grande majorité assurés par les enseignants de l’université, la présence d’industriels n’étant pas réellement justifiée au niveau licence (les enseignants offrant un niveau d’expertise théorique et technique tout à fait adapté aux compétences développées).


%---------------------- % % % Personnalisation des couleurs % % % ----------- BLEU --------
\definecolor{couleurFonce}{RGB}{0,92,133} % Couleur du Code APOGEE
\definecolor{couleurClaire}{RGB}{100,151,186} % Couleur du fond de la bande
\definecolor{couleurTexte}{RGB}{255,255,255} % Couleur du texte de la bande
%------------------------------------------------------------------------------------------


\arrayrulecolor{couleurFonce}% Couleur des lignes séparatrices du tableau
\renewcommand{\arraystretch}{1.2}% Coeff appliqué à la hauteur des cellules
%\rowcolors[\hline]{ligneDébut}{couleurPaire}{couleurImpaire}% Alternance de couleur (need package xcolor)
\begin{tabular}{c|m{6cm}|cm{1cm}|cm{1cm}|cm{1cm}|cm{1cm}|}
\cline{2-6}

&
\cellcolor{couleurFonce} \color{white}\bfseries Intitul\'e & \cellcolor{couleurFonce} \color{white}\bfseries ECTS & \cellcolor{couleurFonce} \color{white}\bfseries CM & \cellcolor{couleurFonce} \color{white}\bfseries TD & \cellcolor{couleurFonce} \color{white}\bfseries TP\\ \cline{2-6}

\hline \multirow{6}{*}{\rotatebox{90}{\color{couleurFonce}\bfseries  SEMESTRE 1}}
 & \color{black} \mbox{Algorithmique} \mbox{1} \mbox{et} \mbox{programmation}  & \color{black} 6 & \color{black} 45 & \color{black}  & \color{black} 15 \\ \cline{2-6}
 & \cellcolor{couleurClaire} \color{couleurTexte} \mbox{Atelier} \mbox{1} \mbox{de} \mbox{l'informaticien}  & \cellcolor{couleurClaire} \color{couleurTexte} 3 & \cellcolor{couleurClaire} \color{couleurTexte} 24 & \cellcolor{couleurClaire} \color{couleurTexte}  & \cellcolor{couleurClaire} \color{couleurTexte}  \\ \cline{2-6}
 & \color{black} \mbox{Introduction} \mbox{mathématiques} \mbox{au} \mbox{raisonnement}  & \color{black} 6 & \color{black} 60 & \color{black}  & \color{black}  \\ \cline{2-6}
 & \cellcolor{couleurClaire} \color{couleurTexte} \mbox{Suites} \mbox{réelles} \mbox{et} \mbox{fonctions}  & \cellcolor{couleurClaire} \color{couleurTexte} 6 & \cellcolor{couleurClaire} \color{couleurTexte} 60 & \cellcolor{couleurClaire} \color{couleurTexte}  & \cellcolor{couleurClaire} \color{couleurTexte}  \\ \cline{2-6}
 & \color{black} \mbox{Arithmétique}  & \color{black} 3 & \color{black} 24 & \color{black}  & \color{black}  \\ \cline{2-6}
 & \cellcolor{couleurClaire} \color{couleurTexte} \mbox{Anglais} \mbox{1}  & \cellcolor{couleurClaire} \color{couleurTexte} 3 & \cellcolor{couleurClaire} \color{couleurTexte} 24 & \cellcolor{couleurClaire} \color{couleurTexte}  & \cellcolor{couleurClaire} \color{couleurTexte}  \\ \cline{2-6}
 & \color{black} \mbox{Préparation} \mbox{au} \mbox{C2I}  & \color{black} 3 & \color{black}  & \color{black} 24 & \color{black}  \\ \cline{2-6}
\hline
\end{tabular}



%---------------------- % % % Personnalisation des couleurs % % % ----------- BLEU --------
\definecolor{couleurFonce}{RGB}{0,92,133} % Couleur du Code APOGEE
\definecolor{couleurClaire}{RGB}{100,151,186} % Couleur du fond de la bande
\definecolor{couleurTexte}{RGB}{255,255,255} % Couleur du texte de la bande
%------------------------------------------------------------------------------------------


\arrayrulecolor{couleurFonce}% Couleur des lignes séparatrices du tableau
\renewcommand{\arraystretch}{1.2}% Coeff appliqué à la hauteur des cellules
%\rowcolors[\hline]{ligneDébut}{couleurPaire}{couleurImpaire}% Alternance de couleur (need package xcolor)
\begin{tabular}{c|m{6cm}|cm{1cm}|cm{1cm}|cm{1cm}|cm{1cm}|}
\cline{2-6}

&
\cellcolor{couleurFonce} \color{white}\bfseries Intitul\'e & \cellcolor{couleurFonce} \color{white}\bfseries ECTS & \cellcolor{couleurFonce} \color{white}\bfseries CM & \cellcolor{couleurFonce} \color{white}\bfseries TD & \cellcolor{couleurFonce} \color{white}\bfseries TP\\ \cline{2-6}

\hline \multirow{6}{*}{\rotatebox{90}{\color{couleurFonce}\bfseries  SEMESTRE 2}}
 & \color{black} \mbox{Algorithmique} \mbox{2} \mbox{et} \mbox{programmation}  & \color{black} 6 & \color{black} 60 & \color{black}  & \color{black}  \\ \cline{2-6}
 & \cellcolor{couleurClaire} \color{couleurTexte} \mbox{Outils} \mbox{l'informatique} \mbox{mathématiques} \mbox{pour}  & \cellcolor{couleurClaire} \color{couleurTexte} 4 & \cellcolor{couleurClaire} \color{couleurTexte} 48 & \cellcolor{couleurClaire} \color{couleurTexte}  & \cellcolor{couleurClaire} \color{couleurTexte}  \\ \cline{2-6}
 & \color{black} \mbox{Modélisation}  & \color{black} 3 & \color{black} 24 & \color{black}  & \color{black}  \\ \cline{2-6}
 & \cellcolor{couleurClaire} \color{couleurTexte} \mbox{Projet} \mbox{informatique} \mbox{1}  & \cellcolor{couleurClaire} \color{couleurTexte} 3 & \cellcolor{couleurClaire} \color{couleurTexte}  & \cellcolor{couleurClaire} \color{couleurTexte}  & \cellcolor{couleurClaire} \color{couleurTexte}  \\ \cline{2-6}
 & \color{black} \mbox{Mathématiques}  & \color{black} 5 & \color{black} 60 & \color{black}  & \color{black}  \\ \cline{2-6}
 & \cellcolor{couleurClaire} \color{couleurTexte} \mbox{Anglais} \mbox{2}  & \cellcolor{couleurClaire} \color{couleurTexte} 3 & \cellcolor{couleurClaire} \color{couleurTexte} 24 & \cellcolor{couleurClaire} \color{couleurTexte}  & \cellcolor{couleurClaire} \color{couleurTexte}  \\ \cline{2-6}
 & \color{black} \mbox{Unité} \mbox{d'ouverture}  & \color{black} 3 & \color{black}  & \color{black} 24 & \color{black}  \\ \cline{2-6}
 & \cellcolor{couleurClaire} \color{couleurTexte} \mbox{Projet} \mbox{professionnel} \mbox{personnel} \mbox{et}  & \cellcolor{couleurClaire} \color{couleurTexte} 3 & \cellcolor{couleurClaire} \color{couleurTexte} 12 & \cellcolor{couleurClaire} \color{couleurTexte}  & \cellcolor{couleurClaire} \color{couleurTexte}  \\ \cline{2-6}
\hline
\end{tabular}



%---------------------- % % % Personnalisation des couleurs % % % ----------- BLEU --------
\definecolor{couleurFonce}{RGB}{0,92,133} % Couleur du Code APOGEE
\definecolor{couleurClaire}{RGB}{100,151,186} % Couleur du fond de la bande
\definecolor{couleurTexte}{RGB}{255,255,255} % Couleur du texte de la bande
%------------------------------------------------------------------------------------------


\arrayrulecolor{couleurFonce}% Couleur des lignes séparatrices du tableau
\renewcommand{\arraystretch}{1.2}% Coeff appliqué à la hauteur des cellules
%\rowcolors[\hline]{ligneDébut}{couleurPaire}{couleurImpaire}% Alternance de couleur (need package xcolor)
\begin{tabular}{c|m{6cm}|cm{1cm}|cm{1cm}|cm{1cm}|cm{1cm}|}
\cline{2-6}

&
\cellcolor{couleurFonce} \color{white}\bfseries Intitul\'e & \cellcolor{couleurFonce} \color{white}\bfseries ECTS & \cellcolor{couleurFonce} \color{white}\bfseries CM & \cellcolor{couleurFonce} \color{white}\bfseries TD & \cellcolor{couleurFonce} \color{white}\bfseries TP\\ \cline{2-6}

\hline \multirow{6}{*}{\rotatebox{90}{\color{couleurFonce}\bfseries  SEMESTRE 3}}
 & \color{black} \mbox{Algorithmique} \mbox{3} \mbox{(programmation} \mbox{orientée} \mbox{objet)} \mbox{et} \mbox{programmation}  & \color{black} 6 & \color{black} 24 & \color{black} 36 & \color{black}  \\ \cline{2-6}
 & \cellcolor{couleurClaire} \color{couleurTexte} \mbox{Bases} \mbox{et} \mbox{Internet} \mbox{de} \mbox{données}  & \cellcolor{couleurClaire} \color{couleurTexte} 5 & \cellcolor{couleurClaire} \color{couleurTexte} 12 & \cellcolor{couleurClaire} \color{couleurTexte}  & \cellcolor{couleurClaire} \color{couleurTexte} 24 \\ \cline{2-6}
 & \color{black} \mbox{Atelier} \mbox{2} \mbox{de} \mbox{l'informaticien}  & \color{black} 4 & \color{black} 12 & \color{black}  & \color{black} 24 \\ \cline{2-6}
 & \cellcolor{couleurClaire} \color{couleurTexte} \mbox{Architecture} \mbox{des} \mbox{ordinateurs}  & \cellcolor{couleurClaire} \color{couleurTexte} 4 & \cellcolor{couleurClaire} \color{couleurTexte} 12 & \cellcolor{couleurClaire} \color{couleurTexte} 12 & \cellcolor{couleurClaire} \color{couleurTexte} 6 \\ \cline{2-6}
 & \color{black} \mbox{Applications} \mbox{de} \mbox{l'algèbre}  & \color{black} 5 & \color{black}  & \color{black} 48 & \color{black}  \\ \cline{2-6}
 & \cellcolor{couleurClaire} \color{couleurTexte} \mbox{Anglais} \mbox{3}  & \cellcolor{couleurClaire} \color{couleurTexte} 3 & \cellcolor{couleurClaire} \color{couleurTexte}  & \cellcolor{couleurClaire} \color{couleurTexte} 25 & \cellcolor{couleurClaire} \color{couleurTexte}  \\ \cline{2-6}
 & \color{black} \mbox{Unité} \mbox{d'ouverture}  & \color{black} 3 & \color{black}  & \color{black} 22 & \color{black}  \\ \cline{2-6}
 & \cellcolor{couleurClaire} \color{couleurTexte} \mbox{Projet} \mbox{professionnel} \mbox{personnel} \mbox{et}  & \cellcolor{couleurClaire} \color{couleurTexte} 2 & \cellcolor{couleurClaire} \color{couleurTexte} 12 & \cellcolor{couleurClaire} \color{couleurTexte}  & \cellcolor{couleurClaire} \color{couleurTexte}  \\ \cline{2-6}
\hline \multirow{6}{*}{\rotatebox{90}{\color{couleurFonce}\bfseries  SEMESTRE 4}}
 & \cellcolor{couleurClaire} \color{couleurTexte} \mbox{Programmation} \mbox{fonctionnelle}  & \cellcolor{couleurClaire} \color{couleurTexte} 6 & \cellcolor{couleurClaire} \color{couleurTexte} 24 & \cellcolor{couleurClaire} \color{couleurTexte} 36 & \cellcolor{couleurClaire} \color{couleurTexte}  \\ \cline{2-6}
 & \color{black} \mbox{Algorithmique} \mbox{des} \mbox{structures} \mbox{discrètes} \mbox{et} \mbox{combinatoire}  & \color{black} 6 & \color{black} 24 & \color{black} 36 & \color{black}  \\ \cline{2-6}
 & \cellcolor{couleurClaire} \color{couleurTexte} \mbox{Projet} \mbox{(Conception} \mbox{et} \mbox{projet)} \mbox{informatique} \mbox{2}  & \cellcolor{couleurClaire} \color{couleurTexte} 5 & \cellcolor{couleurClaire} \color{couleurTexte} 12 & \cellcolor{couleurClaire} \color{couleurTexte}  & \cellcolor{couleurClaire} \color{couleurTexte} 24 \\ \cline{2-6}
 & \color{black} \mbox{Probabilités}  & \color{black} 5 & \color{black}  & \color{black} 48 & \color{black} 30 \\ \cline{2-6}
 & \cellcolor{couleurClaire} \color{couleurTexte} \mbox{Anglais} \mbox{4}  & \cellcolor{couleurClaire} \color{couleurTexte} 3 & \cellcolor{couleurClaire} \color{couleurTexte}  & \cellcolor{couleurClaire} \color{couleurTexte} 24 & \cellcolor{couleurClaire} \color{couleurTexte}  \\ \cline{2-6}
 & \color{black} \mbox{Bases} \mbox{comptable} \mbox{du} \mbox{système}  & \color{black} 5 & \color{black}  & \color{black} 30 & \color{black}  \\ \cline{2-6}
 & \cellcolor{couleurClaire} \color{couleurTexte} \mbox{Programmation} \mbox{impérative}  & \cellcolor{couleurClaire} \color{couleurTexte} 5 & \cellcolor{couleurClaire} \color{couleurTexte} 12 & \cellcolor{couleurClaire} \color{couleurTexte} 20 & \cellcolor{couleurClaire} \color{couleurTexte}  \\ \cline{2-6}
\hline
\end{tabular}


\input{Tableau_Info_LicenceMIAGE_S3S4}

\section*{Aspects pédagogiques}
\textcolor{couleurFonce}{\textbf{\Large{L}}}a licence d’informatique dispose d’une équipe de formation, incluant des enseignants, des représentant des personnels IATOSS et des étudiants. Par ailleurs, les étudiants sont suivis, notamment en première année, par un enseignant référent.

\section*{Echanges internationaux}
\textcolor{couleurFonce}{\textbf{\Large{L}}}es étudiants peuvent effectuer une année à l’étranger (notamment la troisième année). A l’heure actuelle, le contenu des parcours est établi au cas par cas, suivant les matières offertes par l’université d’accueil. Inversement, les étudiants étrangers souhaitant effectuer une année d’informatique à l’université d’Orléans, peuvent suivre tout ou partie de leurs enseignements dans notre licence. 
}

\chapter*{Organisation et fonctionnement de la mention}
\section*{Les parcours de formation}
\section*{Passerelles et réorientations offertes}
Passerelles possibles entre les différents parcours de formation :

Les trois premiers trimestres sont communs. Au cours du quatrième trimestre les étudiants peuvent choisir des modules orientés vers le parcours miage ou STIC. Néanmoins, la véritable différenciation des deux parcours se fait en troisième année. De ce fait, la notion de passerelle ne prend de sens ici que pour un étudiant redoublant sa troisième année et souhaitant changer de parcours.
Le recrutement en parcours MIAGE au semestre 5 est effectué sur dossier pour tous les candidats. 
Le recrutement en parcours E-MIAGE est effectué sur dossier par une commission de recrutement nationale. 

Passerelles possibles vers d’autres mentions de licence au sein du même domaine
La licence Informatique est  conçue en étroite collaboration avec la licence de Mathématiques. Ces licences comportent un socle commun de mathématiques et d’informatique. Des passerelles naturelles existent entre ces deux licences jusqu'à la troisième année.
En fin de deuxième année, les étudiants peuvent être orientés vers la licence professionnelle Réseaux et Télécommunication. Afin de préparer cette réorientation, un parcours présentant un module spécifique est mis en place au quatrième semestre.
Une réorientation en DUT, dès la première année, est possible sous réserve d'accord des responsables des parcours concernés.

Passerelles offertes permettant l’accueil, en cours de cursus, d’étudiants issus d’autres formations : (DUT, CPGE, BTS, …) :
Sur examen du dossier, les étudiants issus d'autres formations peuvent être admis, suivant leur niveau, en deuxième ou troisième année. Typiquement, l'admission des étudiants issus de DUT informatique se fait en troisième année.

Modalités de réorientation en cours d’études :
La réorientation se fait à la demande de l'étudiant et sur avis des directeurs d'études. Concernant la réorientation entre les deux parcours de la licence Informatique, en troisième année, les transferts de crédits se font sur la base des unités communes aux deux parcours.

\section*{Formation continue}

  \end{spacing}
\end{document}
