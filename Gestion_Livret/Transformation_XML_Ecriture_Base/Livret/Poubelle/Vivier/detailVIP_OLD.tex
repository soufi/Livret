
%==========================================================================================
% M2 VIP : PARTIE QUI TRAINE D'UN ANCIEN DOCUMENT !
%==========================================================================================
\module[codeApogee={UE 31}, 
titre={Génie logiciel}, 
COURS={}, 
TD={}, 
TP={}, 
CTD={}, 
TOTAL={0}, 
SEMESTRE={}, 
COEFF={12}, 
ECTS={12}, 
MethodeEval={Contrôle continue et terminal}, 
ModalitesCCSemestreUn={CC et CT}, 
ModalitesCCSemestreDeux={CT}, 
%CalculNFSessionUne={$\frac{(CC+2*CT)}{3}$}, 
%CalculNFSessionDeux={CT}, 
NoteEliminatoire={7}, 
nomPremierResp={Prénom NOM}, 
emailPremierResp={Prenom.NOM@univ-orleans.fr}, 
nomSecondResp={}, 
emailSecondResp={}, 
langue={Français}, 
nbPrerequis={0}, 
descriptionCourte={true}, 
descriptionLongue={true}, 
objectifs={true}, 
ressources={true}, 
bibliographie={false}] 
{
petite-DESCRIPTION 
} 
{
1.	Compléments de programmation : programmation événementielle, DLL, sérialisation, threads, aide intégrée, déploiement.
2.	Concepts supplémentaires de programmation : Technologie COM. Interfaçage d'applications avec des bases de données.
3.	Qualité industrielle : Impératifs de qualité, normes de qualité, qualité de processus, qualité des produits, exigence qualité, gestion de configuration, modèles de maturité.
4.	Technologies Web : Programmation XML / Web.
5.	UML : présentation d'UML. 
6.	Gestion de bases de données : Modélisation, tables, SQL, SQL server, Interfaces d'accés (PHP, ASP, ACCESS), Etude de cas. 
} 
{Bases de données et Programmation orientée objet} 
{\begin{itemize}
\ObjItem Ce module complète et finalise la formation des étudiants dans différents secteurs du logiciel, grâce entre autre aux interventions de professionnels du milieu industriel
\ObjItem l s'agit de donner aux étudiants une formation générale sur la qualité en traitant les domaines suivants : la qualité du logiciel, les normes et les organismes de normalisation, UML, le Cycle de vie, la qualité de produit, la qualité de processus, la gestion de projet fiabilité, les procédures et vecteurs de tests, la gestion de configuration et de déploiement et la gestion des bases de données.
\end{itemize} 
} 
{Ressources} 
{Biblio} 
 
\vfill

%==========================================================================================
\module[codeApogee={UE32  }, 
titre={Informatique temps réel}, 
COURS={}, 
TD={}, 
TP={}, 
CTD={}, 
TOTAL={0}, 
SEMESTRE={}, 
COEFF={12}, 
ECTS={12}, 
MethodeEval={Contrôle continue et terminal}, 
ModalitesCCSemestreUn={CC et CT}, 
ModalitesCCSemestreDeux={CT}, 
%CalculNFSessionUne={$\frac{(CC+2*CT)}{3}$}, 
%CalculNFSessionDeux={CT}, 
NoteEliminatoire={7}, 
nomPremierResp={Prénom NOM}, 
emailPremierResp={Prenom.NOM@univ-orleans.fr}, 
nomSecondResp={}, 
emailSecondResp={}, 
langue={Français}, 
nbPrerequis={1}, 
descriptionCourte={true}, 
descriptionLongue={true}, 
objectifs={true}, 
ressources={true}, 
bibliographie={false}] 
{
petite-DESCRIPTION 
} 
{
1.	Les concepts temps réel : le multi-tâche, la notion de processus, les appels système, les communications (synchronisation par sémaphores, synchronisation par événements, communication par boîtes aux lettres), la répartition des processus, le temps partagé.
2.	Les spécificités temps réel.
3.	Présentation des architectures parallèles : analyse des besoins (évaluation des architectures). exemple du traitement d'images, classification des architectures (architectures SISD, MISD, SIMD et MIMD), problèmes de classification, mémoire et communication : architectures à mémoire partagée et architectures à mémoire distribuée, parallélisme algorithmique (algorithmique répartie, parallélisme de données et parallélisme de flux).
4.	Applications Java embarquées sur carte à puce. 
} 
{Programmation parallèle} 
{\begin{itemize} 
\ObjItem Ce module présentent les différentes méthodes mises en \oe uvre pour aboutir à des systèmes informatisés répondant aux contraintes temporelles imposées par l'application.
\ObjItem l traite également
\ObjItem la mise en \oe uvre matérielle d'architectures parallèles dédiées à des applications de traitement d'image.
l'étude logicielle des adéquations algorithme-architecture.
l'étude des systèmes d'exploitation temps réel pour des applications avec contraintes sévères.
les solutions partagées par Java embarqué.
\end{itemize} 
} 
{Ressources} 
{Biblio} 
 
\vfill

%==========================================================================================
\module[codeApogee={UE 33}, 
titre={Système électroniques communicants}, 
COURS={}, 
TD={}, 
TP={}, 
CTD={}, 
TOTAL={0}, 
SEMESTRE={}, 
COEFF={12}, 
ECTS={12}, 
MethodeEval={Contrôle continue et terminal}, 
ModalitesCCSemestreUn={CC et CT}, 
ModalitesCCSemestreDeux={CT}, 
%CalculNFSessionUne={$\frac{(CC+2*CT)}{3}$}, 
%CalculNFSessionDeux={CT}, 
NoteEliminatoire={7}, 
nomPremierResp={Prénom NOM}, 
emailPremierResp={Prenom.NOM@univ-orleans.fr}, 
nomSecondResp={}, 
emailSecondResp={}, 
langue={Français}, 
nbPrerequis={1}, 
descriptionCourte={true}, 
descriptionLongue={true}, 
objectifs={true}, 
ressources={true}, 
bibliographie={false}] 
{
petite-DESCRIPTION 
} 
{
1.	Les systèmes communicants. Les OS pour l'embarqué. Présentation OS linux..  
2.	Modèles de l'ISO, modèle IEEE, description des couches, protocoles et services. 
3.	Constitution physique des réseaux.
4.	Outils d'analyses matérielle et réseaux. 
5.	TCP/IP embarqué : contraintes de mise en \oe uvre, état de l'art des solutions industrielles.
6.	Bus et réseaux industriels et automobile.
7.	Contrôle-Commande, voix, données, images : Exemple CAMERA IP  motorisée  et transmission d'alarmes.
8.	Travaux dirigés : protocoles d'applications (SMTP, FTP, NFS,...), mise en \oe uvre OS embarqué, étiquettes RFID, VOIP, streaming audio et vidéo. 
} 
{Réseaux} 
{\begin{itemize}
\ObjItem Il s'agit dans ce module d'aborder, d'un point de vue théorique et expérimental, les systèmes électroniques communicants.
\ObjItem Ces notions seront appliquées au fonctionnement de systèmes complexes mettant en \oe uvre du logiciel, du matériel et des piles protocolaires pour le fonctionnement en réseau filaire et non filaire.
\ObjItem Ce module permettra de comprendre les contraintes industrielles de chaque segment de marché des systèmes communicants (automobile, grand public, instrumentation-mesure,...).
\end{itemize} 
} 
{Ressources} 
{Biblio} 
 
\vfill

%==========================================================================================
\module[codeApogee={UE 34}, 
titre={visionique}, 
COURS={}, 
TD={}, 
TP={}, 
CTD={}, 
TOTAL={0}, 
SEMESTRE={}, 
COEFF={12}, 
ECTS={12}, 
MethodeEval={Contrôle continue et terminal}, 
ModalitesCCSemestreUn={CC et CT}, 
ModalitesCCSemestreDeux={CT}, 
%CalculNFSessionUne={$\frac{(CC+2*CT)}{3}$}, 
%CalculNFSessionDeux={CT}, 
NoteEliminatoire={7}, 
nomPremierResp={Prénom NOM}, 
emailPremierResp={Prenom.NOM@univ-orleans.fr}, 
nomSecondResp={}, 
emailSecondResp={}, 
langue={Français}, 
nbPrerequis={1}, 
descriptionCourte={true}, 
descriptionLongue={true}, 
objectifs={true}, 
ressources={true}, 
bibliographie={false}] 
{
petite-DESCRIPTION 
} 
{
1.	Compléments logiciels : Extraction de primitives, mise en correspondance, contours et surfaces actives.
2.	Métrologie par vision : Étalonnage d'une chaîne de mesure.
3.	Signaux Vidéo.
4.	Conception d'un capteur d'image : Technologie des capteurs CCD et mise en \oe uvre (horloges, conditionnement du signal,...). Intensificateurs.
5.	Perception 3D : Capteurs actifs (télémètres, triangulation active, interférométrie holographique). Capteurs passifs (stéréovision, vision dynamique monoculaire, mise en correspondance). Analyse de données 3D.
6.	Normes et compression d'images : TCD, JPEG, MPEG, TV numérique. Imagerie médicale et norme DICOM.
7.	Vision Industrielle : Marché de la vision, solutions matérielles, adéquation algorithme/architecture, méthodologie de développement d'une application, présentation d'applications, acteurs du marché de la vision industrielle, le métier de l'ingénieur en vision. 
} 
{Programmation graphique (Traitement numérique)} 
{\begin{itemize}
\ObjItem Ce module, orienté vers la vision industrielle et ses applications, notamment en productique, est destiné à donner aux élèves un panorama du marché aussi bien en termes de besoins et contraintes qu'en termes de solutions, matérielles et logicielles.
\end{itemize} 
} 
{Ressources} 
{Biblio} 
 
\vfill

%==========================================================================================
\module[codeApogee={UE 35}, 
titre={Filtrage et codage}, 
COURS={}, 
TD={24}, 
TP={}, 
CTD={}, 
TOTAL={24}, 
SEMESTRE={3}, 
COEFF={3}, 
ECTS={3}, 
MethodeEval={Contrôle continue et terminal}, 
ModalitesCCSemestreUn={CC et CT}, 
ModalitesCCSemestreDeux={CT}, 
%CalculNFSessionUne={$\frac{(CC+2*CT)}{3}$}, 
%CalculNFSessionDeux={CT}, 
NoteEliminatoire={7}, 
nomPremierResp={Prénom NOM}, 
emailPremierResp={Prenom.NOM@univ-orleans.fr}, 
nomSecondResp={}, 
emailSecondResp={}, 
langue={Français}, 
nbPrerequis={1}, 
descriptionCourte={true}, 
descriptionLongue={true}, 
objectifs={true}, 
ressources={true}, 
bibliographie={false}] 
{
petite-DESCRIPTION 
} 
{
1.	Synthèse de filtres en virgule fixe (gestion de la dynamique et des troncatures)
2.	Mise en \oe uvre de filtres multicadences (filtres polyphases, décimation, interpolation, filtres CIC, banc de filtres polyphases)
3.	Ondelettes
4.	Introduction au filtrage adaptatif (LMS et Kalman)
5.	Passage en revue exhaustif des normes et des techniques de codage en image
6.	Codage du signal de parole.
} 
{Programmation graphique et traitement numérique} 
{\begin{itemize}
\ObjItem Optimisation de l'implantation de filtres numériques (gestion de la dynamique de calcul et des troncatures, exploitation des possibilités de multicadence et d'adaptation, ondelettes)
\ObjItem Application aux techniques de codage de l'image et de la parole.
\end{itemize} 
} 
{Ressources} 
{Biblio} 
 
\vfill
