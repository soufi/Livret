
%---------------------- % % % Personnalisation des couleurs % % % ----------- MOUTARDE --------
\definecolor{couleurFonce}{RGB}{170,110,0} % Couleur du Code APOGEE
\definecolor{couleurClaire}{RGB}{222,190,137} % Couleur du fond de la bande
\definecolor{couleurTexte}{RGB}{255,255,255} % Couleur du texte de la bande
%------------------------------------------------------------------------------------------



%==========================================================================================
% Semestre 3
%==========================================================================================
\module[codeApogee={UE 31}, 
titre={Architectures applicatives réparties}, 
COURS={20}, 
TD={20}, 
TP={10}, 
CTD={}, 
TOTAL={50}, 
SEMESTRE={semestre 3}, 
COEFF={4}, 
ECTS={4}, 
MethodeEval={Contrôle continue et terminal}, 
ModalitesCCSemestreUn={CC et CT}, 
ModalitesCCSemestreDeux={CT}, 
%CalculNFSessionUne={$\frac{(CC+2*CT)}{3}$}, 
%CalculNFSessionDeux={CT}, 
NoteEliminatoire={7},
nomPremierResp={Frédéric MOAL}, 
emailPremierResp={Frederic.MOAL@univ-orleans.fr}, 
nomSecondResp={Matthieu EXBRAYAT}, 
emailSecondResp={Matthieu.EXBRAYAT@univ-orleans.fr}, 
langue={Français},
nbPrerequis={1},
descriptionCourte={true}, 
descriptionLongue={true}, 
objectifs={true}, 
ressources={true}, 
bibliographie={false}] 
{
Unité obligatoire. 
} 
{
Ce module se compose de deux parties complémentaires, portant sur les
aspects théoriques et pratiques des systèmes d'information répartis:
Dans l'optique d'une approche fondée sur la pratique, le module aborde
dans un premier temps les outils disponibles, autour du langage Java et de
la plateforme Java Enterprise Edition (JavaEE) :
\begin{itemize} 
\item Communication répartie par invocation distante (RMI) et messagerie
(JMS).
\item Persistance d'objets (utilisation de différents frameworks, dont hibernate)
\item Concept de bean métier (EJB)
\item Intégration des différents types d'EJB, interfaçage avec la couche
présentation.
\item Bases de l'administration du serveur d'applications\\
Lorsque les aspects pratiques ont été manipulés, l'étudiant est amené à les
mettre en perspective et à rationaliser leur mise en oeuvre, par l'étude des
aspects "théoriques" suivants :
  \begin{itemize} 
  \item Typologie des SI et exemples significatifs
  \item UML et processus de développement unifié (rappels)
  \item Patron de conception ("Design Patterns")
  \item Concept de transaction répartie
  \end{itemize} 
\end{itemize} 
} 
{Compétences en bases de données, programmation java, conception
d'applications web (servlets/JSP, modèle MVC2).} 
{\begin{itemize} 
  \ObjItem Fournir à l'étudiant les outils nécessaires à l'analyse, la mise en place et
l'exploitation de systèmes d'informations répartis.
  \ObjItem Apporter une solide
formation sur la répartition des données et des traitements dans un
Système d'Information (SI), suivant deux axes : outils avancés pour la
modélisation et la gestion des SI (UML, design patterns, etc.) et SI
distribués contemporains (architectures multi-tiers, plateformes
applicatives).
\end{itemize} 
} 
{Ressources} 
{Biblio} 
 
\vfill

%==========================================================================================
\module[codeApogee={UE 32}, 
titre={Extraction de connaissances dans les BD}, 
COURS={20}, 
TD={20}, 
TP={10}, 
CTD={}, 
TOTAL={50}, 
SEMESTRE={semestre 3}, 
COEFF={4}, 
ECTS={4}, 
MethodeEval={Contrôle continue et terminal}, 
ModalitesCCSemestreUn={CC et CT}, 
ModalitesCCSemestreDeux={CT}, 
%CalculNFSessionUne={$\frac{(CC+2*CT)}{3}$}, 
%CalculNFSessionDeux={CT}, 
NoteEliminatoire={7}, 
nomPremierResp={Christel VRAIN}, 
emailPremierResp={Christel.VRAIN@univ-orleans.fr}, 
nomSecondResp={}, 
emailSecondResp={}, 
langue={Français}, 
nbPrerequis={1}, 
descriptionCourte={true}, 
descriptionLongue={true}, 
objectifs={true}, 
ressources={true}, 
bibliographie={false}] 
{
Unité obligatoire. 
} 
{
\begin{itemize} 
\item Les différents types d'apprentissage et les différentes tâches
\item Classification supervisée : arbre de décision, modèles probabilistes,
machines à vecteur support, noyaux
\item Evaluation des modèles
\item Classification non supervisée : par partitionnement, hiérarchique,
conceptuelle
\item Recherche de règles d'association
\end{itemize} 
Savoir-faire:
\begin{itemize} 
\item Utilisation d'outils : Weka, RapidMiner
\item Définir le problème d'apprentissage : modèle à acquérir, données
nécessaires, techniques applicables
\item Appliquer des techniques d'apprentissage.
Etudes de cas guidés par des applications.
\end{itemize} 
} 
{UE Intelligence Artificielle - UE Analyse de données.} 
{\begin{itemize} 
  \ObjItem Acquérir les notions de base permettant de résoudre un problème
d'Extraction de Connaissances dans les Bases de Données : modélisation de la tâche à effectuer, préparation des données, choix des outils.\\
Le Data
Mining est devenu une discipline fondamentale, justifiée par le
développement de grandes bases de données, la nécessité de constituer
des mémoires d'entreprises et le besoin d'outils intelligents pour exploiter
les flux d'information sur le web.
\end{itemize} 
} 
{Ressources} 
{Biblio} 
 
\vfill

%==========================================================================================
\module[codeApogee={UE 33}, 
titre={Introduction aux Métiers du Social et de l'Assurance}, 
COURS={30}, 
TD={}, 
TP={}, 
CTD={}, 
TOTAL={30}, 
SEMESTRE={semestre 3}, 
COEFF={2}, 
ECTS={2}, 
MethodeEval={Contrôle continue et terminal}, 
ModalitesCCSemestreUn={CC et CT}, 
ModalitesCCSemestreDeux={CT}, 
%CalculNFSessionUne={$\frac{(CC+2*CT)}{3}$}, 
%CalculNFSessionDeux={CT}, 
NoteEliminatoire={7}, 
nomPremierResp={Prénom NOM}, 
emailPremierResp={Prenom.NOM@univ-orleans.fr}, 
nomSecondResp={}, 
emailSecondResp={}, 
langue={Français}, 
nbPrerequis={0}, 
descriptionCourte={true}, 
descriptionLongue={true}, 
objectifs={true}, 
ressources={true}, 
bibliographie={false}] 
{
Unité obligatoire. 
} 
{
\begin{itemize} 
\item Structure juridique des entreprises.
\item Description des métiers.
\item Aspects réglementaires et juridiques.
\item Organismes tiers en présence.
\item Intersections et points communs.
\item Circuit de vente.
\item Cartographie générale par métier.
\item Perspectives des métiers.
\item Métiers ouverts associés à ces formations.
\item Les apports d'une démarche ISO9001 face à la complexité des métiers du
social et de l'Assurance.
\item La CNIL, un organisme incontournable pour traiter de l'Assurance.
\item La dimension internationale dans un SI Assurances.
\item Un SI Assurances dans un SI Bancaire : les particularités de la banque
Assurance.
\end{itemize} 
} 
{} 
{\begin{itemize} 
  \ObjItem Avoir une culture générale des métiers des entreprises du secteur de la
Retraite, de l'Assurance et de la Prévoyance, et connaitre la cartographie
des différents SI du domaine.
  \ObjItem Etre sensibilisé à la règlementation du stockage des données à caractère
personnel.
  \ObjItem Appréhender le contexte international dans les projets informatiques
Assurances et dans la gestion des SI Assurances. Connaître la banque
Assurance.
\end{itemize} 
} 
{Ressources} 
{Biblio} 
 
\vfill

%==========================================================================================
\module[codeApogee={UE 34}, 
titre={SI de la Retraite}, 
COURS={30}, 
TD={10}, 
TP={}, 
CTD={}, 
TOTAL={40}, 
SEMESTRE={semestre 3}, 
COEFF={4}, 
ECTS={4}, 
MethodeEval={Contrôle continue et terminal}, 
ModalitesCCSemestreUn={CC et CT}, 
ModalitesCCSemestreDeux={CT}, 
%CalculNFSessionUne={$\frac{(CC+2*CT)}{3}$}, 
%CalculNFSessionDeux={CT}, 
NoteEliminatoire={7}, 
nomPremierResp={Prénom NOM}, 
emailPremierResp={Prenom.NOM@univ-orleans.fr}, 
nomSecondResp={}, 
emailSecondResp={}, 
langue={Français}, 
nbPrerequis={0}, 
descriptionCourte={true}, 
descriptionLongue={true}, 
objectifs={true}, 
ressources={true}, 
bibliographie={false}] 
{
Unité obligatoire. 
} 
{
\begin{itemize} 
\item Particularités du référentiel des métiers de la retraite.
\item Le régime général de la Sécurité Sociale et les régimes complémentaires.
\item La gestion des adhésions des entreprises.
\item Les déclarations nominatives.
\item L'appel des cotisations, le recouvrement et la gestion comptable.
\item Le calcul des droits cotisés et non cotisés : constiution de la carrière.
\item La gestion des allocataires jusqu'à l'extinction des droits.
\end{itemize}
} 
{} 
{\begin{itemize} 
  \ObjItem Maîtriser le fonctionnement des processus liés aux régimes de Retraite
(régime général Sécurité Sociale et régimes complémentaires ARRCO et
AGIRC) et les grandes entités existantes dans un SI Retraite.
\end{itemize} 
} 
{Ressources} 
{Biblio} 
 
\vfill

%==========================================================================================
\module[codeApogee={UE 35}, 
titre={SI de l'Assurance}, 
COURS={30}, 
TD={10}, 
TP={}, 
CTD={}, 
TOTAL={40}, 
SEMESTRE={semestre 3}, 
COEFF={4}, 
ECTS={4}, 
MethodeEval={Contrôle continue et terminal}, 
ModalitesCCSemestreUn={CC et CT}, 
ModalitesCCSemestreDeux={CT}, 
%CalculNFSessionUne={$\frac{(CC+2*CT)}{3}$}, 
%CalculNFSessionDeux={CT}, 
NoteEliminatoire={7}, 
nomPremierResp={Prénom NOM}, 
emailPremierResp={Prenom.NOM@univ-orleans.fr}, 
nomSecondResp={}, 
emailSecondResp={}, 
langue={Français}, 
nbPrerequis={0}, 
descriptionCourte={true}, 
descriptionLongue={true}, 
objectifs={true}, 
ressources={true}, 
bibliographie={false}] 
{
Unité obligatoire. 
} 
{
Ce module se compose de deux parties complémentaires, couvrant les
domaines d'activité de l'assurance que sont la branche IARD (Incendie - Automobile - Risques Divers) et la branche
Vie. La partie Vie est ensuite scindée en deux parties : une partie
compréhension du métier et une partie SI d'une société d'assurance Vie.
\begin{description}
\item[DOMAINE IARD] : Particularités du référentiel, catalogue de produit,
gestion du réseau et des intervenants commerciaux, contrat, cotisations et
recouvrement, sinistres, circuit des encaissements / décaissements,
rémunération des apporteurs (Agents, Courtiers, Salariés), résultats
techniques.
\item[DOMAINE VIE] :
\begin{itemize}
\item Généralités sur le contrat d'assurance Vie : ses caractéristiques et ses
diverses catégories
\item L'assurance Vie en quelques chiffres
\item Les divers frais : théorie, modes de calcul et illustration par exemples
\item La vie du contrat : sa formation, le paiement des primes, l'affectation des
primes, l'avance, les rachats, le règlement des prestations
\item La tarification du contrat : prime pure, frais, taxes ; tables de mortalité
\item La constitution des provisions mathématiques
\item Histoire de l'Assurance Vie
\item Construction des SI d'une société d'assurance Vie
\item Le régime fiscal du contrat : impôts sur les revenus et droits de
transmission
\item Les assurances collectives : tout ce qui diffère du contrat individuel ; les
contrats d'épargne retraite et de prévoyance complémentaire ; les contrats
à titre privé ou professionnel ; les contrats homme-clé et les contrats
emprunteur.
\end{itemize}
} 
{} 
{\begin{itemize} 
  \ObjItem Maîtriser les SI et le fonctionnement des processus liés à un contrat IARD
(Incendie - Automobile - Risques Divers) et à un contrat d'assurance vie.
\end{itemize} 
} 
{Ressources} 
{Biblio} 
 
\vfill

%==========================================================================================
\module[codeApogee={UE 36}, 
titre={Gestion de projet et qualité}, 
COURS={20}, 
TD={20}, 
TP={10}, 
CTD={}, 
TOTAL={50}, 
SEMESTRE={semestre 3}, 
COEFF={4}, 
ECTS={4}, 
MethodeEval={Contrôle continue et terminal}, 
ModalitesCCSemestreUn={CC et CT}, 
ModalitesCCSemestreDeux={CT}, 
%CalculNFSessionUne={$\frac{(CC+2*CT)}{3}$}, 
%CalculNFSessionDeux={CT}, 
NoteEliminatoire={7}, 
nomPremierResp={Prénom NOM}, 
emailPremierResp={Prenom.NOM@univ-orleans.fr}, 
nomSecondResp={}, 
emailSecondResp={}, 
langue={Français}, 
nbPrerequis={0}, 
descriptionCourte={true}, 
descriptionLongue={true}, 
objectifs={true}, 
ressources={true}, 
bibliographie={false}] 
{
Unité obligatoire. 
} 
{
\begin{itemize}
\item Les étapes d'un projet
\item Les livrables de la conduite de projet
\item Les rôles et responsabilités des acteurs (client, décideurs, maîtrise d'ouvrage, maîtrise d'oeuvre, assurance qualité)
\item Le travail en équipe
\item Méthodes de prévision et d'estimation des charges et de délais
\item Tests
\item Normalisation ISO / maturité CMM
\item La gestion des risques
\item La gestion des évolutions
\item Le plan de communication
\item Les outils du chef de projet.
\end{itemize}
} 
{} 
{\begin{itemize} 
  \ObjItem Etudier la conduite de projet et la mise en place d'un système qualité.
\end{itemize} 
} 
{Ressources} 
{Biblio} 
 
\vfill

%==========================================================================================
\module[codeApogee={UE 37}, 
titre={Gestion des ressources humaines}, 
COURS={10}, 
TD={20}, 
TP={}, 
CTD={}, 
TOTAL={30}, 
SEMESTRE={semestre 3}, 
COEFF={2}, 
ECTS={2}, 
MethodeEval={Contrôle continue et terminal}, 
ModalitesCCSemestreUn={CC et CT}, 
ModalitesCCSemestreDeux={CT}, 
%CalculNFSessionUne={$\frac{(CC+2*CT)}{3}$}, 
%CalculNFSessionDeux={CT}, 
NoteEliminatoire={7}, 
nomPremierResp={Prénom NOM}, 
emailPremierResp={Prenom.NOM@univ-orleans.fr}, 
nomSecondResp={}, 
emailSecondResp={}, 
langue={Français}, 
nbPrerequis={0}, 
descriptionCourte={true}, 
descriptionLongue={true}, 
objectifs={true}, 
ressources={true}, 
bibliographie={false}] 
{
Unité obligatoire. 
} 
{
\begin{itemize}
\item Introduction à la gestion des ressources humaines
\item La DRH en entreprise
\item Le management des hommes et des équipes
\item Evaluation des postes
\item Le recrutement
\item Réforme de la formation professionnelle.
\end{itemize}
} 
{} 
{\begin{itemize} 
  \ObjItem Appréhender la gestion des ressources humaines dans le cadre de l'entreprise.  
\end{itemize}
} 
{Ressources} 
{Biblio} 
 
\vfill

%==========================================================================================
\module[codeApogee={UE 38}, 
titre={Anglais professionnel}, 
COURS={}, 
TD={24}, 
TP={}, 
CTD={}, 
TOTAL={24}, 
SEMESTRE={semestre 3}, 
COEFF={3}, 
ECTS={3}, 
MethodeEval={Contrôle continue et terminal}, 
ModalitesCCSemestreUn={CC et CT}, 
ModalitesCCSemestreDeux={CT}, 
%CalculNFSessionUne={$\frac{(CC+2*CT)}{3}$}, 
%CalculNFSessionDeux={CT}, 
NoteEliminatoire={7}, 
nomPremierResp={Prénom NOM}, 
emailPremierResp={Prenom.NOM@univ-orleans.fr}, 
nomSecondResp={}, 
emailSecondResp={}, 
langue={Français}, 
nbPrerequis={1}, 
descriptionCourte={true}, 
descriptionLongue={true}, 
objectifs={true}, 
ressources={true}, 
bibliographie={false}] 
{
Unité obligatoire. 
} 
{
Travailler la langue écrite et orale afin de perfectionner ses capacités à
communiquer en milieu professionnel. Suivi interactif et personnalisé du
projet industriel.

Suivi de projet industriel :
\begin{itemize}
\item l'étudiant produit plusieurs devoirs pour lequel il reçoit une correction individuelle.
\item Présentation orale
\item Devoir écrit (compréhension et synthèse)
\item Poursuite du travail sur le CV, la lettre de motivation et l'entretien.
\end{itemize}
Supports : Publications les sciences et technologies informatiques, documents
sonores, centre de ressources multimedia.
} 
{Avoir suivi les modules d'anglais de Master 1 ou 550 heures de formation
équivalente.} 
{\begin{itemize} 
  \ObjItem Se rapprocher des compétences demandées au niveau C1; possibilité de
valider le CLES 3.
  \ObjItem L'étudiant comprend des textes longs dans sa discipline, rend compte
oralement et par écrit de documents professionnels, s'exprime dans une
langue claire et structurée sur des sujets complexes à caractère social et
académique.
\end{itemize} 
} 
{Ressources} 
{Biblio} 
 
\vfill

%==========================================================================================
\module[codeApogee={UE 39}, 
titre={Projet industriel}, 
COURS={10}, 
TD={}, 
TP={}, 
CTD={}, 
TOTAL={}, 
SEMESTRE={semestre 3}, 
COEFF={3}, 
ECTS={3}, 
MethodeEval={Contrôle continue et terminal}, 
ModalitesCCSemestreUn={CC et CT}, 
ModalitesCCSemestreDeux={CT}, 
%CalculNFSessionUne={$\frac{(CC+2*CT)}{3}$}, 
%CalculNFSessionDeux={CT}, 
NoteEliminatoire={7}, 
nomPremierResp={Catherine JULIÉ-BONNET}, 
emailPremierResp={Catherine.JULIE-BONNET@univ-orleans.fr}, 
nomSecondResp={},
emailSecondResp={}, 
langue={Français}, 
nbPrerequis={0}, 
descriptionCourte={true}, 
descriptionLongue={true}, 
objectifs={true}, 
ressources={true}, 
bibliographie={false}] 
{
Unité obligatoire. 
} 
{
Ce projet met en oeuvre les connaissances acquises au cours du semestre
par le biais de l'étude et de la réalisation d'une application J2EE ou de la
réalisation d'un projet d'architecture applicative répartie dont le sujet est
défini en collaboration avec un partenaire industriel.
} 
{} 
{\begin{itemize} 
  \ObjItem Maîtriser les aspects fonctionnels et techniques de la mise en oeuvre d'un
système d'information réparti.
\end{itemize} 
} 
{Ressources} 
{Biblio} 
 
\vfill

%==========================================================================================
\module[codeApogee={UE 41}, 
titre={Web Services et interopérabilité}, 
COURS={15}, 
TD={15}, 
TP={10}, 
CTD={}, 
TOTAL={40}, 
SEMESTRE={semestre 4}, 
COEFF={3}, 
ECTS={3}, 
MethodeEval={Contrôle continue et terminal}, 
ModalitesCCSemestreUn={CC et CT}, 
ModalitesCCSemestreDeux={CT}, 
%CalculNFSessionUne={$\frac{(CC+2*CT)}{3}$}, 
%CalculNFSessionDeux={CT}, 
NoteEliminatoire={7}, 
nomPremierResp={Frédéric MOAL},
emailPremierResp={Frederic.MOAL@univ-orleans.fr}, 
nomSecondResp={}, 
emailSecondResp={}, 
langue={Français}, 
nbPrerequis={1}, 
descriptionCourte={true}, 
descriptionLongue={true}, 
objectifs={true}, 
ressources={true}, 
bibliographie={false}] 
{
Unité obligatoire. 
} 
{
Ce module permet de comprendre les technologies et les architectures
sous-jacentes mises en oeuvre dans les architectures de type SOA :
\begin{itemize} 
\item Implementation des Services Web (SOAP, WSDL, UDDI...) ;
\item Exemple de mise en oeuvre en Java et .NET ;
\item Interopérabilité entre SI hétérogènes ;
\item Intégration d'application - Utilisation de bus ESB.
\end{itemize} 
} 
{Maîtrise des architectures distribuées, JEE/.NET} 
{\begin{itemize} 
  \ObjItem Comprendre les architectures orientées services et les technologies sousjacentes.
  \ObjItem Concevoir une architecture d'intégration de SI hétérogènes.
\end{itemize} 
} 
{Ressources} 
{Biblio} 
 
\vfill

%==========================================================================================
\module[codeApogee={UE 42}, 
titre={Intégration de SI}, 
COURS={30}, 
TD={10}, 
TP={}, 
CTD={}, 
TOTAL={40}, 
SEMESTRE={semestre 4}, 
COEFF={3}, 
ECTS={3}, 
MethodeEval={Contrôle continue et terminal}, 
ModalitesCCSemestreUn={CC et CT}, 
ModalitesCCSemestreDeux={CT}, 
%CalculNFSessionUne={$\frac{(CC+2*CT)}{3}$}, 
%CalculNFSessionDeux={CT}, 
NoteEliminatoire={7}, 
nomPremierResp={Prénom NOM}, 
emailPremierResp={Prenom.NOM@univ-orleans.fr}, 
nomSecondResp={}, 
emailSecondResp={}, 
langue={Français}, 
nbPrerequis={0}, 
descriptionCourte={true}, 
descriptionLongue={true}, 
objectifs={true}, 
ressources={true}, 
bibliographie={false}] 
{
Unité obligatoire. 
} 
{
\begin{description}
\item[Architecture d'Entreprise]
\begin{itemize}
\item Principes, Gouvernance, la Stratégie (métier) de l'entreprise
\item Capacités (organisation, acteurs, ...)
\item Architecture fonctionnelle, architecture du système d'information
\item Analyse de l'environnement courant et collecte de l'information
\item Architecture applicative, Modèle fonctionnel,
\item Entrepôts de données, Placement des données et fonctions,
\item Architecture technique, Analyse des gaps et planification des transitions.
\item TD : Définition d'une stratégie d'évolution ou de transformation, analyse
des capacités puis proposition d'architecture fonctionnelle et
d'adaptation organisationnelle.
\end{itemize}
\item[Architecture de solutions]
\begin{itemize}
\item Introduction à la notion d'Architecture de solutions.
\item Les besoins et contraintes, diagramme d'architecture général
\item Les aspects fonctionnels, les unités de déploiement, les aspects
opérationnels
\item Architecture Reference and Architecture Patterns
\item Les contraintes et qualités d'une Architecture (performance/capacité/disponibilité/...)
\item Documentation.
\end{itemize}
\item[Migration de données]
\end{description}
}
{} 
{\begin{itemize} 
  \ObjItem Introduction à la notion d'architecture et aux disciplines du rôle
d'architecte.
\ObjItem La partie "Architecture de solutions" a pour objectif de
donner un aperçu des concepts utilisés lors de la réalisation d'une
Architecture.
\end{itemize} 
} 
{Ressources} 
{Biblio} 
 
\vfill

%==========================================================================================
\module[codeApogee={UE 43}, 
titre={SI Assurance de personnes}, 
COURS={30}, 
TD={10}, 
TP={}, 
CTD={}, 
TOTAL={40}, 
SEMESTRE={semestre 4}, 
COEFF={3}, 
ECTS={3}, 
MethodeEval={Contrôle continue et terminal}, 
ModalitesCCSemestreUn={CC et CT}, 
ModalitesCCSemestreDeux={CT}, 
%CalculNFSessionUne={$\frac{(CC+2*CT)}{3}$}, 
%CalculNFSessionDeux={CT}, 
NoteEliminatoire={7}, 
nomPremierResp={Prénom NOM}, 
emailPremierResp={Prenom.NOM@univ-orleans.fr}, 
nomSecondResp={}, 
emailSecondResp={}, 
langue={Français}, 
nbPrerequis={0}, 
descriptionCourte={true}, 
descriptionLongue={true}, 
objectifs={true}, 
ressources={true}, 
bibliographie={false}] 
{
Unité obligatoire. 
} 
{
\begin{itemize} 
\item Particularités du référentiel
\item Le régime général Maladie
\item Catalogue de produits
\item Contrat
\item Cotisations
\item Sinistres
\item Résultats techniques.
\end{itemize} 
} 
{} 
{\begin{itemize} 
  \ObjItem Maîtriser les spécifités des SI et du fonctionnement des processus liés à un
contrat Prévoyance Lourde, à un contrat Santé et à un contrat Epargne,
dans le cadre de contrats individuels ou de contrats collectifs.
\end{itemize} 
} 
{Ressources} 
{Biblio} 
 
\vfill

%==========================================================================================
\module[codeApogee={UE 44}, 
titre={Management des SI}, 
COURS={30}, 
TD={10}, 
TP={}, 
CTD={}, 
TOTAL={40}, 
SEMESTRE={semestre 4}, 
COEFF={3}, 
ECTS={3}, 
MethodeEval={Contrôle continue et terminal}, 
ModalitesCCSemestreUn={CC et CT}, 
ModalitesCCSemestreDeux={CT}, 
%CalculNFSessionUne={$\frac{(CC+2*CT)}{3}$}, 
%CalculNFSessionDeux={CT}, 
NoteEliminatoire={7}, 
nomPremierResp={Prénom NOM}, 
emailPremierResp={Prenom.NOM@univ-orleans.fr}, 
nomSecondResp={}, 
emailSecondResp={}, 
langue={Français}, 
nbPrerequis={1}, 
descriptionCourte={true}, 
descriptionLongue={true}, 
objectifs={true}, 
ressources={true}, 
bibliographie={false}] 
{
Unité obligatoire. 
} 
{
Plusieurs intervenants professionnels du domaine abordent différentes
thématiques liées au management du SI dans l'entreprise :
\begin{itemize}
\item Dimensionner et aligner le système d'information (SI) - système
d'information informatisé (SII) au sein des processus - organisation du
Métier (MOA - AMOA) et Informatique (MOE) avec la stratégie de
l'Entreprise en environnement sous contraintes ;
\item Pilotage du SI SII par horizon (stratégique - tactique - opérationnel) et
contrainte (Valeur - Coût) ;
\item Intégration et évolution du SII au sein du SI - SII existant.
\item La gestion des services de production informatique.
\item Modèles, normes et standards existants (ITIL, ISO 17799 / 27002, ISO
20000, COBIT, CMMI).
\item TD portant sur les aspects méthodologiques de la mise en production
d'une application informatique.
\end{itemize} 
} 
{\begin{itemize} 
\item Ingénierie des SI.
\item Gestion de projets et qualité.
\item Analyse financière.
\end{itemize} 
}
{\begin{itemize} 
  \ObjItem Identifier les tendances, les évolutions, connaître les méthodes, tableaux
de bords et indicateurs permettant au décideur de piloter le SI (mesurer
l'atteinte des objectifs et analyser les écarts en termes de qualité,
supervision, sécurité, dimensionnement).
  \ObjItem Dimensionner - préparer - suivre - contrôler - optimiser le SI, SII.
  \ObjItem Avoir un aperçu des enjeux et des métiers de la production informatique
et à travers la présentation des principaux processus de gestion du service informatique et
les grand référentiels et normes reconnus sur le marché, représentant les
"bonnes pratiques" d'une DSI nécessaire pour assurer la conformité aux
multiples exigences actuelles.
\end{itemize} 
} 
{Ressources} 
{Biblio} 
 
\vfill

%==========================================================================================
\module[codeApogee={UE 45}, 
titre={Stratégie commerciale autour de l'Internet}, 
COURS={20}, 
TD={10}, 
TP={}, 
CTD={}, 
TOTAL={30}, 
SEMESTRE={semestre 4}, 
COEFF={3}, 
ECTS={3}, 
MethodeEval={Contrôle continue et terminal}, 
ModalitesCCSemestreUn={CC et CT}, 
ModalitesCCSemestreDeux={CT}, 
%CalculNFSessionUne={$\frac{(CC+2*CT)}{3}$}, 
%CalculNFSessionDeux={CT}, 
NoteEliminatoire={7}, 
nomPremierResp={Prénom NOM}, 
emailPremierResp={Prenom.NOM@univ-orleans.fr}, 
nomSecondResp={}, 
emailSecondResp={}, 
langue={Français}, 
nbPrerequis={0}, 
descriptionCourte={true}, 
descriptionLongue={true}, 
objectifs={true}, 
ressources={true}, 
bibliographie={false}] 
{
Unité obligatoire. 
} 
{
Pilotage stratégique de l'entreprise, avec Internet comme outil de pouvoir :
\begin{itemize} 
\item Les différents types d'environnement concurrentiels.
\item Analyse de l'entreprise et évaluation de l'impact d'Internet.
\item Choisir la bonne stratégie.
\item Web design et audit de sites.
\end{itemize} 
} 
{} 
{\begin{itemize} 
  \ObjItem Maîtriser les aspects stratégiques et marketing pour l'utilisation d'Internet.
\end{itemize} 
} 
{Ressources} 
{Biblio} 
 
\vfill

%==========================================================================================
\module[codeApogee={UE 46}, 
titre={Projet industriel}, 
COURS={10}, 
TD={}, 
TP={}, 
CTD={}, 
TOTAL={10}, 
SEMESTRE={semestre 4}, 
COEFF={3}, 
ECTS={3}, 
MethodeEval={Contrôle continue et terminal}, 
ModalitesCCSemestreUn={CC et CT}, 
ModalitesCCSemestreDeux={CT}, 
%CalculNFSessionUne={$\frac{(CC+2*CT)}{3}$}, 
%CalculNFSessionDeux={CT}, 
NoteEliminatoire={7}, 
nomPremierResp={Catherine JULIÉ-BONNET}, 
emailPremierResp={Catherine.JULIE-BONNET@univ-orleans.fr}, 
nomSecondResp={}, 
emailSecondResp={}, 
langue={Français}, 
nbPrerequis={0}, 
descriptionCourte={true}, 
descriptionLongue={true}, 
objectifs={true}, 
ressources={true}, 
bibliographie={false}] 
{
Unité obligatoire. 
} 
{
Ce projet permet de synthétiser toutes les connaissances acquises au
cours de l'année par le biais de l'étude et de la réalisation d'une
intégration d'applications dans un contexte hétérogène. Le sujet de ce
projet est défini en collaboration avec un partenaire industriel.
} 
{} 
{\begin{itemize} 
  \ObjItem Maîtriser tous les aspects (logiciels, matériels, sécurité, etc) de la mise en oeuvre d'un système d'information réparti et hétérogène.
\end{itemize} 
} 
{Ressources} 
{Biblio} 
 
\vfill

%==========================================================================================
\module[codeApogee={UE 47}, 
titre={Stage professionnel}, 
COURS={}, 
TD={}, 
TP={}, 
CTD={}, 
TOTAL={6 mois}, 
SEMESTRE={semestre 4}, 
COEFF={12}, 
ECTS={12}, 
MethodeEval={Contrôle continue et terminal}, 
ModalitesCCSemestreUn={CC et CT}, 
ModalitesCCSemestreDeux={CT}, 
%CalculNFSessionUne={$\frac{(CC+2*CT)}{3}$}, 
%CalculNFSessionDeux={CT}, 
NoteEliminatoire={7}, 
nomPremierResp={Catherine JULIÉ-BONNET}, 
emailPremierResp={Catherine.JULIE-BONNET@univ-orleans.fr}, 
nomSecondResp={}, 
emailSecondResp={}, 
langue={Français}, 
nbPrerequis={0}, 
descriptionCourte={true}, 
descriptionLongue={true}, 
objectifs={true}, 
ressources={true}, 
bibliographie={false}] 
{
Unité obligatoire. 
} 
{
Stage professionnel en entreprise de 6 mois.
} 
{} 
{\begin{itemize} 
  \ObjItem Finaliser son projet professionnel et préparer son recrutement via le stage.
\end{itemize} 
} 
{Ressources} 
{Biblio} 
 
\vfill
