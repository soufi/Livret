
%---------------------- % % % Personnalisation des couleurs % % % ----------- MOUTARDE --------
\definecolor{couleurFonce}{RGB}{170,110,0} % Couleur du Code APOGEE
\definecolor{couleurClaire}{RGB}{222,190,137} % Couleur du fond de la bande
\definecolor{couleurTexte}{RGB}{255,255,255} % Couleur du texte de la bande
%------------------------------------------------------------------------------------------



%==========================================================================================
% Semestre 3
%==========================================================================================
\module[codeApogee={UE 31}, 
titre={Architectures applicatives réparties}, 
COURS={20}, 
TD={20}, 
TP={10}, 
CTD={}, 
TOTAL={50}, 
SEMESTRE={semestre 3}, 
COEFF={4}, 
ECTS={4}, 
MethodeEval={Contrôle continue et terminal}, 
ModalitesCCSemestreUn={CC et CT}, 
ModalitesCCSemestreDeux={CT}, 
%CalculNFSessionUne={$\frac{(CC+2*CT)}{3}$}, 
%CalculNFSessionDeux={CT}, 
NoteEliminatoire={7},
nomPremierResp={Frédéric MOAL}, 
emailPremierResp={Frederic.MOAL@univ-orleans.fr}, 
nomSecondResp={Matthieu EXBRAYAT}, 
emailSecondResp={Matthieu.EXBRAYAT@univ-orleans.fr}, 
langue={Français},
nbPrerequis={1},
descriptionCourte={true}, 
descriptionLongue={true}, 
objectifs={true}, 
ressources={true}, 
bibliographie={false}] 
{
Unité obligatoire. 
} 
{
Ce module se compose de deux parties complémentaires, portant sur les
aspects théoriques et pratiques des systèmes d'information répartis:
Dans l'optique d'une approche fondée sur la pratique, le module aborde
dans un premier temps les outils disponibles, autour du langage Java et de
la plateforme Java Enterprise Edition (JavaEE) :
\begin{itemize} 
\item Communication répartie par invocation distante (RMI) et messagerie (JMS).
\item Persistance d'objets (utilisation de différents frameworks, dont hibernate).
\item Concept de bean métier (EJB).
\item Intégration des différents types d'EJB, interfaçage avec la couche présentation.
\item Bases de l'administration du serveur d'applications\\
Lorsque les aspects pratiques ont été manipulés, l'étudiant est amené à les
mettre en perspective et à rationaliser leur mise en oeuvre, par l'étude des
aspects "théoriques" suivants :
  \begin{itemize} 
  \item Typologie des SI et exemples significatifs
  \item UML et processus de développement unifié (rappels)
  \item Patron de conception ("Design Patterns")
  \item Concept de transaction répartie
  \end{itemize} 
\end{itemize} 
} 
{Compétences en bases de données, programmation java, conception d'applications web (servlets/JSP, modèle MVC2).} 
{\begin{itemize} 
  \ObjItem Fournir à l'étudiant les outils nécessaires à l'analyse, la mise en place et
l'exploitation de systèmes d'informations répartis.
  \ObjItem Apporter une solide
formation sur la répartition des données et des traitements dans un
Système d'Information (SI), suivant deux axes : outils avancés pour la
modélisation et la gestion des SI (UML, design patterns, etc.) et SI
distribués contemporains (architectures multi-tiers, plateformes
applicatives).
\end{itemize} 
} 
{Ressources}
{Biblio}

\vfill

%==========================================================================================
\module[codeApogee={UE 32}, 
titre={Extraction de connaissances dans les BD}, 
COURS={20}, 
TD={20}, 
TP={10}, 
CTD={}, 
TOTAL={50}, 
SEMESTRE={semestre 3}, 
COEFF={4}, 
ECTS={4}, 
MethodeEval={Contrôle continue et terminal}, 
ModalitesCCSemestreUn={CC et CT}, 
ModalitesCCSemestreDeux={CT}, 
%CalculNFSessionUne={$\frac{(CC+2*CT)}{3}$}, 
%CalculNFSessionDeux={CT}, 
NoteEliminatoire={7}, 
nomPremierResp={Christel VRAIN}, 
emailPremierResp={Christel.VRAIN@univ-orleans.fr}, 
nomSecondResp={}, 
emailSecondResp={}, 
langue={Français}, 
nbPrerequis={1}, 
descriptionCourte={true}, 
descriptionLongue={true}, 
objectifs={true}, 
ressources={true}, 
bibliographie={false}] 
{
Unité obligatoire. 
} 
{
\begin{itemize} 
\item Les différents types d'apprentissage et les différentes tâches
\item Classification supervisée : arbre de décision, modèles probabilistes,
machines à vecteur support, noyaux
\item Evaluation des modèles
\item Classification non supervisée : par partitionnement, hiérarchique,
conceptuelle
\item Recherche de règles d'association
\end{itemize} 
Savoir-faire:
\begin{itemize} 
\item Utilisation d'outils : Weka, RapidMiner
\item Définir le problème d'apprentissage : modèle à acquérir, données
nécessaires, techniques applicables
\item Appliquer des techniques d'apprentissage.
Etudes de cas guidés par des applications.
\end{itemize} 
} 
{UE Intelligence Artificielle - UE Analyse de données.} 
{\begin{itemize} 
  \ObjItem Acquérir les notions de base permettant de résoudre un problème
d'Extraction de Connaissances dans les Bases de Données : modélisation de la tâche à effectuer, préparation des données, choix des outils.\\
Le Data
Mining est devenu une discipline fondamentale, justifiée par le
développement de grandes bases de données, la nécessité de constituer
des mémoires d'entreprises et le besoin d'outils intelligents pour exploiter
les flux d'information sur le web.
\end{itemize} 
} 
{Ressources} 
{Biblio} 
 
\vfill

%==========================================================================================
\module[codeApogee={UE 33}, 
titre={Sécurité et protocoles}, 
COURS={20}, 
TD={20}, 
TP={}, 
CTD={}, 
TOTAL={40}, 
SEMESTRE={semestre 3}, 
COEFF={4}, 
ECTS={4}, 
MethodeEval={Contrôle continue et terminal}, 
ModalitesCCSemestreUn={CC et CT}, 
ModalitesCCSemestreDeux={CT}, 
%CalculNFSessionUne={$\frac{(CC+2*CT)}{3}$}, 
%CalculNFSessionDeux={CT}, 
NoteEliminatoire={7}, 
nomPremierResp={Prénom NOM}, 
emailPremierResp={Prenom.NOM@univ-orleans.fr}, 
nomSecondResp={}, 
emailSecondResp={}, 
langue={Français}, 
nbPrerequis={1}, 
descriptionCourte={true}, 
descriptionLongue={true}, 
objectifs={true}, 
ressources={true}, 
bibliographie={false}] 
{
petite-DESCRIPTION 
} 
{
Ce module introduit et familiarise les étudiants avec des notions de
sécurité relatives aux communications.

Après un survol historique de la compétition perpétuelle entre
cryptographie et cryptanalyse, avec comme point clé la seconde guerre
mondiale et le système Enigma, les standards actuels de chiffrements
symétriques et asymétriques sont étudiés en profondeur. Les technologies
actuelles permettent des communications synchrones sur des distances de
plusieurs centaines ou milliers de kilomètres. De plus ces communications
peuvent contenir des informations confidentielles et peuvent également
nécessiter une authentification des personnes en communication
(communication entre un tiers et un centre de paiement par exemple). Des
protocoles de sécurité sont développés afin de garantir les propriétés
mentionnées précédemment.

Ce module présente les mécanismes d'authentification, de confidentialité
et d'intégrité de données et ainsi que quelques protocoles comme
Kerberos. Enfin, le dernier point traité dans ce module démontre que des
algorithmes de chiffrement parfaits ne suffisent pas pour garantir la
sécurité d'un protocole de communication. Les étudiants sont invités à
manipuler un outil de simulation et de vérification de protocoles de
sécurité afin de détecter d'éventuelles failles logiques de conception.
} 
{Notions de mathématiques discrètes, réseaux.} 
{\begin{itemize} 
  \ObjItem Comprendre en profondeur les mécanismes garantissant la sécurité des
systèmes et réseaux en termes de confidentialité, d'authentification et de
disponibilité.
  \ObjItem Modéliser un protocole à partir d'une spécification textuelle et
manipuler un outil de simulation et de vérification.
\end{itemize} 
} 
{Ressources} 
{Biblio} 
 
\vfill

%==========================================================================================
\module[codeApogee={UE 34}, 
titre={Réseaux et infrastructures de production}, 
COURS={10}, 
TD={}, 
TP={30}, 
CTD={}, 
TOTAL={40}, 
SEMESTRE={semestre 3}, 
COEFF={4}, 
ECTS={4}, 
MethodeEval={Contrôle continue et terminal}, 
ModalitesCCSemestreUn={CC et CT}, 
ModalitesCCSemestreDeux={CT}, 
%CalculNFSessionUne={$\frac{(CC+2*CT)}{3}$}, 
%CalculNFSessionDeux={CT}, 
NoteEliminatoire={7}, 
nomPremierResp={Prénom NOM}, 
emailPremierResp={Prenom.NOM@univ-orleans.fr}, 
nomSecondResp={}, 
emailSecondResp={}, 
langue={Français}, 
nbPrerequis={1}, 
descriptionCourte={true}, 
descriptionLongue={true}, 
objectifs={true}, 
ressources={true}, 
bibliographie={false}] 
{
Unité obligatoire. 
} 
{
Ce cours explore les différentes architectures que l'on peut mettre en
oeuvre dans le cadre d'une informatique Client/Serveur. Il fait l'inventaire
de la plupart des problèmes rencontrés et décrit les technologies
(Applicatives, Systèmes et Réseaux) qui composent les infrastructures et
qui permettent le déploiement harmonieux et la gestion efficace d'un
système d'information moderne.

Réseaux de communication :
\begin{itemize}
\item réseaux locaux : architectures, techniques d'accès protocoles de réseau, protocoles de transport
\item interconnexion de réseaux hétérogènes
\item administration des réseaux
\item conférences
\end{itemize} 
Applications réparties :
\begin{itemize}
\item programmation des applications réparties
\item protocoles d'applications
\item spécification formelle et validation de protocoles
\item allocation de ressource, synchronisation
\item les systèmes transactionnels
\item techniques internet.
\end{itemize} 
} 
{Modules Réseaux du Master 1.} 
{\begin{itemize} 
  \ObjItem Maîtriser l'installation, le déploiement et la maintenance d'un SI.
\end{itemize} 
} 
{Ressources} 
{Biblio} 
 
\vfill

%==========================================================================================
\module[codeApogee={UE 35}, 
titre={Gestion de projet et qualité}, 
COURS={20}, 
TD={20}, 
TP={10}, 
CTD={}, 
TOTAL={50}, 
SEMESTRE={semestre 3}, 
COEFF={4}, 
ECTS={4}, 
MethodeEval={Contrôle continue et terminal}, 
ModalitesCCSemestreUn={CC et CT}, 
ModalitesCCSemestreDeux={CT}, 
%CalculNFSessionUne={$\frac{(CC+2*CT)}{3}$}, 
%CalculNFSessionDeux={CT}, 
NoteEliminatoire={7}, 
nomPremierResp={Prénom NOM}, 
emailPremierResp={Prenom.NOM@univ-orleans.fr}, 
nomSecondResp={}, 
emailSecondResp={}, 
langue={Français}, 
nbPrerequis={0}, 
descriptionCourte={true}, 
descriptionLongue={true}, 
objectifs={true}, 
ressources={true}, 
bibliographie={false}] 
{
Unité obligatoire. 
} 
{
\begin{itemize}
\item Les étapes d'un projet
\item Les livrables de la conduite de projet
\item Les rôles et responsabilités des acteurs (client, décideurs, maîtrise d'ouvrage, maîtrise d'oeuvre, assurance qualité)
\item Le travail en équipe
\item Méthodes de prévision et d'estimation des charges et de délais
\item Tests
\item Normalisation ISO / maturité CMM
\item La gestion des risques
\item La gestion des évolutions
\item Le plan de communication
\item Les outils du chef de projet.
\end{itemize}
} 
{} 
{\begin{itemize} 
  \ObjItem Etudier la conduite de projet et la mise en place d'un système qualité.
\end{itemize} 
} 
{Ressources} 
{Biblio} 
 
\vfill

%==========================================================================================
\module[codeApogee={UE 36}, 
titre={Gestion des ressources humaines}, 
COURS={10}, 
TD={20}, 
TP={}, 
CTD={}, 
TOTAL={30}, 
SEMESTRE={semestre 3}, 
COEFF={2}, 
ECTS={2}, 
MethodeEval={Contrôle continue et terminal}, 
ModalitesCCSemestreUn={CC et CT}, 
ModalitesCCSemestreDeux={CT}, 
%CalculNFSessionUne={$\frac{(CC+2*CT)}{3}$}, 
%CalculNFSessionDeux={CT}, 
NoteEliminatoire={7}, 
nomPremierResp={Prénom NOM}, 
emailPremierResp={Prenom.NOM@univ-orleans.fr}, 
nomSecondResp={}, 
emailSecondResp={}, 
langue={Français}, 
nbPrerequis={0}, 
descriptionCourte={true}, 
descriptionLongue={true}, 
objectifs={true}, 
ressources={true}, 
bibliographie={false}] 
{
Unité obligatoire. 
} 
{
\begin{itemize}
\item Introduction à la gestion des ressources humaines
\item La DRH en entreprise
\item Le management des hommes et des équipes
\item Evaluation des postes
\item Le recrutement
\item Réforme de la formation professionnelle.
\end{itemize}
} 
{} 
{\begin{itemize} 
  \ObjItem Appréhender la gestion des ressources humaines dans le cadre de l'entreprise.  
\end{itemize}
} 
{Ressources} 
{Biblio} 
 
\vfill

%==========================================================================================
\module[codeApogee={UE 37}, 
titre={Création d'entreprise}, 
COURS={20}, 
TD={10}, 
TP={}, 
CTD={}, 
TOTAL={30}, 
SEMESTRE={semestre 3}, 
COEFF={2}, 
ECTS={2}, 
MethodeEval={Contrôle continue et terminal}, 
ModalitesCCSemestreUn={CC et CT}, 
ModalitesCCSemestreDeux={CT}, 
%CalculNFSessionUne={$\frac{(CC+2*CT)}{3}$}, 
%CalculNFSessionDeux={CT}, 
NoteEliminatoire={7}, 
nomPremierResp={Prénom NOM}, 
emailPremierResp={Prenom.NOM@univ-orleans.fr}, 
nomSecondResp={}, 
emailSecondResp={}, 
langue={Français}, 
nbPrerequis={0}, 
descriptionCourte={true}, 
descriptionLongue={true}, 
objectifs={true}, 
ressources={true}, 
bibliographie={false}] 
{
Unité obligatoire. 
} 
{
Présentation complète du cadre économique, juridique, financier et
commercial des deux modalités de la création d'entreprise, la création "ex
nihilo" et la "reprise" d'une entreprise dans le cadre d'une transmission.
} 
{} 
{\begin{itemize} 
  \ObjItem Etre capable de réaliser un dossier économique de création d'une
entreprise afin de se familiariser avec la présentation dynamique d'un
projet à des investisseurs potentiels.
\end{itemize} 
} 
{Ressources} 
{Biblio} 
 
\vfill

%==========================================================================================
\module[codeApogee={UE 38}, 
titre={Anglais professionnel}, 
COURS={}, 
TD={24}, 
TP={}, 
CTD={}, 
TOTAL={24}, 
SEMESTRE={semestre 3}, 
COEFF={3}, 
ECTS={3}, 
MethodeEval={Contrôle continue et terminal}, 
ModalitesCCSemestreUn={CC et CT}, 
ModalitesCCSemestreDeux={CT}, 
%CalculNFSessionUne={$\frac{(CC+2*CT)}{3}$}, 
%CalculNFSessionDeux={CT}, 
NoteEliminatoire={7}, 
nomPremierResp={Prénom NOM}, 
emailPremierResp={Prenom.NOM@univ-orleans.fr}, 
nomSecondResp={}, 
emailSecondResp={}, 
langue={Français}, 
nbPrerequis={1}, 
descriptionCourte={true}, 
descriptionLongue={true}, 
objectifs={true}, 
ressources={true}, 
bibliographie={false}] 
{
Unité obligatoire. 
} 
{
Travailler la langue écrite et orale afin de perfectionner ses capacités à
communiquer en milieu professionnel. Suivi interactif et personnalisé du
projet industriel.

Suivi de projet industriel :
\begin{itemize}
\item l'étudiant produit plusieurs devoirs pour lequel il reçoit une correction individuelle.
\item Présentation orale
\item Devoir écrit (compréhension et synthèse)
\item Poursuite du travail sur le CV, la lettre de motivation et l'entretien.
\end{itemize}
Supports : Publications les sciences et technologies informatiques, documents
sonores, centre de ressources multimedia.
} 
{Avoir suivi les modules d'anglais de Master 1 ou 550 heures de formation
équivalente.} 
{\begin{itemize} 
  \ObjItem Se rapprocher des compétences demandées au niveau C1; possibilité de
valider le CLES 3.
  \ObjItem L'étudiant comprend des textes longs dans sa discipline, rend compte
oralement et par écrit de documents professionnels, s'exprime dans une
langue claire et structurée sur des sujets complexes à caractère social et
académique.
\end{itemize} 
} 
{Ressources} 
{Biblio} 
 
\vfill

%==========================================================================================
\module[codeApogee={UE 39}, 
titre={Projet industriel}, 
COURS={10}, 
TD={}, 
TP={}, 
CTD={}, 
TOTAL={}, 
SEMESTRE={semestre 3}, 
COEFF={3}, 
ECTS={3}, 
MethodeEval={Contrôle continue et terminal}, 
ModalitesCCSemestreUn={CC et CT}, 
ModalitesCCSemestreDeux={CT}, 
%CalculNFSessionUne={$\frac{(CC+2*CT)}{3}$}, 
%CalculNFSessionDeux={CT}, 
NoteEliminatoire={7}, 
nomPremierResp={Catherine JULIÉ-BONNET}, 
emailPremierResp={Catherine.JULIE-BONNET@univ-orleans.fr}, 
nomSecondResp={},
emailSecondResp={}, 
langue={Français}, 
nbPrerequis={0}, 
descriptionCourte={true}, 
descriptionLongue={true}, 
objectifs={true}, 
ressources={true}, 
bibliographie={false}] 
{
Unité obligatoire. 
} 
{
Ce projet met en oeuvre les connaissances acquises au cours du semestre
par le biais de l'étude et de la réalisation d'une application J2EE ou de la
réalisation d'un projet d'architecture applicative répartie dont le sujet est
défini en collaboration avec un partenaire industriel.
} 
{} 
{\begin{itemize} 
  \ObjItem Maîtriser les aspects fonctionnels et techniques de la mise en oeuvre d'un
système d'information réparti.
\end{itemize} 
} 
{Ressources} 
{Biblio} 
 
\vfill

%==========================================================================================
\module[codeApogee={UE 41}, 
titre={Web Services et interopérabilité}, 
COURS={15}, 
TD={15}, 
TP={10}, 
CTD={}, 
TOTAL={40}, 
SEMESTRE={semestre 4}, 
COEFF={3}, 
ECTS={3}, 
MethodeEval={Contrôle continue et terminal}, 
ModalitesCCSemestreUn={CC et CT}, 
ModalitesCCSemestreDeux={CT}, 
%CalculNFSessionUne={$\frac{(CC+2*CT)}{3}$}, 
%CalculNFSessionDeux={CT}, 
NoteEliminatoire={7}, 
nomPremierResp={Frédéric MOAL},
emailPremierResp={Frederic.MOAL@univ-orleans.fr}, 
nomSecondResp={}, 
emailSecondResp={}, 
langue={Français}, 
nbPrerequis={1}, 
descriptionCourte={true}, 
descriptionLongue={true}, 
objectifs={true}, 
ressources={true}, 
bibliographie={false}] 
{
Unité obligatoire. 
} 
{
Ce module permet de comprendre les technologies et les architectures
sous-jacentes mises en oeuvre dans les architectures de type SOA :
\begin{itemize} 
\item Implementation des Services Web (SOAP, WSDL, UDDI...) ;
\item Exemple de mise en oeuvre en Java et .NET ;
\item Interopérabilité entre SI hétérogènes ;
\item Intégration d'application - Utilisation de bus ESB.
\end{itemize} 
} 
{Maîtrise des architectures distribuées, JEE/.NET} 
{\begin{itemize} 
  \ObjItem Comprendre les architectures orientées services et les technologies sousjacentes.
  \ObjItem Concevoir une architecture d'intégration de SI hétérogènes.
\end{itemize} 
} 
{Ressources} 
{Biblio} 
 
\vfill

%==========================================================================================
\module[codeApogee={UE 42}, 
titre={Fouille de données et de textes}, 
COURS={15}, 
TD={15}, 
TP={10}, 
CTD={}, 
TOTAL={40}, 
SEMESTRE={semestre 4}, 
COEFF={3}, 
ECTS={3}, 
MethodeEval={Contrôle continue et terminal}, 
ModalitesCCSemestreUn={CC et CT}, 
ModalitesCCSemestreDeux={CT}, 
%CalculNFSessionUne={$\frac{(CC+2*CT)}{3}$}, 
%CalculNFSessionDeux={CT}, 
NoteEliminatoire={7}, 
nomPremierResp={Christel VRAIN}, 
emailPremierResp={Christel.VRAIN@univ-orleans.fr}, 
nomSecondResp={}, 
emailSecondResp={}, 
langue={Français}, 
nbPrerequis={1}, 
descriptionCourte={true}, 
descriptionLongue={true}, 
objectifs={true}, 
ressources={true}, 
bibliographie={false}] 
{
Unité obligatoire. 
} 
{
\begin{enumerate} 
\item Algorithme de fouille de données (approfondissement)
\begin{itemize} 
\item apprentissage statistique et analyse de données
\item apprentissage symbolique (induction, reformulation, ...)
\end{itemize} 
\item Extraction de Connaissances à partir de textes.
\end{enumerate} 
} 
{UE Extraction de connaissances dans les bases de données.} 
{\begin{itemize} 
\ObjItem Approfondir certaines techniques présentées dans le module Extraction de Connaissances dans les BD.
\ObjItem Élargir la problématique à des types de données complexes comme les données spatiales ou textuelles.
\end{itemize} 
} 
{Ressources} 
{Biblio} 
 
\vfill

%==========================================================================================
\module[codeApogee={UE 43}, 
titre={Sécurité des Réseaux}, 
COURS={20}, 
TD={}, 
TP={15}, 
CTD={}, 
TOTAL={35}, 
SEMESTRE={semestre 4}, 
COEFF={3}, 
ECTS={3}, 
MethodeEval={Contrôle continue et terminal}, 
ModalitesCCSemestreUn={CC et CT}, 
ModalitesCCSemestreDeux={CT}, 
%CalculNFSessionUne={$\frac{(CC+2*CT)}{3}$}, 
%CalculNFSessionDeux={CT}, 
NoteEliminatoire={7}, 
nomPremierResp={Prénom NOM}, 
emailPremierResp={Prenom.NOM@univ-orleans.fr}, 
nomSecondResp={}, 
emailSecondResp={}, 
langue={Français}, 
nbPrerequis={0}, 
descriptionCourte={true}, 
descriptionLongue={true}, 
objectifs={true}, 
ressources={true}, 
bibliographie={false}] 
{
Unité obligatoire. 
} 
{
L'enseignement est assuré par plusieurs intervenants extérieurs qui
abordent des thèmes comme : les réseaux ATM, le protocole IPv6, les
réseaux mobiles, la protection des réseaux contre le piratage, les ACL, les
architectures réseaux sécurisées, les infrastructures à clés publiques, la
sécurité dans les réseaux de télécommunications, dans les réseaux
mobiles (GSM, UMTS, ...), les services à cartes (télécarte, carte bancaire,
télévision à péage, ...), les protocoles IPSEC/IKE, les failles de sécurité,
l'analyse de risques, les alertes, les remontées d'alarmes, la détection
d'intrusions, les techniques de piratage, les firewalls.
} 
{} 
{\begin{itemize} 
    \ObjItem **************
\end{itemize} 
} 
{Ressources} 
{Biblio} 
 
\vfill

%==========================================================================================
\module[codeApogee={UE 44}, 
titre={Management des SI}, 
COURS={30}, 
TD={10}, 
TP={}, 
CTD={}, 
TOTAL={40}, 
SEMESTRE={semestre 4}, 
COEFF={3}, 
ECTS={3}, 
MethodeEval={Contrôle continue et terminal}, 
ModalitesCCSemestreUn={CC et CT}, 
ModalitesCCSemestreDeux={CT}, 
%CalculNFSessionUne={$\frac{(CC+2*CT)}{3}$}, 
%CalculNFSessionDeux={CT}, 
NoteEliminatoire={7}, 
nomPremierResp={Prénom NOM}, 
emailPremierResp={Prenom.NOM@univ-orleans.fr}, 
nomSecondResp={}, 
emailSecondResp={}, 
langue={Français}, 
nbPrerequis={1}, 
descriptionCourte={true}, 
descriptionLongue={true}, 
objectifs={true}, 
ressources={true}, 
bibliographie={false}] 
{
Unité obligatoire. 
} 
{
Plusieurs intervenants professionnels du domaine abordent différentes
thématiques liées au management du SI dans l'entreprise :
\begin{itemize}
\item Dimensionner et aligner le système d'information (SI) - système
d'information informatisé (SII) au sein des processus - organisation du
Métier (MOA - AMOA) et Informatique (MOE) avec la stratégie de
l'Entreprise en environnement sous contraintes ;
\item Pilotage du SI SII par horizon (stratégique - tactique - opérationnel) et
contrainte (Valeur - Coût) ;
\item Intégration et évolution du SII au sein du SI - SII existant.
\item La gestion des services de production informatique.
\item Modèles, normes et standards existants (ITIL, ISO 17799 / 27002, ISO
20000, COBIT, CMMI).
\item TD portant sur les aspects méthodologiques de la mise en production
d'une application informatique.
\end{itemize} 
} 
{\begin{itemize} 
\item Ingénierie des SI.
\item Gestion de projets et qualité.
\item Analyse financière.
\end{itemize} 
}
{\begin{itemize} 
  \ObjItem Identifier les tendances, les évolutions, connaître les méthodes, tableaux
de bords et indicateurs permettant au décideur de piloter le SI (mesurer
l'atteinte des objectifs et analyser les écarts en termes de qualité,
supervision, sécurité, dimensionnement).
  \ObjItem Dimensionner - préparer - suivre - contrôler - optimiser le SI, SII.
  \ObjItem Avoir un aperçu des enjeux et des métiers de la production informatique
et à travers la présentation des principaux processus de gestion du service informatique et
les grand référentiels et normes reconnus sur le marché, représentant les
"bonnes pratiques" d'une DSI nécessaire pour assurer la conformité aux
multiples exigences actuelles.
\end{itemize} 
} 
{Ressources} 
{Biblio} 
 
\vfill

%==========================================================================================
\module[codeApogee={UE 45}, 
titre={Stratégie commerciale autour de l'Internet}, 
COURS={20}, 
TD={10}, 
TP={}, 
CTD={}, 
TOTAL={30}, 
SEMESTRE={semestre 4}, 
COEFF={3}, 
ECTS={3}, 
MethodeEval={Contrôle continue et terminal}, 
ModalitesCCSemestreUn={CC et CT}, 
ModalitesCCSemestreDeux={CT}, 
%CalculNFSessionUne={$\frac{(CC+2*CT)}{3}$}, 
%CalculNFSessionDeux={CT}, 
NoteEliminatoire={7}, 
nomPremierResp={Prénom NOM}, 
emailPremierResp={Prenom.NOM@univ-orleans.fr}, 
nomSecondResp={}, 
emailSecondResp={}, 
langue={Français}, 
nbPrerequis={0}, 
descriptionCourte={true}, 
descriptionLongue={true}, 
objectifs={true}, 
ressources={true}, 
bibliographie={false}] 
{
Unité obligatoire. 
} 
{
Pilotage stratégique de l'entreprise, avec Internet comme outil de pouvoir :
\begin{itemize} 
\item Les différents types d'environnement concurrentiels.
\item Analyse de l'entreprise et évaluation de l'impact d'Internet.
\item Choisir la bonne stratégie.
\item Web design et audit de sites.
\end{itemize} 
} 
{} 
{\begin{itemize} 
  \ObjItem Maîtriser les aspects stratégiques et marketing pour l'utilisation d'Internet.
\end{itemize} 
} 
{Ressources} 
{Biblio} 
 
\vfill

%==========================================================================================
\module[codeApogee={UE 46}, 
titre={Projet industriel}, 
COURS={10}, 
TD={}, 
TP={}, 
CTD={}, 
TOTAL={10}, 
SEMESTRE={semestre 4}, 
COEFF={3}, 
ECTS={3}, 
MethodeEval={Contrôle continue et terminal}, 
ModalitesCCSemestreUn={CC et CT}, 
ModalitesCCSemestreDeux={CT}, 
%CalculNFSessionUne={$\frac{(CC+2*CT)}{3}$}, 
%CalculNFSessionDeux={CT}, 
NoteEliminatoire={7}, 
nomPremierResp={Catherine JULIÉ-BONNET}, 
emailPremierResp={Catherine.JULIE-BONNET@univ-orleans.fr}, 
nomSecondResp={}, 
emailSecondResp={}, 
langue={Français}, 
nbPrerequis={0}, 
descriptionCourte={true}, 
descriptionLongue={true}, 
objectifs={true}, 
ressources={true}, 
bibliographie={false}] 
{
Unité obligatoire. 
} 
{
Ce projet permet de synthétiser toutes les connaissances acquises au
cours de l'année par le biais de l'étude et de la réalisation d'une
intégration d'applications dans un contexte hétérogène. Le sujet de ce
projet est défini en collaboration avec un partenaire industriel.
} 
{} 
{\begin{itemize} 
  \ObjItem Maîtriser tous les aspects (logiciels, matériels, sécurité, etc) de la mise en oeuvre d'un système d'information réparti et hétérogène.
\end{itemize} 
} 
{Ressources} 
{Biblio} 
 
\vfill

%==========================================================================================
\module[codeApogee={UE 47}, 
titre={Stage professionnel}, 
COURS={}, 
TD={}, 
TP={}, 
CTD={}, 
TOTAL={6 mois}, 
SEMESTRE={semestre 4}, 
COEFF={12}, 
ECTS={12}, 
MethodeEval={Contrôle continue et terminal}, 
ModalitesCCSemestreUn={CC et CT}, 
ModalitesCCSemestreDeux={CT}, 
%CalculNFSessionUne={$\frac{(CC+2*CT)}{3}$}, 
%CalculNFSessionDeux={CT}, 
NoteEliminatoire={7}, 
nomPremierResp={Catherine JULIÉ-BONNET}, 
emailPremierResp={Catherine.JULIE-BONNET@univ-orleans.fr}, 
nomSecondResp={}, 
emailSecondResp={}, 
langue={Français}, 
nbPrerequis={0}, 
descriptionCourte={true}, 
descriptionLongue={true}, 
objectifs={true}, 
ressources={true}, 
bibliographie={false}] 
{
Unité obligatoire. 
} 
{
Stage professionnel en entreprise de 6 mois.
} 
{} 
{\begin{itemize} 
  \ObjItem Finaliser son projet professionnel et préparer son recrutement via le stage.
\end{itemize} 
} 
{Ressources} 
{Biblio} 
 
\vfill


%==========================================================================================
