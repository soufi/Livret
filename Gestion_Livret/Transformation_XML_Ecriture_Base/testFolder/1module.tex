\module[codeApogee={6ST01}, 
titre={Projet de fin d'étude}, 
COURS={}, 
TD={}, 
TP={}, 
CTD={4}, 
TOTAL={4}, 
SEMESTRE={Semestre 6}, 
COEFF={8}, 
ECTS={8}, 
MethodeEval={Contrôle continu et terminal}, 
ModalitesCCSemestreUn={CC et CT}, 
ModalitesCCSemestreDeux={CT}, 
%CalculNFSessionUne={$\frac{(CC+2*CT)}{3}$}, 
%CalculNFSessionDeux={CT}, 
NoteEliminatoire={}, 
nomPremierResp={Pierre Debs}, 
emailPremierResp={pierre.debs@univ-orleans.fr}, 
nomSecondResp={},
emailSecondResp={}, 
langue={Français}, 
nbPrerequis={0}, 
descriptionCourte={true}, 
descriptionLongue={true}, 
objectifs={true}, 
ressources={false}, 
bibliographie={false}] 
{
Unité obligatoire. 
} 
{}
{
\begin{description}
\item[Microéconomie] Le cours présente les principes de comportement des entreprises en concurrence imparfaite (monopole discriminant, concurrence monopolistique, duopole, oligopole). Il présente des instruments permettant d'analyser la concurrence sur un marché, et étudie les conséquences des imperfections de concurrence, tant normatives (intervention publique) que positives (rigidité macroéconomique des prix). Ce cours constitue une introduction à l'économie industrielle.
\item[Comportements stratégiques] Ce cours dispensé en travaux dirigés se proposent d'illustrer l'enseignement de microéconomie au moyen de jeux économiques (dilemme du prisonnier, enchères, « concours de beauté »). Il se termine par une étude sectorielle d'économie industrielle.
\end{description}
}
{}
{Ressources} 
{Biblio} 
 
\vfill
